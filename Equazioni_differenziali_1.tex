\documentclass[a4paper,11pt]{report}

% Geometria del foglio
\usepackage[a4paper, top=2.5cm, bottom=2cm, left=1.5cm, right=1.5cm]{geometry}

% Lingua documento
\usepackage[italian]{babel}
\usepackage[utf8]{inputenc}

\usepackage{lipsum}
\usepackage{datetime}

% Frontespizio, Head e footer delle pagine
\usepackage[]{fancyhdr}
%\pagestyle{fancy}
\fancyhead{} % cancella tutti i campi

\fancyfoot[C]{\thepage}


% Includere immagini 
\usepackage{graphicx}
\usepackage{adjustbox}
\usepackage{floatflt}
\usepackage{float}
\usepackage{wrapfig}
\usepackage{caption}
\usepackage{subcaption}
\usepackage[]{xcolor}

% Pacchetti per la matematica
\usepackage{amsmath}
\usepackage{amsfonts}
\usepackage{amssymb}
\usepackage{gensymb}

\usepackage{multicol}
\usepackage{framed}

\title{Dispense di Differenziali}
\author{Luca Colombo Gomez}
\date{}

\newcommand{\R}{\mathbb{R}}
\newcommand{\Rn}{\mathbb{R}^n}
\newcommand{\fourier}{\emph{F}}
\newcommand{\x}{\bar{x}}
\newcommand{\xp}{\bar{x}'}
\newcommand{\y}{\bar{y}}
\newcommand{\yp}{\bar{y}'}
\newcommand{\kk}{\bar{k}}
\newcommand{\kp}{\bar{k}'}
\newcommand{\z}{\bar{z}}
\newcommand{\n}{\bar{n}}

\makeindex

\begin{document}
\tableofcontents
\chapter{Distribuzioni e trasformate di Fourier}
\section{Distribuzioni}
le distribuzioni ( funzioni generalizzate) sono degli oggetti che generalizzano le funzione e le distribuzioni di probabilità. 
Estendono il concetto di derivata a tutte le funzioni continue e oltre.
le distribuzioni sono importanti in fisica (p.e. distribuzione delta di Dirac).

\subsection{Idea di base}
$f:\R\rightarrow\R $ funzione integrabile
$ \phi : \R \rightarrow \R $ smooth $\left(C^{\infty}\right)$, con supporto compatto.
$\rightarrow \int f \phi \mathrm{dx} \in \mathbb{R}$, dipende linearmente e in un modo continuo da $\phi$.
$\Rightarrow$ f $\grave{e}$ un funzionale lineare continuo sullo spazio di tutte le "funzioni test"$\phi.$ Questa è la definizione di una distribuzione.\\
Le distribuzioni possono essere moltiplicate con dei numeri reali, e possono essere sommate $\rightarrow$ formano uno spazio vettoriale reale.
\subsection{Derivata di una distribuione}
Considera prima il caso di una funzione $f:\R \rightarrow \R$ differenziabile. Se $\phi$ è una funzione test, abbiamo:

$$\int_{\R}f'\phi dx = -\int_{\R}f\phi ' dx$$

non c'è un termine di bordo perchè $\phi $ ha supporto compatto. $\rightarrow $ suggerisce la seguente definizione della derivata S' di una distribuzione 
S : S' = funzionale lineare che manda la funzione test $\phi$ in $-S\left(\phi\right)$

\subsection{Delta di Dirac}
("Funzione delta di Dirac") $\delta(x)$ è la distribuzione che manda la funzione test $\phi$ in $\phi(0)$. È la derivata della funzione step di Heaviside.
$$
H\left(x\right)=\left\{ \begin{matrix}
0 & x < 0 \\
1 & x \geq 0
\end{matrix}\right. $$
La derivata della delta di Dirac è la distribuzione che manda $\phi$ in $-\phi '(0)$. la delta è un esempio di una distribuzione che non è una funzione, ma può essere definita come limite di una seuenza di funzioni, p.e.
$$ \delta\left(x\right) = \lim_{a\to 0}\delta_a\left(x\right) $$
$$ \delta_a\left(x\right) = \left\{\begin{matrix}
\dfrac{1}{2a} && -a\leq x\leq a\\
0 && |x| > a
\end{matrix}\right.$$

Dimostrazione

$$
\int_R \delta_a (x)\Phi(x) dx= \int_{-a}^{a}\dfrac{1}{2a}\Phi(x) dx = \dfrac{1}{2a}\left(\psi(a)-\psi(-a)\right)
\quad \psi= \int \Phi,\quad \psi '=\Phi \qquad\blacksquare$$
$$
\Rightarrow \lim_{a\to 0}\int_{R}\delta_a(x)\Phi(x)dx=lim_{a\to 0}\dfrac{\psi(a)-\psi(-a)}{2a}=\psi '(0)
$$

\subsection{Definizione formale}
Def: una funzione $\Phi : U \rightarrow \mathbb{R} $ ha \underline{supporto compatto} se esiste un sottoinsieme compatto K di U tale che $\Phi(x)=0 \forall x \in U\backslash K $\\
Le funzioni $ C^{\infty} \Phi :U\rightarrow \mathbb{R}$ con supporto compatto formano uno spazio vettoriale topologico $\mathbf{D}(U)$\\

\underline{Def:} lo spazio delle \underline{distribuizioni} su $U \subseteq \mathbb{R}^n$ è il \underline{duale} $D\sp\prime(U)$ dello spazio vettoriale topologico D(U) di funzioni $C^{\infty}$ con supporto compatto in U.\\

Notazione: $$S \in D\sp\prime(U), \quad
\phi \in D(U), \quad
S : D(U)\rightarrow \mathbb{R}, \quad 
	\Phi \mapsto S(\Phi) = <S|\Phi>
$$

Una funzione integrabile f definisce una distribuzione $\tilde{f}$ su $\mathbb{R}^n$ tramite
\begin{equation}
<\tilde{f},\Phi> := \int_{\Rn} f \Phi d^{n}x
\forall \Phi \in D(U) 
\end{equation}


Si dice che $\tilde{f}$ è la distribuzione associata alla funzione f, o che la distribuzione $\tilde{f}$ è equivalente alla funzione f.\\
La distribuzione di Dirac (o misura di Dirac) è definita da 
\begin{equation}
<\delta,\Phi> := \Phi(0) 
\end{equation}

\underline{Teorema:} la distribuzione di Dirac non può essere rappresentata da una funzione integrabile (senza dim). Nonostante ciò scriveremo in seguito formalmente
\begin{equation}
<\delta , \Phi> = \int_{\Rn} \delta (x) \phi(x) d^n x= \Phi(0) 
\end{equation}

\underline{Es } (n=1)\\
Si dimostri che 
\begin{equation}
\delta' (x) =-\dfrac{\delta(x)}{x}
\end{equation}
Si ha $$\int x\delta(x)\Phi(x)dx = \left.x\Phi(x)\right\|_{x=0}=0 \forall \Phi \Rightarrow x\delta(x)=0 $$
$$\Rightarrow \delta(x) + x\delta ' (x) \Rightarrow \delta ' (x) = -\dfrac{\delta(x)}{x}  \qquad \blacksquare$$

Un'altra identità utile è
\begin{equation}
\delta(g(x)) = \sum_{i}\dfrac{\delta(x-x_i)}{|g'(x_i)|} 
\end{equation} 
dove $x_i$ sono gli zeri della fuznione g(x).\\
Rappresentazione della delta: $\delta(x) = \lim_{a\to 0} \delta_a(x)$, con 
\begin{subequations}
\begin{equation}
\delta_a(x)=\left\{\begin{matrix}
\dfrac{1}{2a} & -a \leq x \leq a \\
0 & |x| > a
\end{matrix}\right.
\end{equation}
\begin{equation}
\delta_a(x) = \dfrac{1}{\pi}\dfrac{a}{a^2+x^2}
\end{equation}
\begin{equation}
\delta_a(x) = \dfrac{1}{a\sqrt[]{\pi}}\exp{-\dfrac{x^2}{a^2}}
\end{equation}
\begin{equation}
\delta_a(x) = \dfrac{1}{2\pi} \int_{-\infty}^{\infty}\exp^{ikx-a|k|}dk
\end{equation}
\end{subequations}

Dimostrazione della (1.6b):

$$\int_{-\infty}^{\infty} \delta_a(x)\Phi(x)dx=
\dfrac{1}{\pi}\int_{-\infty}^{\infty}\dfrac{a}{a^2+x^2}\Phi(x)dx = (x=at) = 
\dfrac{1}{\pi}\int_{-\infty}^{\infty}\dfrac{1}{1+t^2}\Phi(at)dt$$
$$lim_{a\to 0} \int_{-\infty}^{\infty}\delta_a(x)\Phi(x)dx=\dfrac{1}{\pi}\lim_{a\to 0}\int_{-\infty}^{\infty}\Phi(at)\dfrac{dt}{1+t^2}$$

usa il \underline{teorema della convergenza dominata}:
se ${f_k}_{k\in\mathbb{N}} \grave{e}$ una successione di funzioni misurabili con limite puntuale f, e se esiste una funzione integrabile g tale che $|f_k|\leq g \forall k$ allora f $\grave{e}$ integrabile e $\lim_{k\to \infty}\int f_k dx = \int f dx$\\
Da noi $f_k(t)\hat{=}\dfrac{\Phi(at)}{1+t^2}\rightarrow$ funzione g esiste, perchè $\Phi$ ha supporto compatto.\\
Quindi : $$
\lim_{a\to 0}\int_{-\infty}{\infty}\delta_a(x)\Phi(x)dx=\dfrac{1}{\pi}\int_{-\infty}^{\infty} \lim_{a\to 0} \Phi(at) \dfrac{dt}{1+t^2}=\dfrac{1}{\pi} \Phi(0) arctan(t)|_{-\infty}^{\infty}=\Phi(0) q.e.d.$$

%(1.6.c) e (1.6.d) a casa
In N dimensioni: coordinate cartesiane $x_1,x_2,\dots,x_n \Rightarrow \delta(x)= \delta(x_1)\dots \delta(x_n)$

\subsection{Proprietà della delta di Dirac}
$$\delta(-\bar{r})=\delta(\hat{r})$$
\begin{equation}
\int_{\Re^n} d^n\hat{r}\delta(\hat{r}-\hat{r'})f(\hat{r}) = \int_{\Re^n}d^n\hat{\rho}\delta(\hat{\rho})f(\hat{\rho}+\hat{r'})=\left.f(\hat{\rho}+\hat{r'}\right|_{\hat{\rho=0}}=f(\hat{r'})
\end{equation}

Scegli f=1 $\Rightarrow$ formalmente 
\begin{equation}
\int_{\Re^n}d^n\hat{r}\delta(\hat{r}-\hat{r'})=1
\end{equation}


$\delta$ in coordinate curvilinee?\\
la quantità invariante per trasformazione di coordinate è $d^n\hat{r}\delta(\hat{r}-\hat{r'})$
Coord $\alpha_i (x_i,\dots,x_n),i=1,\dots,n$
$x_j$ sono le coordinate cartesiane

Jacobiano
$$
J(x_i,\xi_j)=\left(\begin{matrix}
\dfrac{\partial x_1}{\partial \xi_1} & \dots & \dfrac{\partial x_1}{\partial \xi_n}\\
\vdots & & \vdots \\
\dfrac{\partial x_n}{\partial \xi_1} & \dots &\dfrac{\partial x_n}{\partial \xi_n}
\end{matrix}\right)dx_1\dots dx_n \delta(x_1-x_1')\dots \delta(x_n - x_n')
$$
$$
|J|d\xi_1\dots d\xi_n \delta(x_1 - x_1')\dots \delta(x_n-x_n')=d\xi_1 \dots d\xi_n\delta(\xi_1\dots \xi_1')\dots \delta(\xi_n - \xi_n')
$$
\begin{equation}
\delta(\xi_1-\xi_1')\dots\delta(\xi_n-\xi_n')=|J|\delta(x_1-x_1')\dots\delta(x_n-x_n')
\end{equation}
Esempio coordinate sferiche 3-dim
$$
x=r\cos\varphi\sin\theta \qquad y=r\sin\varphi\sin\theta \qquad z=r\cos\theta \quad (\xi_1=r,\xi_2=\theta,\xi_3=\varphi)
$$
$$\left|\dfrac{\partial x_i}{\partial \xi_j}\right|=\left|\begin{matrix}
\dfrac{\partial x}{\partial r} & \dfrac{\partial x}{\partial \theta} & \dfrac{\partial x}{\partial \varphi} \\
\dfrac{\partial y}{\partial r} & \dfrac{\partial y}{\partial \theta} & \dfrac{\partial y}{\partial \varphi}\\
\dfrac{\partial z}{\partial r} & \dfrac{\partial z}{\partial \theta} & \dfrac{\partial z}{\partial \varphi}
\end{matrix}\right|=r^2\sin\theta
$$
\begin{equation}
\Rightarrow \delta(r-r')\delta(\theta - \theta')\delta(\varphi - \varphi') = r^2\sin\theta \delta(x-x')\delta(y-y')\delta(z-z')
\end{equation}
Esercizio: calcolare J in coordinate cilindriche

Densità di carica di un insieme discreto di N cariche puntiformi
\begin{equation}
\rho(\hat{r}) = \sum_{i=1}^{N} q_i\delta(\hat{r}-\hat{r_i})
\end{equation}
%r_i posizione carica i-esima 

carica Q uniformemente distribuita su una superficie sferica con raggio R. $\rho(\underline{r})=?$\\
Chiamo $\rho(\underline{r})=A Q\delta(r-R)$ con A da determinare.
$$\int d^3\underline{r}\rho(\underline{r})= 4\pi \int r^2 sin\theta dr d\theta d\phi A Q \delta(r-R) = 4\pi AQ \int r^2 dr \delta(r-R)=4\pi AQR^2 $$
Normalizzando $4\pi AQR^2 =Q$ ottengo $A=\dfrac{1}{4\pi R^2}$
$$ \rho(\underline{r})=\dfrac{Q}{4\pi R^2}\delta(r-R)$$

\section{Trasformate di Fourier}
\underline{Definizione:} spazi $L^P$ \\
Sia $\chi$ uno spazio di misura con misura m positiva. (Possiamo prendere la misura di Lebesgue come esempio)\\
$L^P(\chi):=$ spazio di funzioni su $\chi$ tale che $|f|^p$ sia integrabile, e $\int_{\chi}|f|^p dm < \infty$\\
Si dimostra che per p$\geq$1, $L^P(\chi)$ è uno spazio vettoriale e $\left\|f\right\| := \left\{ \int_{\chi} |f|^p dm\right\}^{\frac{1}{p}}$ è una norma su questo spazio.\\
A noi interessa il caso in cui m è la \underline{misura di Lebesgue}; in tal caso gli elementi di $L^P(\chi)$ sono le funzioni f con $\int_{\chi}|f|^p dx < \infty$.
Il caso p=2 trova applicazioni in meccanica quantistica.\\
% aggiunta seconda lezione da appunti Serena
\underline{Definizione:} Sia f una funzione $f\in L^1 (\Rn)$, la \underline{trasformata di Fourier} \fourier f è una funzione in $\Rn$ (in realtà sul duale di $\R^n$, ma coincide con $\R^n$) definita da 
\begin{equation}
\left(\left(\fourier \right)f\right)(\kk) = \int_{\Rn}e^{-\bar{k}\bar{x}}f(\bar{x})d^n\bar{x} 
\end{equation}
Si osserva che se $f\in L'(\Rn) \Rightarrow \exists (\fourier f)(\bar{k})$. In seguito verrà usata la notazione $\hat{f}(\bar{k})=(\fourier f)(\bar{k})$\\
Una possibile rappresentazione della $\delta$ di Dirac è (per n=1)
$$
	\delta(x)=\lim_{a \to 0} \dfrac{1}{2\pi} \int_{-\infty}^{+\infty} e^{ikx-a\left | k\right |}dk
$$
che è una Trasformata di Fourier. Formalmente si può scrivere
\begin{equation}
\delta(\x)=\left(\dfrac{1}{2\pi}\right)^n\int_{\Rn}e^{i\kk\x}d^n\kk
\end{equation}

Se conosco la trasformata di Fourier $\hat{f}$ posso determinare la funzione f applicando la \underline{antitrasformata} di Fourier:
\begin{multline*}
\left(\dfrac{1}{2\pi}\right)^n\int_{\Rn}d^n\bar{k}e^{i\bar{k}\bar{x}}\hat{f}(\bar{k})=\\
=\left(\dfrac{1}{2\pi}\right)^n \int_{\Rn}d^n\bar{k}e^{i\bar{k}\bar{x}}\int_{\Rn}e^{-i\bar{k}\bar{x'}}f(\bar{x'})d^n\bar{x'}= \left(\dfrac{1}{2\pi}\right)^n\int_{\Rn}d^n\bar{x'}f(\bar{x'})\int_{\Rn}d^n\bar{k}e^{i\bar{k}\left(\bar{x}-\bar{x'}\right)}=\\
=\left(\dfrac{1}{2\pi}\right)^n\int_{\Rn}d^n\bar{x'}f(\bar{x'})\delta(\bar{x}-\bar{x'})
\end{multline*}

\begin{equation}
\Rightarrow f(\bar{x})=\left(\dfrac{1}{2\pi}\right)^n\int_{\Rn}d^n\bar{k}e^{i\bar{k}\bar{x}}\hat{f}(\bar{k})
\end{equation}

paragonando quest'ultima espressione con (1.13) si trova che
\begin{equation}
(\fourier\delta)(\kk)=1
\end{equation}
Definiamo ora una famiglia di funzioni
\begin{equation}
\varphi_{\kk}(\x)=\left(\dfrac{1}{2\pi}\right)^{\dfrac{1}{2}}e^{i\kk\x}
\end{equation}

Allora possiamo riscrivere la funzione $f(\x)$ come:
$$
f(\x)=\left(\dfrac{1}{2\pi}\right)^{\dfrac{n}{2}}\int d^n\kk\hat{f}(\kk\varphi_{\kk}(\x))
$$

$\Rightarrow$ se dimostro che $\varphi_{\bar{k}}(\bar{x})$ è \underline{ortonormale} e \underline{completo} allora posso sviluppare qualsiasi funzione su $\varphi_{\kk}(\x)$ che chiamo \underline{funzioni di base}.\\
\underline{Dimostrazione completezza:}
\begin{equation}
\int d^n\kk \varphi_{\kk}(\x)\varphi_{\kk}^{*}(\xp)=\left(\dfrac{1}{2\pi}\right)^n\int d^n\kk e^{i\kk(\x-\xp)}=\delta(\x-\xp)
\end{equation}

Che equivale alla condizione di completezza, infatti:
\begin{multline*}
f(\x)=\int d^n\xp f(\xp)\delta(\x-\xp)=\\
=\int d^n\bar{x'}f(\bar{x'})\int d^n\kk\varphi_{\kk}(\x)\varphi_{\kk}^{*}(\xp)=\int d^n\kk\varphi_{\kk}(\x)\int d^n\xp f(\xp)\varphi_{\kk}^{*}(\xp)=\\
=\int d^n\kk\varphi_{\kk}(\x)\left(\dfrac{1}{2\pi}\right)^{\dfrac{n}{2}}\hat{f}(\x)
\end{multline*}


$$
\Rightarrow f(\bar{x})=\left(\dfrac{1}{2\pi}\right)^{\dfrac{n}{2}}\int d^n\bar{k}\varphi_{\bar{k}}(\bar{x})\hat{f}(\bar{x})
$$

quindi qualunque f è sviluppabile in una base di $\varphi_{\bar{k}}$\\
\underline{Dimostrazione ortogonalita:}\\
\begin{equation}
\int d^n\x\varphi_{k}(\x)\varphi_{\kp}^{*}(\xp)=\left(\dfrac{1}{2\pi}\right)^{n}\int d^n e^{i(\kk-\kp)\x}=\delta (\x-\xp)
\end{equation}
\underline{Osservazione:} $\{\varphi_{\bar{k}}(\bar{x})\}$ formano una \underline{base} di $\Rn$, non di tutti i sottoinsiemi di $\Rn$. Sulla superficie sferica, le armoniche sferiche sono le combinazioni lineari di $\{\varphi_{\kk}(\x)\}$

\section{Funzioni di Green}
\underline{Definizione:} un \underline{nucleo} (detto anche \underline{Kernel}) su $\Rn$ è una distribuzione su $\Rn \wedge \Rn$, ossia un elemento del duale $D'(\Rn \wedge \Rn)$ di funzioni $C^{\infty}$ con supporto compatto in $\Rn \wedge \Rn$.
$$
K:D(\Rn \wedge \Rn)\rightarrow\Rn \qquad K\in D'(\Rn \wedge \Rn)
$$
\underline{Definizione:} un \underline{nucleo fondamentale} (o elementare) E di un operatore differenziale lineare D su $\Rn$ con coefficienti $a_j(\bar{x})\in C^{\infty}$;
$$
D=\sum_{|j|\leq m}a_j(\bar{x})D^j
$$
è un nucleo, che soddisfa
\begin{equation}
D=\sum_{|j|\leq m}a_j(\bar{x})D^jE(\bar{x},\bar{y})=\delta(\bar{x}-\bar{y}) (1.19)
\end{equation}
osservazione sulla notazione: j è un indice multiplo
$$
j=(j_1,\dots ,j_n) \qquad |j|=\sum_{i=1}^n j_i
$$
con m:= ordine dell'operatore differenziale $D^j$
$$
D^j=\left(\dfrac{\partial}{\partial x_i}\right)^{j_i}\dots \left(\dfrac{\partial}{\partial x_n}\right)^{j_n}
$$
\underline{Esempio} m=3,n=2 $\Rightarrow j=(j_1,j_2) \quad \bar{x}=(\bar{x_1},\bar{x_2})$
$$
D=a_{21}(\bar{x})\left(\dfrac{\partial}{\partial x_1}\right)^2\dfrac{\partial}{\partial x_2}+a_{20}(\bar{x})\left(\dfrac{\partial}{\partial x_1}\right)^2
$$
\emph{è un esempio, non è l'unico operatore possibile}\\
Dalla definizione abbiamo che 
$$
D E(\x-\y)=\delta(\x  - \y)=\prod_{i=1}^{n}\delta(x_i-y_i)
$$
Dato il nucleo fondamentale E, una soluzione dell'equazione differenziale $Dx=B$ può essere trovata tramite
\begin{equation}
X(\x)=\int_{\Rn}E(\x,\y)B(\y)d^ny (1.20)
\end{equation}

\underline{dimostrazione:}\\
$$
DX(\x)=\int_{\Rn}DE(\x,\y)B(\y)d^n(\y)=\int_{\Rn}\delta(\x-\y)B(\y)d^ny
$$
$$
\Rightarrow DX(\x)=B(\x) \blacksquare
$$
\underline{Definizione:} una funzione di green è un nucleo elementare per l'operatore differenziale
\begin{equation}
-\dfrac{\nabla^2}{4\pi}\Rightarrow-\nabla^2G(\x,\y)=-4\pi\delta(\x-\y) (1.21)
\end{equation}

Si può dimostrare che $G(\x,\y)=G(\y,\x)$.\\
\underline{Esempio:} si consideri una carica puntiforme in $\y$ con carica $q=1$. $\rho=q\delta(\x-\y)$ è la densità di carica. Dalle equazioni di Maxwell si ha $\nabla^2\phi(\x)=-4\pi\rho(\x)=-4\pi\delta(\x-\y)\rightarrow$ definizione di Funzione di Green.\\
Una possibile Funzione di Green è $\phi(\x)=\dfrac{1}{|\x-\y|}$\\
Se è nota una funzione di Green $\Rightarrow$ una soluzione dell'equazione di Poisson è:
$$
\phi(|x)=\int G(\x,\y)\rho(\y)d^3y (1.23)
$$
Quindi passo da un'equazione differenziale ad un integrale.\\
Nel nostro caso $\phi(\x)=\int \rho(\y)d^3\y$, come in elettromagnetismo, in generale

\begin{subequations}
\begin{equation}
G(\x,\y)=\dfrac{1}{|\x-\y|}+F(\x,\y) (1.24a)
\end{equation}
\begin{equation}
\nabla^2F(\x,\y)=0 (1.24b)
\end{equation}
\end{subequations}
$\Rightarrow$ F dipende dalle condizioni di bordo.
\underline{Teorema:} In assenza di superficio di bordo la funzione di Green $G(\x,\y)$ dipende solo dalla differenza $\x-\y$
$$
G(\x,\y)=G(\x-\y)\rightarrow\mathrm{Invarianza traslazionale} (1.25)
$$
Nota: non dimostrato ma intuibile, $\delta(\x-\y)$ invariante se non ho condizioni di bordo; $\nabla^2$ invariante; $G(\x-\y)$ invariante per traslazioni $\x\rightarrow\x-\bar{a}$, $\y\rightarrow-\bar{a}$\\
Se $G(\x,\y)=G(\x-\y)$ è più facile trovarla. Per n=3
$$
\nabla^2G(\x-\y)=-4\pi\delta(\x-\y)
$$
$$
\dfrac{\nabla^2}{(2\pi)^3}\int d^3k e^{i\kk(\x-\y)}\tilde{G}(\kk)=-\dfrac{4\pi}{(2\pi)^3}\int d^3ke^{i\kk(\x-\y)}
$$
Quindi il primo passaggio è scrivere G e $\delta$ come trasformate di Fourier
$$
\int d^3ke^{i\kk(\x-\y)}(-k^2\tilde{G}(\kk)+4\pi)=0
$$
il secondo passaggio è osservare che $\x$ compare solo come esponente ($\Rightarrow -k^2$) e porto tutto da una parte.\\
$e^{i\kk(\x-\y)}=\psi_k(\x)$ è linearmente indipendete, quindi deve annullarsi il coefficiente
$$
-k^2\tilde{G}(\kk)+4\pi =0 \Rightarrow \tilde{G}(\kk)=\dfrac{4\pi}{\kk^2}
$$
Applico l'antitrasformata di Fourier
$$
G(\x)=\dfrac{1}{2\pi^3}\int d^3\kk e^{i\kk\x}\dfrac{4\pi}{\kk^2}
$$
Cambiamento di coordinate: coordinate sferiche
%inserisci immagine con x, k e theta
$$
d^3\kk=k^2\sin\theta dkd\theta d\varphi \Rightarrow G(\x)=\dfrac{1}{\pi}\int\sin\theta d\theta dk \dfrac{e^{ik|\x|\cos\theta}}{k^2}
$$
cambio variabile: $u=\cos\theta \rightarrow du=-\sin\theta d\theta$
$$
G(\x)=-\dfrac{1}{\pi}\int dudke^{ik|\x|u}=\int_0^{\infty}\dfrac{2}{k|\x|\pi}\sin(k|\x|)dk= \dfrac{1}{|\x|}
$$

\chapter{Equazione del Calore}
La legge di Fourier della conduzione termica è data da 
\begin{equation}
\bar{q} =-k\bar{\nabla}T
\end{equation}
dove $\bar{q}:=$ densità di flusso termico; $k:=$ conducibilità termica; $T:=$ temperatura.\\
La temperatura può essere riscritta come $ T=\dfrac{\phi}{C_p \rho}$, dove $\rho$ è la denstià, $C_p$ è il calore specifico a pressione costante, e $\phi$ è il calore per unità di volume.\\
L'equazione di continuità è:
\begin{equation}
\dfrac{\partial \phi}{\partial t} + \bar{\nabla}\bar{q}=0
\end{equation}
che ha la forma tipica di una legge di conservazione.\\
In questa formula sostituiscto $\phi$ e $\bar{q}$ e trovo
$$
\dfrac{\partial}{\partial t}(\rho C_p T) + \bar{nabla}\cdot(-k\bar{\nabla T}) = \rho C_p \dfrac{\partial}{\partial t}T - k \nabla^2 T=0
$$
Definendo $\chi=\dfrac{k}{\rho C_p}:=$ coefficiente di conducibilità termica, si ottiene l'\underline{equazione del calore}
\begin{equation}
\dfrac{\partial T}{\partial t}=\chi \nabla^2 T
\end{equation}

%inizio lezione 11/10

paragona con l'equazione di diffusione 
\begin{equation}
\dfrac{\partial u}{\partial t}=D\cdot \Delta u
\end{equation}

D: Coefficiente di diffusione\\
u: densità del materiale che si diffonde\\

%ha riscritto 2.3 e 2.4 come 3 e 4
la (2.4) segue dalla 1 legge di Fick sulla corrente di diffusione
\begin{equation}
\overrightarrow{q}_{D}=-D\overrightarrow{\nabla}u
\end{equation}

più l'equazione di continuità (il materiale non viene creato o distrutto)
$$
\dfrac{\partial u}{\partial t}+\overrightarrow{\nabla}\cdot\overrightarrow{q_D}=0
$$
La (2.4) viene anche chiamata 2 legge di Fick.\\
N.B. anche l'equazione di Black-Scholes per il prezzo di un'opzione può essere riportato nella forma (2.3),(2.4).\\
-Risolviamo la (2.3) in d dimensioni:
\begin{equation}
\Delta=\dfrac{\partial^2}{\partial x_{1}^2}+\cdots+\dfrac{\partial^2}{\partial x_{d}^2} 
\end{equation}

Faccio una trasformata di Fourier:
\begin{equation}
T(\bar{x},t)=\dfrac{1}{(2\pi)^d}\int d^d\bar{k}e^{i\bar{k}\bar{x}}\tilde{T}(\bar{k},t) 
\end{equation}

$$(3)\Rightarrow\dfrac{1}{(2\pi)^d}\int d^d \bar{k} e^{i\bar{k}\bar{x}}\left(\dfrac{\partial}{\partial t}\tilde{T}(\bar{k},t) + \chi \bar{k}^2\tilde{T}(\bar{k},t)\right)=0 
$$
dato che gli esponenziali sono lnearmente indipendenti, devo annullare i coefficienti
\begin{equation}
\dfrac{\partial}{\partial t}\tilde{T}+\chi \bar{k}^2\tilde{T}=0 \Rightarrow\tilde{T}(\bar{k},t)=e^{-\chi\bar{k}t }\tilde{T}_0(\bar{k}) 
\end{equation}

faccio una trasformata di fourier inversa
\begin{equation}
\int d^d \bar{y}e^{-i\bar{k}\bar{y}}T_0(\bar{y}) 
\end{equation}

sostituisco (8),(9) nella (7):
$$
T(\bar{x},t)=\dfrac{1}{(2\pi)^d}\int d^d \bar{y}T_0(\bar{y})\int d^d\bar{k}e^{i\bar{k}(\bar{x}-\bar{y})-\chi\bar{k}t}
$$
definisco
\begin{equation}
\int d^d\bar{k}e^{i\bar{k}(\bar{x}-\bar{y})-\chi\bar{k}t }=:(2\pi)^d G(\bar{x}-\bar{y},t)
\end{equation}

G: propagatore/nucleo di calore ("heat kernel"). Propaga le condizioni iniziali di $T_0(\bar{y})$. Quindi abbiamo la convoluzione:
\begin{equation}
T(\bar{x},t)=\int d^d\bar{y}G(\bar{x}-\bar{y},t)T_0(\bar{y}) (11)
\end{equation}

Caso particolare: $T_0(\bar{y})=\delta(\bar{y})$
$$
\Rightarrow T(\bar{x},t)=G(\bar{x},t)
$$
il propagatore è soluzione dell'equazione del calore corrispondente a un dato iniziale deltiforme. Per questo motivo, il propagatore è anche chiamato soluzione fondamentale, perchè esso è una soluzione e con esso si costruiscono tutte le altre per convoluzione.\\
Calcoliamo G:
$$
G(\bar{z},t)=\dfrac{1}{(2\pi)^d}\int d^d \bar{k}e^{i\bar{k}\bar{z}-\chi \bar{k}t}
$$
passo agli esponenti, usando il teorema dei residui sposto l'asse reale nel piano complesso in alto o in basso
$$
-\chi t\left(\bar{k}-\dfrac{i\bar{z}}{2\chi t}\right)^2-\dfrac{\bar{z}^2}{4\chi t}
$$
$$
\left(\bar{k}-\dfrac{i\bar{z}}{2\chi t}\right)^2=:\bar{k'}
$$
$$
\Rightarrow G(\bar{z},t)=\dfrac{1}{(2\pi)^d}\exp\left(\dfrac{-\bar{z}^2}{4\chi t}\right)\int d^d \bar{k} e^{-\chi t \bar{k'}^2}
$$
l'integrale è Gaussiano (più precisamente prodotto di integrali Gaussiani)
$$
G(\bar{z},t)=\dfrac{1}{(2\pi)^d}\exp\left(-\dfrac{\bar{z}^2}{4\chi t}\right)\sqrt{\dfrac{\pi}{\chi t}}^d
$$
Se $Re(\chi t)>0\Rightarrow t>0$, la soluzione esiste solo per $t>0$, cioé per tempi posteriori all'essegnazione del dato iniziale (soluzione ritardata)
\begin{equation}
\Rightarrow G(\bar{z},t)=\dfrac{1}{(4\pi\chi t)^{\dfrac{d}{2}}}\exp\left(-\dfrac{\bar{z}^2}{4\chi t}\right) (12)
\end{equation}

-Soluzione della (11) per d=1 per dato iniziale localizzato:
$$
T_0(x)=\left\{\begin{matrix}
 \hat{T}_0 & |x|\leq L \\ 
 0 & |x| > L
\end{matrix}\right.
 % disegno della barriera
$$
%sopra la freccia c'è scritto (12)
$$
(11)\Rightarrow T(\bar{x},t)=\int_{L}^{L}dy\hat{T}_0 \dfrac{1}{(4\pi\chi t)^{1/2}}exp\left(-\dfrac{(x-y)^2}{4\chi t}\right)
$$
definisco $z:=x-y$ 
$$
=\dfrac{\hat{T}_0}{(4\pi\chi t)^{1/2}}\int_{x-L}^{x+L}dz exp\left(-\dfrac{z^2}{4\chi t}\right)
$$
definisco $r:=\dfrac{z}{(4\chi t)^{1/2}}$
$$
=\dfrac{\hat{T}_0}{\pi}\int_{\dfrac{x-L}{2\sqrt{\chi t}}}^{\dfrac{x+L}{2\sqrt{\chi t}}} e^{-r^2}dr=\dfrac{\hat{T}_0}{\sqrt{\pi}}\left(\int_{\dfrac{x-L}{2\sqrt{\chi t}}}^{0}e^{-r^2}dr + \int_0^{\dfrac{x+L}{2\sqrt{\chi t}}}\right)
$$
$$
=\dfrac{\hat{T}_0}{\sqrt{\pi}}\left(-\dfrac{\sqrt{pi}}{2}\right)
$$
\begin{equation}
=\dfrac{\hat{T}_0}{2}\left(erf\dfrac{x+L}{2\sqrt{\chi t}}-erf\dfrac{x-L}{2\sqrt{\chi t}}\right) (13)
\end{equation}

dove la funzione degli errori (di Gauss) e definita da 
\begin{equation}
erf(s):=\dfrac{2}{\sqrt{\pi}}\int_0^s e^{-r^2}dr (14)
\end{equation}

dato che la soluzione e pari possiamo limitarci a $x\geq 0$
%inserisci immagine 
Il punto fondamentale è che, sebben il dato iniziale sia nonnullo solo in una regione localizzata, appena comincia al'evoluzione la funzione è maggiore di zero \underline{ovunque}, per quanto lontano dal supporto del dato iniziale. $\grave{E}$ questo il \underline{comportamento diffusivo} che contrasta con la propagazione per onde.\\
-Dimostrazione che la (11) soddisfa il dato iniziale per $t\rightarrow 0$. A tal fine dimostriamo che :
$$
\lim_{a \to 0} \dfrac{1}{a\sqrt{\pi}}e^{-\dfrac{x^2}{a^2}}=\delta(x)=:\delta_a(x)
$$
$$
\int_{-\infty}^{\infty}\delta_a(x)\phi(x)dx=\int_{-\infty}^{\infty}\dfrac{1}{a\sqrt{\pi}}e^{-\dfrac{x^2}{a^2}}\phi(x)dx= (\dfrac{x}{a}=:y)=\dfrac{1}{\sqrt{\pi}}\int_{-\infty}^{\infty}e^{-y^2}\phi(ya)dy
$$
$$
\lim_{a\to 0}\int_{-\infty}^{\infty}\delta_a(x)\phi(x)dx=\dfrac{1}{\sqrt{\pi}}\lim_{a\to 0}\int_{-\infty}^{\infty}e^{-y^2}\phi(ya)dy
=(\text{teo conv dominata})=\dfrac{1}{\sqrt{\pi}}\int_{-\infty}{\infty}\lim_{a\to 0}e^{-y^2}\phi(ya)dy=$$
$$=\dfrac{\phi(0)}{\sqrt{\pi}}\int_{-\infty}^{\infty}e^{-y^2}dy=\phi(0)
$$
Con $a=2\sqrt{\chi t}:$
\begin{equation}
\lim_{t \to 0}\dfrac{1}{2\sqrt{\pi\chi t}}e^{-\dfrac{x^2}{4\chi t}}=\delta(x) (15)
\end{equation}

Quindi (12)$\Rightarrow$
$$
\lim_{t\to 0} G(\bar{z},t)=\delta(z_1)\cdot\dots\cdot\delta(z_d)=\delta(\bar{z})
$$
e la (11) implica 
$$
\lim_{t\to 0}T(\bar{x},t)=\int d^d\bar{y}\lim_{t\to 0}G(\bar{x}-\bar{y},t)T_0(\bar{y})=\int d^d \bar{y}\delta(\bar{x}-\bar{y})T_0(\bar{y})=T_0(\bar{x}) \quad \blacksquare
$$
-Flusso di calore con produzione di calore:
$$
\dfrac{\partial T}{\partial t} -\chi\Delta T=S(\bar{x},t) \quad t>0
$$
S è il termine di sorgente, rende l'equazione lineare non omogenea\\
-Soluzione particolare dell'equazione non omogenea:\\
definiamo una funzione di Green G tramite una convoluzione spaziale e temporale
\begin{equation}
\left(\partial_t-\chi\Delta\right)G(\bar{x}-\bar{x'},t-t')=\delta(\bar{x}-\bar{x'})\delta(t-t') (16)
\end{equation}

$$
\Rightarrow T(\bar{x},t)=\int d^n\bar{x'}dt'
$$
\begin{equation}
G(\bar{x}-\bar{x'},t-t')S(\bar{x'},t') (17)
\end{equation}

Check: 
$$
(\partial_t-\chi\Delta_{\bar{x}})T(\bar{x},t)=\int d^n\bar{x'}dt'(\partial_t-\chi\Delta_{\bar{x}})G(\bar{x}-\bar{x'},t-t')S(\bar{x'},t')=S(\bar{x},t)
$$
$$
(\partial_t-\chi\Delta_{\bar{x}})G(\bar{x}-\bar{x'},t-t')=\delta(\bar{x}-\bar{x'})\delta(t-t')
$$
trasformata di Fourier, definisco $\bar{z}:=\bar{x}-\bar{x'}, \tau := t-t'$
$$
G(\bar{z},\tau)=\dfrac{1}{(2\pi)^{n+1}}\int d^n\bar{k}d\omega e^{i(\bar{k}\bar{z}-\omega\tau)}\tilde{G}{\bar{k},\omega}\delta(\bar{z})\delta(\tau)=\dfrac{1}{(2\pi)^{n+1}}\int d^n\bar{k}d\omega e^{i(\bar{k}\bar{z}-\omega\tau)}
$$
sostituito nella (16) mi da
$$
(-i\omega+\chi\bar{k}^2)\tilde{G}(\bar{k},\omega)=1
$$
$$
\Rightarrow\tilde{G}(\bar{k},\omega)=\dfrac{1}{-i\omega+\chi\bar{k}^2}
$$
$$
\Rightarrow G(\bar{z},\tau)=\dfrac{1}{(2\pi)^{n+1}}\int d^n\bar{k}dw e^{i(\bar{k}\bar{z}-\omega\tau)}\dfrac{1}{-i\omega+\chi\bar{k}^2}
$$
calcolo l'integrale in $d\omega$ con il teorema dei residui, chiudendo sopra se $\tau$ è positiva, e viceversa
$$
i\omega+\chi\bar{k}^2=0 \Rightarrow\omega=-i\chi\bar{k}^2
$$
%immagine del cammino complesso dei residui
$$
\int d\omega\dfrac{e^{-i\omega\tau}}{-i(\omega+i\chi\bar{k}^2)}=:\int F(\omega) d\omega
$$
$$
e^{-i\omega\tau}=e^{-i(Re\omega+iIm\omega)\tau}
$$
$$
\begin{matrix}
 \tau > 0 & \Rightarrow Im\omega <0\\ 
 \tau < 0 & \Rightarrow Im\omega>0 \Rightarrow G(\bar{z},t)=0
\end{matrix}
$$
per $\tau<=$, cioè $t-t'<0$
$$
F(\omega)dw=\dfrac{e^{-i\tau(\rho\cos\varphi+i\rho\sin\varphi)}}{-i(\rho e^{i\varphi}+i\chi\bar{k}^2)}\rho e^{i\varphi}id\varphi=:f(\varphi)d\varphi
$$
vado a risolvere l'integrale
$$
\left|\int_{0}^{\pi}f(\varphi)d\varphi\right|\leq\int_{0}^{\pi}\left|f(\varphi)\right|d\varphi=\int_{0}^{\pi}\dfrac{e^{\tau\rho\sin\varphi}\rho d\varphi}{\left|\rho e^{i\varphi}+i\chi\bar{k}^2\right|}=$$
$$=\int_{0}^{\pi}\dfrac{e^{\tau\rho\sin\varphi}\rho d\varphi}{\sqrt{(\rho\cos\varphi)^2+(\rho\sin\varphi+\chi\bar{k}^2)^2}}=\int_{0}^{\pi}\dfrac{e^{\tau\rho\sin\varphi}\rho d\varphi}{\sqrt{\rho^2+2\rho\sin\varphi\chi\bar{k}^2+\chi^2 k^4}}
$$
posso minorare il denominatore con $\rho^2$ e ottengo
$$
\leq\int_{0}^{\pi}\dfrac{e^{\tau\rho\sin\varphi}\rho d\varphi}{\rho}=2\int_{0}^{\pi/2}e^{\tau\rho\sin\varphi d\varphi}
$$

%inizio lezione 12/10

So che in $[0,\dfrac{\pi}{2}]: \sin\varphi\geq\dfrac{2\varphi}{\pi}$

%inserisci immagine

$$
(\tau<0)\Rightarrow \tau\rho\sin\varphi \leq \tau\rho\dfrac{2\varphi}{\pi}
$$
$$
\Rightarrow 2\int_{0}{\pi/2}e^{\tau\rho\sin\varphi}d\varphi \leq 2\int_{0}^{\pi/2}e^{\tau\rho\dfrac{2\varphi}{\pi}}=2\left[\dfrac{\pi}{2\tau\rho}e^{\tau\rho\dfrac{2\varphi}{\pi}}\right]_{0}^{\dfrac{\pi}{2}}=\dfrac{\pi}{\tau\rho}(e^{\tau\rho}-1)\rightarrow (\rho \to \infty) 0
$$

Analogamente si dimostra che anche per il cammino sotto l'integrale per $\rho\to\infty$ tende a 0. Restano da calcolare i residui.
$$
ResF(\omega)=\lim_{\omega \to -i\chi\bar{k}^2} F(\omega)\cdot (\omega + i\chi\bar{k}^2)=ie^{-i\tau(-i\chi\bar{k}^2)}=ie^{-\tau\chi\bar{k}^2}
$$
Per $\tau>0$ chiudo l'integrale sotto. Il valore dell'integrale sull'asse reale è la differenza tra l'integrale sul cammino chiuso e quello sul solo semicerchio sotto.
%inserisco i simboli di integrali di cammino descritti sopra
$$
\tau>0:\int_{-\infty}^{\infty}d\omega F(\omega)=(residui)=-2\pi i ResF(\omega)=-2\pi i e^{-\tau\chi\bar{
k}^2}=2\pi e^{-\tau\chi\bar{k}^2}
$$
$$
\Rightarrow G(\bar{z},t)=\dfrac{1}{(2\pi)^n}\int d^n\bar{k}e^{i\bar{k}\bar{z}-\tau\chi\bar{k}^2} =\dfrac{1}{(2\pi)^n}exp\left(\dfrac{-\bar{z}^2}{4\tau\chi}\right) \int d^n\bar{k} e^{-\tau\chi\left(\bar{k}-\dfrac{i\bar{z}}{2\tau\chi}\right)}=
$$
\begin{equation}
=\dfrac{1}{(2\pi)^n}exp\left(-\dfrac{\bar{z}^2}{4\tau\chi}\right)\sqrt{\dfrac{\pi}{\chi\tau}}^n=\dfrac{1}{(4\pi\tau\chi)^{n/2}}exp\left(-\dfrac{\bar{z}^2}{4\chi\tau}\right) (18)
\end{equation}

$\rightarrow$ nucleo di calore!
\begin{equation}
G(\bar{z},\tau)=0,\tau>0 (19)
\end{equation}

$\rightarrow$ soluzione particolare dell'equazione del calore con sorgente:
$$
T(\bar{x},t)=(17)=\int d^n\bar{x'}\int_{-\infty=0}^{t}dt'G(\bar{x}-\bar{x'},t-t')S(\bar{x'},t')
$$
Soluzione generale dell'equazione omogenea
$$
T_{om}(\bar{x},t)=(11)=\int d^n\bar{x'}G(\bar{x}-\bar{x'},t)T_0(\bar{x'})
$$
Soluzione generale dell'equazione non omogenea
$$
T(\bar{x},t)=T_p(\bar{x},t)+T_{om}(\bar{x},t)
$$
supponi $S(\bar{x'},t')=0$ per $t'<0$ $\Rightarrow T_p(\bar{x},t)=0$, e quindi abbiamo che $T(\bar{x},0)=T_0(\bar{x})$\\
-Considera il problema di Dirichlet in un dominio connesso ( o su una varietà curva con bordo) U. $\lambda_n$ : autovalori del problema di Dirichlet, $\phi :$ autofunzioni di $\Delta$
$$\begin{matrix}
\Delta \phi + \lambda\phi =0 & \text{in } U \\
\phi=0 & \text{su }\partial U
\end{matrix}
$$
\begin{equation}
G(t,\bar{x},\bar{y})=\sum_{n}e^{-\lambda_n\chi t}\phi_n(\bar{x})\phi_n(\bar{y}) (20)
\end{equation}

\begin{equation}
T(\bar{x},t)=\int d^d\bar{y}G(t,\bar{x},\bar{y})t_0(\bar{y})=\int d^d \bar{y}\sum_{n}e^{-\lambda_n\chi t}\phi_n(\bar{x})\phi_n(\bar{y})T_0(\bar{y})(21)
\end{equation}


La (20) è un esempio di una funzione di Green che non dipende solo la $\bar{x}-\bar{y}$, ma da $\bar{x}$ e $\bar{y}$ separatamente. Motivo: rottura dell'invarianza per traslazioni a causa del bordo $\partial U$\\
Check: $T(\bar{x},0)=(21)=$
$$
=\int d^d \bar{y}\sum_{n}\phi_n(\bar{x})\phi_n(\bar{y})T_0(\bar{y})=\int d^d\bar{y}\delta(\bar{x}-\bar{y})T_0(\bar{y})=T_0(\bar{x})
$$
$$
\left . T(\bar{x},t)\right|_{\bar{x}\in\partial U}=\int d^d\bar{y}\sum_n e^{-\lambda n \chi t}
\left . \phi_n(\bar{x})\right|_{\bar{x}\in\partial U}\phi_n(\bar{y})t_0(\bar{y})=0
$$
$$
\dfrac{\partial}{\partial t}T(\bar{x},t)=\int d^d\bar{y}\sum_n (-\lambda_n \chi)e^{-\lambda_n\chi t}\cdot \phi_n(\bar{x})\phi_n(\bar{y})T_0(\bar{y})
$$
$$
\chi\Delta_{\bar{x}}T(\bar{x},t)=\int d^d \bar{y} \sum_n e^{-\lambda_n \chi t}\chi\Delta_{\bar{x}}\phi_n(\bar{x})\phi_n(\bar{y})T_0(\bar{y})=(\chi\Delta_{\bar{x}}\phi_n(\bar{x})=-\lambda_n\phi_n(\bar{x}))=\dfrac{\partial}{\partial t}T(\bar{x},t)
$$
Esempio: $U=[0,L] \rightarrow$ da risolvere $\partial_t T=\chi\partial_{x}^2T, $ $x\in[0,L],t>0$
condizioni al contorno $T(0,t)=0=T(L,t)$ , $T(x,0)=T_0(x)$
$$
\partial^{2}_{x}\phi=-\lambda\phi\rightarrow\phi=\phi_0\sin\dfrac{n\pi}{L}x, n=0,1,2,\ddots, \lambda=\dfrac{n^2\pi^2}{L^2}
$$
$$
\int_0^L dx \phi_n(x)^2=1\Rightarrow\phi_0=\sqrt{\dfrac{2}{L}}
$$
\begin{equation}
(20)\Rightarrow G(t,x,y)=\dfrac{2}{L}\sum_{n=0}^{\infty}e^{-n^2\pi^2\dfrac{\chi t}{L^2}}\cdot \sin\dfrac{n\pi}{L}x\sin\dfrac{n\pi}{L}y (22)
\end{equation}

\begin{equation}
(21)\Rightarrow T(x,t)=\int_0^L dy\sum_{n=1}^{\infty}e^{-n^2\pi^2\dfrac{\chi t}{L^2}}\cdot \dfrac{2}{L}\sin\dfrac{n\pi}{L}x\sin\dfrac{n\pi}{L}y T_0(y) (23)
\end{equation}

Sviluppo $T_0(y)$ in serie di Fourier: 
\begin{equation}
T_0(y)=\sum_{n=1}^{\infty}b_n\sin\dfrac{n\pi}{L}y (24)
\end{equation}

con 
\begin{equation}
b_n=\dfrac{2}{L}\int_0^LT_0(y)\sin\dfrac{n\pi}{L}ydy (25)
\end{equation}

Usando la (25), posso riscrivere la (23)
\begin{equation}
T(x,t)=\sum_{n=1}^{\infty}e^{-n^2\pi^2\dfrac{\chi t}{L^2}}b_n\sin\dfrac{n\pi}{L} (26)
\end{equation}

N.B.: la (20) vale anche per varietà compatte senza bordo (p.e. $S^2$)\\
Compiti a casa:
i) risolvere $\partial_t T=\chi\Delta T$ nel cubo
$$
\begin{matrix}
\partial_t T=\chi\Delta T\\
0\leq x\leq L & 0\leq y\leq L & 0\leq z\leq L & t\geq 0\\
T(\bar{x},0)=T_0(\bar{x})\\
T(\bar{x},t)=0 & x=0,L & y=0,L & z=0,L 
\end{matrix}
$$
ii) le (20),(21) valgono anche per il problema di Neumann $\Delta\phi + \lambda\phi =0$ in Um $\bar{n}\cdot\bar{\nabla}\phi=0$ su $\partial U$ ($\bar{n}$: versore normale al bordo). Perchè? (p.e. verificare che $\left . \bar{n}\cdot\bar{\nabla}T\right|_{\bar{x}\in \partial U}=0$
risolvere l'equzione del calore nel cubo con pareti isolati (nessun flusso termico attraverso le pareti, vedi equazione (1)
$$
\begin{matrix}
T(\bar{x},0)=T_0(\bar{x})\\
\partial_x T=0 & x=0,L \\
\partial_y T=0 & y=0,L \\
\partial_z T=0 & z=0,L \\
\end{matrix}
$$
%cambia solo che i sensi di i) diventano coseni, dato che sono passato alla derivata
 iii) Risolvere l'equzione del calore sulla 2-sfera. Suggerimento: scrivere il laplaciano in coordinate sferiche
\begin{equation}
\Delta =\dfrac{1}{r^2}\partial_r(r^2\partial_r)+\dfrac{1}{r^2\sin\theta}\partial_{\theta} (\sin\theta\partial_{\theta}+\dfrac{1}{r^2\sin^2\theta}\partial^2\varphi (27)
\end{equation}

porre $r=cost$ e usare 
\begin{equation}
\dfrac{1}{\sin\theta}\partial_{\theta}(\sin\theta Y^m_l)+\dfrac{1}{\sin^2\theta}\partial^2_{\varphi}Y_l^m=-\lambda Y_l^m (28)
\end{equation}

con $\lambda=l(l+1),Y_l^m$ armoniche sferiche. Sostituire le (24),(25) col corrispondente sviluppo in armoniche sferiche.\\
-Parentesi: soluzione di $\Delta_{\bar{x}}G(\bar{x},\bar{y})=-\delta(\bar{x}-\bar{y})$ in d dimensioni.\\
consideriamo un problema un po' più generale:
\begin{equation}
(\Delta_{\bar{x}}-m^2)G(\bar{x}-\bar{y})=-\delta(\bar{x}-\bar{y}) (29)
\end{equation}

$\rightarrow$ G è il nucleo dell'equazione di \underline{Helmholtz} 
\begin{equation}
(\Delta_{\bar{x}}-m^2)f=-S (30)
\end{equation}

(soluzione: $f(\bar{x})=\int d^d\bar{x'}G(\bar{x}-\bar{x'})S(\bar{x'})$)
G è il propagatore per un campo scalare in di dimensioni Euclidee ($\rightarrow$ teoria quantistica dei campi). Uso la trasformata di Fourier per riportarmi ad un'equazione algebrica
$$
G(\bar{x}-\bar{y})=\dfrac{1}{(2\pi)^d}\int d^d \bar{k}e^{i\bar{k}(\bar{x}-\bar{y})}\tilde{G}(\bar{k})
$$
$$
\delta(\bar{x}-\bar{y})=\dfrac{1}{(2\pi)^d}\int d^d \bar{k}e^{i\bar{k}(\bar{x}-\bar{y})} \Rightarrow (-k^2-m^2)\tilde{G}(\bar{k})=-1\Rightarrow \tilde{G}(\bar{K})=\dfrac{1}{\bar{k}^2+m^2}
$$
\begin{equation}
\Rightarrow G(\bar{z})=\dfrac{1}{(2\pi)^d}\int d^d \bar{k} \dfrac{e^{i\bar{k}\bar{z}}}{\bar{k}^2+m^2}
\end{equation}

Uso 
\begin{equation}
\dfrac{1}{\bar{k}^2+m^2}=\int_0^{\infty}exp\left( -\tau(\bar{k}^2+m^2)\right)d\tau (32)
\end{equation}
$$
\Rightarrow G(\bar{z})=\dfrac{1}{(2\pi)^d}\int_0^{\infty}d\tau e^{-\tau m^2}\cdot\int d^d \bar{k} e^{i\bar{k}\bar{z}-\tau\bar{k}^2}
$$

%%%%%%%%%%%%%%%%%%%%%%%%%%%%%%%%%%%%%%%%%%%%%%%%%%%%%%%%%%%%%%%%%%
%inizio lezione 18/10
%%%%%%%%%%%%%%%%%%%%%%%%%%%%%%%
$$
i\kk\bar{z}-\tau\kk^2=-\tau(\kk-\dfrac{i\bar
z}{2\tau})^2-\dfrac{\bar{z}^2}{4\tau}
$$
$$
\kp:=\kk-\dfrac{i\bar
z}{2\tau}
$$
$$
\Rightarrow G(\bar{z})=\dfrac{1}{(2\pi)^d}\int_0^{\infty}d\tau e^{-\tau m^2-\frac{\bar{z}^2}{4\tau}}\cdot\int d^d\kp e^{-\tau\kp^2}
$$
$$
\int d^d\kp e^{-\tau\kp^2}=\left(\dfrac{\pi}{\tau}\right)^{d/2} \quad (\tau>0)
$$
$$
\dfrac{1}{(4\pi)^{d/2}}\int_{0}^{\infty}\tau^{-d/2}e^{-\tau m^2 - \frac{\bar{z}^2}{4\tau}}d\tau
$$
$$
\tau:=\dfrac{m^{-1}}{2}|\bar{z}|e^t, \quad d\tau=\dfrac{m^{-1}}{2}|\bar{z}|e^tdt
$$
$$
\tau^{-\dfrac{d}{2}}=\left(\dfrac{m^{-1}|\bar{z}}{2}\right)^{-\dfrac{d}{2}}e^{-\dfrac{d}{2}t}
$$
$$
-\tau m^2-\dfrac{\bar{z}^2}{4\tau}=-\dfrac{m^-1}{2}|\bar{z}|e^t m^2
$$
$$
-\dfrac{\bar{z}^2}{4\tau}\dfrac{2m}{|\bar{z}|}e^{-t}=-m|\bar{z}|\cosh t
$$
$$
\Rightarrow G(\bar{z})=\dfrac{1}{(4\pi)^{d/2}}\int_{-\infty}^{\infty}dty\left(\dfrac{|\bar{z}|}{2m}\right)^{1-d/2}e^{(1-d/2)t}e^{-m|\z|\cosh t} =\int_{-\infty}^{0}+\int_{0}^{\infty} 
$$
$$
\dfrac{1}{(4\pi)^{d/2}}\int_{-\infty}^{0}dt\left(\dfrac{|\z|}{2m}\right)^{1-d/2}e^{(1-d/2)t}e^{-m|\z|\cosh t}
$$
$$
(t'=-t)=\dfrac{1}{(4\pi)^{d/2}}\int_0^\infty dt'\left(\dfrac{|\bar{z}|}{2m}\right)^{1-d/2}e^{-(1-d/2)t'}e^{-m|\bar{z}|\cosh t'}
$$
$$
(t'\rightarrow t)\Rightarrow G(\bar{z})=\dfrac{1}{(4\pi)^{d/2}}\int_0^{\infty}dt\left(\dfrac{|\bar{z}|}{2m}\right)^{1-d/2}\cdot 2\cosh ((1-d/2)t)e^{-m|\bar{z}|\cosh t}
$$
funzione di Bessel modificata del 2 tipo:
\begin{equation}
K_{\nu}(x)=\int_0^{\infty}e^{-x\cosh t}\cosh(\nu t)dt (33)
\end{equation}

$$
Rex>0 \Rightarrow K_{-\nu}(x)=K_{\nu}(x)
$$
$$
\Rightarrow G(\bar{z})=\dfrac{1}{(4\pi)^{d/2}}\left(\dfrac{|\bar{z}|}{2m}\right)^{1-d/2}\cdot 2 K_{1-d/2}(m|\bar{z}|)
$$
\begin{equation}
\Rightarrow G(\z)=\dfrac{1}{(2\pi)^{d/2}}m^{d-2}(|\z|m)^{1-d/2}\cdot K_{1-d/2}(m|\z|) (34)
\end{equation}

$m\to 0:$ usa
\begin{equation}
K_{\nu}(x) (x\to 0)\left\{\begin{matrix}
-\gamma-\ln \dfrac{x}{2} & \nu=0 \\
\dfrac{\Gamma(\nu)}{2}\left(\dfrac{2}{x}\right)^{\nu} & \nu >0
\end{matrix}\right.(35)
\end{equation}

dove $\gamma=\lim_{n\to\infty}\left(\sum_{k=1}^{n}\dfrac{1}{k}-\ln n\right)\approx 0,577$ costante di Eulero-Mascheroni.\\
$\Gamma(\nu)$: funzione gamma di Eulero, estende il concetto di fattoriale ai numeri completti, el senso che per ogni numero intero non negativo n si ha $\Gamma (n)=(n-1)!$

\begin{equation}
\Gamma(z)=\int_0^{\infty}t^{z-1}e^{-t}dt,\quad Rez>0 (36)
\end{equation}

Esercizio: integrando per parti, dimostrare che 
\begin{equation}
\Gamma(z+1)=z\Gamma(z)(37)
\end{equation}
$\Rightarrow \Gamma(z)=\dfrac{\Gamma(z+1)}{z}$
Usando 	questa la definizione della $\Gamma$ può essere estesa al piano $Rez<0\Rightarrow$ per $d+2$ e $m\to 0$, la (34) diventa
\begin{equation}
G(\z)\rightarrow \dfrac{1}{(2\pi)^{d/2}}m^{\dfrac{d-2}{2}}|\z|^{1-d/2}\dfrac{\Gamma(\dfrac{d}{2}-1)}{2}\left(\dfrac{2}{m|\z|}\right)^{d/2-1}=\dfrac{\Gamma(\dfrac{d}{2}-1)}{4\pi^{d/2}|\z|^{d-2}} (38)
\end{equation}

corrisponde al potenziale di una carica puntiforme in d-dimensioni.\\
Caso $d=3$:
\begin{equation*}\label{2.38'}
\begin{matrix}
\Gamma\left(\dfrac{1}{2}\right)=\sqrt{\pi} \\
\Rightarrow G(\z)=\dfrac{1}{4\pi|\z|}(38')
\end{matrix}
\end{equation*}

Caso limite $d=2$:
\begin{equation}
(34)\Rightarrow G(\z)=\dfrac{1}{2\pi}K_0(m|\z|) (39)
\end{equation}

$$
(35) (m\to 0)\rightarrow \dfrac{1}{2\pi}\left(-\gamma -\ln \dfrac{m|\z|}{2}\right)=\dfrac{1}{2\pi}(-\gamma -\ln m + \ln 2 - ln|\z|)
$$
Chiaro: G definita a meno di una costante additiva nel caso $m\to 0$.
\begin{equation}
\Delta_{\x}G(\x-\y)=-\delta(|x-\y) \Rightarrow G(\z)=-\dfrac{\ln |\z|}{2\pi} (40)
\end{equation}

Check: $\Delta_{\x}G(\x)=-\delta(\x)$\\
A causa dell'invarianza per rotazioni, G dipende solo da $|\x|$, se le condizioni al contorno non rompono l'invarianza. Passo in coordinate polari per controllare.\\
Coordinate polari:
$$
\Delta =\dfrac{\partial^2}{\partial r^2}+\dfrac{1}{r}\dfrac{\partial}{\partial r}+\dfrac{1}{r^2}\dfrac{\partial ^2}{\partial \varphi^2}
$$
considero $r\neq 0$, $\dfrac{\partial G}{\partial \varphi}=0$
$$
\left(\dfrac{\partial^2}{\partial r^2}+\dfrac{1}{r}\dfrac{\partial}{\partial r}\right)\dfrac{-\ln r}{2\pi}=\dfrac{1}{2\pi}\left(\dfrac{1}{r^2} + \dfrac{1}{r}\left(-\dfrac{1}{r}\right)\right)=0\qquad \checkmark
$$

Per verificare che $\Delta_{\x}G(\x)=-\delta(\x)$ integro $\Delta G$ su un disco D centrato in zero, con raggio $\epsilon$
$$
\int_D \Delta G d^2\x=\int_D \underline{\nabla}\cdot(\underline{\nabla}G)d^2\x
$$
Uso Gauss per passare a un integrale di bordo
$$
=\int_{\partial D} \bar{n} \cdot \underline{\nabla}G ds \quad \bar{n}=\bar{e_r}, \quad ds=rd\varphi
$$
$$
\int_{\partial D}\bar{n}\cdot \underline{\nabla}G ds=\int_{0}^{2\pi}\left(\dfrac{\partial}{\partial r}G\right)rd\varphi=\int_0^{2\pi}-\dfrac{1}{2\pi r}r d\varphi =-1 =-\int_D \delta(\x)d^2\x \quad \checkmark
$$
ho ottenuto il nucleo del laplaciano in due dimensioni, che corrisponde al potenziale elettrostatico in due dimensioni
\section{Flusso di calore in un cilindro infinito}
%immagine cilindro, a fianco le condizioni
$$
\begin{matrix}
\dfrac{\partial}{\partial t}T=\chi\Delta T \\
T(\x,t)=0 & \x\text{ sul bordo}\\
T(\x,0)=T_0(\x)
\end{matrix}
$$
la simmetria del problema suggerisc edi usare le coordinate cilindriche: $r,\varphi,z$, con $x=r\cos \varphi$, $y=r\sin \varphi$
$$
\Delta = \dfrac{\partial ^2}{\partial r^2} +\dfrac{1}{r}\dfrac{\partial^2}{\partial \varphi ^2} + \dfrac{\partial ^2}{\partial z^2}
$$
$0\leq r \leq L, \quad 0\leq \varphi \leq 2\pi, \quad -\infty <z<\infty$
\subsection{Separazione delle variabili}
$$
T(r,\varphi,z,t)=\tau(t)R(r)\Phi(\varphi)Z(z)\Rightarrow \dfrac{\partial T}{\partial t}=\tau '(t)R\Phi Z
$$
$$
\Delta T=\tau(R'' +\dfrac{1}{r}R')\Phi Z + \tau R \dfrac{1}{r^2}\Phi''Z+\tau R \Phi Z'' = \dfrac{1}{\chi}\tau'R\Phi Z \Rightarrow \dfrac{1}{R}(R'' + \dfrac{1}{r}R') + \dfrac{1}{r^2}\dfrac{\Phi''}{\Phi}+\dfrac{Z''}{Z}-\dfrac{1}{\chi}\dfrac{\tau'}{\tau}=0
$$
la funzione di Z è costante $=C_2$, e analogamente la funzione di t $=C_2$, di conseguenza la parte rimanente in $r,\varphi=C_1-C_2$
$$
\begin{matrix}
\tau'=\chi\tau C_1 & \tau=\tau_0 e^{\chi C_1 t} \\
Z''=C_2Z & Z=Z_0e^{\sqrt{C_2}z}+Z_1e^{-\sqrt{C_2}z}
\end{matrix}
$$
$$
r^2\dfrac{1}{R}(R''+\dfrac{1}{r}R')+\dfrac{\Phi''}{\Phi}=(C_1-C_2)r^2 \Rightarrow \dfrac{\Phi''}{\Phi}=cost:=-\lambda^2
$$
$$
^2\dfrac{1}{R}(R''+\dfrac{1}{r}R')+ (C_2-C_1)r^2=\lambda^2 \Rightarrow \Phi=\Phi_0 \cos\lambda\varphi + \Phi_1\sin\lambda\varphi=\Phi_0\cos\lambda(\varphi+2\pi)+\Phi_1\sin\lambda(\varphi+2\pi)=\phi_0(\cos\lambda\varphi\cos 2\pi\lambda-\sin\lambda\varphi\sin 2\pi\lambda)+\phi_1(\sin\lambda\varphi\cos 2\pi\lambda + \cos \lambda\varphi\sin 2\pi\lambda)
%aggiunta a mano
$$
$$
\Rightarrow \begin{matrix}
\phi_0\cos 2\pi\lambda + \phi_1 \sin 2\pi\lambda=\phi_0\\
-\phi_0\sin 2\pi\lambda + \phi_1 \cos 2\pi\lambda=\phi_1
\end{matrix}
\Rightarrow
\left(\begin{matrix}
\cos 2\pi\lambda -1 & \sin 2\pi\lambda \\
-\sin 2\pi\lambda & \cos 2\pi\lambda -1
\end{matrix}\right)
\left(\begin{matrix}
\phi_0\\
\phi_1
\end{matrix}\right)
=0
$$
ho una soluzione non banale se il determinante è nullo $\implies \cos 2\pi\lambda=1,\quad \sin 2\pi\lambda=0, \quad \lambda=m,m \in \mathbb{Z}$. Basta prendere $m\in \mathbb{N}_0$, per cui le funzioni $\Phi$ formano un sistema completo.\\
Passo all'equazione radiale:
$$
r^2 R'' + r R' + ((C_2-C_1)r^2-m^2)R=0
$$
Supponiamo per semplicità che $T_0(\x)$ dipenda solo da $r,\varphi$.
$$
T(r,\varphi,z,0)=\tau_0R\Phi Z'=T_0(r,\varphi)\Rightarrow Z=cost \Rightarrow C_2=0
$$
$$
Z=Z_0e^{\sqrt{C_2}z}+Z_1e^{-\sqrt{C_2}z}
$$
senza perdere la generalità, poniamo $Z=1, C_1<0$ altrimenti T diverge per $t\to \infty \quad(\tau=\tau_0e^{\chi C_1 t})$
(Inoltre si può far vedere che la soluzione dell'equzione radiale (con $C_2=0$) diverge nell'origine (per le nostre condizioni al contorno) se $C_1>0$)
Pongo $x:=\sqrt{|C_1|}r$
\begin{equation}
\Rightarrow x^2\dfrac{d^2 R}{dx^2}+x\dfrac{dR}{dx}+(x^2-m^2)R=0 (41)
\end{equation}

equazione differenziale di Bessel. Cerca soluzioni della forma

\begin{equation}
R(x)=x^\alpha \sum_{n=0}^{\infty} a_nx^n(42)
\end{equation}

$$
(a_0\neq 0)\Rightarrow R'(x)=\alpha x^{\alpha -1}\sum_n a_nx^n + x^\alpha\sum_n na_nx^{n-1}
$$
$$
R''(x)=\alpha(\alpha-1)x^{\alpha-2}\sum_na_nx^n + 2\alpha x^{\alpha-1}\sum_n a_n x^{n-1}+x^\alpha\sum_n n(n-1)a_nx^{n-2}
$$
$$
(41)\Rightarrow \alpha(\alpha-1)x^\alpha\sum_n a_n x^n+2\alpha x^\alpha \sum_n n a_n x^n + x^\alpha \sum_n n(n-1)a_n x^n + \alpha x^\alpha\sum_n a_n x^n + x^\alpha \sum_n n a_n x^n + (x^2-m^2)x^\alpha \sum_n a_n x^n=0
$$
studio il prefattore di $x^0$: $\alpha^2 a_0-m^2a_0=0\Rightarrow \alpha=\pm m$. Scarto la soluzione $\alpha=-m$ perchè non fisica (divergerebbe sull'asse del cilindro).\\
Soluzione regolare in $x=0: \alpha=m$. In tal caso:

\begin{equation}
2m\sum_n n a_n x^n + \sum_n n^2 a_n x^n + \sum_n a_n x^{n+2}=0 (43)
\end{equation}

sostituisco nell'ultimo pezzo $n+2=n'$
$$
\sum_{n'} a_{n'-2} x^{n'}
$$
$$
x^1: 2ma_1+a_1=0 \Rightarrow a_1=0 
$$
$\Rightarrow la (43) diventa$
\begin{equation}
\sum_{n=2}^{\infty}x^n(a_n(2mn+n^2)+a_{n-2})=0 \Rightarrow a_n := - \dfrac{a_n-2}{2mn+n^2} (44)
\end{equation}

$\rightarrow$ relazione di ricorrenza\\

%%%%%%%%%%%%%%%%%%%%%%%%%%%%%%%%%%%%%%
% inizio lezione 19/10
%%%%%%%%%%%%%%%%%%%%%%%%%%%%%%%%%%%
scegliendo $a_0=\dfrac{2^{-m}}{m!}$ per motivi di normalizzazione, si ottiene la soluzione

\begin{equation}
J_m(x)=\sum_{k=0}^{\infty}\dfrac{(-1)^k\left(\dfrac{x}{2}\right)^{m+2k}}{k!(m+k)!} (45)
\end{equation}

dove per m non interi si definisce $m!:=\Gamma(m+1)$
$J_m:$ \underline{funzione di Bessel del 1 tipo}\\
% disegno funzioni di bessel m=0,1,2
$\Rightarrow R=J_m(\sqrt{|C_1|}r)$, impongo le condizioni al contorno $
T(r=L,\varphi,z,t)=0 $
$$\Rightarrow R(r=L)=0 \Rightarrow J_m(\sqrt{|C_1|}L)=0 \implies \sqrt{|C_1|}L=j_k^{(m)}, \quad k=1,2,\dots \implies C_1=-\left(\dfrac{j_k^{(m)}}{L}\right)^2
$$
dove $j_k^{(m)}$ sono gli zeri positivi di $J_m(x)$, noti numericamente.
$$
\Rightarrow T(r,\varphi,z,t)=\tau_0e^{-\chi \left(\dfrac{j_k^{(m)}}{L}\right)^2 t}\cdot J_m\left(j_k^{(m)}\dfrac{r}{L}\right)(\phi_0\cos m\varphi + \phi_1 \sin r\varphi)
$$
la costante $\tau_0$ si può riassorbire in $\phi_0,\phi_1$, quindi la pongo =1. La soluzione generale sarà uan combinazione lineare di queste soluzioni.

\begin{equation}
T(r,\varphi,z,t)=\sum_{m=0}^{\infty}\sum_{k=1}^\infty e^{-\chi\left(\dfrac{j_k^{(m)}}{L}\right)^2t}J_m\left(j_k^{(m)}\dfrac{r}{L}\right)(C_{km}\cos m\varphi + S_{km}\sin m\varphi) (46)
\end{equation}

\begin{equation}
T(r,\varphi,z,0)=\sum_{m=0}^{\infty}\sum_{k=1}^\infty J_m\left(j_k^{(m)}\dfrac{r}{L}\right)(C_{km}\cos m\varphi + S_{km}\sin m\varphi)=T_0(r\varphi) (47)
\end{equation}

la (47) forma un sistema di funzioni completo nel cilindro $\rightarrow$ qualsiasi funzione $T_0(r,\varphi)$ con $T_0(L,\varphi)=0$ possiede uno sviluppo di questo tipo. Devono essere scelte delle costanti $C_{km} e S_{km}$ appropriate, invertendo la (47) 

\section{Problemi non omogenei}
spesso la separazione delle variabili riducce delle equazioni differenziali alle derivate parziali a delle equazioni differenziali ordinarie, come p.e.
\begin{equation}
a(x)u'' + b(x)u'+c(x)u=f(x) (48)
\end{equation}

Ipotesi: $a(x)$ continuamente differenziabile; b,c continue. Moltiplica la (48) con $\dfrac{1}{a(x)}exp\left(\int_\alpha^x\dfrac{b(\xi)}{a(\xi)}d\xi\right)$ e definisco
$$
p(x):= exp\left(\int_\alpha^x \dfrac{b(\xi)}{a(\xi)}d\xi\right)
$$
$$
q(x):=\dfrac{c(x)}{a(x)}p(x),\quad f(x):=\dfrac{F(x)}{a(x)}p(x)
$$
posso scrivere la "forma autoaggiunta" della (48)

\begin{equation}
(48)\Rightarrow \dfrac{d}{dx}\left(p\dfrac{du}{dx}\right) + qu = f(x) (49)
\end{equation}

%aggiunta non legata al problema vero e proprio
digressione:\\
$$
p\dfrac{d}{dx} \left( p\dfrac{d}{dx}u\right)+pqu=pf
$$
definisco $pq:=\omega^2(x)$, $pf:=g$ e y attraverso 
$$
\dfrac{d}{dy}=p(x)\dfrac{d}{dx} \Rightarrow \dfrac{dx}{p(x)}=dy \Rightarrow y=\int \dfrac{dx}{p(x)} 
$$ 
ottengo
$$
\dfrac{d^2u}{dy^2}+\omega^2(y)u=g
$$
è un'equazione di oscillatore armonico con frequenza $\omega$ che dipende dal "tempo" g, dove g è una forzante. Viene chiamato "oscillatore di Ermakoff", e trova applicazioni in Cosmologia (equazione di Sasaki-Mukhonov) per quanto riguarda la teoria dell'inflazione.\\

Equazione omogenea: 
\begin{equation}
\dfrac{d}{dx}\left(p\dfrac{d}{dx}\right)+qv=0 (50)
\end{equation}

possiede 2 soluzioni $v_1, v_2$ linearmente indipendenti $\implies$ soluzione generale:
$$
v(x)=c_1v_1(x) + c_2v_2(x), \quad c_1, c_2 cost
$$

considera la funzione
\begin{equation}
w(x)=v_1(x)\int_\alpha^x v_2(\xi)f(\xi)d\xi - v_2(x)\int_\alpha^x v_1(\xi)f(\xi)d\xi (51)
\end{equation}

(paragona con il metodo della variazione delle costanti)
$$
\Rightarrow w'(x)=v_1'(x)\int_\alpha^x v_2(\xi)f(\xi)d\xi - v_2'(x)\int_\alpha^x v_1(\xi)f(\xi)d\xi +v_1(x)v_2(x)f(x)-v_2(x)v_1(x)f(x) = 
$$
$$
= v_1'(x)\int_\alpha^x v_2(\xi)f(\xi)d\xi - v_2'(x)\int_\alpha^x v_1(\xi)f(\xi)d\xi \Rightarrow
$$
\begin{multline*}
\dfrac{d}{dx}\left(p(x)\dfrac{dw}{dx}\right)=\\
\dfrac{d}{dx}(p(x)v_1'(x))\int_\alpha^x v_2(\xi)f(\xi)d\xi - \dfrac{d}{dx}(p(x)v_2'(x))\int_\alpha^x v_1(\xi)f(\xi)d\xi + p(x)v_1'(x)v_2(x)f(x)-p(x)v_2'(x)v_1(x)f(x)
\end{multline*}

i coefficienti davanti agli integrali corrispondono rispettiamente a $-qv_1$ e $-qv_2$
$$
-q(x)w(x) + p(x)(v_1'(x)v_2(x)-v_2'(x)v_1(x))f(x)
$$
Inoltre:
$$
\dfrac{d}{dx}\left\{p(x)(v_1'(x)v_2(x)-v_2'(x)v_1(x))\right\}
$$
deve avere il Wronskiano 
$$
\left| \begin{matrix}
v_1 & v_2\\
v_1' & v_2'
\end{matrix}
\right| \neq 0
$$
$$
=\dfrac{d}{dx}(p v_1')v_2-\dfrac{d}{dx}(pv_2')v_1 + pv_1'v_2' - p v_2'v_1'=0
$$
$$
\Rightarrow p(x)(v_1'(x)v_2(x)-v_2'(x)v_1(x))=K\quad costante
$$
Quindi: 
\begin{equation}
\dfrac{d}{dx}\left(p\dfrac{dw}{dx}\right)+qw=Kf (52)
\end{equation}

Inoltre, se $v_1',v_2'$ sono limitati per $x\to\alpha$, $w(\alpha)=w'(\alpha)=0$
\begin{equation}
(52):K \Rightarrow \dfrac{w(x)}{K}=u(x)=\int_\alpha^xR(x,\xi)f(\xi)d\xi (53)
\end{equation}

con 
\begin{equation}
R(x,\xi):=\dfrac{v_1(x)v_2(\xi)-v_2(x)v_1(\xi)}{p(x)(v_1'(x)v_2(x)-v_2'(x)v_1(x))} (54)
\end{equation}

è una soluzione del problema ai valori iniziali
\begin{equation}
\left\{\begin{matrix}
\dfrac{d}{dx}\left(p\dfrac{du}{dx}\right) + qu=f(x) & x>\alpha\\
u(\alpha)=u'(\alpha)=0 &
\end{matrix}\right. (55)
\end{equation}

Il denominatore della (54) è costante $\implies R(x,\xi)$ soddisfa l'equazione omogenea (50) sia come funzione di x che di $\xi$. (NB: $R(x,\xi)=-R(\xi,x)$). Per $\xi$ fissato: $R(x,\xi)$ è la soluzione del problema omogeneo ai valori iniziali
$$
\dfrac{d}{dx}\left(p(x)\dfrac{dR}{dx}\right) + q(x)R=0, \quad x>\xi
$$
\begin{equation}
R |_{x=\xi}=0 \qquad \left.\dfrac{dR}{dx}\right|_{x=\xi}=(54)=\dfrac{1}{p(\xi)} (56)
\end{equation}

$R(x,\xi)$: funzione di Green (one-sided).\\
esempio: oscillatore armonico invertito
$$
\left\{\begin{matrix}
u'' - u =f(x) & x>0 \\
u(0)=u'(x)=0&
\end{matrix}\right.
$$
soluzione per $\xi$ fissato, $R(x,\xi)$ soddisfa 
$$
\dfrac{d^2R}{dx^2}-R=0, \quad x>\xi
$$
$$
R|_{x=\xi}=0, \qquad \left.\dfrac{dR}{dx}\right|_{x=\xi}=1
$$
$$
\Rightarrow R= A(\xi)\sinh(x)+B(\xi)\cosh(x)
$$
$$
R|_{x=\xi}=A\sinh\xi + B \cosh \xi =0
$$
$$
\left.\dfrac{dR}{dx}\right|_{x=\xi}=A\cosh\xi + B \sinh \xi =1
$$
$$
\Rightarrow A=\cosh \xi, \quad B=-\sinh \xi, \Rightarrow R=\sinh(x-\xi)
$$
$$
(53)\Rightarrow u(x)=\int_0^x f(\xi)\sinh(x-\xi)d\xi
$$
N.B. 
i) Se $u(\alpha),u'(\alpha)\neq 0 \implies$ aggiungi soluzione $c_1v_1(x)+c_2v_2(x)$ dell'equazione omogenea, in modo tale da soddisfare le nuove condizioni iniziali. Nell'esempio sopra:
$$
u(x)=\int_0^x f(\xi)\sinh(x-\xi)d\xi + c_1\sinh(x)+ c_2\cosh(x)
$$
soddisfa $u(0)=c_2,u'(0)=c_1$\\
ii) $(53)\Rightarrow$ Il valore di $u(x)$ dipende solo da $f(\xi)$ per  $\xi<x$. Comportamento molto simile a quelle delle equazioni alle derivate parziali iperboliche (vedi piu tardi) %inserisci riferimento dall'indice
\section{Problema ai valori al contorno}
Risolvi p.e.
$$\left\{\begin{matrix}
\dfrac{d}{dx}\left(p\dfrac{du}{dx}\right)+qu=-f(x) & \alpha <x<\beta\\
u(\alpha)=u(\beta)=0 &
\end{matrix}\right.
$$
(ndr il - davanti a $f(x)$ viene messo per convenienza)\\
soluzione generale:
\begin{equation}
u(x)=-\int_\alpha^x R(\xi,x)f(\xi)d\xi + c_1v_1(x)+c_2v_2(x) 
\end{equation}

\begin{equation}
\left\{\begin{matrix}
u(\alpha)=c_1v_1(\alpha)+c_2v_2(\alpha)=0\\
u(\beta)=-\int_{\alpha}^\beta R(\beta,\xi)f(\xi)d\xi + c_1v_1(\beta)+c_2v_2(\beta)=0
\end{matrix}\right. (58)
\end{equation}

La (58) ha una soluzione per $c_1,c_2$ se $D:=v_1(\alpha)v_2(\beta)-v_2(\alpha)v_1(\beta)\neq 0$. In tal caso
$$
c_1=-\dfrac{v_2(\alpha)}{D}\int_\alpha^\beta R(\beta,\xi)f(\xi)d\xi= -\dfrac{v_2(\alpha)}{D}\int_\alpha^x R(\beta,\xi)f(\xi)d\xi -\dfrac{v_2(\alpha)}{D}\int_x^\beta  R(\beta,\xi)f(\xi)d\xi
$$
$$
c_2= \dfrac{v_1(\alpha)}{D}\int_\alpha^\beta  R(\beta,\xi)f(\xi)d\xi= \dfrac{v_1(\alpha)}{D}\int_\alpha^x R(\beta,\xi)f(\xi)d\xi +\dfrac{v_1(\alpha)}{D}\int_x^\beta  R(\beta,\xi)f(\xi)d\x 
$$

%%%%%%%%%%%%%%%%%%%%%%%%%%%%%%%%%%%%%%%%%%%%%%%%%%%
%inizio lezione 25/10
%%%%%%%%%%%%%%%%%%%%%%%%%%%%%%%%%%%%%%%%%%%%%%%%%%%%%

% questo va probabilmente inserito prima delle soluzioni esplicite per c1 c2

$$
\Rightarrow u(x)=-\int_\alpha^x\left[R(x,\xi)+\dfrac{v_2(\alpha)v_1(x) - v_1(\alpha)v_2(x)}{D}R(\beta,\xi)\right]f(\xi)d\xi-\int_x^\beta \dfrac{v_2(\alpha)v_1(x) - v_1(\alpha)v_2(x)}{D} R(\beta,\xi)f(\xi)d\xi
$$

\begin{multline*}
R(x,\xi)+\dfrac{v_2(\alpha)v_1(x)-v_1(\alpha)v_2(x)}{D}R(\beta,\xi)=(54)=\\
\dfrac{v_1(x)v_2(\xi)-v_2(x)v_1(\xi)}{K D}(v_1(\alpha)v_2(\beta)-v_2(\alpha)v_1(\beta)) + \dfrac{v_2(\alpha)v_1(x)-v_1(\alpha)v_2(x)}{D} \dfrac{v_1(\beta)v_2(\xi)-v_2(\beta)v_1(\xi)}{K}
\end{multline*}

$$
\dfrac{1}{KD}(v_1(\alpha)v_2(\xi)-v_2(\alpha)v_1(\xi))(v_1(x)v_2(\beta)-v_2(x)v_1(\beta))
$$

Definisco G
\begin{equation}
G(x,\xi):=\left\{\begin{matrix}
\dfrac{1}{KD}(v_1(\xi)v_2(\alpha)-v_2(\xi)v_1(\alpha))(v_1(x)v_2(\beta)-v_2(x)v_1(\beta)) & \xi\leq x \\
\dfrac{1}{KD}(v_1(x)v_2(\alpha)-v_2(x)v_1(\alpha))(v_1(\xi)v_2(\beta)-v_2(\xi)v_1(\beta)) & x\leq \xi
\end{matrix}\right. (59)
\end{equation}

$\implies$ soluzione del problema ai valori al contorno (57):
\begin{equation}
u(x)=\int_\alpha^\beta G(x,\xi)f(\xi)d\xi (60)
\end{equation}

$G(x,\xi)$ è una funzione di Green
\begin{equation}
G(x,\xi)=G(\xi,x) (61)
\end{equation}


per determinare G notiamo che per ogni $\xi$ soddisfa il problma ai valori al contorno
\begin{equation}
\begin{matrix}
\dfrac{d}{dx}\left(p(x)\dfrac{dG}{dx}\right) + q(x)G = 0 \quad x\neq \xi \\
G|_{x=\alpha}=G|_{x=\beta}=0 \\
G_{x=\xi+0}==G|_{\xi-0} \\
G continua in \xi \\
\dfrac{dG}{dx}|_{x=\xi+0}-\dfrac{dG}{dx}|_{x=\xi-0}=-\dfrac{1}{p(\xi)}
\end{matrix} (62)
\end{equation}


$\frac{dG}{dx}$ discontinua in $x=\xi$. (Usare la (59). Per ricavare l'ultima equazione bisogna usare anche la definizione di K)

\underline{Esempio:} 
$$
((1+x)^2u')'-u=f(x), \quad 0<x<1, \quad u(0)=u(1)=0
$$
$$
\Rightarrow \dfrac{d}{dx}\left((1+x)^2 \dfrac{dG(x,\xi)}{dx}\right)-G(x,\xi)=0
$$
Prova $G(x,\xi)=c(\xi)(1+x)^\alpha \implies \dfrac{dG}{dx}=c\alpha(1+x)^{\alpha-1}$
$$
\left((1+x)^2 \dfrac{dG}{dx}\right)'=\left(c\alpha(1+x)
^{\alpha+1}\right)'=c\alpha(\alpha +1)(1+x)^\alpha = G = c(1+x)^\alpha
$$
$$
\implies \alpha^2 + \alpha - 1 =0 \Rightarrow \alpha=\dfrac{1}{2}(1\pm \sqrt{5}):=\alpha_\pm
$$
devo separe i casi $x>\xi$ e $x<\xi$
$$
\Rightarrow  G(x,\xi)=\left\{\begin{matrix}
c_+(\xi)(1+x)^{\alpha_+} + c_-(\xi)(1+x)^{\alpha_-} & x<\xi \\
\tilde{c}_+(\xi)(1+x)^{\alpha_+} + \tilde{c}_-(\xi)(1+x)^{\alpha_-} & x>\xi
\end{matrix}\right.
$$
$$
\begin{matrix}
G|_{x=0}=c_+ + c_- = 0 & \implies c_- = -c_+ \\
G|_{x=1}=\tilde{c}_+ 2^\alpha + \tilde{c}_-2^\alpha = 0 & \implies \tilde{c}_- = -\tilde{c}_+2^{\sqrt{5}} 
\end{matrix}
$$
%% da qui mancano dei pezzi
$$
\Rightarrow G(x,\xi)=\left\{\begin{matrix}
c_+(\xi)((1+x)^{\alpha_+}
\end{matrix}\right\}
$$
$$
G(x,\xi)=\left\{\begin{matrix}
c_+(x)((1+\xi)
\end{matrix}\right\}
$$
$$
\Rightarrow G(x,\xi)=\left\{\begin{matrix}
\lambda((1+\xi)^{\alpha_+}-2^{\sqrt{5}}(1+\xi)^{\alpha_-})((1+x)^{\alpha_+
}-(1+x)^{\alpha_-}) & x<\xi \\
\lambda((1+\xi)^{\alpha_+}-(1+\xi)^{\alpha_-})((1+x)^{\alpha_+}-2^{\sqrt{5}}(1+x)^{\alpha_-}) & x>\xi \\
\end{matrix}\right\}
$$
$\Rightarrow G|_{x=\xi+0}=G|_{x=\xi-0}$
$$
\left.\dfrac{dG}{dx}\right|_{x=\xi+0}=\lambda((1+\xi)^{\alpha_+}-(1+\xi)^{\alpha_-})(\alpha_+(1+\xi)^{\alpha_+ +1}-2^{\sqrt{5}}\alpha_-(1+\xi)^{\alpha_- -1})
$$
$$
\left.\dfrac{dG}{dx}\right|_{x=\xi-0}=\lambda((1+\xi)^{\alpha_+}-2^{\sqrt{5}}(1+\xi)^{\alpha_-})(\alpha_+(1+\xi)^{\alpha_+ +1}-\alpha_-(1+\xi)^{\alpha_- -1})
$$
$ \left.\frac{dG}{dx}\right|_{x=\xi+0}-\frac{dG}{dx}_{x=\xi-0}$

\begin{multline*}
=\lambda \left[ \alpha_+ (1+\xi)^{2\alpha_+ -1}-2^{\sqrt{5}}\alpha_-(1+\xi)^{\alpha_+ + \alpha_- -1} - \alpha_+(1+\xi)^{\alpha_-+\alpha_+-1} + 2^{\sqrt{5}}\alpha_- (1+\xi)^{2\alpha_- -1}- \right.\\
\left. \alpha_+(1+\xi)^{2\alpha_+ -1}+ \alpha_- (1+\xi)^{\alpha_+ + \alpha_- - 1} + 2^{\sqrt{5}}\alpha_+(1+\xi)^{\alpha_- + \alpha_+ -1}-2^{\sqrt{5}}\alpha_-(1+\xi)^{2\alpha_- -1} \right]=-\dfrac{1}{p(\xi)}=-\dfrac{1}{(1+\xi)^2}
\end{multline*}

$$
\alpha_+ + \alpha_- -1=-2
$$
$$
\Rightarrow \lambda\left[ -2^{\sqrt{5}}\alpha_- - \alpha_+ + \alpha_- + 2^{\sqrt{5}}\alpha_+\right]=-1
$$
$\rightarrow$ determina $\lambda$\\ %nel senso che la roba sopra lo determina
\underline{Problema ai valori al contorno piu generale:}
%sottosezione? è il problema di dirichlet piu generale
%dirichlet -> ho fissato al contorno solo la funzione
%neumann-> ho fissato al contorno la funzione e la derivata
%robin-> fisso con una combinazione lineare di u e u', in pratica vincolo posizione e velocità lungo la traiettoria

\begin{equation}
(pu')'+qu=-f,\quad \alpha<x<\beta, \quad u(\alpha)=a, u(\beta)=b (63)
\end{equation}

A tal fine: nota che $\frac{\partial G}{\partial \xi}(x\alpha)$ soddisfa
$$
\dfrac{d}{dx}\left[p(x)\dfrac{d}{dx}\left(\dfrac{\partial G}{\partial \xi}(x,\alpha)\right)\right]+q(x)\dfrac{\partial G}{\partial \xi}(x,\alpha)=0, \quad \alpha<x<\beta
$$
$$
\dfrac{\partial G}{\partial \xi}(\alpha,\alpha)=\dfrac{1}{p(\alpha)}, \quad \dfrac{\partial G}{\partial \xi}(\beta,\alpha)=0
$$
mentre
$$
\dfrac{d}{dx}\left[p(x)\dfrac{d}{dx}\left(\dfrac{\partial G}{\partial \xi}(x,\beta)\right)\right]+q(x)\dfrac{\partial G}{\partial \xi}(x,\beta)=0, \quad \alpha<x<\beta
$$
$$
\dfrac{\partial G}{\partial \xi}(\alpha,\beta)=0, \quad \dfrac{\partial G}{\partial \xi}(\beta,\beta)=-\dfrac{1}{p(\beta)}
$$
(seguono dalla definizione (59) $\rightarrow$ compito)\\
il problema (63) ha la soluzione
\begin{equation}
u(x)=\int_\alpha^\beta G(x,\xi)f(|xi)d\xi + ap(\alpha)\dfrac{\partial G}{\partial \xi}(x,\alpha)-bp(\beta)\dfrac{\partial G}{\partial \xi}(x,\beta) (64)
\end{equation}

combinazione lineare di soluzione particolare (l'integrale) e soluzione dell'omogenea(il resto).\\
Nella soluzione del problema (57) abbiamo dovuto assumere $D\neq 0$.\\
Caso $D=0$: le equazioni
$$
c_1v_1(\alpha)+c_2v_2(\alpha)=0
$$
$$
c_1v_1(\beta)+c_2 v_2(\beta)=0
$$
ammettono soluzione non banale $\Rightarrow v(x)=c_1v_1(x) + c_2 v_2(x)$ soddisfa
$$
(pv')'+qv=0, \quad \alpha<x<\beta, \quad v(\alpha)=v(\beta)=0
$$
Se u è una soluzione del problema (63), lo è anche u+cv, $\forall c$ costante.$\Rightarrow$ il problema (63) non può avere soluzione unica.
Inoltre:\\
moltiplica la (63) conn v e integra da $\alpha$ a $ \beta $:
$$
-\int_\alpha^\beta f(x)v(x)dx=\int_\alpha^\beta v(x)((pu')' + qu)dx
$$
integro due volte per parti
$$
=\left[vpu' - v'pu\right]_\alpha^\beta+\int_\alpha^\beta u\left[(pv')'+qv dx\right]=p(\alpha)v'(\alpha)a-p(\beta)v'(\beta)b
$$
dove l'integranda è nulla.
$v(pu')' \rightarrow -v'pu' \rightarrow (v'p)'u$\\
Se il problema (63) deve avere una soluzione, la funzione f e le due costanti a,b devono soddisfare
$$
p(\alpha)v'(\alpha)a-p(\beta)v'(\beta)b=-\int_\alpha^\beta v(x)f(x)dx
$$
altrimenti non ci può essere una soluzione del problema.\\
$\Rightarrow$ nel caso D=0, il problema (63) può avere nessuna soluzione o molte soluzioni, ma mai una sola soluzione.
\underline{Problemi ai valori al contorno ancora piu generali:}
\begin{equation}
\begin{matrix}
(pu')' + qu = -f(x), \quad \alpha<x<\beta\\
-\mu_1 u'(\alpha) + \sigma_1 u(\alpha)=a,\\
\mu_2 u'(\beta) + \sigma_2 u(\beta)=a
\end{matrix} (65)
\end{equation}

(condizione al contorno di Robin).\\
La funzione di Green $G(x,\xi)$ si ricava come prima se
$$
D:=\left[-\mu_1 v_1'(\alpha)+\sigma_1 v_1(\alpha)\right]\cdot \left[\mu_2 v_2'(\beta)+\sigma_2 v_2(\beta)\right] - \left[-\mu_1 v_2'(\alpha)+\sigma_1 v_2(\alpha)\right] \cdot \left[\mu_2 v_1'(\beta)+\sigma_2 v_1(\beta)\right]\neq 0
$$
(esercizio)
%nota che è in pratica un determinante
%prendo sol part dell'omogeneo con la R, aggiungo la generale del problema omogeneo (c1v1(x)+c2v2(x)). devo determinare questa ultima con c.c. diverse, anziche avere D=\= 0 ho uesta condizione sopra
$G(x,\xi)$ è la soluzione del problema
\begin{equation}
\begin{matrix}
\dfrac{d}{dx}\left(p(x)\dfrac{dG}{dx}\right) + q(x)G = 0 \quad x\neq \xi \\
-\mu_1\dfrac{dG}{dx}|_{x=\alpha}+\sigma_1 G|_{x=\alpha}=\mu_2\dfrac{dG}{dx}|_{x=\beta} + \sigma_2 G|_{x=\beta}=0 \\
G|_{x=\xi+0}=G|_{x=\xi-0}
\dfrac{dG}{dx}|_{x=\xi+0}-\dfrac{dG}{dx}|_{x=\xi-0}=-\dfrac{1}{p(\xi)}
\end{matrix} (66)
\end{equation}

(paragona con le (62), qui le condizioni al contorno su G sono più generali). G soddisfa ancora $G(x,\xi)=G(\xi,x)$\\
-Soluzione di (65):
\begin{equation}
u(x)=\int_\alpha^\beta G(x,\xi)f(\xi)d\xi + \dfrac{p(\alpha)}{\mu_1}aG(x,\alpha) + \dfrac{p(\beta)}{\mu_2}bG(x,\beta) (67)
\end{equation}

$(\mu_1,\mu_2\neq 0)$
se $\mu_1=0$ sostituisci $\dfrac{1}{\mu_1}G(x,\alpha) con \dfrac{1}{\sigma_1}\dfrac{\partial G}{\partial \xi}(x,\alpha)$\\
se $\mu_2=0$ sostituisci $\dfrac{1}{\mu_2}G(x,\beta) con \dfrac{1}{\sigma_2}\dfrac{\partial G}{\partial \xi}(x,\alpha)$\\
(Siccome $D\neq 0$ non può essere $\mu_1=\sigma_1=0$ oppure $\mu_2=\sigma_2=0$)\\
Caso D=0: il problema (65) avrà nessuna soluzione o molte soluzione ed è quindi ben bosto se e solo se $D\neq 0$. In tal caso la soluzione è data dalla (67)

%%%%%%%%%%%%%%%%%%%%%%%%%%%%%%%%%%%%%
%% inizio lezione 26/10
%%%%%%%%%%%%%%%%%%%%%%%%%%%%%%%%%%%%%%%%
NB:\\
i) In un problema ai valori al contorno, il valore di u in un dato punto dipende dai valori di f(x) nell'intero intervallo $\alpha,\beta)$. $\rightarrow$ Comportamento simile a quello delle equazionni alle derivate parziali ellittiche (vedi piu tardi).\\ %inserisci riferimento capitolo
ii) Se una soluzione particolare dell'equazione differenziale non omogenea può essere indovinata, non è necessario usare le funzioni di Green

\subsection{Applicazione del formalismo imparato}
Conduzione di calore in un intervallo con sorgente
\begin{equation}
\begin{matrix}
\partial_t T = \chi \partial_x^2 T + S(x,t), \quad x \in [0,L], \quad t\geq 0\\
T(0,t)=T(L,t)=0, \quad T(x,0)=0
\end{matrix} (68)
\end{equation}

Sviluppa $T(x,t)$ in serie di Fourier
\begin{equation}
T(x,t)=\sum_{n=1}^\infty b_n(t)\sin\dfrac{n\pi x}{L} (69)
\end{equation}

con 
\begin{equation}
b_n(t)=\dfrac{2}{L}\int_0^L T(x,t) \sin \dfrac{n\pi x}{L}dx (70), vedi (25)
\end{equation}

\begin{subequations}
\begin{equation}
S(x,t)=\sum_{n=1}^\infty s_n(t)\sin\dfrac{n\pi x}{L} (71a)
\end{equation}
\begin{equation}
s_n(t)=\dfrac{2}{L}\int_0^L S(x,t) \sin \dfrac{n\pi x}{L}dx (71b)
\end{equation}
\end{subequations}

inserisco (69),(71a) nella (68)
\begin{equation}
\Rightarrow \partial_t b_n(t)=\chi \left(-\dfrac{n^2 \pi^2}{L^2}\right)b_n(t) + s_n(t)
\end{equation}

Questa si risolve facilmente col metodo di variazione delle costanti arbitrarie. Invece con la funzione di Green: Riscrivi la (72) nella forma 
$$
(p(t)u'(t))'=f(t)
$$
con $p(t)=exp\left(\dfrac{\chi n^2 \pi^2}{L^2}t\right), \quad u'(t)=b_n(t),\quad f(t)=p(t)s_n(t)$
$$
T(x,0)=0 \Rightarrow b_n(t)=0 \Rightarrow u'(t)=0
$$
$u(t$) è definita a meno di una costante additiva $\rightarrow$ scegli $u(t)=0 \rightarrow$ problema ai valori iniziali (55). La funzione di Green $R(t,\xi)$ dalla (56):
$$
\dfrac{d}{dt}\left(p(t)\dfrac{dR}{dt}\right)=0, \quad t>\xi
$$
$$
\Rightarrow p(t)\dfrac{dR}{dt}=C(\xi) \Rightarrow R=-C(\xi)\dfrac{L^2}{\chi n^2\pi^2} \exp \left(-\dfrac{\chi n^2 \pi^2}{L^2}t\right)+\tilde{C}(\xi)
$$
$$
R|_{t=\xi}=-C(\xi)\dfrac{L^2}{\chi n^2\pi^2} \exp \left(-\dfrac{\chi n^2 \pi^2}{L^2}\xi \right)+\tilde{C}(\xi) =0
$$
$$
\tilde{C}(\xi)=C\dfrac{L^2}{\chi n^2 \pi^2}\exp \left(-\dfrac{\chi n^2 \pi^2}{L^2}\xi \right)=\gamma(\xi)\exp\left(-\dfrac{\chi n^2 \pi^2}{L^2}\xi \right)
$$
$$
\Rightarrow R(t,\xi)=\gamma(\xi)\left(\exp\left(-\dfrac{\chi n^2 \pi^2}{L^2}\xi \right) - \exp\left(-\dfrac{\chi n^2 \pi^2}{L^2}t \right)\right)
$$
$$
\left.\dfrac{dR}{dt}\right|_{t=\xi}=\gamma \dfrac{\chi n^2 \pi^2}{L^2}\exp\left(-\dfrac{\chi n^2 \pi^2}{L^2}\xi \right)=\dfrac{1}{p(\xi)}=\exp\left(-\dfrac{\chi n^2 \pi^2}{L^2}\xi \right)
$$
$$
\gamma=\dfrac{L^2}{\chi n^2 \pi^2} \implies \Rightarrow R(t,\xi)=\gamma(\xi)\left(\exp\left(-\dfrac{\chi n^2 \pi^2}{L^2}\xi \right) - \exp\left(-\dfrac{\chi n^2 \pi^2}{L^2}t \right)\right)
$$
$$
u(t)=(53)=\int_0^t \dfrac{L^2}{\chi n^2 \pi^2}\left( \exp\left(-\dfrac{\chi n^2 \pi^2}{L^2}\xi \right) - \exp\left(-\dfrac{\chi n^2 \pi^2}{L^2}t \right)\right)e^{\dfrac{\chi n^2 \pi^2}{L^2}\xi} s_n(\xi)d\xi
$$
$$
=\dfrac{L^2}{\chi n^2 \pi^2}\int_0^t\left( 1-\exp\left( -\dfrac{\chi n^2 \pi^2}{L^2}(t-\xi)\right)\right)\cdot s_n(\xi)d\xi \equiv \int_0^t H(t,\xi)d\xi
$$
$b_n(t)=u'(t)$
%inserisci appunti a mano, sono solo una digressione fatta per spiegare i due contributi di questo caso particolare
$$
\int_0^t H(t,\xi)d\xi=H(t,t) + \int_0^t \dfrac{\partial}{\partial t}H(t,\xi)d\xi=\dfrac{L^2}{\chi n^2\pi^2}\int_0^t \dfrac{\chi n^2 \pi^2}{L^2}\exp \left(-\dfrac{\chi n^2 \pi^2}{L^2}(t-\xi)\right) s_n(t)d\xi =
$$
$$
=\int_0^t \exp \left(-\dfrac{\chi n^2 \pi^2}{L^2}(t-\xi)\right) s_n(t)d\xi
$$
Nella (69):
$$
T(x,t)=\sum_{n=1}^{\infty}\int_0^t\exp \left(-\dfrac{\chi n^2 \pi^2}{L^2}(t-\xi)\right) s_n(t)d\xi \sin\dfrac{n\pi x}{L}
$$
riscrivo $s_n(t)$ usando la (71b)
$$
\Rightarrow T(x,t)=\int_0^t \int_0^L \sum_{n=1}^{\infty}\dfrac{2}{L}\exp \left(-\dfrac{\chi n^2 \pi^2}{L^2}(t-\xi)\right) \sin \dfrac{n\pi x}{L}\sin \dfrac{n\pi y}{L}S(y,\xi)dyd\xi
$$
\begin{equation}
\Rightarrow T(x,t)=\int_0^t \int_0^L G(t-\xi,x,y)S(y,\xi)dyd\xi
\end{equation}

\begin{equation}
G(t-\xi,x,y):=\dfrac{2}{L}\sum_{n=1}^{\infty}\dfrac{2}{L}\exp \left(-\dfrac{\chi n^2 \pi^2}{L^2}(t-\xi)\right) \sin \dfrac{n\pi x}{L}\sin \dfrac{n\pi y}{L}
\end{equation}

è identica alla (22).\\%dipende solo dalla differenza t-xi, perchè non è rotta l'invarianza per traslazioni temporali, ma dipende separatamente da x e  perche essendoci un bordo l'invarianza traslazionale spaziale è rotta
NB: se $T(x,0)=T_0(x)$ anziché 0, sostituisci $ b_n(0)=0$ con
$$
b_n(0)=\dfrac{2}{L}\int_0^L T_0(x)\sin\dfrac{n\pi x}{L}dx
$$
$$
b_n(t)=\int_0^t\exp\left(-\dfrac{\chi n^2 \pi^2}{L^2}(t-\xi)\right)s_n(\xi)d\xi + \underbrace{b_n(0)\exp\left(-\dfrac{\chi n^2 \pi^2}{L^2}(t)\right)}
$$
Soluzione dell'equazione omogenea $b_n'(t)=-\dfrac{\chi n^2 \pi^2}{L^2}b_n(t)$
$$
\Rightarrow T(x,t) = \sum_{n=1}^\infty \int_0^t\exp\left(-\dfrac{\chi n^2 \pi^2}{L^2}(t-\xi)\right)s_n(\xi)d\xi\sin\dfrac{n\pi x}{L}+\sum_{n=1}^\infty b_n(0)\exp\left(-\dfrac{\chi n^2 \pi^2}{L^2}t\right)\sin\dfrac{n\pi x}{L}
$$
\begin{equation}
T(x,t)=\int_0^t \int_0^t G(t-\xi,x,y)S(y,\xi)dyd\xi + \int_0^L G(t,x,y)T_0(y)dy
\end{equation}

-Abbiamo visto che la (74) coincide con la (22), ottenuta dalla (20). Questo vale in generale.: vogliamo risolvere
\begin{equation}
\begin{matrix}
\partial_t T=\chi\Delta T + S(\x,t), \quad \x \in U\\
T(\x,t)|_{\x\in U}=0
\end{matrix} (76)
\end{equation}

A tal fine: funzione di Green tale che 
\begin{equation}
(\partial_t \chi \Delta_{\x})G(t-t',\x,\y)=\delta(t-t')\delta(\x-\y) (77)
\end{equation}

\begin{equation}
\Rightarrow T(\x,t)=\int_{-\infty}^{\infty}dt'\int_{U}d^d\y G(t-t',\x,\y)S(\y,t') (78)
\end{equation}

Sviluppa G e $\delta$ secondo
\begin{equation}
\begin{matrix}
G(t-t',\x,\y)=\dfrac{1}{2\pi}\int d\omega\sum_{n,m}e^{-i\omega(t-t')}\cdot \varphi_n (\x)\varphi_m(\y)\tilde{G}_{n m}(\omega)\\
\delta(t-t')\delta(\x-\y)=\dfrac{1}{2\pi}\int d\omega e^{-i\omega(t-t')}\sum_{n,m}\varphi_n{\x}\varphi_m(\y)\delta_{nm}\end{matrix} (79)
\end{equation}

a causa del bordo l'integrale viene discretizzato.\\
$\varphi$ soddisfa $\Delta \varphi_n+\lambda_n\varphi_n=0$ in U, $\varphi(\x)|_{\x\in\partial U}=0$ formano un sistema completo in U. Motro completezza
$$
\sum_{n,m}\varphi_n(\x)\varphi_m(\y)\delta_n,m=\sum_n\varphi_n(\x)\varphi_n(\y)=\delta(\x-\y)
$$
sostituisco la (79)nella (77)
$$
\dfrac{1}{2\pi}\int d\omega \sum_{n,m}(-i\omega + \chi \lambda_n)e^{-i\omega(t-t')}\cdot \varphi_n(\x)\varphi_m(\y)\tilde{G}_{n,m}(\omega)=\dfrac{1}{2\pi}\int d\omega \sum_{n,m}(-i\omega + \chi \lambda_n)e^{-i\omega(t-t')}\cdot \varphi_n(\x)\varphi_m(\y) \delta_{n m}
$$
$$
\Rightarrow \tilde{G}_{nm}(\omega)=\dfrac{\delta_{nm}}{-i\omega + \chi \lambda_n}
$$
Nella (79): $\Rightarrow$
$$
G(\tau,\x,\y)=\dfrac{1}{2\pi}\int d\omega \sum_{n,m}e^{-i\omega \tau} \varphi_n(\x)\varphi_m(\y)\dfrac{\delta_{nm}}{-i\omega + \chi \lambda_n}
$$
Considera $\int_{-\infty}^{\infty} d\omega \dfrac{e^{-i\omega \tau}}{-i\omega +\chi\lambda_n}$
Affermazione: titti gli autovalori $\lambda_n$ sono positivi:\\
Dimostrazione
$$
0=\int_{U}\varphi_n(\Delta\varphi_n+\lambda_n\varphi_n)d^d\x=\int_{U}\left[\bar{\nabla}\cdot(\varphi_n\bar{\nabla}\varphi_n)-\bar{\nabla}\varphi_n \cdot \bar{nabla}\varphi_n + \lambda_n\varphi_n^2\right]d^d\x
$$
uso Gauss
$$
=\int_{\partial U}\n(\varphi_n\bar{\nabla}\varphi_n)d^{d-1}\x + \int_{U}\left[-(\bar{\nabla}\varphi_n)^2 +\lambda_n\varphi_n^2 \right]d^d\x
$$
il primo integrale è nullo per le condizioni al contorno, $\varphi_n|_{\partial U}=0$
$$
\lambda_n=\dfrac{\int_{U}(\bar{\nabla}\varphi_n)^2d^d\x}{\int_u\varphi_n^2 d^d\x}>0, \quad q.e.d.
$$
$\rightarrow$ polo nel semipiano inferiore:
%immagine di chiusura di percorsi
$$
e^{-i\omega\tau}=e^{-i(Re\omega + i Im\omega)\tau}=e^{-i\tau Re\omega}e^{\tau Im\omega}
$$
Abbiamo gia dimostrato che il contributo del semicerchio è nullo (vedi )%inserisci riferimento
\begin{equation}
\Rightarrow G(\tau,\x,\y)=0, \quad \tau<0 (80)
\end{equation}

$\tau=t-t', \quad \tau<0\Rightarrow t<t' \rightarrow$ contributi solo dal passato nella convoluzione

%%%%%%%%%%%%%%%%%%%%%%%%%%%%%%%%%%%%%%%%%%%%%%%%%%%%%%%%%%%%%%
%% inizio lezione 2/11
%%%%%%%%%%%%%%%%%%%%%%%%%%%%%%%%%%%

$$
\tau >0 : \int_{-\infty}^{\infty} d\omega \dfrac{e^{-i\omega t}}{-i(\omega +i\chi\lambda_n)}=-2\pi i \cdot i e^{-i(-i\chi \lambda_n)\tau} = 2\pi e^{-\chi \lambda_n \tau}
$$
\begin{equation}
\Rightarrow G(\tau,\x,\y)=\sum_{n} e^{-\chi\lambda_n \tau} \varphi_n (\x)\varphi_n (\y) (81) v(20)
\end{equation}

$\Rightarrow$ L'equazione del calore $(\partial_t - \chi \Delta)T=S$ in U ha la soluzione

\begin{equation}
T(\x,t)=\int_0^t dt' \int_U d^d\y G(t-t',\x,\y)S(\y,t') + \int_U d^d\y G(t,\x,\y)T_0(\y) (82)
\end{equation}

Soluzione particolare dell'equazione non omogenea (v.(78)), tenendo conto della (80) e di $S(\y,t')=0$ per $t'<0$ sommata a soluzone dell'equazione omogenea in modo tale che $T(\x,0)=T_0(\x)$

-Altro esempio: equazione di laplace (Poisson) nella palla di raggio R\\
\begin{equation}
\begin{matrix}
\Delta \Phi=-F(r,\theta,\varphi) & r<R\\
\Phi(R,\theta,\varphi=0
\end{matrix} 
\end{equation}
(Elettrostatica: $-F=4\pi\rho$)
$$
\Delta = \partial_{r}^2 + \dfrac{2}{r}\partial_r + \dfrac{1}{r^2\sin\theta}\partial_\theta(\sin \theta \partial_\theta) + \dfrac{1}{r^2\sin^2\theta}\partial^2_{\varphi}
$$
Sviluppo di $\Phi,F$ in armoniche sferiche (reali, non complesse):
$$
\Phi(r,\theta,\varphi)=\sum_{l=0}^{\infty}\left(\dfrac{1}{2}a_{l0}(r)P^0_l(\cos\theta) + \sum_{m=1}^l\left(a_{lm}(r)P_l^m(\cos\theta)\cos m\varphi + b_{lm}(r)P_l^m (\cos\theta) \sin m\varphi\right)\right)
$$
\begin{equation}
F(r,\theta\varphi)=\sum_{l=0}^\infty\left(\dfrac{1}{2} A_{l0}(r)P_l^0(\cos\theta) + \sum_{m=1}^l \left(A_{lm}(r)P_l^m(\cos\theta)\cos m\varphi +B_{lm}(r)P_l^m(\cos\theta)\sin m\varphi\right)\right)
\end{equation}
Sostituisco nella (83)
\begin{equation*}
\sum_{l=0}^\infty \left(\dfrac{1}{2}a''_{l0}(r)P_l^0(\cos\theta) + \dfrac{2}{r}\dfrac{1}{2}a'_{l0}(r)P_l^0(\cos\theta) + \dfrac{1}{r^2}(-l)(l+1)\dfrac{a_{l0}(r)}{2}P_l^0(\cos\theta)\right) +\sum_{l=1}^\infty \sum_{m>0}\left( a''_{lm}(r)P_l^m(\cos\theta)\cos m\varphi + b''_{lm}(r)P_l^m(\cos\theta)\sin m\varphi + \dfrac{2}{r} a'_{lm}(r) P_l^m (\cos\theta)\cos m\varphi + \dfrac{2}{r}b'_{lm}(r)P_l^m(\cos\theta) \sin m\varphi + \dfrac{1}{r^2}a_{lm}(r)(-l)(l+1)P_l^m(\cos\theta)\cos m\varphi + \dfrac{1}{r^2} b_{lm}(r)(-l)(l+1)P_l^m(\cos\theta)\sin m\varphi\right)
\end{equation*}
$$
=-\sum_l \dfrac{1}{2}A_{l0}(r)P_l^0(\cos\theta) - \sum_{l,m>0}\left(A_{lm}(r) P_l^m(\cos\theta)\cos m \varphi + B_{lm}(r)P_l^m(\cos\theta)\sin m\varphi\right)
$$
(usato: $\frac{1}{\sin\theta}\partial_\theta(\sin \theta\partial_\theta(P_l^m(\cos\theta)cos m \varphi)+\frac{1}{\sin^2\theta} \partial^2_\varphi(P_l^m(\cos\theta)\cos m\varphi)=-l(l+1)P_l^m(\cos\theta)\cos m\varphi$. Sfrutto il fatto che le armoniche sferiche $P_l^m(\cos\theta)\cos m\varphi$ e $P_l^m(\cos\theta)\sin m\varphi$ sono linearmente indipendenti
\begin{equation}
\begin{matrix}
\implies a''_{lm}(r) +\dfrac{2}{r}a'_{lm}(r) - \dfrac{l(l+1)}{r^2}a_{lm}(r)=-A_{lm}(r)\\
 b''_{lm}(r) +\dfrac{2}{r}b'_{lm}(r) - \dfrac{l(l+1)}{r^2}b_{lm}(r)=-B_{lm}(r)\\
a_{lm}(R)=b_{lm}(R)=0
\end{matrix}
\end{equation}
Entrambe le equazioni possono essere riscritte nella forma $(pu')' + qu=-f$, con $p=r^2$, $q=-l(l+1)$, $f=A_{lm}(r)r^2$ oppure $B_{lm}(r)r^2$, $u=a_{lm}(r)$ oppure $b_{lm}(r)$. 
Funzione di Green $G(r,\xi)$. Soddisfa l'equazione omogenea $\frac{d}{dr}\left(r^2\frac{dG}{dr}\right)-l(l+1)G=0$. Prova $G(r,\xi)=C(\xi)r^\alpha$, $r\neq \xi$
$$
r^2\dfrac{dG}{dr}=C\alpha r^{\alpha+1}
$$
$$
\implies C\alpha(\alpha+1)r^\alpha - l(l+1)Cr^\alpha=0, \quad \alpha=l \vee \alpha=-l-1
$$
$$
G(r,\xi)=\left\{
\begin{matrix}
c_1(\xi)r^l + c_2(\xi)r^{-l-1} & \xi\leq r\\
\tilde{c}_1(\xi)r^l + \tilde{c}_2(\xi)r^{-l-1} & \xi \geq r
\end{matrix}\right.
$$
N.B. l'estremo r=0 dell'intervallo [0,R] è un punto singolare: $p(r)=r^2$ si annulla in r=0, e la soluzione fondamentale $\sim r^{-l-1}$ diverge.
Al posto di una condizione al contorno inr=0 richiediamo che u (e G) rimangano limitate $\implies \tilde{c}_2=0$
$$
G|_{r=\xi+0}=G|_{r=\xi-0} \implies c_1\xi^l + c_2\xi^{-l-1}=\tilde{c}_1\xi^l
$$
$$
\tilde{c}_10c_1 + c_2 \xi^{-2l-1}
$$
$$
G|_{r=R}=0 \Rightarrow c_1R^l+c_2R^{-l-1}=0 \implies c_2=-c_1R^{2l + 1} \implies \tilde{c}_1=c_1(1-R^{2l+1}\xi^{-2l-1})
$$
$$
\Rightarrow G(r,\xi)=\left\{\begin{matrix}
c_1r^l(1-(\dfrac{R}{r})^{2l+1} & \xi\leq r \\
c_1r^l(1-(\dfrac{R}{\xi})^{2l+1} & \xi\geq r 
\end{matrix}\right.
$$
$$
\dfrac{dG}{dr}|_{r=\xi+0}-\dfrac{dG}{dr}|_{r=\xi-0}=\dfrac{1}{p(\xi)}
$$
$$
c_1=\dfrac{\xi^l}{(2l+1)R^{2l+1}}
$$

\begin{equation}
G_l(r,\xi)=\left\{
\begin{matrix}
\dfrac{1}{(2l+1)R}(\dfrac{\xi}{R})^l[\dfrac{r}{R})^{-l-1} - (\dfrac{r}{R})^l] & \xi\leq r\\
appunti
\end{matrix}\right.
\end{equation}

Inversa della (84):
\begin{equation}
\begin{matrix}
A_{lm}(r)=\dfrac{(2l+1)(l-m)!}{2\pi(l+m)!}\int_{-\pi}^\pi \int_0^\pi F(r,\theta,\varphi)P_l^m(\cos\theta) \cos m\varphi \sin \theta d\theta d\varphi\\

B_{lm}(r)=\dfrac{(2l+1)(l-m)!}{2\pi(l+m)!}\int_{-\pi}^\pi \int_0^\pi F(r,\theta,\varphi)P_l^m(\cos\theta) \cos m\varphi \sin \theta d\theta d\varphi
\end{matrix}(88)
\end{equation}

(usa ortogonalità delle armoniche sferiche).\\
Con la (87) e (88), la (84) diventa
\begin{equation}
\Phi(r,\theta,\varphi) =\int_{-\pi}^\pi \int_0^\pi \int_0^R G(r,\theta,\varphi,r',\theta',\varphi')F(r',\theta',\varphi')r'^2\sin\theta'dr'd\theta'd\varphi' (89)
\end{equation}
dove, per r'<r

\begin{equation*}
G(r,\theta,\varphi,r',\theta',\varphi')=\sum_{l,m>0}\dfrac{(2l+1)(l-m)!}{2\pi(l+m)!}\dfrac{1}{(2l+1)R}(\dfrac{r'}{R})^l[(\dfrac{r}{R})^{-l-1}-(\dfrac{r}{R})^l] P_L^m(\cos\theta)P_l^m(\cos\theta')(\cos m\varphi \cos m\varphi' \sin m\varphi \sin m\varphi') + \sum_l \dfrac{1}{2}\dfrac{2l+1}{2\pi}\dfrac{1}{(2l+1)R}(\dfrac{r'}{R})^l[(\dfrac{r}{R})^{-l-1} - (\dfrac{r}{R})^l]P^0_l (\cos \theta)P_l^0(\cos\theta')
\end{equation*}

(per r'>r : scambia r con r')

\begin{equation*}
\sum_{m=1}^l \dfrac{(2l+1)(l-m)!}{2\pi(l+m)!}P_l^m(\cos\theta)P_l^m(\cos\theta')\underbrace{(\cos m\varphi \cos m\varphi' + \sin m\varphi + \sin m\varphi')} + \dfrac{1}{2}\dfrac{2l+1}{2\pi}P_l^0(\cos\theta)P_l^0(\cos\theta')
\end{equation*}
$$
\dfrac{1}{2}\left(e^{im(\varphi - \varphi')} + e^{-im(\varphi - \varphi')}\right)
$$
\begin{equation*}
Y_l^m(\theta,\varphi)=(-1)^m\sqrt{\dfrac{(2l+1)(l-m)!}{4\pi(l+m)!}}P_l^m(\cos\theta)e^{im\varphi}
\end{equation*}
Riscrivo passando alle armoniche complesse
\begin{equation*}
=\sum_{m=1}^l \left(Y_l^m(\theta,\varphi)Y_l^{m*}(\theta',\varphi') + Y_l^{m*}(\theta,\varphi)Y_l^m(\theta',\varphi')  \right) + Y_l^0(\theta,\varphi)Y_l^{0*}(\theta',\varphi')
\end{equation*}
$$
Y_l^{m*}(\theta,\varphi) = (-1)^m Y_l^{-m}(\theta,\varphi)
$$
$$
Y_l^{m}(\theta',\varphi') = (-1)^m Y_l^{-m*}(\theta',\varphi')
$$
\begin{equation*}
=\sum_{m=-l}^l Y_l^m (\theta,\varphi)Y_l^{m*}(\theta',\varphi')=\dfrac{2l+1}{4\pi}P_l(\cos\gamma)
\end{equation*}

con $\cos\gamma=\cos\theta\cos\theta' + \sin\theta \sin\theta' \cos(\varphi-\varphi')$. Quindi
$$
G(r,\theta,\varphi;r',\theta',\varphi')=\sum_{l=0}^{\infty}\dfrac{1}{(2l+1)R}(\dfrac{r'}{R})^l [(\dfrac{r}{R})^{-l-1} - (\dfrac{r}{R})^l]\dfrac{2l+1}{4\pi}P_l(\cos\gamma)
$$

%%%%%%%%%%%%%%%%%%%%%%%%%%%%%%%%%%%%%%%%%%%%%%%%%%%%%%
%% inizio lezione 8/11
%%%%%%%%%%%%%%%%%%%%%%%%%%%%%%

-funzione generatrice dei polinomi di Legendre:
\begin{equation}
\dfrac{1}{\sqrt{1-2xt + t^2}} = \sum_{n=0}^{\infty} P_n(x)t^n (90)
\end{equation}

\begin{equation*}
G(r,\theta,\varphi;r',\theta',\varphi')=\dfrac{1}{4\pi r} \sum_{l=0}^\infty P_l(\cos \gamma)\left(\dfrac{r'}{r}\right)^l - \dfrac{1}{4\pi R}\sum_{l=0}^\infty P_l(\cos\gamma)\left(\dfrac{rr'}{R^2}\right)^l
\end{equation*}

\begin{equation*}
(90)=\dfrac{1}{4\pi r}\dfrac{1}{\sqrt{1-2\dfrac{r'}{r}\cos\gamma + \dfrac{r'^2}{r^2}}} - \dfrac{1}{4\pi R}\dfrac{1}{\sqrt{1-2\dfrac{rr'}{R^2}\cos\gamma + \dfrac{r^2r'^2}{R^4}}}
\end{equation*}
\begin{equation}
=\dfrac{1}{4\pi}(r^2 + r'^2 - 2rr'\cos\gamma)^{-\dfrac{1}{2}} - \dfrac{1}{4\pi}(R^2 + \dfrac{r^2r'^2}{R^2} - 2rr'\cos\gamma)^{-\dfrac{1}{2}} (91)
\end{equation}
è già simmetrica in $r,r' \rightarrow$ stesso risultato per $r' >r$.\\
in coordinate cartesiane:
\begin{equation}
G(x,y,z,x',y',z')=\dfrac{1}{4\pi}\left\{(x-x')^2 + (y-y')^2 + (z-z')^2\right\}^{-\dfrac{1}{2}} - \dfrac{1}{4\pi}\dfrac{R}{r'}\left\{(x-x'\dfrac{R^2}{r'^2})^2 + (y-y'\dfrac{R^2}{r'^2})^2 + (z-z'\dfrac{R^2}{r'^2})^2\right\}^{-\dfrac{1}{2}} (91')
\end{equation}
usando
\begin{equation*}
R^2  + \dfrac{r^2 r'^2}{R^2} - 2rr'\cos\gamma = \dfrac{r'^2}{R^2}\left(r^2 + \dfrac{R^4}{r'^2} - \dfrac{2 r R^2}{r'}\cos\gamma \right)=\dfrac{r'^2}{R^2}(\bar{r}-\bar{\rho})^2
\end{equation*}

$$
\bar{\rho} := \dfrac{R^2}{r'^2}\bar{r}'
$$
1 termine della (91'): carica puntiforme in $\bar{r} '$.\\
2 termine : carica immagine in $\bar{\rho} =\dfrac{R^2}{r'^2}\bar{r}'$ (che è fuori dalla palla)

%%%%%%  Il capitolo sulle cariche immagine è stato fatto qui, mistero sul perchè lo abbia messo in nuovo capitolo ricominciando la numerazione

\section{Il kernel di Schroedinger}

equazione di Schroedinger

$$
i\hbar \partial_t \psi = \hat{H}\psi =(p. libera)= -\dfrac{\hbar^2}{2m}\Delta \psi 
$$
\begin{equation}
\Rightarrow -i\partial_t \psi =\dfrac{\hbar}{2m}=\chi \Delta \psi (92)
\end{equation}
risulta dall'equazione del calore $\partial_t \psi =\chi \Delta \psi$ tramite la rotazione di Wick $t \rightarrow it$ $(\Rightarrow \partial _t \rightarrow -i\partial_t)$\\
Supponiamo di essere in d dimensioni spaziali. Trasformata di fourier
$$
\psi(\x,t)=\dfrac{1}{(2\pi)^d}\int d^d\kk e^{i\kk \x}\tilde{\psi}(\kk,t)
$$
$$
(92)\Rightarrow -i\partial_t \tilde{\psi}(\kk,t)=\chi (-\kk ^2)\tilde{\psi}(\kk,t)
$$
$$
\Rightarrow \tilde{psi}(\kk,t) = e^{-i\chi\kk t^2}\tilde{\psi}_0(\kk)
$$
$$
\tilde{psi}_0(\kk) = \int d^d \y e^{-i\kk \y} \psi_0(\y)
$$
$$
\Rightarrow \psi(\x,t) = \dfrac{1}{(2\pi)^d}\int d^d \y \psi_0(\y) \cdot \int d^d\kk e^{i\kk(\x-\y) - i\chi \kk^2 t}
$$
definisco $\z :=\x - \y$, \quad $(2\pi)^d G(\z,t) :=\int d^d\kk e^{i\kk(\x-\y) - i\chi \kk^2 t}$
$$
i\kk \z - i \chi \kk^2 t = -i\chi t\left(\kk - \dfrac{\z}{2\chi t}\right)^2 + \dfrac{i\z^2}{4\chi t}
$$
$Re(i\chi t)=0$ $\Rightarrow$ regolarizza $t\rightarrow t-i\epsilon$, $\epsilon>0$ $\implies Re(i\chi(t-i\epsilon))=\chi\epsilon >0 \quad \checkmark$
$$
\Rightarrow G(\z,t)=\dfrac{1}{(2\pi)^d}\exp \left(\dfrac{i\z}{4\chi(t-i\epsilon)}\right)\cdot \int d^d\kk e^{-i\chi(t-i\epsilon)\left( \kk -\dfrac{\z}{2\chi(t-i\epsilon)}\right)^2}
$$
$$
=\dfrac{1}{(2\pi)^d}\exp\left(\dfrac{i\z^2}{4\chi(t-i\epsilon)}\right)\sqrt[d]{\dfrac{\pi}{i\chi(t-i\epsilon)}}
$$
\begin{equation}
\rightarrow (\epsilon \to 0) \left(\dfrac{m}{2\pi i \hbar t}\right)^{\dfrac{d}{2}}\exp \left(-\dfrac{m\z^2}{2i\hbar t}  \right) (93)
\end{equation}
"Schroedinger Kernel" esiste per tutti i $t\in \R$. (l'equazione di Schroedinger è reversibile, quella del calore no)
$$
-i\partial_t \psi =\chi \Delta \psi
$$
$$
(t\to -t)\Rightarrow i\partial_t\psi=\chi\Delta \psi
$$
$$
(\psi \to \psi^*)\Rightarrow i\partial_t\psi^*=\chi\Delta \psi^*
$$
$$
(c.contorno)\Rightarrow -i\partial_t\psi=\chi\Delta\psi
$$
Commenti:
i) $G(\x-\y,t)$ soddisfa l'equazione di Schroedinger $-\partial_t \psi = \chi \Delta_{\x}G$\\
ii) $\lim_{t\to 0} G(\z,t)=\delta(\z)$\\
Più in generale (non necessariamente particella libera):\\
cerchiamo $G^R(\x,\y,t) t.c. $
\begin{equation}
\psi(\x,t)=i\int d^d\y \psi_0(\y) G^R(\x,\y,t) (94)
\end{equation}
$G^R(\x,\y,t)$: funzione di Green ritardata, $G^R(\x,\y,t)$=0 per t<0. Rappresenta l'ampiezza di probabilità che la particella cada dal punto $(\y,t=0)$ al punto $(\x,t)$.\\
$\varphi_n$ autofunzioni di $\hat{H}$
\begin{equation}
\hat{H}\varphi_n=E_n \varphi_n (95)
\end{equation}

\begin{equation}
\Rightarrow G^R(\x,\y,t)=-i\sum_n e^{-iE_nt/\hbar}\varphi_n(\x)\varphi_n^* (\y), \quad t\geq 0(96)
\end{equation}
vedi le (20) e (81)
$$
(94)\Rightarrow \psi(\x,t)=\int d^d\y \sum_n e^{-iE_nt/\hbar}\varphi_n(\x)\varphi_n^* (\y)\psi_0(\y)
$$
sooddisfa l'equazione di Schroedinger:
$$
i\hbar\partial_t\psi(\x,t)=\int d^d\y \sum_n i\hbar \dfrac{-iE_nt}{\hbar} e^{-iE_nt/\hbar}\varphi_n(\x)\varphi_n^* (\y)\psi_0(\y)
$$
$$
\hat{H}_{\x}\psi(\x,t)=\int d^d\y \sum_n e^{-iE_nt/\hbar}\underbrace{\hat{H}_{\x}\varphi_n(\x)}\varphi_n^* (\y)\psi_0(\y)
$$
$$
E_n\varphi_n(\x)
$$
Inoltre: $\psi(\x,0)=\int d^d\y \underbrace{\sum_n \varphi_n(\x)\varphi_n^* (\y)}\psi_0(\y)=\psi_0(\x) \qquad \checkmark$
$$
\delta(\x-\y)
$$
Trasformata di Fourier:
$$
\tilde{G}^R(\x,\y,E)=\int_{-\infty}^{\infty}dt e^{iEt/\hbar}G^R(\x,\y,t)= \int_0^\infty dt e^{iEt/\hbar}G^R(\x,\y,t)=(96)=-i\sum_n\int_0^\infty dt e^{i(E-E_n)t/\hbar}\varphi_n(\x)\varphi_n^*(\y)
$$
l'ultimo integrale non è ben definito se $E\in \R \rightarrow$ regolarizza tramite $E\to E+i\epsilon,\quad \epsilon >0$
\begin{equation}
\Rightarrow \tilde{G}^R(\x,\y,E)=\sum_n\dfrac{\varphi_n(\x)\varphi_n^*(\y)}{E+i\epsilon - E_n}(97)
\end{equation}

%%%%%%%%%%%%%%%%%%%%%%%%%%%%%%%%%%%%
%% inizio lezione 9/11
%%%%%%%%%%%%%%%%%%%%%%%%%%%%%%%%%%%%

Abbiamo 
\begin{equation}
(E + i\epsilon - \hat{H}_{\x})\tilde{G}(\x,\y,E)=\sum_n \dfrac{(E + i\varepsilon - \hat{H}_{\x}) \varphi_n(\x)\varphi_n^*(\y)}{E + i\varepsilon - E_n} = \sum_n \varphi_n(\x)\varphi_n^*(\y)=\delta(\x-\y)
\end{equation}
Funzione di Green \underline{avanzata}: 
$$
G^A(\x,\y,t)
$$
$G^A(\x,\y,t)=0$ per t>0. Affinché la sua trasformata di fourier
$$
\tilde{G}^A(\x,\y,E)=\int_{-\infty}^0 dt e^{iEt/\hbar}G^A(\x,\y,t)
$$
esista, dobbiamo sostituire $E\to E-i\varepsilon, \varepsilon >0$
$$
\tilde{G}^A(\x,\y,E)=\sum_n \dfrac{\varphi_n(\x)\varphi_n^*(\y)}{E - i\varepsilon - E_n}
$$
Esercizio: calcolare G per l'oscillatore armonico.
$$
\hat{H}\varphi_n = -\dfrac{\hbar^2}{2m}\varphi''(x) + \dfrac{1}{2}m\omega^2 x^2 \varphi_n(x)=E_n\varphi_n(x)
$$
$$
E_n=\hbar\omega\left(n+\dfrac{1}{2}\right);\quad n=0,1,2,\dots 
$$
$$
\varphi_n(x)=(\sqrt{\pi}2^n n!)^{-\dfrac{1}{2}}\exp\left(-\dfrac{m\omega x^2}{2\hbar}\right)\cdot\underbrace{H_n\left(\left(\dfrac{m\omega}{\hbar}\right)^{\dfrac{1}{2}}x\right)}
$$
polinomi di Hermite
$$
(96)\Rightarrow -i\sum_{n=0}^\infty e^{-iE_n t/\hbar}\varphi_n(x)\varphi_n^*(y)
$$
$$
=-i\sum_{n=0}^\infty e^{-i\hbar\omega(n+1/2)} \dfrac{1}{\sqrt{\pi} 2^n n!} \exp\left(-\dfrac{m\omega}{2\hbar}(x^2+y^2)\right) H_n\left(\left(\dfrac{m\omega}{\hbar}\right)^{\dfrac{1}{2}}x\right)H_n\left(\left(\dfrac{m\omega}{\hbar}\right)^{\dfrac{1}{2}}y\right)
$$
Usa 
\begin{equation}
\sum_{n=0}^\infty \dfrac{H_n (x)H_n (y)}{n!} \left(\dfrac{u}{2}\right)^n=\dfrac{1}{\sqrt{1-u^2}}\exp\{\dfrac{2u}{1+u}xy - \dfrac{u^2}{1-u^2}(x-y)^2\}(99)
\end{equation}
$$
(u=e^{-i\omega t}) \Rightarrow -i\sum_{n=0}^\infty e^{-iE_n t/\hbar} \varphi_n(x)\varphi_n^*(y) = 
$$
$$
-i e^{-i\omega t/2}\pi^{-\dfrac{1}{2}}\exp\left\{-\dfrac{m\omega}{2\hbar}(x^2+y^2)\right\}\dfrac{1}{\sqrt{1-e^{-2i\omega t}}}\exp\left\{\dfrac{2e^{-i\omega t}}{1+e^{-i\omega t}} \dfrac{m\omega}{\hbar}xy - \dfrac{e^{-2i\omega t}}{1-e^{-2i\omega t}}\dfrac{m\omega}{\hbar}(x-y)^2\right\}
$$
\begin{equation}
=-\sqrt{\dfrac{i}{2\pi\sin \omega t}}\exp\left\{ \dfrac{i\omega t}{\hbar}\left[\dfrac{x^2+y^2}{2}\cot \omega t - xy \mathrm{cosec} \omega t\right]\right\}(100)
\end{equation}
"Mehler Kernel"

\section{L'equazione dei telegrafisti} %ha detto lui stesso che è un nuovo paragrafo
equazioni di Maxwell in materia, sistema CGS:
\begin{subequations}
\begin{equation}
div \bar{D} = 4\pi \rho_{macr.}(101a)
\end{equation}
\begin{equation}
rot \bar{H}=\dfrac{4\pi}{c}\bar{j}_{macr.} + \dfrac{1}{c}\dfrac{\partial \bar{D}}{\partial t}(101b)
\end{equation}
\begin{equation}
div\bar{B}=0 (101c)
\end{equation}
\begin{equation}
rot\bar{E}=-\dfrac{1}{c}\dfrac{\partial \bar{B}}{\partial t}(101d)
\end{equation}
\end{subequations}
Legge di materiale (caso più semplice)
\begin{equation}
\begin{matrix}
\bar{D} = \varepsilon \bar{E}
\bar{B} = \mu \bar{H}
\end{matrix} (102)
\end{equation}
legge di Ohm, $\sigma$ rappresenta la conducibilità elettrica
\begin{equation}
\bar{j}_{macr} = \sigma \bar{E}(103)
\end{equation}
$$
(101b)\Rightarrow rot Rot \bar{H}  \dfrac{4\pi}{c}rot \bar{j}_macr + \dfrac{i}{c}\dfrac{\partial}{\partial t}rot \bar{D}
$$
=0
$$
(103),(101d),(101c)\Rightarrow\overbrace{grad div \bar{H}} - \Delta \bar{H} = \dfrac{4\pi\sigma}{c}\underbrace{rot\bar{E}} - \dfrac{\varepsilon}{c^2}\dfrac{\partial^2}{\partial t^2} \bar{B}
$$
$-\dfrac{1}{c}\dfrac{\partial }{\partial t}\bar{B}$
\begin{subequations}
\begin{equation}
\Rightarrow \Delta \bar{B}=\dfrac{4\pi\sigma \mu}{c^2}\dfrac{\partial \bar{B}}{\partial t} + \dfrac{\epsilon \mu}{c^2}\dfrac{\partial^2 \bar{B}}{\partial t?2}(104a)
\end{equation}
\text{equazione dei telegrafisti}
$$
(101d)\Rightarrow rot rot \bar{E} = -\dfrac{1}{c}\dfrac{\partial}{\partial t}rot\bar{B}=-\dfrac{\mu}{c}\dfrac{\partial}{\partial t}rot \bar{H}
$$
$$
(101b)=-\dfrac{\mu}{c}\dfrac{\partial}{\partial t}\left(\dfrac{4\pi}{c}\sigma \bar{E} + \dfrac{\epsilon}{c}\dfrac{\partial}{\partial t}\bar{E}\right)=grad\underbrace{div \bar{E}} - \Delta \bar{E}
$$
\text{=0 se $j_{macr}=0$}

\begin{equation}
\Rightarrow \Delta \bar{E}=\dfrac{4\pi\sigma\mu}{c^2}\dfrac{\partial \bar{E}}{\partial t} + \dfrac{\varepsilon \mu}{c^2}\dfrac{\partial^2 \bar{E}}{\partial t^2}(104b)
\end{equation}
\end{subequations}
è equivalente all'equazione delle onde, con l'aggiunta del termine dissipativo.\\
Casi limite: $\sigma=0$ equazione delle onde. $\epsilon=0$: equazione del calore con $\chi=\dfrac{c^2}{4\pi \sigma \mu}$

Esercizio: risolvere l'equazione dei telegrafisti $\Delta u = a\partial_t u + b \partial^2_t u$ (a,b costanti positive) nell'intervallo $0\leq x\leq L$, $t\geq 0$, con $u(0,t)=u(L,t)=0$, $u(x,0)=0$, $\dfrac{\partial u}{\partial t}(x,0) = g(x)$
Soluzione: serie di Fourier
$$
u(x,t)=\sum_{n=1}^\infty u_n(t) \sin \dfrac{n\pi x}{L}
$$
con 
$$
u_n(t)=\dfrac{2}{L}\int_0^Lu(x,t)\sin \dfrac{n\pi x}{L}dx \Rightarrow -\dfrac{n^2\pi^2}{L^2}u_n(t)=au_n'(t) + b u_n''(t)
$$

ansatz $u_n(t)=\alpha e^{\lambda_n t}$
$$
\Rightarrow -\dfrac{n^2\pi^2}{L^2}=a\lambda_n + b \lambda_n^2
$$
$$
\Rightarrow \lambda_n = -\dfrac{a}{2b} \pm \sqrt{\dfrac{a^2}{4b^2}-\dfrac{n^2\pi^2}{L^2 b}}\equiv \lambda_n^\pm
$$
$\rightarrow$ comportamento oscillatorio se $\dfrac{n^2\pi^2}{L^2}>\dfrac{a^2}{4b}$
$$
u_n(t)=\alpha_n^+ e^{\lambda_n^+ t} + \alpha_n^-e^{\lambda_n^- t}
$$
$$
u(x,0)=0 \Rightarrow u_n(0)=0 \Rightarrow \alpha_n^- = \alpha_n^+
$$
$$
\Rightarrow u(x,t)=\sum_{n=1}^\infty\alpha_n^+(e^{\lambda_n^+ t}-e^{\lambda_n^- t})\sin\dfrac{n\pi x}{L}
$$
$$
\dfrac{\partial u(x,t)}{\partial t} =\sum_{n=1}^\infty \left(\lambda_n^+ e^{\lambda_n^+ t}-\lambda_n^-e^{\lambda_n^- t}  \right)\sin \dfrac{n\pi x}{L}
$$
$$
\left.\dfrac{\partial u(x,t)}{\partial t}\right|_{t=0}= \sum_{n=1}^\infty \alpha_n^+(\lambda_n^+ - \lambda_n^-)\sin\dfrac{n\pi x}{L} =: g(x) = \sum_{n=1}^\infty g_n \sin \dfrac{n\pi x}{L}
$$
$$
\alpha_n^+ = \dfrac{g_n}{\lambda_n^+ - \lambda_n^-}=\dfrac{g_n}{2\sqrt{\dfrac{a^2}{4b^2}-\dfrac{n^2\pi^2}{L^2 b}}}=\dfrac{1}{2\sqrt{\dfrac{a^2}{4b^2}-\dfrac{n^2\pi^2}{L^2 b}}} \dfrac{2}{L}\int_{0}
^{L}g(x')\sin\dfrac{n\pi x'}{L} dx'
$$

e quindi

$$
u(x,t)=\dfrac{1}{L}\int_0^L dx' \sum_{n=1}^\infty \dfrac{e^{\lambda_n^+ t} - e^{\lambda_n^- t}}{\sqrt{\dfrac{a^2}{4b^2}-\dfrac{n^2\pi^2}{L^2 b}}}\sin\dfrac{n\pi x'}{L}\sin \dfrac{n\pi x}{L}g(x')=\int_0^L G(x,x',t)g(x')dx'
$$

con la funzione di Green
$$
G(x,x',t)=\dfrac{1}{L} \sum_{n=1}^\infty \dfrac{e^{\lambda_n^+ t} - e^{\lambda_n^- t}}{\sqrt{\dfrac{a^2}{4b^2}-\dfrac{n^2\pi^2}{L^2 b}}}\sin\dfrac{n\pi x'}{L}\sin \dfrac{n\pi x}{L}
$$
N.B. se esiste $n_0$ tale che
$$
\lambda_{n_0}^+ = \lambda_{n_0}^-
$$
cioè 
$$
\sqrt{\dfrac{a^2}{4b^2}-\dfrac{n^2\pi^2}{L^2 b}}=0
$$
scrivi
$$
\sqrt{\dfrac{a^2}{4b^2}-\dfrac{n^2\pi^2}{L^2 b}}= \dfrac{e^{\left(-\dfrac{a}{2b}+\sqrt{\cdots}\right)t}- e^{\left(-\dfrac{a}{2b}-\sqrt{\cdots}\right)t}}{\sqrt{\cdots}}
$$
$$
\dfrac{e^{-\dfrac{at}{2b}} 2\sinh (\sqrt{\cdots} t)}{\sqrt{\cdots}}\xrightarrow{\sqrt{\cdots}\to 0} \dfrac{e^{-\dfrac{at}{2b}}2\sqrt{\cdots}t}{\sqrt{\cdots}}=2te^{-\dfrac{at}{2b}}
$$

\chapter{Classificazione delle equazioni differenziali alle derivate parziali lineari del 2 ordine}
per semplicità ci limitiamo alle equazioni in 2 variabili, 
\begin{equation}
a_{11}(x,y)u_{xx} + 2a_{12}(x,y)u_{xy} a_{22}(x,y)u_{yy} + b_{1}(x,y)u_{x} + b_2(x,y)u_y + c(x,y)u = d(x,y) (105)
\end{equation}
La classificazione è basata sulla sola parte contenente le derivate seconde, $a_{11}u_{xx} + 2a_{12}(x,y)u_{xy} a_{22}(x,y)u_{yy}$, detta \underline{parte principale} dell'equazione.\\
\underline{Definizione}: Sia $\delta(x,y)=a_{12}^2 - a_{11}a_{22}$, il \underline{discriminante} della parte principale\\
Se $\delta >0$ l'equazione (105) si dice \underline{iperbolica}\\
Se $\delta =0$ l'equazione (105) si dice \underline{parabolica}\\
Se $\delta <0$ l'equazione (105) si dice \underline{ellittica}\\
\underline{N.B.} la definizione si estende anche alle equazioni \underline{quasilinearli} (nelle quali i coefficienti $a_{ij}$ possono dipendere anche da $u, u_x,u_y$) se si riesce a stabilire il segno di $\delta$.\\
Esempio: \underline{superfici minime}\\
Considera superficie in $\R^3$ definita da $z=u(x,y)$, $x,y \in \Omega \subset \R^2$
%immagine superfici minime
$$
\R^3 \ni \x = \left( \begin{matrix}
x\\
y\\
u(x,y)
\end{matrix}\right)
$$
vettori tangenti:
$$
\partial_x \x = \left( \begin{matrix}
1\\
0\\
u_x
\end{matrix}\right) ,\quad 
\partial_y \y = \left( \begin{matrix}
0\\
1\\
u_y
\end{matrix}\right)
$$
vettore normale
$$
\bar{N}=\partial_x \x \times \partial_y\y=\left| \begin{matrix}
e_x & e_y & e_z \\
1 & 0 & u_x \\
0 & 1 & u_y
\end{matrix}\right| = e_z-u_xe_x - u_y e_y = \left( \begin{matrix}
-u_x \\
-u_y \\
1
\end{matrix}\right)
$$

Elemento di superficie:
$$
dA = |\bar{N}|dxdy=\sqrt{u_x^2 u_y^2 +1}dxdy 
$$

$\Rightarrow$ area della superficie:
\begin{equation}
A=\int_{\Omega} \sqrt{1+(\nabla u)^2} dxdy\equiv \int_{\Omega} \mathcal{L}dxdy (106)
\end{equation}

Superfici minime: $\delta A=0$
Equazione di Eulero-Lagrange:
\begin{equation}
\partial_i \dfrac{\mathcal{L}}{\partial \partial_i u}-\dfrac{\partial \mathcal{L}}{u}=0 (107)
\end{equation}
(somma su i da 1 a 2 sottintesa)
$$
\partial_i=\dfrac{\partial}{\partial_i} \qquad i=1,2 \qquad x_1=x,\quad x_2=y
$$
[paragona con $\dfrac{d}{dt}\dfrac{\partial \mathcal{L}}{\partial \dot{u}}-\dfrac{\partial\mathcal{L}}{\partial u}=0$]
$$
(107)\Rightarrow \partial_i \left( \dfrac{\partial_i u}{\sqrt{1+(\nabla u)^2}}\right)=0
$$
%pezzo che non ricordo dove vada, credo qui.
$$
\dfrac{\partial\mathcal{L}}{\partial \partial_i u} = \dfrac{1}{2}(1 +(\nabla u)^2)^{-\dfrac{1}{2}} 2\partial_i u=\dfrac{\partial_i u}{\sqrt{(1+(\nabla u)^2}}, \qquad \dfrac{\partial \mathcal{L}}{\partial u}=0 
$$
oppure
\begin{equation}
\nabla \cdot \left( \dfrac{\partial_i u}{\sqrt{1+(\nabla u)^2}}\right)=0 (108)
\end{equation}
"equazione delle superfici minime"

%%%%%%%%%%%%%%%%%%%%%%%%%%%%%%%%%%%
%% inizio lezione 15/11
%%%%%%%%%%%%%%%%%%%%%%%%%%%%%%%%%%
(Senza usare la (107):
$$
\delta\mathcal{L}=\delta\sqrt{1+(\nabla u)^2}=\dfrac{1}{2}\left(1+(\nabla u)^2\right)^{-\dfrac{1}{2}}2\nabla u \underset{\nabla \delta u}{\underbrace{\delta\nabla u}} \overset{\text{int. parti}}{\underset{\delta u|_{bordo}=0}{\longrightarrow}}
 -\delta u \nabla \left(\dfrac{\nabla u}{\sqrt{1+(\nabla u)^2}}\right)\overset{!}{=}0
$$
$$
(108)\Leftrightarrow \partial_i \left( \dfrac{\partial_i u}{\sqrt{1+(\nabla u)^2}}\right) =0 \Rightarrow \dfrac{\partial_i \partial_i u}{\sqrt{1+(\nabla u)^2}} + \partial_i u \left(-\dfrac{1}{2}\right) (1+(\nabla u)^2)^{-\frac{3}{2}} \cdot \underset{\partial_i(\partial_j u \partial_j u)}{\underbrace{\partial_i (\nabla u)^2}} =0
$$
$$
\partial_i(\partial_j u \partial_j u)=2(\partial_i\partial_j u)\partial_j u 
$$
$$
\Rightarrow (1+\underset{(\partial_x u)^2 + (\partial_y u)^2}{\underbrace{(\nabla u)^2}})\underset{\partial_x^2 u + \partial_y^2 u}{\underbrace{\Delta u}} =- \partial_x u \partial_x u \partial_x^2 u - \partial_x u\partial_y u \partial_x\partial_y u - \partial_yu \partial_x u \partial_y \partial_x u - \partial_y u\partial_y u \partial_y^2 u =0
$$
$$
\Rightarrow u_{xx} + u_{yy} + u_x^2 u_{yy} + u_y^2 u_{xx} - 2u_xu_y u_{xy}=0
$$
\begin{equation}
\Rightarrow (1+u_y^2)u_{xx} + (1+u_x^2) u_{yy} - 2u_x u_y u_{xy}=0 (108')
\end{equation}
$$
a_{11}=1+u_y^2, \qquad a_{22}=1+u_x^2, \qquad a_{12}=-u_x u_y
$$
$$
\delta = a_{12}^2 -a_{11}a_{22} = u_x^2 u_y^2 - (1+u_x^2 u_y^2 + u_x^2u_y^2)=-(1+(\nabla u)^2) <0
$$
$\Rightarrow$  equazione ellittica

def. $ A:=\left(\begin{matrix}
a_{11} & a_{12}\\
a_{12} & a_{22}
\end{matrix}\right)
$, $det A=-\delta$\\
iperbolica ($\delta >0$): $detA<0$, 2 autovalori di segno opposto\\
parabolica ($\delta =0$): 1 autovalore  =0\\
ellittica ($\delta <0$): 2 autovalori di segno uguale\\
$\grave{E}$ il parallelismo col comportamento della forma quadratica generata da A a suggerire la medesima denominazione delle coniche. \\
Esempi:\\
Equazioni ellittiche: laplace, Poisson (parte principale $\Delta u$)\\
Equazioni paraboliche: equazione del calore $\partial_t u = \chi \Delta u$ nel caso di 2 variabili: $\partial_t u = \chi \partial_x^2 u \Rightarrow a_{11}=\chi, \quad a_{12}=0 = a_{22} $\\
Iperboliche: equazione delle onde e dei telegrafisti ( quest'ultima per $\epsilon \neq 0$, altrimenti è parabolica), equazione di Klein-Gordon.

\section{Il problema di Cauchy}
La classificazione ha una prima motivazione nel problema di Cauchy: trovare una funzione $u\in C^2$ che soddisfi la (105) e i dati di Cauchy, ossia:
\begin{equation}
u|_{\gamma}=\Phi|_{\gamma}, \quad \left.\dfrac{\partial u}{\partial n}\right|_{\gamma}=\left.\Psi\right|_{\gamma} (109)
\end{equation}

dove $\Phi$, $\Psi$ sono funzioni assegnate e $\gamma$ è una data urva regolare con normale $\bar{n}$, $\dfrac{\partial n}{\partial n}=\bar{n}\cdot\bar{\nabla} u$.\\
%immagine curva regolare con normale%
Possibile modo di tentare di risolvere il problema: supporre che le $a_{ij}$, la $\gamma$ e i dati $\Phi$, $\Psi$ siano analitici e cercare \\
i) di calcolare tutte le derivate di u su $\gamma$\\
ii) di dimostrare che lo sviluppo di Taylor di u è convergente in un intorno di $\gamma$.\\
Se il procedimento ha successo si è costruita una soluzione analitica. \\
Questo è l'obiettivo del \underline{teorema di Cauchy - Kowalevski}: se i coefficienti $a_{ij}$, $b_i$, c, d sono analitici in un dominio D, se $\gamma \subset D$ è analitica e i dati di Cauchy $\Phi$, $\Psi$ sono analitici in D, il problema di cauchy (105) con dati iniziali (109) ammette una e una sola soluzione analitica in un opportuno intorno $I\subset D$ di $\gamma$, purché la normale $\bar{n}$ a $\gamma$ verifichi la condizione 
\begin{equation}
\bar{n}\cdot A \bar{n}\neq 0 (110).
\end{equation}
(senza dim.)\\
$\rightarrow$ la (110) chiama in causa la classificazione.
Il ruolo della (110) è quello di consentire il calcolo delle derivate seconde:
$$
\bar{n}\equiv (\alpha,\beta)
$$
Deriva tangenzialmente il primo dato di cauchy
%immagine derivata tangenzialmente gamma%
$$
\dfrac{\partial}{\partial\bar{\tau}} \equiv \bar{\tau}\cdot \bar{\nabla}, \qquad \bar{\tau}= (-\beta,\alpha), \qquad \left.\dfrac{\partial u}{\partial \tau}\right|_{\gamma} = \left. \dfrac{\partial \Phi}{\partial \tau}\right|_{\gamma}, \qquad \alpha^2 + \beta^2 =1
$$
\begin{equation}
\Rightarrow (-\beta u_x +\alpha u_y)|_{\gamma}=\left.\dfrac{\partial \Phi}{\partial \tau}\right|_{\gamma} (111)
\end{equation}
Secondo dato di Cauchy:
\begin{equation}
(\alpha u_x + \beta u_y)|_{\gamma} = \Psi|_{\gamma} (112)
\end{equation}
$$
(111),(112) \underset{\alpha^2 + \beta^2 \neq 0}{\implies} 
$$
\begin{eqnarray}
u_x|_{\gamma}=p_0(s),\\
u_y|_{\gamma}=q_0(s) (113)
\end{eqnarray}
dove $p_0$, $q_0$ sono espresse tramite i dati e s è il parametro naturale su $\gamma$ (lunghezza della curva).\\
Deriva (113) lungo $\gamma$:
$$
\bar{\tau}\cdot \bar{\nabla} u_x|_{\gamma}= p_0'(s)
$$
$$
\bar{\tau}\cdot \bar{\nabla}=\dfrac{d}{ds}
$$
$$
\bar{\tau}\cdot \bar{\nabla}u_y|_{\gamma}=q_0'(s),
$$
$$
\Leftrightarrow \left\{\begin{matrix}
(-\beta u_{xx} + \alpha u_{xy})|_{\gamma}=p_0'\\
(-\beta u_{xy} + \alpha u_{yy})|_{\gamma}=q_0'
\end{matrix}\right.
$$
Anche la parte principale nella (105) è esprimibile su $\gamma$ in base ai dati, perchè tutti i termmini rimanenti sono calcolabili:
$$
(a_{11}u_{xx}+2a_{12}u_{xy}+a_{22}u_{yy})|_{\gamma}=r_0(s)
$$
$\Rightarrow$ sistema di 3 equazioni lineare con determinante
\begin{equation}\left| \begin{matrix}

a_{11} & 2a_{12} & a_{22}\\
-\beta & \alpha & 0 \\
0 & -\beta & \alpha 
\end{matrix}\right| = \bar{n}\cdot A \bar{n}(114)
\end{equation}
$\Rightarrow$ la condizione (110) consente di calcolare le derivate seconde. Nello stesso modo di calcolano, sempre grazie alla (110), tutte le derivate successive.\\
Interpretazione geometrica della negazione della (110)?
\begin{equation}
\bar{n}\cdot A\bar{n}=a_{11}\alpha^2+2a_{12}\alpha\beta + a_{22}\beta^2=0 (115)
\end{equation}
Autovalori di A: $\underset{\in\R}{\underbrace{\lambda_1,\lambda_2}}$; con autovettori ortogonali $\bar{\Theta_1},\bar{\Theta_2}$ (A simmetrica).
$$
\bar{n}=: \eta_1 \bar{\Theta_1} + \eta_2\bar{\Theta_2} \Rightarrow \bar{n}\cdot A \bar{n}=\lambda_1\eta_1^2 + \lambda_2\eta_2^2
$$
$\rightarrow$ nel caso ellittico, questo è sempre $\neq 0$.\\
Caso parabolico: $A\bar{n}=0$ per $\bar{n}$ nell'autospazio corrispondente all'autovalore nullo.\\
Caso iperbolico: $\bar{n}\cdot A\bar{n} = 0$ per $\dfrac{\eta_1}{\eta_2}=\pm \sqrt{\left|\dfrac{\lambda_1}{\lambda_2}\right|} \rightarrow$ 2 campi vettoriali
$$
\bar{w}_1 = \sqrt{|\lambda_2|}\bar{\Theta}_1 + \sqrt{|\lambda_1|}\bar{\Theta}_2, \qquad (\eta_2=\sqrt{|\lambda_1|}, \eta_1=\sqrt{|\lambda_2|})
$$ 
$$
\bar{w}_2 = \sqrt{|\lambda_2|}\bar{\Theta}_1 + \sqrt{|\lambda_1|}\bar{\Theta}_2, \qquad (\eta_2=\sqrt{|\lambda_1|}, \eta_1=-\sqrt{|\lambda_2|})
$$
$\cdot$ curve $\perp$ a questi campi:
$$
\chi_{1,2}(x,y)=\text{cost}
$$
%immagine curva perp%

$\implies \bar{n} \propto \bar{\nabla}\chi \Rightarrow (\alpha,\beta) \propto (\chi_x, \chi_y)$ e quindi (115) $\Rightarrow$
\begin{equation}
a_{11}\chi_x^2 + 2a_{12}\chi_x\chi_y + a_{22}\chi_y^2 =0 (116)
\end{equation}
Poni $\chi =: f(x)-y \Rightarrow \chi_x = f', \quad \chi_y=-1 $
$$
(116)\Rightarrow a_{11}f'^2 - 2 a_{12}f' + a_{22}=0 (116')
$$
equazione di 2 grado per f'.\\
nel caso parabolico ci si riduce ad una sola famiglia, nel caso ellittico non ci sono curve di questo genere.\\
\underline{Definizione}: le curve $\chi(x,y)=$cost. soddisfacenti la (116) si chiamano \underline{curve caratteristiche} dell'equazione (105)\\
Esempi:\\
i) equazione d'onda in 1+1 dimensioni
$$
c^2 u_{xx}-u_{tt}=0, \qquad a_{11}=c^2, \quad a_{12}=0, \quad a_{22}=-1
$$
$$
(116')\Rightarrow c^2f'^2 =1 \implies f=\pm \dfrac{x}{c} + f_0
$$
$$
\chi=f(x) - t= \pm \dfrac{x}{c}+f_0 -t \overset{!}{=}cost \Rightarrow x-x_0 \pm c(t-t_0)=0
$$
ii) equazione del calore 
$$
\mathcal{H}u_{xx}- u_t=0, \qquad a_{11}=\mathcal{H}, \quad a_{12}=a_{22}=0
$$
$$
(116')\Rightarrow \mathcal{H} f'^2 =0 \Rightarrow f=cost=:f_0, \quad \chi = f_0-t=cost \Rightarrow t=cost. \text{ho delle rette}
$$
-Cambiamento di coordinate:
Sia $\xi=\xi(x,y)$, $\eta=\eta(x,y)$ cambiamento di coordinate con determinante Jacobiano $\neq 0$.
$$
v(\xi,\eta):=u(x,y)
$$
Come si trasforma la parte principale della (105)?
$$
u_x=v_\xi \xi_x + v_\eta \eta_x,\quad u_y=v_\xi \xi_y + v_\eta \eta_y
$$
$$
u_{xx}=v_{\xi \xi}\xi_x^2 + 2v_{\xi\eta}\xi_x\eta_x + v_{\eta\eta}\eta_x^2 + v_xi\eta_{xx} +v_\eta \eta_{xx}
$$
$$
u_{xy}= v_{\xi \xi}\xi_x\xi_y + 2v_{\xi\eta}(\xi_x\eta_y + \xi_y\eta_x) + v_{\eta\eta} \eta_x\eta_y + v_\xi \xi_{xy} + v_\eta \eta_{xy}
$$
$$
u_{yy}= v_{\xi\xi }\xi_y^2 + 2v_{\xi\eta}\xi_y\eta_y + v_{\eta\eta}\eta_y^2 + v_\xi \xi_{yy} + v_\eta \eta_{yy}
$$
$$
\dfrac{\partial}{\partial x} = \dfrac{\partial \xi}{\partial x}\dfrac{\partial}{\partial \xi} + \dfrac{\partial \eta}{\partial x}\dfrac{\partial}{\partial \eta}
$$
$$
\dfrac{\partial}{\partial y} = \dfrac{\partial \xi}{\partial y}\dfrac{\partial}{\partial \xi} + \dfrac{\partial \eta}{\partial y}\dfrac{\partial}{\partial \eta}
$$
%%%%%%%%%%%%%%%%%%%%%%%%%%%%%%%%%%%%%%%%%%
% inizio lezione 16/11
%%%%%%%%%%%%%%%%%%%%%%%%%%%%%%%%%%%%%%%%%%%%

$\rightarrow$ la parte principale della (105) diventa $ \tilde{a}_{11} v_{\xi\xi} + 2 \tilde{a}_{12}v_{\xi\eta} + \tilde{a}_{22}v_{\eta\eta}$, con 
$$
\tilde{a}_{11}= a_11 \xi_x^2 + 2a_{12}\xi_x\xi_y + a_{22} \xi_y^2 = \bar{\nabla}\xi \cdot A\bar{\nabla}\xi
$$
$$
\tilde{a}_{12}= a_11 \xi_x\eta_x + a_{12}(\xi_x\eta_y+\xi_y\eta_x) + a_{22} \xi_y\eta_y = \bar{\nabla}\eta \cdot A\bar{\nabla}\xi = \bar{\nabla}\xi \cdot A\bar{\nabla}\eta
$$
$$
\tilde{a}_{22}= a_11 \eta_x^2 + 2a_{12}\eta_x\eta_y + a_{22} \eta_y^2 = \bar{\nabla}\eta \cdot A\bar{\nabla}\eta
$$
\begin{equation}
\implies \tilde{A} = J A J^{-1}, \quad \text{J = Jacobiano} (117)
\end{equation}
$$
\Rightarrow det \tilde{A} = det A\underset{>0}{\underbrace{(det J)^2}}
$$
$\Rightarrow$ le trasformazioni invertibili di coordinate lasciano invariante il carattere dell'equazione.\\
Esempio: equazione d'onda $c^2u_{xx} - u_{tt}=0$. Prendi $\xi=x-ct$, $\eta = x+ct$, ("coordinate di cono luce")  poni $v(\xi,\eta) = u(x,y) \implies \tilde{A}=\left(\begin{matrix}
0 & 1\\
1 & 0
\end{matrix}\right)$, e l'equazione diventa $v_{\xi\eta}=0$. ($\rightarrow$ soluzione generale: $v(\xi,\eta)=v_1(\xi) + v_2(\eta)$, con $v_1$, $v_2$ funzioni arbitrarie.\\
\underline{- riduzione alla forma canonica}:
forme canoniche:\\
$u_{xx} + u_{yy} \quad$ ellittica\\
$u_{xx} (- u_t) \quad$ parabolica\\
$u_{xx} - u_{tt} \quad$ iperbolica (in alternativa $u_{xt}$)\\
N.B. costanti come per esempio $c^2$ in $c^2 u_{xx}-u_{tt}$ possno essere riassobite nella scala delle variabili.\\
Trasformazione di una parte principale generica nella forma canonica:\\
\underline{Caso iperbolico}:\\
L'esempio sopra suggerisce di prendere le curve caratteristiche come nuove linee coordinate $\xi=\chi_1(x,y),\quad \eta=\chi_2(x,y)$
%questa matrice era scritta in forma piu compatta, con uan croce a separare i 4 blocchi
$$
(117)\Rightarrow \tilde{A}\left(\begin{matrix}
a_{11}\chi_{1x}^2 + 2a_{12}\chi_{1x}\chi_{1y} + a_{22}\chi_{1y}^2  & a_{11}\chi_{1x}\chi_{2x} + a_{12}(\chi_{1x}\chi_{2y} + \chi_{1y}\chi_{2x}) + a_{22}\chi_{1y}\chi_{2y} \\
a_{11}\chi_{1x}\chi_{2x} + a_{12}(\chi_{1x}\chi_{2y} + \chi_{1y}\chi_{2x}) + a_{22}\chi_{1y}\chi_{2y}  & a_{11}\chi_{2x}^2 + 2a_{12}\chi_{2x}\chi_{2y} + a_{22}\chi_{2y}^2
\end{matrix} \right)=
$$
$$
\overset{(116)}{=}\left(\begin{matrix}
0 & *\\
* & 0
\end{matrix}\right) \rightarrow \text{parte principale riconducibile alla forma canonica $u_{xy}$}
$$
$$
[\Rightarrow 2 * v_{\xi\eta} + \dots =0|: (2*) \Rightarrow \text{parte princ }v_{\xi\eta} ]
$$
\underline{Caso parabolico}: \\
prendi $\xi=x$, $\eta=\chi(x,y) \Rightarrow J=\left(\begin{matrix}
\xi_x & \xi_y \\
\eta_x & \eta_y
\end{matrix}\right)=\left( \begin{matrix}
1 & 0\\
\chi_{x} & \chi_y
\end{matrix}\right)$\\
N.B. detA=0 $\Rightarrow$ se uno dei coefficienti $a_{11}, a_{22}$ è sero lo è anche $a_{12}$ ($a_{11}a_{22}-a_{12}^2=0$). In tal caso siamo già nella forma canonica.\\
$\rightarrow $ supponiamo $a_{11}a_{22}>0$ ( non può essere $<0$, perchè $a_{12}^2 = a_{11}a_{22}$).\\
Equazione delle caratteristiche (116):
$$
a_{11}\chi_x^2+2\sqrt{a_{11}a_{22}}\chi_x\chi_y + a_{22}\chi_y^2 =0
$$
(se $a_{12}=-\sqrt{a_{11}a_{22}}$: manda y in -y).\\
Senza perdere la generalità:\\
$a_{11}>0$ (altrimenti moltiplica la (105) con -1) $\Rightarrow a_{22}>0$. $\rightarrow$ abbiamo $(\sqrt{a_{11}}\chi_x + \sqrt{a_{22}}\chi_y)^2=0$
\begin{equation}
\Rightarrow\chi_y = - \sqrt{\dfrac{a_{11}}{a_{22}}}\chi_x (118)
\end{equation}
$$
\tilde{A}=\left(\begin{matrix}
1 & 0\\
\chi_x & \chi_y
\end{matrix}\right)\left(\begin{matrix}
a_11 & \sqrt{a_{11}a_{22}}\\
\sqrt{a_{11}a_{22}} & a_22
\end{matrix}\right)\left(\begin{matrix}
1 & \chi_x \\
0 & \chi_y
\end{matrix}\right)
$$
%anche questa matrice sotto era con la croce in mezzo inn forma compatta
$$
=\left(\begin{matrix}
a_{11} & a_{11}\chi_x + \sqrt{a_{11}a_{22}}\chi_y \\
a_{11}\chi_x + \sqrt{a_{11}a_{22}}\chi_y & a_{11}\chi_x^2 + 2\sqrt{a_{11}a_{22}}\chi_x\chi_y + a_{22}\chi_y^2
\end{matrix}\right)\overset{(118)}{=}\left(\begin{matrix}
a_{11} & 0\\
0 & 0
\end{matrix}\right) \rightarrow \text{forma canonica}
$$
\underline{Caso ellittico}: il metodo seguito per le equazioni iperboliche e paraboliche non è applicabile (non disponiamo di curve caratteristiche) $\rightarrow$ torna alla (117) e imponi direttamente $\tilde{a}_{12}=0,\tilde{a}_{11}=\tilde{a}_{22}$
\begin{equation}
a_{11}\xi_x \eta_x + a_{12}(\xi_x\eta_y + \xi_y\eta_x) + a_{22}\xi_y\eta_y=0(119)
\end{equation}
\begin{equation}
a_{11}(\xi_x^2 - \eta_x^2) + 2a_{12}(\xi_x\xi_y - \eta_x\eta_y) + a_{22}(\xi_y^2 - \eta_y^2)=0
(119')
\end{equation}
$$
2i\cdot(119) + (119'):
$$
\begin{equation}
\implies a_{11}(\xi_x + i\eta_x)^2 + 2a_{12}(\xi_x + i\eta_x)(\xi_y + i\eta_y) + a_{22}(\xi_y + i\eta_y)^2 =0 (119'')
\end{equation}
Si ha $a_{11}a_{22}>0$ nel caso ellittico (altrimenti $\delta =a_{12}^2 - a_{11}a_{22}$ non può mai essere <0). $\rightarrow$ la (119'') è un'equazione di 2 grado non degenere per $\rho:= \dfrac{\xi_x + i\eta_x}{\xi_y + i\eta_y}$, con soluzioni $\rho_{\pm}=-(a_{12} + i\sqrt{|\delta|})/a_{11}$
Prendi per esempio $\rho_+$:
$$
\Rightarrow a_{11} (\xi_x + i\eta_x)=-(a_{12}+i\sqrt{|\delta|})(\xi_y + i \eta_y)
$$
e quindi 
\begin{equation}
\begin{matrix}
\xi_x=(a_{12}\eta_x + a_{22}\eta_y)/\sqrt{|\delta|} \\
\xi_y=-(a_{11}\eta_x + a_{12}\eta_y)/\sqrt{|\delta|} 
\end{matrix}(120)
\end{equation}
"equazioni di Beltrami", caratterizzano le trasformazioni che conducono la parte principale ad un'espressione proporzionale all'operatore di Laplace. $\grave{E}$ stato dimostrato (risultato non banale) che le (120) ammettono soluzioni nell'intera regione di ellitticità.\\
N.B. le (120) possono essere scritte nella forma\\
\begin{equation}
\dfrac{\partial w}{\partial \z}=\mu\dfrac{\partial w}{\partial z} (120')
\end{equation}
con $z=x+iy$, $w==\xi+i\eta$, $\mu=\dfrac{a_{11}+i a_{12 - \sqrt{|\delta|}}}{-a_{11}+i a_{12 - \sqrt{|\delta|}}}$ (esercizio)\\
Commento: un'applicazione $w(z,\z)$ che soddisfa la (120') viene chiamata mappa \underline{quasiconforme}\\
(Caso particolare $\mu=0$: mappa conforme, $w=w(z)$ funzione olomorfa).

\section{La questione della buona posizione del problema di Cauchy}
\underline{Definizione}: un prblema al contorno per un equazione alle derivate parziali di sice \underline{ben posto secondo Hadamard} se possiede una e una sola soluzione ed essa dipende in modo continuo dai dati.\\
Commenti:\\
i) L'enunciato corretto di un problema al contorno contiente non solo l'equazione differenziale e i dati al contorno, ma anche la precisazione dello spazio funzionale in cui si cerca la soluzione. Questa scelta è in grado di influenzare ad esempio l'unicità.\\
ii) Cosa si intende per dipendenza continua? $\rightarrow$ richiede una metrica nello spazio delle soluzione e una metrica nello spazio dei dati\\
X: spazio delle soluzioni, con norma $\| \cdot \|_X$\\
Y: spazio dei dati, con norma $\| \cdot \|_Y$\\
$\rightarrow$ dipendenza continua significa:\\
Sia $\{\bar{\delta}_n\}$ successione di dati (ordinabili in un vettore) t.c.  $\lim_{n\to\infty} \| \bar{\delta}_n - \bar{\delta}\|_Y=0$, e sia $\{u_n\}$ successione di soluzioni, u: soluzione corrispondente al dato limite $\bar{\delta}$. Si ha dipendenza continua se $\lim_{n\to \infty}\|u_n - u \|=0$.\\
-Le curve caratteristiche di un'equazione differenziale hanno un ruolo critico nel problema di Cauchy, perchè limitano la scelta delle curve portanti i dati.\\
$\rightarrow$ equazioni ellittica avvantaggiate dall'assenza di curve caratteristiche? E' così per la questione dell'esistenza e unicità, ma non per la dipendenza continua (idem per le equazioni paraboliche).\\
N.B. dipendenza continua dai dati è proprietà irrinunciavile per un modello matematico sensato. Infatti i dati sono di solito sperimentali e ciò non deve essere causa di un comportamento imprevedibile della soluzione.\\
Inoltre: mancanza della dipendenza continua impedisce il calcolo numerico della soluzione, a causa della ripercussione incontrollabile degli errori di arrotondamento.\\
-La non buona posizione del problema di Cauchy per equazioni ellittiche e paraboliche è messa in luce dai seguenti esempi:\\
i)Successione di problemi di Cauchy\\
$$ u_{xx}^{(n)} + u_{yy}^{(n)}=0, \quad u^{(n)}(0,y)=0, \quad u_x^{(n)}(0,y)=A_n\cos ny; n=1,2,\dots $$
$$
\bar{n}\cdot \bar{\nabla} = \left(\begin{matrix}
1 \\ 0
\end{matrix}\right) \cdot \left( \begin{matrix}
\partial_x \\ \partial_y
\end{matrix}\right)=\partial_x
$$
$\rightarrow$ unica soluzione: $u^{(n)}(x,y)=\dfrac{A_n}{2n}(e^{nx}-e^{-nx})\cos ny $
(si può trovare con un ansatz di separazione delle variabili, $u^{(n)}(x,y)=f(x)g(y)$ ($\rightarrow$ esercizio!)). Prendi per esempio $A_n=e^{-\sqrt{n}} \Rightarrow n\to \infty$ \\
-i dati di Cauchy tendono uniformemente a 0.\\
-per qualunque $x\neq 0$ le $u^{(n)}$ non solo non tendono a 0, ma non restano limitate.
ii) Successione 
$$
u_{xx}^{(n)} - u_t^{n}=0, \quad u^{(n)}(0,t)=A_n\sin nt, \quad u_x^{(n)}(0,t)=0;n=1,2,\dots
$$
$\rightarrow$ soluzione: 
$$
u^{(n)}(x,t) =\dfrac{1}{2}A_n\left(e^{\sqrt{\dfrac{n}{2}}x}\sin\left(nt+\sqrt{\dfrac{n}{2}x}\right) + e^{-\sqrt{\dfrac{n}{2}}x}\sin\left(nt-\sqrt{\dfrac{n}{2}x}\right)\right) 
$$
(esercizio!)\\
Prendi per esempio $A_n=n^{-k}, k>0 \rightarrow$ stessa situazione dell'esempio i)
















\chapter{Carica immagine}
problema: sfera con raggio R, messa a terra, con carica puntiforme q in $\bar{r}_0$\\
%immagine sfera a terra con cariche, un casino

Per motivi di simmetria, la carica immagine giace sulla retta che collega O e q.\\
Potenziale in P
\begin{equation}
\Phi = \dfrac{q}{a}+\dfrac{q'}{b} = 0
\end{equation}
prendi p=A:
\begin{equation}
\dfrac{q}{R-r_0} + \dfrac{q'}{d}=0
\end{equation}

$$
(2)\Rightarrow \dfrac{q'}{q}=-\dfrac{d}{R-r_0}
$$
prendo p=B:
\begin{equation}
\dfrac{q}{R+r_0}+\dfrac{q'}{d+2R}=0
\end{equation}
$$
(3)\Rightarrow \dfrac{q'}{q}=\dfrac{d+2R}{R+r_0}
$$
$$
\Rightarrow \dfrac{d}{R-r_0}= \dfrac{d+2R}{R+r_0}
$$
$$
\Rightarrow d(R+r_0)=(d+2R)(R-r_0)%manca roba?
$$
$$
\Rightarrow dR + 2dr_0 = dR - dr_0 - 2R^2 - 2Rr_0
$$
\begin{equation}
\Rightarrow d=\dfrac{R(R-r_0)}{r_0}
\end{equation}
e quindi
\begin{equation}
q' = \dfrac{dq}{Rr_0} = -\dfrac{q}{Rr_0}\dfrac{R(R-r_0)}{r_0}=-\dfrac{qR}{r_0}
\end{equation}
Verificare che la (1) vale $\forall p$
$$
(1)\Leftrightarrow \dfrac{q'}{q} = -\dfrac{b}{a}=(5)= \dfrac{R}{r_0} \Leftrightarrow \dfrac{b}{a}=\dfrac{R}{r_0}
$$
Calcolo $\alpha$ : $a^2=r_0^2 + R^2 = 2r_0 R\cos\alpha \implies \cos\alpha =\dfrac{r_0^2 + R^2 - a^2}{2r_0R}$\\
D'altra parte: 
$$
\cos\alpha=\dfrac{r_0+l}{R} \implies \dfrac{r_0+l}{R}=\dfrac{r_0^2 +R^2 - a^2}{2r_0 R} \implies l=\dfrac{-r_0^2 + R^2 - a^2}{2r_0}
$$
$$
c^2=R^2 - (r_0+l)^2 = R^2 - \left(\dfrac{r_0^2 + R^2 - a^2}{2r_0} \right)^2=b^2 - (d +R-r_0 -l)^2 = b^2 -(d+R)^2 - (r_o+l)^2 + 2(d+R)(r_0+l)
$$
$$
\Rightarrow R^2 = b^2 - (d+R)^2 + 2\underbrace{(d+R)}\underbrace{(r_0+l)}
$$
$$
\dfrac{R^2}{r_0^2} \qquad \qquad \dfrac{r_0^2 + R^2 - a^2}{2r_0R}
$$
$$
\Rightarrow R^2 = b^2 - \dfrac{R^4}{r_0^2} + \dfrac{R^2}{r_0^2}(r_0^2 + R^2 - a^2) \implies \dfrac{R}{r_0^2}a^2 = b^2 \implies \dfrac{b}{a}=\dfrac{R}{r_0} \qquad \checkmark
$$

\end{document}









