\documentclass[a4paper,11pt]{report}

% Geometria del foglio
\usepackage[a4paper, top=2.5cm, bottom=2cm, left=1.5cm, right=1.5cm]{geometry}

% Lingua documento
\usepackage[italian]{babel}
\usepackage[utf8]{inputenc}

\usepackage{lipsum}
\usepackage{datetime}

% Frontespizio, Head e footer delle pagine
\usepackage[]{fancyhdr}
%\pagestyle{fancy}
\fancyhead{} % cancella tutti i campi

\fancyfoot[C]{\thepage}


% Includere immagini 
\usepackage{graphicx}
\usepackage{adjustbox}
\usepackage{floatflt}
\usepackage{float}
\usepackage{wrapfig}
\usepackage{caption}
\usepackage{subcaption}
\usepackage[]{xcolor}

% Pacchetti per la matematica
\usepackage{amsmath}
\usepackage{amsfonts}
\usepackage{amssymb}
\usepackage{gensymb}

\usepackage{multicol}
\usepackage{framed}

\title{Appunti del corso di Equazioni Differenziali I}
\author{raccolti da \\Luca Colombo Gomez}
\date{AA 2017/2018}

\newcommand{\R}{\mathbb{R}}
\newcommand{\Rn}{\mathbb{R}^n}
\newcommand{\fourier}{\emph{F}}
\newcommand{\x}{\bar{x}}
\newcommand{\xp}{\bar{x}'}
\newcommand{\y}{\bar{y}}
\newcommand{\yp}{\bar{y}'}
\newcommand{\kk}{\bar{k}}
\newcommand{\kp}{\bar{k}'}
\newcommand{\z}{\bar{z}}
\newcommand{\n}{\bar{n}}
\newcommand{\Backlund}{B$\ddot{\text{a}}$cklund }

\makeindex

\begin{document}
\titlepage
\maketitle

\tableofcontents
\chapter{Distribuzioni e trasformate di Fourier}
\section{Distribuzioni}
le distribuzioni ( funzioni generalizzate) sono degli oggetti che generalizzano le funzione e le distribuzioni di probabilità. 
Estendono il concetto di derivata a tutte le funzioni continue e oltre.
le distribuzioni sono importanti in fisica (p.e. distribuzione delta di Dirac).

\subsection{Idea di base}
$f:\R\rightarrow\R $ funzione integrabile
$ \phi : \R \rightarrow \R $ smooth $\left(C^{\infty}\right)$, con supporto compatto.
$\rightarrow \int f \phi \mathrm{dx} \in \mathbb{R}$, dipende linearmente e in un modo continuo da $\phi$.
$\Rightarrow$ f $\grave{e}$ un funzionale lineare continuo sullo spazio di tutte le "funzioni test"$\phi.$ Questa è la definizione di una distribuzione.\\
Le distribuzioni possono essere moltiplicate con dei numeri reali, e possono essere sommate $\rightarrow$ formano uno spazio vettoriale reale.
\subsection{Derivata di una distribuione}
Considera prima il caso di una funzione $f:\R \rightarrow \R$ differenziabile. Se $\phi$ è una funzione test, abbiamo:

$$\int_{\R}f'\phi dx = -\int_{\R}f\phi ' dx$$

non c'è un termine di bordo perchè $\phi $ ha supporto compatto. $\rightarrow $ suggerisce la seguente definizione della derivata S' di una distribuzione 
S : S' = funzionale lineare che manda la funzione test $\phi$ in $-S\left(\phi\right)$

\subsection{Delta di Dirac}
("Funzione delta di Dirac") $\delta(x)$ è la distribuzione che manda la funzione test $\phi$ in $\phi(0)$. È la derivata della funzione step di Heaviside.
$$
H\left(x\right)=\left\{ \begin{matrix}
0 & x < 0 \\
1 & x \geq 0
\end{matrix}\right. $$
La derivata della delta di Dirac è la distribuzione che manda $\phi$ in $-\phi '(0)$. la delta è un esempio di una distribuzione che non è una funzione, ma può essere definita come limite di una seuenza di funzioni, p.e.
$$ \delta\left(x\right) = \lim_{a\to 0}\delta_a\left(x\right) $$
$$ \delta_a\left(x\right) = \left\{\begin{matrix}
\dfrac{1}{2a} && -a\leq x\leq a\\
0 && |x| > a
\end{matrix}\right.$$

Dimostrazione

$$
\int_R \delta_a (x)\Phi(x) dx= \int_{-a}^{a}\dfrac{1}{2a}\Phi(x) dx = \dfrac{1}{2a}\left(\psi(a)-\psi(-a)\right)
\quad \psi= \int \Phi,\quad \psi '=\Phi \qquad\blacksquare$$
$$
\Rightarrow \lim_{a\to 0}\int_{R}\delta_a(x)\Phi(x)dx=lim_{a\to 0}\dfrac{\psi(a)-\psi(-a)}{2a}=\psi '(0)
$$

\subsection{Definizione formale}
Def: una funzione $\Phi : U \rightarrow \mathbb{R} $ ha \underline{supporto compatto} se esiste un sottoinsieme compatto K di U tale che $\Phi(x)=0 \forall x \in U\backslash K $\\
Le funzioni $ C^{\infty} \Phi :U\rightarrow \mathbb{R}$ con supporto compatto formano uno spazio vettoriale topologico $\mathbf{D}(U)$\\

\underline{Def:} lo spazio delle \underline{distribuizioni} su $U \subseteq \mathbb{R}^n$ è il \underline{duale} $D\sp\prime(U)$ dello spazio vettoriale topologico D(U) di funzioni $C^{\infty}$ con supporto compatto in U.\\

Notazione: $$S \in D\sp\prime(U), \quad
\phi \in D(U), \quad
S : D(U)\rightarrow \mathbb{R}, \quad 
	\Phi \mapsto S(\Phi) = <S|\Phi>
$$

Una funzione integrabile f definisce una distribuzione $\tilde{f}$ su $\mathbb{R}^n$ tramite
\begin{equation}
<\tilde{f},\Phi> := \int_{\Rn} f \Phi d^{n}x
\forall \Phi \in D(U) 
\end{equation}


Si dice che $\tilde{f}$ è la distribuzione associata alla funzione f, o che la distribuzione $\tilde{f}$ è equivalente alla funzione f.\\
La distribuzione di Dirac (o misura di Dirac) è definita da 
\begin{equation}
<\delta,\Phi> := \Phi(0) 
\end{equation}

\underline{Teorema:} la distribuzione di Dirac non può essere rappresentata da una funzione integrabile (senza dim). Nonostante ciò scriveremo in seguito formalmente
\begin{equation}
<\delta , \Phi> = \int_{\Rn} \delta (x) \phi(x) d^n x= \Phi(0) 
\end{equation}

\underline{Es } (n=1)\\
Si dimostri che 
\begin{equation}
\delta' (x) =-\dfrac{\delta(x)}{x}
\end{equation}
Si ha $$\int x\delta(x)\Phi(x)dx = \left.x\Phi(x)\right\|_{x=0}=0 \forall \Phi \Rightarrow x\delta(x)=0 $$
$$\Rightarrow \delta(x) + x\delta ' (x) \Rightarrow \delta ' (x) = -\dfrac{\delta(x)}{x}  \qquad \blacksquare$$

Un'altra identità utile è
\begin{equation}
\delta(g(x)) = \sum_{i}\dfrac{\delta(x-x_i)}{|g'(x_i)|} 
\end{equation} 
dove $x_i$ sono gli zeri della fuznione g(x).\\
Rappresentazione della delta: $\delta(x) = \lim_{a\to 0} \delta_a(x)$, con 
\begin{subequations}
\begin{equation}
\delta_a(x)=\left\{\begin{matrix}
\dfrac{1}{2a} & -a \leq x \leq a \\
0 & |x| > a
\end{matrix}\right.
\end{equation}
\begin{equation}
\delta_a(x) = \dfrac{1}{\pi}\dfrac{a}{a^2+x^2}
\end{equation}
\begin{equation}
\delta_a(x) = \dfrac{1}{a\sqrt[]{\pi}}\exp{-\dfrac{x^2}{a^2}}
\end{equation}
\begin{equation}
\delta_a(x) = \dfrac{1}{2\pi} \int_{-\infty}^{\infty}\exp^{ikx-a|k|}dk
\end{equation}
\end{subequations}

Dimostrazione della (1.6b):

$$\int_{-\infty}^{\infty} \delta_a(x)\Phi(x)dx=
\dfrac{1}{\pi}\int_{-\infty}^{\infty}\dfrac{a}{a^2+x^2}\Phi(x)dx = (x=at) = 
\dfrac{1}{\pi}\int_{-\infty}^{\infty}\dfrac{1}{1+t^2}\Phi(at)dt$$
$$lim_{a\to 0} \int_{-\infty}^{\infty}\delta_a(x)\Phi(x)dx=\dfrac{1}{\pi}\lim_{a\to 0}\int_{-\infty}^{\infty}\Phi(at)\dfrac{dt}{1+t^2}$$

usa il \underline{teorema della convergenza dominata}:
se ${f_k}_{k\in\mathbb{N}} \grave{e}$ una successione di funzioni misurabili con limite puntuale f, e se esiste una funzione integrabile g tale che $|f_k|\leq g \forall k$ allora f $\grave{e}$ integrabile e $\lim_{k\to \infty}\int f_k dx = \int f dx$\\
Da noi $f_k(t)\hat{=}\dfrac{\Phi(at)}{1+t^2}\rightarrow$ funzione g esiste, perchè $\Phi$ ha supporto compatto.\\
Quindi : $$
\lim_{a\to 0}\int_{-\infty}{\infty}\delta_a(x)\Phi(x)dx=\dfrac{1}{\pi}\int_{-\infty}^{\infty} \lim_{a\to 0} \Phi(at) \dfrac{dt}{1+t^2}=\dfrac{1}{\pi} \Phi(0) arctan(t)|_{-\infty}^{\infty}=\Phi(0) q.e.d.$$

%(1.6.c) e (1.6.d) a casa
In N dimensioni: coordinate cartesiane $x_1,x_2,\dots,x_n \Rightarrow \delta(x)= \delta(x_1)\dots \delta(x_n)$

\subsection{Proprietà della delta di Dirac}
$$\delta(-\bar{r})=\delta(\hat{r})$$
\begin{equation}
\int_{\Re^n} d^n\hat{r}\delta(\hat{r}-\hat{r'})f(\hat{r}) = \int_{\Re^n}d^n\hat{\rho}\delta(\hat{\rho})f(\hat{\rho}+\hat{r'})=\left.f(\hat{\rho}+\hat{r'}\right|_{\hat{\rho=0}}=f(\hat{r'})
\end{equation}

Scegli f=1 $\Rightarrow$ formalmente 
\begin{equation}
\int_{\Re^n}d^n\hat{r}\delta(\hat{r}-\hat{r'})=1
\end{equation}


$\delta$ in coordinate curvilinee?\\
la quantità invariante per trasformazione di coordinate è $d^n\hat{r}\delta(\hat{r}-\hat{r'})$
Coord $\alpha_i (x_i,\dots,x_n),i=1,\dots,n$
$x_j$ sono le coordinate cartesiane

Jacobiano
$$
J(x_i,\xi_j)=\left(\begin{matrix}
\dfrac{\partial x_1}{\partial \xi_1} & \dots & \dfrac{\partial x_1}{\partial \xi_n}\\
\vdots & & \vdots \\
\dfrac{\partial x_n}{\partial \xi_1} & \dots &\dfrac{\partial x_n}{\partial \xi_n}
\end{matrix}\right)dx_1\dots dx_n \delta(x_1-x_1')\dots \delta(x_n - x_n')
$$
$$
|J|d\xi_1\dots d\xi_n \delta(x_1 - x_1')\dots \delta(x_n-x_n')=d\xi_1 \dots d\xi_n\delta(\xi_1\dots \xi_1')\dots \delta(\xi_n - \xi_n')
$$
\begin{equation}
\delta(\xi_1-\xi_1')\dots\delta(\xi_n-\xi_n')=|J|\delta(x_1-x_1')\dots\delta(x_n-x_n')
\end{equation}
Esempio coordinate sferiche 3-dim
$$
x=r\cos\varphi\sin\theta \qquad y=r\sin\varphi\sin\theta \qquad z=r\cos\theta \quad (\xi_1=r,\xi_2=\theta,\xi_3=\varphi)
$$
$$\left|\dfrac{\partial x_i}{\partial \xi_j}\right|=\left|\begin{matrix}
\dfrac{\partial x}{\partial r} & \dfrac{\partial x}{\partial \theta} & \dfrac{\partial x}{\partial \varphi} \\
\dfrac{\partial y}{\partial r} & \dfrac{\partial y}{\partial \theta} & \dfrac{\partial y}{\partial \varphi}\\
\dfrac{\partial z}{\partial r} & \dfrac{\partial z}{\partial \theta} & \dfrac{\partial z}{\partial \varphi}
\end{matrix}\right|=r^2\sin\theta
$$
\begin{equation}
\Rightarrow \delta(r-r')\delta(\theta - \theta')\delta(\varphi - \varphi') = r^2\sin\theta \delta(x-x')\delta(y-y')\delta(z-z')
\end{equation}
Esercizio: calcolare J in coordinate cilindriche

Densità di carica di un insieme discreto di N cariche puntiformi
\begin{equation}
\rho(\hat{r}) = \sum_{i=1}^{N} q_i\delta(\hat{r}-\hat{r_i})
\end{equation}
%r_i posizione carica i-esima 

carica Q uniformemente distribuita su una superficie sferica con raggio R. $\rho(\underline{r})=?$\\
Chiamo $\rho(\underline{r})=A Q\delta(r-R)$ con A da determinare.
$$\int d^3\underline{r}\rho(\underline{r})= 4\pi \int r^2 sin\theta dr d\theta d\phi A Q \delta(r-R) = 4\pi AQ \int r^2 dr \delta(r-R)=4\pi AQR^2 $$
Normalizzando $4\pi AQR^2 =Q$ ottengo $A=\dfrac{1}{4\pi R^2}$
$$ \rho(\underline{r})=\dfrac{Q}{4\pi R^2}\delta(r-R)$$

\section{Trasformate di Fourier}
\underline{Definizione:} spazi $L^P$ \\
Sia $\chi$ uno spazio di misura con misura m positiva. (Possiamo prendere la misura di Lebesgue come esempio)\\
$L^P(\chi):=$ spazio di funzioni su $\chi$ tale che $|f|^p$ sia integrabile, e $\int_{\chi}|f|^p dm < \infty$\\
Si dimostra che per p$\geq$1, $L^P(\chi)$ è uno spazio vettoriale e $\left\|f\right\| := \left\{ \int_{\chi} |f|^p dm\right\}^{\frac{1}{p}}$ è una norma su questo spazio.\\
A noi interessa il caso in cui m è la \underline{misura di Lebesgue}; in tal caso gli elementi di $L^P(\chi)$ sono le funzioni f con $\int_{\chi}|f|^p dx < \infty$.
Il caso p=2 trova applicazioni in meccanica quantistica.\\
% aggiunta seconda lezione da appunti Serena
\underline{Definizione:} Sia f una funzione $f\in L^1 (\Rn)$, la \underline{trasformata di Fourier} \fourier f è una funzione in $\Rn$ (in realtà sul duale di $\R^n$, ma coincide con $\R^n$) definita da 
\begin{equation}
\left(\left(\fourier \right)f\right)(\kk) = \int_{\Rn}e^{-\bar{k}\bar{x}}f(\bar{x})d^n\bar{x} 
\end{equation}
Si osserva che se $f\in L'(\Rn) \Rightarrow \exists (\fourier f)(\bar{k})$. In seguito verrà usata la notazione $\hat{f}(\bar{k})=(\fourier f)(\bar{k})$\\
Una possibile rappresentazione della $\delta$ di Dirac è (per n=1)
$$
	\delta(x)=\lim_{a \to 0} \dfrac{1}{2\pi} \int_{-\infty}^{+\infty} e^{ikx-a\left | k\right |}dk
$$
che è una Trasformata di Fourier. Formalmente si può scrivere
\begin{equation}
\delta(\x)=\left(\dfrac{1}{2\pi}\right)^n\int_{\Rn}e^{i\kk\x}d^n\kk
\end{equation}

Se conosco la trasformata di Fourier $\hat{f}$ posso determinare la funzione f applicando la \underline{antitrasformata} di Fourier:
\begin{multline*}
\left(\dfrac{1}{2\pi}\right)^n\int_{\Rn}d^n\bar{k}e^{i\bar{k}\bar{x}}\hat{f}(\bar{k})=\\
=\left(\dfrac{1}{2\pi}\right)^n \int_{\Rn}d^n\bar{k}e^{i\bar{k}\bar{x}}\int_{\Rn}e^{-i\bar{k}\bar{x'}}f(\bar{x'})d^n\bar{x'}= \left(\dfrac{1}{2\pi}\right)^n\int_{\Rn}d^n\bar{x'}f(\bar{x'})\int_{\Rn}d^n\bar{k}e^{i\bar{k}\left(\bar{x}-\bar{x'}\right)}=\\
=\left(\dfrac{1}{2\pi}\right)^n\int_{\Rn}d^n\bar{x'}f(\bar{x'})\delta(\bar{x}-\bar{x'})
\end{multline*}

\begin{equation}
\Rightarrow f(\bar{x})=\left(\dfrac{1}{2\pi}\right)^n\int_{\Rn}d^n\bar{k}e^{i\bar{k}\bar{x}}\hat{f}(\bar{k})
\end{equation}

paragonando quest'ultima espressione con (1.13) si trova che
\begin{equation}
(\fourier\delta)(\kk)=1
\end{equation}
Definiamo ora una famiglia di funzioni
\begin{equation}
\varphi_{\kk}(\x)=\left(\dfrac{1}{2\pi}\right)^{\dfrac{1}{2}}e^{i\kk\x}
\end{equation}

Allora possiamo riscrivere la funzione $f(\x)$ come:
$$
f(\x)=\left(\dfrac{1}{2\pi}\right)^{\dfrac{n}{2}}\int d^n\kk\hat{f}(\kk\varphi_{\kk}(\x))
$$

$\Rightarrow$ se dimostro che $\varphi_{\bar{k}}(\bar{x})$ è \underline{ortonormale} e \underline{completo} allora posso sviluppare qualsiasi funzione su $\varphi_{\kk}(\x)$ che chiamo \underline{funzioni di base}.\\
\underline{Dimostrazione completezza:}
\begin{equation}
\int d^n\kk \varphi_{\kk}(\x)\varphi_{\kk}^{*}(\xp)=\left(\dfrac{1}{2\pi}\right)^n\int d^n\kk e^{i\kk(\x-\xp)}=\delta(\x-\xp)
\end{equation}

Che equivale alla condizione di completezza, infatti:
\begin{multline*}
f(\x)=\int d^n\xp f(\xp)\delta(\x-\xp)=\\
=\int d^n\bar{x'}f(\bar{x'})\int d^n\kk\varphi_{\kk}(\x)\varphi_{\kk}^{*}(\xp)=\int d^n\kk\varphi_{\kk}(\x)\int d^n\xp f(\xp)\varphi_{\kk}^{*}(\xp)=\\
=\int d^n\kk\varphi_{\kk}(\x)\left(\dfrac{1}{2\pi}\right)^{\dfrac{n}{2}}\hat{f}(\x)
\end{multline*}


$$
\Rightarrow f(\bar{x})=\left(\dfrac{1}{2\pi}\right)^{\dfrac{n}{2}}\int d^n\bar{k}\varphi_{\bar{k}}(\bar{x})\hat{f}(\bar{x})
$$

quindi qualunque f è sviluppabile in una base di $\varphi_{\bar{k}}$\\
\underline{Dimostrazione ortogonalita:}\\
\begin{equation}
\int d^n\x\varphi_{k}(\x)\varphi_{\kp}^{*}(\xp)=\left(\dfrac{1}{2\pi}\right)^{n}\int d^n e^{i(\kk-\kp)\x}=\delta (\x-\xp)
\end{equation}
\underline{Osservazione:} $\{\varphi_{\bar{k}}(\bar{x})\}$ formano una \underline{base} di $\Rn$, non di tutti i sottoinsiemi di $\Rn$. Sulla superficie sferica, le armoniche sferiche sono le combinazioni lineari di $\{\varphi_{\kk}(\x)\}$

\section{Funzioni di Green}
\underline{Definizione:} un \underline{nucleo} (detto anche \underline{Kernel}) su $\Rn$ è una distribuzione su $\Rn \wedge \Rn$, ossia un elemento del duale $D'(\Rn \wedge \Rn)$ di funzioni $C^{\infty}$ con supporto compatto in $\Rn \wedge \Rn$.
$$
K:D(\Rn \wedge \Rn)\rightarrow\Rn \qquad K\in D'(\Rn \wedge \Rn)
$$
\underline{Definizione:} un \underline{nucleo fondamentale} (o elementare) E di un operatore differenziale lineare D su $\Rn$ con coefficienti $a_j(\bar{x})\in C^{\infty}$;
$$
D=\sum_{|j|\leq m}a_j(\bar{x})D^j
$$
è un nucleo, che soddisfa
\begin{equation}
D=\sum_{|j|\leq m}a_j(\bar{x})D^jE(\bar{x},\bar{y})=\delta(\bar{x}-\bar{y}) (1.19)
\end{equation}
osservazione sulla notazione: j è un indice multiplo
$$
j=(j_1,\dots ,j_n) \qquad |j|=\sum_{i=1}^n j_i
$$
con m:= ordine dell'operatore differenziale $D^j$
$$
D^j=\left(\dfrac{\partial}{\partial x_i}\right)^{j_i}\dots \left(\dfrac{\partial}{\partial x_n}\right)^{j_n}
$$
\underline{Esempio} m=3,n=2 $\Rightarrow j=(j_1,j_2) \quad \bar{x}=(\bar{x_1},\bar{x_2})$
$$
D=a_{21}(\bar{x})\left(\dfrac{\partial}{\partial x_1}\right)^2\dfrac{\partial}{\partial x_2}+a_{20}(\bar{x})\left(\dfrac{\partial}{\partial x_1}\right)^2
$$
\emph{è un esempio, non è l'unico operatore possibile}\\
Dalla definizione abbiamo che 
$$
D E(\x-\y)=\delta(\x  - \y)=\prod_{i=1}^{n}\delta(x_i-y_i)
$$
Dato il nucleo fondamentale E, una soluzione dell'equazione differenziale $Dx=B$ può essere trovata tramite
\begin{equation}
X(\x)=\int_{\Rn}E(\x,\y)B(\y)d^ny (1.20)
\end{equation}

\underline{dimostrazione:}\\
$$
DX(\x)=\int_{\Rn}DE(\x,\y)B(\y)d^n(\y)=\int_{\Rn}\delta(\x-\y)B(\y)d^ny
$$
$$
\Rightarrow DX(\x)=B(\x) \blacksquare
$$
\underline{Definizione:} una funzione di green è un nucleo elementare per l'operatore differenziale
\begin{equation}
-\dfrac{\nabla^2}{4\pi}\Rightarrow-\nabla^2G(\x,\y)=-4\pi\delta(\x-\y) (1.21)
\end{equation}

Si può dimostrare che $G(\x,\y)=G(\y,\x)$.\\
\underline{Esempio:} si consideri una carica puntiforme in $\y$ con carica $q=1$. $\rho=q\delta(\x-\y)$ è la densità di carica. Dalle equazioni di Maxwell si ha $\nabla^2\phi(\x)=-4\pi\rho(\x)=-4\pi\delta(\x-\y)\rightarrow$ definizione di Funzione di Green.\\
Una possibile Funzione di Green è $\phi(\x)=\dfrac{1}{|\x-\y|}$\\
Se è nota una funzione di Green $\Rightarrow$ una soluzione dell'equazione di Poisson è:
$$
\phi(|x)=\int G(\x,\y)\rho(\y)d^3y (1.23)
$$
Quindi passo da un'equazione differenziale ad un integrale.\\
Nel nostro caso $\phi(\x)=\int \rho(\y)d^3\y$, come in elettromagnetismo, in generale

\begin{subequations}
\begin{equation}
G(\x,\y)=\dfrac{1}{|\x-\y|}+F(\x,\y) (1.24a)
\end{equation}
\begin{equation}
\nabla^2F(\x,\y)=0 (1.24b)
\end{equation}
\end{subequations}
$\Rightarrow$ F dipende dalle condizioni di bordo.
\underline{Teorema:} In assenza di superficio di bordo la funzione di Green $G(\x,\y)$ dipende solo dalla differenza $\x-\y$
$$
G(\x,\y)=G(\x-\y)\rightarrow\mathrm{Invarianza traslazionale} (1.25)
$$
Nota: non dimostrato ma intuibile, $\delta(\x-\y)$ invariante se non ho condizioni di bordo; $\nabla^2$ invariante; $G(\x-\y)$ invariante per traslazioni $\x\rightarrow\x-\bar{a}$, $\y\rightarrow-\bar{a}$\\
Se $G(\x,\y)=G(\x-\y)$ è più facile trovarla. Per n=3
$$
\nabla^2G(\x-\y)=-4\pi\delta(\x-\y)
$$
$$
\dfrac{\nabla^2}{(2\pi)^3}\int d^3k e^{i\kk(\x-\y)}\tilde{G}(\kk)=-\dfrac{4\pi}{(2\pi)^3}\int d^3ke^{i\kk(\x-\y)}
$$
Quindi il primo passaggio è scrivere G e $\delta$ come trasformate di Fourier
$$
\int d^3ke^{i\kk(\x-\y)}(-k^2\tilde{G}(\kk)+4\pi)=0
$$
il secondo passaggio è osservare che $\x$ compare solo come esponente ($\Rightarrow -k^2$) e porto tutto da una parte.\\
$e^{i\kk(\x-\y)}=\psi_k(\x)$ è linearmente indipendete, quindi deve annullarsi il coefficiente
$$
-k^2\tilde{G}(\kk)+4\pi =0 \Rightarrow \tilde{G}(\kk)=\dfrac{4\pi}{\kk^2}
$$
Applico l'antitrasformata di Fourier
$$
G(\x)=\dfrac{1}{2\pi^3}\int d^3\kk e^{i\kk\x}\dfrac{4\pi}{\kk^2}
$$
Cambiamento di coordinate: coordinate sferiche
%inserisci immagine con x, k e theta
$$
d^3\kk=k^2\sin\theta dkd\theta d\varphi \Rightarrow G(\x)=\dfrac{1}{\pi}\int\sin\theta d\theta dk \dfrac{e^{ik|\x|\cos\theta}}{k^2}
$$
cambio variabile: $u=\cos\theta \rightarrow du=-\sin\theta d\theta$
$$
G(\x)=-\dfrac{1}{\pi}\int dudke^{ik|\x|u}=\int_0^{\infty}\dfrac{2}{k|\x|\pi}\sin(k|\x|)dk= \dfrac{1}{|\x|}
$$

\chapter{Equazione del Calore}
La legge di Fourier della conduzione termica è data da 
\begin{equation}
\bar{q} =-k\bar{\nabla}T
\end{equation}
dove $\bar{q}:=$ densità di flusso termico; $k:=$ conducibilità termica; $T:=$ temperatura.\\
La temperatura può essere riscritta come $ T=\dfrac{\phi}{C_p \rho}$, dove $\rho$ è la denstià, $C_p$ è il calore specifico a pressione costante, e $\phi$ è il calore per unità di volume.\\
L'equazione di continuità è:
\begin{equation}
\dfrac{\partial \phi}{\partial t} + \bar{\nabla}\bar{q}=0
\end{equation}
che ha la forma tipica di una legge di conservazione.\\
In questa formula sostituiscto $\phi$ e $\bar{q}$ e trovo
$$
\dfrac{\partial}{\partial t}(\rho C_p T) + \bar{nabla}\cdot(-k\bar{\nabla T}) = \rho C_p \dfrac{\partial}{\partial t}T - k \nabla^2 T=0
$$
Definendo $\chi=\dfrac{k}{\rho C_p}:=$ coefficiente di conducibilità termica, si ottiene l'\underline{equazione del calore}
\begin{equation}
\dfrac{\partial T}{\partial t}=\chi \nabla^2 T
\end{equation}

%inizio lezione 11/10

paragona con l'equazione di diffusione 
\begin{equation}
\dfrac{\partial u}{\partial t}=D\cdot \Delta u
\end{equation}

D: Coefficiente di diffusione\\
u: densità del materiale che si diffonde\\

%ha riscritto 2.3 e 2.4 come 3 e 4
la (2.4) segue dalla 1 legge di Fick sulla corrente di diffusione
\begin{equation}
\overrightarrow{q}_{D}=-D\overrightarrow{\nabla}u
\end{equation}

più l'equazione di continuità (il materiale non viene creato o distrutto)
$$
\dfrac{\partial u}{\partial t}+\overrightarrow{\nabla}\cdot\overrightarrow{q_D}=0
$$
La (2.4) viene anche chiamata 2 legge di Fick.\\
N.B. anche l'equazione di Black-Scholes per il prezzo di un'opzione può essere riportato nella forma (2.3),(2.4).\\
-Risolviamo la (2.3) in d dimensioni:
\begin{equation}
\Delta=\dfrac{\partial^2}{\partial x_{1}^2}+\cdots+\dfrac{\partial^2}{\partial x_{d}^2} 
\end{equation}

Faccio una trasformata di Fourier:
\begin{equation}
T(\bar{x},t)=\dfrac{1}{(2\pi)^d}\int d^d\bar{k}e^{i\bar{k}\bar{x}}\tilde{T}(\bar{k},t) 
\end{equation}

$$(3)\Rightarrow\dfrac{1}{(2\pi)^d}\int d^d \bar{k} e^{i\bar{k}\bar{x}}\left(\dfrac{\partial}{\partial t}\tilde{T}(\bar{k},t) + \chi \bar{k}^2\tilde{T}(\bar{k},t)\right)=0 
$$
dato che gli esponenziali sono lnearmente indipendenti, devo annullare i coefficienti
\begin{equation}
\dfrac{\partial}{\partial t}\tilde{T}+\chi \bar{k}^2\tilde{T}=0 \Rightarrow\tilde{T}(\bar{k},t)=e^{-\chi\bar{k}t }\tilde{T}_0(\bar{k}) 
\end{equation}

faccio una trasformata di fourier inversa
\begin{equation}
\int d^d \bar{y}e^{-i\bar{k}\bar{y}}T_0(\bar{y}) 
\end{equation}

sostituisco (8),(9) nella (7):
$$
T(\bar{x},t)=\dfrac{1}{(2\pi)^d}\int d^d \bar{y}T_0(\bar{y})\int d^d\bar{k}e^{i\bar{k}(\bar{x}-\bar{y})-\chi\bar{k}t}
$$
definisco
\begin{equation}
\int d^d\bar{k}e^{i\bar{k}(\bar{x}-\bar{y})-\chi\bar{k}t }=:(2\pi)^d G(\bar{x}-\bar{y},t)
\end{equation}

G: propagatore/nucleo di calore ("heat kernel"). Propaga le condizioni iniziali di $T_0(\bar{y})$. Quindi abbiamo la convoluzione:
\begin{equation}
T(\bar{x},t)=\int d^d\bar{y}G(\bar{x}-\bar{y},t)T_0(\bar{y}) (11)
\end{equation}

Caso particolare: $T_0(\bar{y})=\delta(\bar{y})$
$$
\Rightarrow T(\bar{x},t)=G(\bar{x},t)
$$
il propagatore è soluzione dell'equazione del calore corrispondente a un dato iniziale deltiforme. Per questo motivo, il propagatore è anche chiamato soluzione fondamentale, perchè esso è una soluzione e con esso si costruiscono tutte le altre per convoluzione.\\
Calcoliamo G:
$$
G(\bar{z},t)=\dfrac{1}{(2\pi)^d}\int d^d \bar{k}e^{i\bar{k}\bar{z}-\chi \bar{k}t}
$$
passo agli esponenti, usando il teorema dei residui sposto l'asse reale nel piano complesso in alto o in basso
$$
-\chi t\left(\bar{k}-\dfrac{i\bar{z}}{2\chi t}\right)^2-\dfrac{\bar{z}^2}{4\chi t}
$$
$$
\left(\bar{k}-\dfrac{i\bar{z}}{2\chi t}\right)^2=:\bar{k'}
$$
$$
\Rightarrow G(\bar{z},t)=\dfrac{1}{(2\pi)^d}\exp\left(\dfrac{-\bar{z}^2}{4\chi t}\right)\int d^d \bar{k} e^{-\chi t \bar{k'}^2}
$$
l'integrale è Gaussiano (più precisamente prodotto di integrali Gaussiani)
$$
G(\bar{z},t)=\dfrac{1}{(2\pi)^d}\exp\left(-\dfrac{\bar{z}^2}{4\chi t}\right)\sqrt{\dfrac{\pi}{\chi t}}^d
$$
Se $Re(\chi t)>0\Rightarrow t>0$, la soluzione esiste solo per $t>0$, cioé per tempi posteriori all'essegnazione del dato iniziale (soluzione ritardata)
\begin{equation}
\Rightarrow G(\bar{z},t)=\dfrac{1}{(4\pi\chi t)^{\dfrac{d}{2}}}\exp\left(-\dfrac{\bar{z}^2}{4\chi t}\right) (12)
\end{equation}

-Soluzione della (11) per d=1 per dato iniziale localizzato:
$$
T_0(x)=\left\{\begin{matrix}
 \hat{T}_0 & |x|\leq L \\ 
 0 & |x| > L
\end{matrix}\right.
 % disegno della barriera
$$
%sopra la freccia c'è scritto (12)
$$
(11)\Rightarrow T(\bar{x},t)=\int_{L}^{L}dy\hat{T}_0 \dfrac{1}{(4\pi\chi t)^{1/2}}exp\left(-\dfrac{(x-y)^2}{4\chi t}\right)
$$
definisco $z:=x-y$ 
$$
=\dfrac{\hat{T}_0}{(4\pi\chi t)^{1/2}}\int_{x-L}^{x+L}dz exp\left(-\dfrac{z^2}{4\chi t}\right)
$$
definisco $r:=\dfrac{z}{(4\chi t)^{1/2}}$
$$
=\dfrac{\hat{T}_0}{\pi}\int_{\dfrac{x-L}{2\sqrt{\chi t}}}^{\dfrac{x+L}{2\sqrt{\chi t}}} e^{-r^2}dr=\dfrac{\hat{T}_0}{\sqrt{\pi}}\left(\int_{\dfrac{x-L}{2\sqrt{\chi t}}}^{0}e^{-r^2}dr + \int_0^{\dfrac{x+L}{2\sqrt{\chi t}}}\right)
$$
$$
=\dfrac{\hat{T}_0}{\sqrt{\pi}}\left(-\dfrac{\sqrt{pi}}{2}\right)
$$
\begin{equation}
=\dfrac{\hat{T}_0}{2}\left(erf\dfrac{x+L}{2\sqrt{\chi t}}-erf\dfrac{x-L}{2\sqrt{\chi t}}\right) (13)
\end{equation}

dove la funzione degli errori (di Gauss) e definita da 
\begin{equation}
erf(s):=\dfrac{2}{\sqrt{\pi}}\int_0^s e^{-r^2}dr (14)
\end{equation}

dato che la soluzione e pari possiamo limitarci a $x\geq 0$
%inserisci immagine 
Il punto fondamentale è che, sebben il dato iniziale sia nonnullo solo in una regione localizzata, appena comincia al'evoluzione la funzione è maggiore di zero \underline{ovunque}, per quanto lontano dal supporto del dato iniziale. $\grave{E}$ questo il \underline{comportamento diffusivo} che contrasta con la propagazione per onde.\\
-Dimostrazione che la (11) soddisfa il dato iniziale per $t\rightarrow 0$. A tal fine dimostriamo che :
$$
\lim_{a \to 0} \dfrac{1}{a\sqrt{\pi}}e^{-\dfrac{x^2}{a^2}}=\delta(x)=:\delta_a(x)
$$
$$
\int_{-\infty}^{\infty}\delta_a(x)\phi(x)dx=\int_{-\infty}^{\infty}\dfrac{1}{a\sqrt{\pi}}e^{-\dfrac{x^2}{a^2}}\phi(x)dx= (\dfrac{x}{a}=:y)=\dfrac{1}{\sqrt{\pi}}\int_{-\infty}^{\infty}e^{-y^2}\phi(ya)dy
$$
$$
\lim_{a\to 0}\int_{-\infty}^{\infty}\delta_a(x)\phi(x)dx=\dfrac{1}{\sqrt{\pi}}\lim_{a\to 0}\int_{-\infty}^{\infty}e^{-y^2}\phi(ya)dy
=(\text{teo conv dominata})=\dfrac{1}{\sqrt{\pi}}\int_{-\infty}{\infty}\lim_{a\to 0}e^{-y^2}\phi(ya)dy=$$
$$=\dfrac{\phi(0)}{\sqrt{\pi}}\int_{-\infty}^{\infty}e^{-y^2}dy=\phi(0)
$$
Con $a=2\sqrt{\chi t}:$
\begin{equation}
\lim_{t \to 0}\dfrac{1}{2\sqrt{\pi\chi t}}e^{-\dfrac{x^2}{4\chi t}}=\delta(x) (15)
\end{equation}

Quindi (12)$\Rightarrow$
$$
\lim_{t\to 0} G(\bar{z},t)=\delta(z_1)\cdot\dots\cdot\delta(z_d)=\delta(\bar{z})
$$
e la (11) implica 
$$
\lim_{t\to 0}T(\bar{x},t)=\int d^d\bar{y}\lim_{t\to 0}G(\bar{x}-\bar{y},t)T_0(\bar{y})=\int d^d \bar{y}\delta(\bar{x}-\bar{y})T_0(\bar{y})=T_0(\bar{x}) \quad \blacksquare
$$
-Flusso di calore con produzione di calore:
$$
\dfrac{\partial T}{\partial t} -\chi\Delta T=S(\bar{x},t) \quad t>0
$$
S è il termine di sorgente, rende l'equazione lineare non omogenea\\
-Soluzione particolare dell'equazione non omogenea:\\
definiamo una funzione di Green G tramite una convoluzione spaziale e temporale
\begin{equation}
\left(\partial_t-\chi\Delta\right)G(\bar{x}-\bar{x'},t-t')=\delta(\bar{x}-\bar{x'})\delta(t-t') (16)
\end{equation}

$$
\Rightarrow T(\bar{x},t)=\int d^n\bar{x'}dt'
$$
\begin{equation}
G(\bar{x}-\bar{x'},t-t')S(\bar{x'},t') (17)
\end{equation}

Check: 
$$
(\partial_t-\chi\Delta_{\bar{x}})T(\bar{x},t)=\int d^n\bar{x'}dt'(\partial_t-\chi\Delta_{\bar{x}})G(\bar{x}-\bar{x'},t-t')S(\bar{x'},t')=S(\bar{x},t)
$$
$$
(\partial_t-\chi\Delta_{\bar{x}})G(\bar{x}-\bar{x'},t-t')=\delta(\bar{x}-\bar{x'})\delta(t-t')
$$
trasformata di Fourier, definisco $\bar{z}:=\bar{x}-\bar{x'}, \tau := t-t'$
$$
G(\bar{z},\tau)=\dfrac{1}{(2\pi)^{n+1}}\int d^n\bar{k}d\omega e^{i(\bar{k}\bar{z}-\omega\tau)}\tilde{G}{\bar{k},\omega}\delta(\bar{z})\delta(\tau)=\dfrac{1}{(2\pi)^{n+1}}\int d^n\bar{k}d\omega e^{i(\bar{k}\bar{z}-\omega\tau)}
$$
sostituito nella (16) mi da
$$
(-i\omega+\chi\bar{k}^2)\tilde{G}(\bar{k},\omega)=1
$$
$$
\Rightarrow\tilde{G}(\bar{k},\omega)=\dfrac{1}{-i\omega+\chi\bar{k}^2}
$$
$$
\Rightarrow G(\bar{z},\tau)=\dfrac{1}{(2\pi)^{n+1}}\int d^n\bar{k}dw e^{i(\bar{k}\bar{z}-\omega\tau)}\dfrac{1}{-i\omega+\chi\bar{k}^2}
$$
calcolo l'integrale in $d\omega$ con il teorema dei residui, chiudendo sopra se $\tau$ è positiva, e viceversa
$$
i\omega+\chi\bar{k}^2=0 \Rightarrow\omega=-i\chi\bar{k}^2
$$
%immagine del cammino complesso dei residui
$$
\int d\omega\dfrac{e^{-i\omega\tau}}{-i(\omega+i\chi\bar{k}^2)}=:\int F(\omega) d\omega
$$
$$
e^{-i\omega\tau}=e^{-i(Re\omega+iIm\omega)\tau}
$$
$$
\begin{matrix}
 \tau > 0 & \Rightarrow Im\omega <0\\ 
 \tau < 0 & \Rightarrow Im\omega>0 \Rightarrow G(\bar{z},t)=0
\end{matrix}
$$
per $\tau<=$, cioè $t-t'<0$
$$
F(\omega)dw=\dfrac{e^{-i\tau(\rho\cos\varphi+i\rho\sin\varphi)}}{-i(\rho e^{i\varphi}+i\chi\bar{k}^2)}\rho e^{i\varphi}id\varphi=:f(\varphi)d\varphi
$$
vado a risolvere l'integrale
$$
\left|\int_{0}^{\pi}f(\varphi)d\varphi\right|\leq\int_{0}^{\pi}\left|f(\varphi)\right|d\varphi=\int_{0}^{\pi}\dfrac{e^{\tau\rho\sin\varphi}\rho d\varphi}{\left|\rho e^{i\varphi}+i\chi\bar{k}^2\right|}=$$
$$=\int_{0}^{\pi}\dfrac{e^{\tau\rho\sin\varphi}\rho d\varphi}{\sqrt{(\rho\cos\varphi)^2+(\rho\sin\varphi+\chi\bar{k}^2)^2}}=\int_{0}^{\pi}\dfrac{e^{\tau\rho\sin\varphi}\rho d\varphi}{\sqrt{\rho^2+2\rho\sin\varphi\chi\bar{k}^2+\chi^2 k^4}}
$$
posso minorare il denominatore con $\rho^2$ e ottengo
$$
\leq\int_{0}^{\pi}\dfrac{e^{\tau\rho\sin\varphi}\rho d\varphi}{\rho}=2\int_{0}^{\pi/2}e^{\tau\rho\sin\varphi d\varphi}
$$

%inizio lezione 12/10

So che in $[0,\dfrac{\pi}{2}]: \sin\varphi\geq\dfrac{2\varphi}{\pi}$

%inserisci immagine

$$
(\tau<0)\Rightarrow \tau\rho\sin\varphi \leq \tau\rho\dfrac{2\varphi}{\pi}
$$
$$
\Rightarrow 2\int_{0}{\pi/2}e^{\tau\rho\sin\varphi}d\varphi \leq 2\int_{0}^{\pi/2}e^{\tau\rho\dfrac{2\varphi}{\pi}}=2\left[\dfrac{\pi}{2\tau\rho}e^{\tau\rho\dfrac{2\varphi}{\pi}}\right]_{0}^{\dfrac{\pi}{2}}=\dfrac{\pi}{\tau\rho}(e^{\tau\rho}-1)\rightarrow (\rho \to \infty) 0
$$

Analogamente si dimostra che anche per il cammino sotto l'integrale per $\rho\to\infty$ tende a 0. Restano da calcolare i residui.
$$
ResF(\omega)=\lim_{\omega \to -i\chi\bar{k}^2} F(\omega)\cdot (\omega + i\chi\bar{k}^2)=ie^{-i\tau(-i\chi\bar{k}^2)}=ie^{-\tau\chi\bar{k}^2}
$$
Per $\tau>0$ chiudo l'integrale sotto. Il valore dell'integrale sull'asse reale è la differenza tra l'integrale sul cammino chiuso e quello sul solo semicerchio sotto.
%inserisco i simboli di integrali di cammino descritti sopra
$$
\tau>0:\int_{-\infty}^{\infty}d\omega F(\omega)=(residui)=-2\pi i ResF(\omega)=-2\pi i e^{-\tau\chi\bar{
k}^2}=2\pi e^{-\tau\chi\bar{k}^2}
$$
$$
\Rightarrow G(\bar{z},t)=\dfrac{1}{(2\pi)^n}\int d^n\bar{k}e^{i\bar{k}\bar{z}-\tau\chi\bar{k}^2} =\dfrac{1}{(2\pi)^n}exp\left(\dfrac{-\bar{z}^2}{4\tau\chi}\right) \int d^n\bar{k} e^{-\tau\chi\left(\bar{k}-\dfrac{i\bar{z}}{2\tau\chi}\right)}=
$$
\begin{equation}
=\dfrac{1}{(2\pi)^n}exp\left(-\dfrac{\bar{z}^2}{4\tau\chi}\right)\sqrt{\dfrac{\pi}{\chi\tau}}^n=\dfrac{1}{(4\pi\tau\chi)^{n/2}}exp\left(-\dfrac{\bar{z}^2}{4\chi\tau}\right) (18)
\end{equation}

$\rightarrow$ nucleo di calore!
\begin{equation}
G(\bar{z},\tau)=0,\tau>0 (19)
\end{equation}

$\rightarrow$ soluzione particolare dell'equazione del calore con sorgente:
$$
T(\bar{x},t)=(17)=\int d^n\bar{x'}\int_{-\infty=0}^{t}dt'G(\bar{x}-\bar{x'},t-t')S(\bar{x'},t')
$$
Soluzione generale dell'equazione omogenea
$$
T_{om}(\bar{x},t)=(11)=\int d^n\bar{x'}G(\bar{x}-\bar{x'},t)T_0(\bar{x'})
$$
Soluzione generale dell'equazione non omogenea
$$
T(\bar{x},t)=T_p(\bar{x},t)+T_{om}(\bar{x},t)
$$
supponi $S(\bar{x'},t')=0$ per $t'<0$ $\Rightarrow T_p(\bar{x},t)=0$, e quindi abbiamo che $T(\bar{x},0)=T_0(\bar{x})$\\
-Considera il problema di Dirichlet in un dominio connesso ( o su una varietà curva con bordo) U. $\lambda_n$ : autovalori del problema di Dirichlet, $\phi :$ autofunzioni di $\Delta$
$$\begin{matrix}
\Delta \phi + \lambda\phi =0 & \text{in } U \\
\phi=0 & \text{su }\partial U
\end{matrix}
$$
\begin{equation}
G(t,\bar{x},\bar{y})=\sum_{n}e^{-\lambda_n\chi t}\phi_n(\bar{x})\phi_n(\bar{y}) (20)
\end{equation}

\begin{equation}
T(\bar{x},t)=\int d^d\bar{y}G(t,\bar{x},\bar{y})t_0(\bar{y})=\int d^d \bar{y}\sum_{n}e^{-\lambda_n\chi t}\phi_n(\bar{x})\phi_n(\bar{y})T_0(\bar{y})(21)
\end{equation}


La (20) è un esempio di una funzione di Green che non dipende solo la $\bar{x}-\bar{y}$, ma da $\bar{x}$ e $\bar{y}$ separatamente. Motivo: rottura dell'invarianza per traslazioni a causa del bordo $\partial U$\\
Check: $T(\bar{x},0)=(21)=$
$$
=\int d^d \bar{y}\sum_{n}\phi_n(\bar{x})\phi_n(\bar{y})T_0(\bar{y})=\int d^d\bar{y}\delta(\bar{x}-\bar{y})T_0(\bar{y})=T_0(\bar{x})
$$
$$
\left . T(\bar{x},t)\right|_{\bar{x}\in\partial U}=\int d^d\bar{y}\sum_n e^{-\lambda n \chi t}
\left . \phi_n(\bar{x})\right|_{\bar{x}\in\partial U}\phi_n(\bar{y})t_0(\bar{y})=0
$$
$$
\dfrac{\partial}{\partial t}T(\bar{x},t)=\int d^d\bar{y}\sum_n (-\lambda_n \chi)e^{-\lambda_n\chi t}\cdot \phi_n(\bar{x})\phi_n(\bar{y})T_0(\bar{y})
$$
$$
\chi\Delta_{\bar{x}}T(\bar{x},t)=\int d^d \bar{y} \sum_n e^{-\lambda_n \chi t}\chi\Delta_{\bar{x}}\phi_n(\bar{x})\phi_n(\bar{y})T_0(\bar{y})=(\chi\Delta_{\bar{x}}\phi_n(\bar{x})=-\lambda_n\phi_n(\bar{x}))=\dfrac{\partial}{\partial t}T(\bar{x},t)
$$
Esempio: $U=[0,L] \rightarrow$ da risolvere $\partial_t T=\chi\partial_{x}^2T, $ $x\in[0,L],t>0$
condizioni al contorno $T(0,t)=0=T(L,t)$ , $T(x,0)=T_0(x)$
$$
\partial^{2}_{x}\phi=-\lambda\phi\rightarrow\phi=\phi_0\sin\dfrac{n\pi}{L}x, n=0,1,2,\ddots, \lambda=\dfrac{n^2\pi^2}{L^2}
$$
$$
\int_0^L dx \phi_n(x)^2=1\Rightarrow\phi_0=\sqrt{\dfrac{2}{L}}
$$
\begin{equation}
(20)\Rightarrow G(t,x,y)=\dfrac{2}{L}\sum_{n=0}^{\infty}e^{-n^2\pi^2\dfrac{\chi t}{L^2}}\cdot \sin\dfrac{n\pi}{L}x\sin\dfrac{n\pi}{L}y (22)
\end{equation}

\begin{equation}
(21)\Rightarrow T(x,t)=\int_0^L dy\sum_{n=1}^{\infty}e^{-n^2\pi^2\dfrac{\chi t}{L^2}}\cdot \dfrac{2}{L}\sin\dfrac{n\pi}{L}x\sin\dfrac{n\pi}{L}y T_0(y) (23)
\end{equation}

Sviluppo $T_0(y)$ in serie di Fourier: 
\begin{equation}
T_0(y)=\sum_{n=1}^{\infty}b_n\sin\dfrac{n\pi}{L}y (24)
\end{equation}

con 
\begin{equation}
b_n=\dfrac{2}{L}\int_0^LT_0(y)\sin\dfrac{n\pi}{L}ydy (25)
\end{equation}

Usando la (25), posso riscrivere la (23)
\begin{equation}
T(x,t)=\sum_{n=1}^{\infty}e^{-n^2\pi^2\dfrac{\chi t}{L^2}}b_n\sin\dfrac{n\pi}{L} (26)
\end{equation}

N.B.: la (20) vale anche per varietà compatte senza bordo (p.e. $S^2$)\\
Compiti a casa:
i) risolvere $\partial_t T=\chi\Delta T$ nel cubo
$$
\begin{matrix}
\partial_t T=\chi\Delta T\\
0\leq x\leq L & 0\leq y\leq L & 0\leq z\leq L & t\geq 0\\
T(\bar{x},0)=T_0(\bar{x})\\
T(\bar{x},t)=0 & x=0,L & y=0,L & z=0,L 
\end{matrix}
$$
ii) le (20),(21) valgono anche per il problema di Neumann $\Delta\phi + \lambda\phi =0$ in Um $\bar{n}\cdot\bar{\nabla}\phi=0$ su $\partial U$ ($\bar{n}$: versore normale al bordo). Perchè? (p.e. verificare che $\left . \bar{n}\cdot\bar{\nabla}T\right|_{\bar{x}\in \partial U}=0$
risolvere l'equzione del calore nel cubo con pareti isolati (nessun flusso termico attraverso le pareti, vedi equazione (1)
$$
\begin{matrix}
T(\bar{x},0)=T_0(\bar{x})\\
\partial_x T=0 & x=0,L \\
\partial_y T=0 & y=0,L \\
\partial_z T=0 & z=0,L \\
\end{matrix}
$$
%cambia solo che i sensi di i) diventano coseni, dato che sono passato alla derivata
 iii) Risolvere l'equzione del calore sulla 2-sfera. Suggerimento: scrivere il laplaciano in coordinate sferiche
\begin{equation}
\Delta =\dfrac{1}{r^2}\partial_r(r^2\partial_r)+\dfrac{1}{r^2\sin\theta}\partial_{\theta} (\sin\theta\partial_{\theta}+\dfrac{1}{r^2\sin^2\theta}\partial^2\varphi (27)
\end{equation}

porre $r=cost$ e usare 
\begin{equation}
\dfrac{1}{\sin\theta}\partial_{\theta}(\sin\theta Y^m_l)+\dfrac{1}{\sin^2\theta}\partial^2_{\varphi}Y_l^m=-\lambda Y_l^m (28)
\end{equation}

con $\lambda=l(l+1),Y_l^m$ armoniche sferiche. Sostituire le (24),(25) col corrispondente sviluppo in armoniche sferiche.\\
-Parentesi: soluzione di $\Delta_{\bar{x}}G(\bar{x},\bar{y})=-\delta(\bar{x}-\bar{y})$ in d dimensioni.\\
consideriamo un problema un po' più generale:
\begin{equation}
(\Delta_{\bar{x}}-m^2)G(\bar{x}-\bar{y})=-\delta(\bar{x}-\bar{y}) (29)
\end{equation}

$\rightarrow$ G è il nucleo dell'equazione di \underline{Helmholtz} 
\begin{equation}
(\Delta_{\bar{x}}-m^2)f=-S (30)
\end{equation}

(soluzione: $f(\bar{x})=\int d^d\bar{x'}G(\bar{x}-\bar{x'})S(\bar{x'})$)
G è il propagatore per un campo scalare in di dimensioni Euclidee ($\rightarrow$ teoria quantistica dei campi). Uso la trasformata di Fourier per riportarmi ad un'equazione algebrica
$$
G(\bar{x}-\bar{y})=\dfrac{1}{(2\pi)^d}\int d^d \bar{k}e^{i\bar{k}(\bar{x}-\bar{y})}\tilde{G}(\bar{k})
$$
$$
\delta(\bar{x}-\bar{y})=\dfrac{1}{(2\pi)^d}\int d^d \bar{k}e^{i\bar{k}(\bar{x}-\bar{y})} \Rightarrow (-k^2-m^2)\tilde{G}(\bar{k})=-1\Rightarrow \tilde{G}(\bar{K})=\dfrac{1}{\bar{k}^2+m^2}
$$
\begin{equation}
\Rightarrow G(\bar{z})=\dfrac{1}{(2\pi)^d}\int d^d \bar{k} \dfrac{e^{i\bar{k}\bar{z}}}{\bar{k}^2+m^2}
\end{equation}

Uso 
\begin{equation}
\dfrac{1}{\bar{k}^2+m^2}=\int_0^{\infty}exp\left( -\tau(\bar{k}^2+m^2)\right)d\tau (32)
\end{equation}
$$
\Rightarrow G(\bar{z})=\dfrac{1}{(2\pi)^d}\int_0^{\infty}d\tau e^{-\tau m^2}\cdot\int d^d \bar{k} e^{i\bar{k}\bar{z}-\tau\bar{k}^2}
$$

%%%%%%%%%%%%%%%%%%%%%%%%%%%%%%%%%%%%%%%%%%%%%%%%%%%%%%%%%%%%%%%%%%
%inizio lezione 18/10
%%%%%%%%%%%%%%%%%%%%%%%%%%%%%%%
$$
i\kk\bar{z}-\tau\kk^2=-\tau(\kk-\dfrac{i\bar
z}{2\tau})^2-\dfrac{\bar{z}^2}{4\tau}
$$
$$
\kp:=\kk-\dfrac{i\bar
z}{2\tau}
$$
$$
\Rightarrow G(\bar{z})=\dfrac{1}{(2\pi)^d}\int_0^{\infty}d\tau e^{-\tau m^2-\frac{\bar{z}^2}{4\tau}}\cdot\int d^d\kp e^{-\tau\kp^2}
$$
$$
\int d^d\kp e^{-\tau\kp^2}=\left(\dfrac{\pi}{\tau}\right)^{d/2} \quad (\tau>0)
$$
$$
\dfrac{1}{(4\pi)^{d/2}}\int_{0}^{\infty}\tau^{-d/2}e^{-\tau m^2 - \frac{\bar{z}^2}{4\tau}}d\tau
$$
$$
\tau:=\dfrac{m^{-1}}{2}|\bar{z}|e^t, \quad d\tau=\dfrac{m^{-1}}{2}|\bar{z}|e^tdt
$$
$$
\tau^{-\dfrac{d}{2}}=\left(\dfrac{m^{-1}|\bar{z}}{2}\right)^{-\dfrac{d}{2}}e^{-\dfrac{d}{2}t}
$$
$$
-\tau m^2-\dfrac{\bar{z}^2}{4\tau}=-\dfrac{m^-1}{2}|\bar{z}|e^t m^2
$$
$$
-\dfrac{\bar{z}^2}{4\tau}\dfrac{2m}{|\bar{z}|}e^{-t}=-m|\bar{z}|\cosh t
$$
$$
\Rightarrow G(\bar{z})=\dfrac{1}{(4\pi)^{d/2}}\int_{-\infty}^{\infty}dty\left(\dfrac{|\bar{z}|}{2m}\right)^{1-d/2}e^{(1-d/2)t}e^{-m|\z|\cosh t} =\int_{-\infty}^{0}+\int_{0}^{\infty} 
$$
$$
\dfrac{1}{(4\pi)^{d/2}}\int_{-\infty}^{0}dt\left(\dfrac{|\z|}{2m}\right)^{1-d/2}e^{(1-d/2)t}e^{-m|\z|\cosh t}
$$
$$
(t'=-t)=\dfrac{1}{(4\pi)^{d/2}}\int_0^\infty dt'\left(\dfrac{|\bar{z}|}{2m}\right)^{1-d/2}e^{-(1-d/2)t'}e^{-m|\bar{z}|\cosh t'}
$$
$$
(t'\rightarrow t)\Rightarrow G(\bar{z})=\dfrac{1}{(4\pi)^{d/2}}\int_0^{\infty}dt\left(\dfrac{|\bar{z}|}{2m}\right)^{1-d/2}\cdot 2\cosh ((1-d/2)t)e^{-m|\bar{z}|\cosh t}
$$
funzione di Bessel modificata del 2 tipo:
\begin{equation}
K_{\nu}(x)=\int_0^{\infty}e^{-x\cosh t}\cosh(\nu t)dt (33)
\end{equation}

$$
Rex>0 \Rightarrow K_{-\nu}(x)=K_{\nu}(x)
$$
$$
\Rightarrow G(\bar{z})=\dfrac{1}{(4\pi)^{d/2}}\left(\dfrac{|\bar{z}|}{2m}\right)^{1-d/2}\cdot 2 K_{1-d/2}(m|\bar{z}|)
$$
\begin{equation}
\Rightarrow G(\z)=\dfrac{1}{(2\pi)^{d/2}}m^{d-2}(|\z|m)^{1-d/2}\cdot K_{1-d/2}(m|\z|) (34)
\end{equation}

$m\to 0:$ usa
\begin{equation}
K_{\nu}(x) (x\to 0)\left\{\begin{matrix}
-\gamma-\ln \dfrac{x}{2} & \nu=0 \\
\dfrac{\Gamma(\nu)}{2}\left(\dfrac{2}{x}\right)^{\nu} & \nu >0
\end{matrix}\right.(35)
\end{equation}

dove $\gamma=\lim_{n\to\infty}\left(\sum_{k=1}^{n}\dfrac{1}{k}-\ln n\right)\approx 0,577$ costante di Eulero-Mascheroni.\\
$\Gamma(\nu)$: funzione gamma di Eulero, estende il concetto di fattoriale ai numeri completti, el senso che per ogni numero intero non negativo n si ha $\Gamma (n)=(n-1)!$

\begin{equation}
\Gamma(z)=\int_0^{\infty}t^{z-1}e^{-t}dt,\quad Rez>0 (36)
\end{equation}

Esercizio: integrando per parti, dimostrare che 
\begin{equation}
\Gamma(z+1)=z\Gamma(z)(37)
\end{equation}
$\Rightarrow \Gamma(z)=\dfrac{\Gamma(z+1)}{z}$
Usando 	questa la definizione della $\Gamma$ può essere estesa al piano $Rez<0\Rightarrow$ per $d+2$ e $m\to 0$, la (34) diventa
\begin{equation}
G(\z)\rightarrow \dfrac{1}{(2\pi)^{d/2}}m^{\dfrac{d-2}{2}}|\z|^{1-d/2}\dfrac{\Gamma(\dfrac{d}{2}-1)}{2}\left(\dfrac{2}{m|\z|}\right)^{d/2-1}=\dfrac{\Gamma(\dfrac{d}{2}-1)}{4\pi^{d/2}|\z|^{d-2}} (38)
\end{equation}

corrisponde al potenziale di una carica puntiforme in d-dimensioni.\\
Caso $d=3$:
\begin{equation*}\label{2.38'}
\begin{matrix}
\Gamma\left(\dfrac{1}{2}\right)=\sqrt{\pi} \\
\Rightarrow G(\z)=\dfrac{1}{4\pi|\z|}(38')
\end{matrix}
\end{equation*}

Caso limite $d=2$:
\begin{equation}
(34)\Rightarrow G(\z)=\dfrac{1}{2\pi}K_0(m|\z|) (39)
\end{equation}

$$
(35) (m\to 0)\rightarrow \dfrac{1}{2\pi}\left(-\gamma -\ln \dfrac{m|\z|}{2}\right)=\dfrac{1}{2\pi}(-\gamma -\ln m + \ln 2 - ln|\z|)
$$
Chiaro: G definita a meno di una costante additiva nel caso $m\to 0$.
\begin{equation}
\Delta_{\x}G(\x-\y)=-\delta(|x-\y) \Rightarrow G(\z)=-\dfrac{\ln |\z|}{2\pi} (40)
\end{equation}

Check: $\Delta_{\x}G(\x)=-\delta(\x)$\\
A causa dell'invarianza per rotazioni, G dipende solo da $|\x|$, se le condizioni al contorno non rompono l'invarianza. Passo in coordinate polari per controllare.\\
Coordinate polari:
$$
\Delta =\dfrac{\partial^2}{\partial r^2}+\dfrac{1}{r}\dfrac{\partial}{\partial r}+\dfrac{1}{r^2}\dfrac{\partial ^2}{\partial \varphi^2}
$$
considero $r\neq 0$, $\dfrac{\partial G}{\partial \varphi}=0$
$$
\left(\dfrac{\partial^2}{\partial r^2}+\dfrac{1}{r}\dfrac{\partial}{\partial r}\right)\dfrac{-\ln r}{2\pi}=\dfrac{1}{2\pi}\left(\dfrac{1}{r^2} + \dfrac{1}{r}\left(-\dfrac{1}{r}\right)\right)=0\qquad \checkmark
$$

Per verificare che $\Delta_{\x}G(\x)=-\delta(\x)$ integro $\Delta G$ su un disco D centrato in zero, con raggio $\epsilon$
$$
\int_D \Delta G d^2\x=\int_D \underline{\nabla}\cdot(\underline{\nabla}G)d^2\x
$$
Uso Gauss per passare a un integrale di bordo
$$
=\int_{\partial D} \bar{n} \cdot \underline{\nabla}G ds \quad \bar{n}=\bar{e_r}, \quad ds=rd\varphi
$$
$$
\int_{\partial D}\bar{n}\cdot \underline{\nabla}G ds=\int_{0}^{2\pi}\left(\dfrac{\partial}{\partial r}G\right)rd\varphi=\int_0^{2\pi}-\dfrac{1}{2\pi r}r d\varphi =-1 =-\int_D \delta(\x)d^2\x \quad \checkmark
$$
ho ottenuto il nucleo del laplaciano in due dimensioni, che corrisponde al potenziale elettrostatico in due dimensioni
\section{Flusso di calore in un cilindro infinito}
%immagine cilindro, a fianco le condizioni
$$
\begin{matrix}
\dfrac{\partial}{\partial t}T=\chi\Delta T \\
T(\x,t)=0 & \x\text{ sul bordo}\\
T(\x,0)=T_0(\x)
\end{matrix}
$$
la simmetria del problema suggerisc edi usare le coordinate cilindriche: $r,\varphi,z$, con $x=r\cos \varphi$, $y=r\sin \varphi$
$$
\Delta = \dfrac{\partial ^2}{\partial r^2} +\dfrac{1}{r}\dfrac{\partial^2}{\partial \varphi ^2} + \dfrac{\partial ^2}{\partial z^2}
$$
$0\leq r \leq L, \quad 0\leq \varphi \leq 2\pi, \quad -\infty <z<\infty$
\subsection{Separazione delle variabili}
$$
T(r,\varphi,z,t)=\tau(t)R(r)\Phi(\varphi)Z(z)\Rightarrow \dfrac{\partial T}{\partial t}=\tau '(t)R\Phi Z
$$
$$
\Delta T=\tau(R'' +\dfrac{1}{r}R')\Phi Z + \tau R \dfrac{1}{r^2}\Phi''Z+\tau R \Phi Z'' = \dfrac{1}{\chi}\tau'R\Phi Z \Rightarrow \dfrac{1}{R}(R'' + \dfrac{1}{r}R') + \dfrac{1}{r^2}\dfrac{\Phi''}{\Phi}+\dfrac{Z''}{Z}-\dfrac{1}{\chi}\dfrac{\tau'}{\tau}=0
$$
la funzione di Z è costante $=C_2$, e analogamente la funzione di t $=C_2$, di conseguenza la parte rimanente in $r,\varphi=C_1-C_2$
$$
\begin{matrix}
\tau'=\chi\tau C_1 & \tau=\tau_0 e^{\chi C_1 t} \\
Z''=C_2Z & Z=Z_0e^{\sqrt{C_2}z}+Z_1e^{-\sqrt{C_2}z}
\end{matrix}
$$
$$
r^2\dfrac{1}{R}(R''+\dfrac{1}{r}R')+\dfrac{\Phi''}{\Phi}=(C_1-C_2)r^2 \Rightarrow \dfrac{\Phi''}{\Phi}=cost:=-\lambda^2
$$
$$
^2\dfrac{1}{R}(R''+\dfrac{1}{r}R')+ (C_2-C_1)r^2=\lambda^2 \Rightarrow \Phi=\Phi_0 \cos\lambda\varphi + \Phi_1\sin\lambda\varphi=\Phi_0\cos\lambda(\varphi+2\pi)+\Phi_1\sin\lambda(\varphi+2\pi)=\phi_0(\cos\lambda\varphi\cos 2\pi\lambda-\sin\lambda\varphi\sin 2\pi\lambda)+\phi_1(\sin\lambda\varphi\cos 2\pi\lambda + \cos \lambda\varphi\sin 2\pi\lambda)
%aggiunta a mano
$$
$$
\Rightarrow \begin{matrix}
\phi_0\cos 2\pi\lambda + \phi_1 \sin 2\pi\lambda=\phi_0\\
-\phi_0\sin 2\pi\lambda + \phi_1 \cos 2\pi\lambda=\phi_1
\end{matrix}
\Rightarrow
\left(\begin{matrix}
\cos 2\pi\lambda -1 & \sin 2\pi\lambda \\
-\sin 2\pi\lambda & \cos 2\pi\lambda -1
\end{matrix}\right)
\left(\begin{matrix}
\phi_0\\
\phi_1
\end{matrix}\right)
=0
$$
ho una soluzione non banale se il determinante è nullo $\implies \cos 2\pi\lambda=1,\quad \sin 2\pi\lambda=0, \quad \lambda=m,m \in \mathbb{Z}$. Basta prendere $m\in \mathbb{N}_0$, per cui le funzioni $\Phi$ formano un sistema completo.\\
Passo all'equazione radiale:
$$
r^2 R'' + r R' + ((C_2-C_1)r^2-m^2)R=0
$$
Supponiamo per semplicità che $T_0(\x)$ dipenda solo da $r,\varphi$.
$$
T(r,\varphi,z,0)=\tau_0R\Phi Z'=T_0(r,\varphi)\Rightarrow Z=cost \Rightarrow C_2=0
$$
$$
Z=Z_0e^{\sqrt{C_2}z}+Z_1e^{-\sqrt{C_2}z}
$$
senza perdere la generalità, poniamo $Z=1, C_1<0$ altrimenti T diverge per $t\to \infty \quad(\tau=\tau_0e^{\chi C_1 t})$
(Inoltre si può far vedere che la soluzione dell'equzione radiale (con $C_2=0$) diverge nell'origine (per le nostre condizioni al contorno) se $C_1>0$)
Pongo $x:=\sqrt{|C_1|}r$
\begin{equation}
\Rightarrow x^2\dfrac{d^2 R}{dx^2}+x\dfrac{dR}{dx}+(x^2-m^2)R=0 (41)
\end{equation}

equazione differenziale di Bessel. Cerca soluzioni della forma

\begin{equation}
R(x)=x^\alpha \sum_{n=0}^{\infty} a_nx^n(42)
\end{equation}

$$
(a_0\neq 0)\Rightarrow R'(x)=\alpha x^{\alpha -1}\sum_n a_nx^n + x^\alpha\sum_n na_nx^{n-1}
$$
$$
R''(x)=\alpha(\alpha-1)x^{\alpha-2}\sum_na_nx^n + 2\alpha x^{\alpha-1}\sum_n a_n x^{n-1}+x^\alpha\sum_n n(n-1)a_nx^{n-2}
$$

\begin{equation*}
\begin{split}
(41)\Rightarrow &\alpha(\alpha-1)x^\alpha\sum_n a_n x^n+2\alpha x^\alpha \sum_n n a_n x^n + x^\alpha \sum_n n(n-1)a_n x^n + \\
& \alpha x^\alpha\sum_n a_n x^n + x^\alpha \sum_n n a_n x^n + (x^2-m^2)x^\alpha \sum_n a_n x^n=0
\end{split}
\end{equation*}

studio il prefattore di $x^0$: $\alpha^2 a_0-m^2a_0=0\Rightarrow \alpha=\pm m$. Scarto la soluzione $\alpha=-m$ perchè non fisica (divergerebbe sull'asse del cilindro).\\
Soluzione regolare in $x=0: \alpha=m$. In tal caso:

\begin{equation}
2m\sum_n n a_n x^n + \sum_n n^2 a_n x^n + \sum_n a_n x^{n+2}=0 ~~(43)
\end{equation}

sostituisco nell'ultimo pezzo $n+2=n'$
$$
\sum_{n'} a_{n'-2} x^{n'}
$$
$$
x^1: 2ma_1+a_1=0 \Rightarrow a_1=0 
$$

$\Rightarrow$ la (43) diventa:
\begin{equation}
\sum_{n=2}^{\infty}x^n(a_n(2mn+n^2)+a_{n-2})=0 \Rightarrow a_n := - \dfrac{a_n-2}{2mn+n^2} ~~ (44)
\end{equation}

$\rightarrow$ relazione di ricorrenza\\

%%%%%%%%%%%%%%%%%%%%%%%%%%%%%%%%%%%%%%
% inizio lezione 19/10
%%%%%%%%%%%%%%%%%%%%%%%%%%%%%%%%%%%
scegliendo $a_0=\dfrac{2^{-m}}{m!}$ per motivi di normalizzazione, si ottiene la soluzione

\begin{equation}
J_m(x)=\sum_{k=0}^{\infty}\dfrac{(-1)^k\left(\dfrac{x}{2}\right)^{m+2k}}{k!(m+k)!} (45)
\end{equation}

dove per m non interi si definisce $m!:=\Gamma(m+1)$
$J_m:$ \underline{funzione di Bessel del 1 tipo}\\
% disegno funzioni di bessel m=0,1,2
$\Rightarrow R=J_m(\sqrt{|C_1|}r)$, impongo le condizioni al contorno $
T(r=L,\varphi,z,t)=0 $
$$\Rightarrow R(r=L)=0 \Rightarrow J_m(\sqrt{|C_1|}L)=0 \implies \sqrt{|C_1|}L=j_k^{(m)}, \quad k=1,2,\dots \implies C_1=-\left(\dfrac{j_k^{(m)}}{L}\right)^2
$$
dove $j_k^{(m)}$ sono gli zeri positivi di $J_m(x)$, noti numericamente.
$$
\Rightarrow T(r,\varphi,z,t)=\tau_0e^{-\chi \left(\dfrac{j_k^{(m)}}{L}\right)^2 t}\cdot J_m\left(j_k^{(m)}\dfrac{r}{L}\right)(\phi_0\cos m\varphi + \phi_1 \sin r\varphi)
$$
la costante $\tau_0$ si può riassorbire in $\phi_0,\phi_1$, quindi la pongo =1. La soluzione generale sarà uan combinazione lineare di queste soluzioni.

\begin{equation}
T(r,\varphi,z,t)=\sum_{m=0}^{\infty}\sum_{k=1}^\infty e^{-\chi\left(\dfrac{j_k^{(m)}}{L}\right)^2t}J_m\left(j_k^{(m)}\dfrac{r}{L}\right)(C_{km}\cos m\varphi + S_{km}\sin m\varphi) (46)
\end{equation}

\begin{equation}
T(r,\varphi,z,0)=\sum_{m=0}^{\infty}\sum_{k=1}^\infty J_m\left(j_k^{(m)}\dfrac{r}{L}\right)(C_{km}\cos m\varphi + S_{km}\sin m\varphi)=T_0(r\varphi) (47)
\end{equation}

la (47) forma un sistema di funzioni completo nel cilindro $\rightarrow$ qualsiasi funzione $T_0(r,\varphi)$ con $T_0(L,\varphi)=0$ possiede uno sviluppo di questo tipo. Devono essere scelte delle costanti $C_{km} e S_{km}$ appropriate, invertendo la (47) 

\section{Problemi non omogenei}
spesso la separazione delle variabili riducce delle equazioni differenziali alle derivate parziali a delle equazioni differenziali ordinarie, come p.e.
\begin{equation}
a(x)u'' + b(x)u'+c(x)u=F(x) (48)
\end{equation}

Ipotesi: $a(x)$ continuamente differenziabile; b,c continue. Moltiplica la (48) con $\dfrac{1}{a(x)}exp\left(\int_\alpha^x\dfrac{b(\xi)}{a(\xi)}d\xi\right)$ e definisco
$$
p(x):= exp\left(\int_\alpha^x \dfrac{b(\xi)}{a(\xi)}d\xi\right)
$$
$$
q(x):=\dfrac{c(x)}{a(x)}p(x),\quad f(x):=\dfrac{F(x)}{a(x)}p(x)
$$
posso scrivere la "forma autoaggiunta" della (48)

\begin{equation}
(48)\Rightarrow \dfrac{d}{dx}\left(p\dfrac{du}{dx}\right) + qu = f(x) (49)
\end{equation}

%aggiunta non legata al problema vero e proprio
digressione:\\
$$
p\dfrac{d}{dx} \left( p\dfrac{d}{dx}u\right)+pqu=pf
$$
definisco $pq:=\omega^2(x)$, $pf:=g$ e y attraverso 
$$
\dfrac{d}{dy}=p(x)\dfrac{d}{dx} \Rightarrow \dfrac{dx}{p(x)}=dy \Rightarrow y=\int \dfrac{dx}{p(x)} 
$$ 
ottengo
$$
\dfrac{d^2u}{dy^2}+\omega^2(y)u=g
$$
è un'equazione di oscillatore armonico con frequenza $\omega$ che dipende dal "tempo" g, dove g è una forzante. Viene chiamato "oscillatore di Ermakoff", e trova applicazioni in Cosmologia (equazione di Sasaki-Mukhonov) per quanto riguarda la teoria dell'inflazione.\\

Equazione omogenea: 
\begin{equation}
\dfrac{d}{dx}\left(p\dfrac{dv}{dx}\right)+qv=0 (50)
\end{equation}

possiede 2 soluzioni $v_1, v_2$ linearmente indipendenti $\implies$ soluzione generale:
$$
v(x)=c_1v_1(x) + c_2v_2(x), \quad c_1, c_2 cost
$$

considera la funzione
\begin{equation}
w(x)=v_1(x)\int_\alpha^x v_2(\xi)f(\xi)d\xi - v_2(x)\int_\alpha^x v_1(\xi)f(\xi)d\xi ~~(51)
\end{equation}

(paragona con il metodo della variazione delle costanti)
$$
\Rightarrow w'(x)=v_1'(x)\int_\alpha^x v_2(\xi)f(\xi)d\xi - v_2'(x)\int_\alpha^x v_1(\xi)f(\xi)d\xi +v_1(x)v_2(x)f(x)-v_2(x)v_1(x)f(x) = 
$$
$$
= v_1'(x)\int_\alpha^x v_2(\xi)f(\xi)d\xi - v_2'(x)\int_\alpha^x v_1(\xi)f(\xi)d\xi \Rightarrow
$$
\begin{multline*}
\dfrac{d}{dx}\left(p(x)\dfrac{dw}{dx}\right)=\\
\dfrac{d}{dx}(p(x)v_1'(x))\int_\alpha^x v_2(\xi)f(\xi)d\xi - \dfrac{d}{dx}(p(x)v_2'(x))\int_\alpha^x v_1(\xi)f(\xi)d\xi + p(x)v_1'(x)v_2(x)f(x)-p(x)v_2'(x)v_1(x)f(x)
\end{multline*}

i coefficienti davanti agli integrali corrispondono rispettiamente a $-qv_1$ e $-qv_2$
$$
-q(x)w(x) + p(x)(v_1'(x)v_2(x)-v_2'(x)v_1(x))f(x)
$$
Inoltre:
$$
\dfrac{d}{dx}\left\{p(x)(v_1'(x)v_2(x)-v_2'(x)v_1(x))\right\}
$$
deve avere il Wronskiano 
$$
\left| \begin{matrix}
v_1 & v_2\\
v_1' & v_2'
\end{matrix}
\right| \neq 0
$$
$$
=\dfrac{d}{dx}(p v_1')v_2-\dfrac{d}{dx}(pv_2')v_1 + pv_1'v_2' - p v_2'v_1'=0
$$
$$
\Rightarrow p(x)(v_1'(x)v_2(x)-v_2'(x)v_1(x))=K\quad costante
$$
Quindi: 
\begin{equation}
\dfrac{d}{dx}\left(p\dfrac{dw}{dx}\right)+qw=Kf (52)
\end{equation}

Inoltre, se $v_1',v_2'$ sono limitati per $x\to\alpha$, $w(\alpha)=w'(\alpha)=0$
\begin{equation}
(52):K \Rightarrow \dfrac{w(x)}{K}=u(x)=\int_\alpha^xR(x,\xi)f(\xi)d\xi (53)
\end{equation}

con 
\begin{equation}
R(x,\xi):=\dfrac{v_1(x)v_2(\xi)-v_2(x)v_1(\xi)}{p(x)(v_1'(x)v_2(x)-v_2'(x)v_1(x))} (54)
\end{equation}

è una soluzione del problema ai valori iniziali
\begin{equation}
\left\{\begin{matrix}
\dfrac{d}{dx}\left(p\dfrac{du}{dx}\right) + qu=f(x) & x>\alpha\\
u(\alpha)=u'(\alpha)=0 &
\end{matrix}\right. (55)
\end{equation}

Il denominatore della (54) è costante $\implies R(x,\xi)$ soddisfa l'equazione omogenea (50) sia come funzione di x che di $\xi$. (NB: $R(x,\xi)=-R(\xi,x)$). Per $\xi$ fissato: $R(x,\xi)$ è la soluzione del problema omogeneo ai valori iniziali
$$
\dfrac{d}{dx}\left(p(x)\dfrac{dR}{dx}\right) + q(x)R=0, \quad x>\xi
$$
\begin{equation}
R |_{x=\xi}=0 \qquad \left.\dfrac{dR}{dx}\right|_{x=\xi}=(54)=\dfrac{1}{p(\xi)} (56)
\end{equation}

$R(x,\xi)$: funzione di Green (one-sided).\\
esempio: oscillatore armonico invertito
$$
\left\{\begin{matrix}
u'' - u =f(x) & x>0 \\
u(0)=u'(x)=0&
\end{matrix}\right.
$$
soluzione per $\xi$ fissato, $R(x,\xi)$ soddisfa 
$$
\dfrac{d^2R}{dx^2}-R=0, \quad x>\xi
$$
$$
R|_{x=\xi}=0, \qquad \left.\dfrac{dR}{dx}\right|_{x=\xi}=1
$$
$$
\Rightarrow R= A(\xi)\sinh(x)+B(\xi)\cosh(x)
$$
$$
R|_{x=\xi}=A\sinh\xi + B \cosh \xi =0
$$
$$
\left.\dfrac{dR}{dx}\right|_{x=\xi}=A\cosh\xi + B \sinh \xi =1
$$
$$
\Rightarrow A=\cosh \xi, \quad B=-\sinh \xi, \Rightarrow R=\sinh(x-\xi)
$$
$$
(53)\Rightarrow u(x)=\int_0^x f(\xi)\sinh(x-\xi)d\xi
$$
N.B. 
i) Se $u(\alpha),u'(\alpha)\neq 0 \implies$ aggiungi soluzione $c_1v_1(x)+c_2v_2(x)$ dell'equazione omogenea, in modo tale da soddisfare le nuove condizioni iniziali. Nell'esempio sopra:
$$
u(x)=\int_0^x f(\xi)\sinh(x-\xi)d\xi + c_1\sinh(x)+ c_2\cosh(x)
$$
soddisfa $u(0)=c_2,u'(0)=c_1$\\
ii) $(53)\Rightarrow$ Il valore di $u(x)$ dipende solo da $f(\xi)$ per  $\xi<x$. Comportamento molto simile a quelle delle equazioni alle derivate parziali iperboliche (vedi piu tardi) %inserisci riferimento dall'indice
\section{Problema ai valori al contorno}
Risolvi p.e.
$$\left\{\begin{matrix}
\dfrac{d}{dx}\left(p\dfrac{du}{dx}\right)+qu=-f(x) & \alpha <x<\beta\\
u(\alpha)=u(\beta)=0 &
\end{matrix}\right.
$$
(ndr il - davanti a $f(x)$ viene messo per convenienza)\\
soluzione generale:
\begin{equation}
u(x)=-\int_\alpha^x R(\xi,x)f(\xi)d\xi + c_1v_1(x)+c_2v_2(x) 
\end{equation}

\begin{equation}
\left\{\begin{matrix}
u(\alpha)=c_1v_1(\alpha)+c_2v_2(\alpha)=0\\
u(\beta)=-\int_{\alpha}^\beta R(\beta,\xi)f(\xi)d\xi + c_1v_1(\beta)+c_2v_2(\beta)=0
\end{matrix}\right. (58)
\end{equation}

La (58) ha una soluzione per $c_1,c_2$ se $D:=v_1(\alpha)v_2(\beta)-v_2(\alpha)v_1(\beta)\neq 0$. In tal caso
$$
c_1=-\dfrac{v_2(\alpha)}{D}\int_\alpha^\beta R(\beta,\xi)f(\xi)d\xi= -\dfrac{v_2(\alpha)}{D}\int_\alpha^x R(\beta,\xi)f(\xi)d\xi -\dfrac{v_2(\alpha)}{D}\int_x^\beta  R(\beta,\xi)f(\xi)d\xi
$$
$$
c_2= \dfrac{v_1(\alpha)}{D}\int_\alpha^\beta  R(\beta,\xi)f(\xi)d\xi= \dfrac{v_1(\alpha)}{D}\int_\alpha^x R(\beta,\xi)f(\xi)d\xi +\dfrac{v_1(\alpha)}{D}\int_x^\beta  R(\beta,\xi)f(\xi)d\x 
$$

%%%%%%%%%%%%%%%%%%%%%%%%%%%%%%%%%%%%%%%%%%%%%%%%%%%
%inizio lezione 25/10
%%%%%%%%%%%%%%%%%%%%%%%%%%%%%%%%%%%%%%%%%%%%%%%%%%%%%

% questo va probabilmente inserito prima delle soluzioni esplicite per c1 c2

$$
\Rightarrow u(x)=-\int_\alpha^x\left[R(x,\xi)+\dfrac{v_2(\alpha)v_1(x) - v_1(\alpha)v_2(x)}{D}R(\beta,\xi)\right]f(\xi)d\xi-\int_x^\beta \dfrac{v_2(\alpha)v_1(x) - v_1(\alpha)v_2(x)}{D} R(\beta,\xi)f(\xi)d\xi
$$

\begin{multline*}
R(x,\xi)+\dfrac{v_2(\alpha)v_1(x)-v_1(\alpha)v_2(x)}{D}R(\beta,\xi)=(54)=\\
\dfrac{v_1(x)v_2(\xi)-v_2(x)v_1(\xi)}{K D}(v_1(\alpha)v_2(\beta)-v_2(\alpha)v_1(\beta)) + \dfrac{v_2(\alpha)v_1(x)-v_1(\alpha)v_2(x)}{D} \dfrac{v_1(\beta)v_2(\xi)-v_2(\beta)v_1(\xi)}{K}
\end{multline*}

$$
\dfrac{1}{KD}(v_1(\alpha)v_2(\xi)-v_2(\alpha)v_1(\xi))(v_1(x)v_2(\beta)-v_2(x)v_1(\beta))
$$

Definisco G
\begin{equation}
G(x,\xi):=\left\{\begin{matrix}
\dfrac{1}{KD}(v_1(\xi)v_2(\alpha)-v_2(\xi)v_1(\alpha))(v_1(x)v_2(\beta)-v_2(x)v_1(\beta)) & \xi\leq x \\
\dfrac{1}{KD}(v_1(x)v_2(\alpha)-v_2(x)v_1(\alpha))(v_1(\xi)v_2(\beta)-v_2(\xi)v_1(\beta)) & x\leq \xi
\end{matrix}\right. (59)
\end{equation}

$\implies$ soluzione del problema ai valori al contorno (57):
\begin{equation}
u(x)=\int_\alpha^\beta G(x,\xi)f(\xi)d\xi (60)
\end{equation}

$G(x,\xi)$ è una funzione di Green
\begin{equation}
G(x,\xi)=G(\xi,x) (61)
\end{equation}


per determinare G notiamo che per ogni $\xi$ soddisfa il problema ai valori al contorno
\begin{equation}
\begin{matrix}
\dfrac{d}{dx}\left(p(x)\dfrac{dG}{dx}\right) + q(x)G = 0 \quad x\neq \xi \\
G|_{x=\alpha}=G|_{x=\beta}=0 \\
G_{x=\xi+0}=G|_{\xi-0} \\
G \mathrm{~continua~in~} \xi \\
\dfrac{dG}{dx}|_{x=\xi+0}-\dfrac{dG}{dx}|_{x=\xi-0}=-\dfrac{1}{p(\xi)}
\end{matrix} (62)
\end{equation}


$\frac{dG}{dx}$ discontinua in $x=\xi$. (Usare la (59). Per ricavare l'ultima equazione bisogna usare anche la definizione di K)

\underline{Esempio:} 
$$
((1+x)^2u')'-u=f(x), \quad 0<x<1, \quad u(0)=u(1)=0
$$
$$
\Rightarrow \dfrac{d}{dx}\left((1+x)^2 \dfrac{dG(x,\xi)}{dx}\right)-G(x,\xi)=0
$$
Prova $G(x,\xi)=c(\xi)(1+x)^\alpha \implies \dfrac{dG}{dx}=c\alpha(1+x)^{\alpha-1}$
$$
\left((1+x)^2 \dfrac{dG}{dx}\right)'=\left(c\alpha(1+x)
^{\alpha+1}\right)'=c\alpha(\alpha +1)(1+x)^\alpha = G = c(1+x)^\alpha
$$
$$
\implies \alpha^2 + \alpha - 1 =0 \Rightarrow \alpha=\dfrac{1}{2}(1\pm \sqrt{5}):=\alpha_\pm
$$
devo separe i casi $x>\xi$ e $x<\xi$
$$
\Rightarrow  G(x,\xi)=\left\{\begin{matrix}
c_+(\xi)(1+x)^{\alpha_+} + c_-(\xi)(1+x)^{\alpha_-} & x<\xi \\
\tilde{c}_+(\xi)(1+x)^{\alpha_+} + \tilde{c}_-(\xi)(1+x)^{\alpha_-} & x>\xi
\end{matrix}\right.
$$
$$
\begin{matrix}
G|_{x=0}=c_+ + c_- = 0 & \implies c_- = -c_+ \\
G|_{x=1}=\tilde{c}_+ 2^\alpha + \tilde{c}_-2^\alpha = 0 & \implies \tilde{c}_- = -\tilde{c}_+2^{\sqrt{5}} 
\end{matrix}
$$
%% da qui mancano dei pezzi
$$
\Rightarrow G(x,\xi)=\left\{\begin{matrix}
c_+(\xi)((1+x)^{\alpha_+}
\end{matrix}\right\}
$$
$$
G(x,\xi)=\left\{\begin{matrix}
c_+(x)((1+\xi)
\end{matrix}\right\}
$$
$$
\Rightarrow G(x,\xi)=\left\{\begin{matrix}
\lambda((1+\xi)^{\alpha_+}-2^{\sqrt{5}}(1+\xi)^{\alpha_-})((1+x)^{\alpha_+
}-(1+x)^{\alpha_-}) & x<\xi \\
\lambda((1+\xi)^{\alpha_+}-(1+\xi)^{\alpha_-})((1+x)^{\alpha_+}-2^{\sqrt{5}}(1+x)^{\alpha_-}) & x>\xi \\
\end{matrix}\right\}
$$
$\Rightarrow G|_{x=\xi+0}=G|_{x=\xi-0}$
$$
\left.\dfrac{dG}{dx}\right|_{x=\xi+0}=\lambda((1+\xi)^{\alpha_+}-(1+\xi)^{\alpha_-})(\alpha_+(1+\xi)^{\alpha_+ +1}-2^{\sqrt{5}}\alpha_-(1+\xi)^{\alpha_- -1})
$$
$$
\left.\dfrac{dG}{dx}\right|_{x=\xi-0}=\lambda((1+\xi)^{\alpha_+}-2^{\sqrt{5}}(1+\xi)^{\alpha_-})(\alpha_+(1+\xi)^{\alpha_+ +1}-\alpha_-(1+\xi)^{\alpha_- -1})
$$
$ \left.\frac{dG}{dx}\right|_{x=\xi+0}-\frac{dG}{dx}_{x=\xi-0}$

\begin{multline*}
=\lambda \left[ \alpha_+ (1+\xi)^{2\alpha_+ -1}-2^{\sqrt{5}}\alpha_-(1+\xi)^{\alpha_+ + \alpha_- -1} - \alpha_+(1+\xi)^{\alpha_-+\alpha_+-1} + 2^{\sqrt{5}}\alpha_- (1+\xi)^{2\alpha_- -1}- \right.\\
\left. \alpha_+(1+\xi)^{2\alpha_+ -1}+ \alpha_- (1+\xi)^{\alpha_+ + \alpha_- - 1} + 2^{\sqrt{5}}\alpha_+(1+\xi)^{\alpha_- + \alpha_+ -1}-2^{\sqrt{5}}\alpha_-(1+\xi)^{2\alpha_- -1} \right]=-\dfrac{1}{p(\xi)}=-\dfrac{1}{(1+\xi)^2}
\end{multline*}

$$
\alpha_+ + \alpha_- -1=-2
$$
$$
\Rightarrow \lambda\left[ -2^{\sqrt{5}}\alpha_- - \alpha_+ + \alpha_- + 2^{\sqrt{5}}\alpha_+\right]=-1
$$
$\rightarrow$ determina $\lambda$\\ %nel senso che la roba sopra lo determina
\underline{Problema ai valori al contorno piu generale:}
%sottosezione? è il problema di dirichlet piu generale
%dirichlet -> ho fissato al contorno solo la funzione
%neumann-> ho fissato al contorno la funzione e la derivata
%robin-> fisso con una combinazione lineare di u e u', in pratica vincolo posizione e velocità lungo la traiettoria

\begin{equation}
(pu')'+qu=-f,\quad \alpha<x<\beta, \quad u(\alpha)=a, u(\beta)=b (63)
\end{equation}

A tal fine: nota che $\frac{\partial G}{\partial \xi}(x\alpha)$ soddisfa
$$
\dfrac{d}{dx}\left[p(x)\dfrac{d}{dx}\left(\dfrac{\partial G}{\partial \xi}(x,\alpha)\right)\right]+q(x)\dfrac{\partial G}{\partial \xi}(x,\alpha)=0, \quad \alpha<x<\beta
$$
$$
\dfrac{\partial G}{\partial \xi}(\alpha,\alpha)=\dfrac{1}{p(\alpha)}, \quad \dfrac{\partial G}{\partial \xi}(\beta,\alpha)=0
$$
mentre
$$
\dfrac{d}{dx}\left[p(x)\dfrac{d}{dx}\left(\dfrac{\partial G}{\partial \xi}(x,\beta)\right)\right]+q(x)\dfrac{\partial G}{\partial \xi}(x,\beta)=0, \quad \alpha<x<\beta
$$
$$
\dfrac{\partial G}{\partial \xi}(\alpha,\beta)=0, \quad \dfrac{\partial G}{\partial \xi}(\beta,\beta)=-\dfrac{1}{p(\beta)}
$$
(seguono dalla definizione (59) $\rightarrow$ compito)\\
il problema (63) ha la soluzione
\begin{equation}
u(x)=\int_\alpha^\beta G(x,\xi)f(|xi)d\xi + ap(\alpha)\dfrac{\partial G}{\partial \xi}(x,\alpha)-bp(\beta)\dfrac{\partial G}{\partial \xi}(x,\beta) (64)
\end{equation}

combinazione lineare di soluzione particolare (l'integrale) e soluzione dell'omogenea(il resto).\\
Nella soluzione del problema (57) abbiamo dovuto assumere $D\neq 0$.\\
Caso $D=0$: le equazioni
$$
c_1v_1(\alpha)+c_2v_2(\alpha)=0
$$
$$
c_1v_1(\beta)+c_2 v_2(\beta)=0
$$
ammettono soluzione non banale $\Rightarrow v(x)=c_1v_1(x) + c_2 v_2(x)$ soddisfa
$$
(pv')'+qv=0, \quad \alpha<x<\beta, \quad v(\alpha)=v(\beta)=0
$$
Se u è una soluzione del problema (63), lo è anche u+cv, $\forall c$ costante.$\Rightarrow$ il problema (63) non può avere soluzione unica.
Inoltre:\\
moltiplica la (63) conn v e integra da $\alpha$ a $ \beta $:
$$
-\int_\alpha^\beta f(x)v(x)dx=\int_\alpha^\beta v(x)((pu')' + qu)dx
$$
integro due volte per parti
$$
=\left[vpu' - v'pu\right]_\alpha^\beta+\int_\alpha^\beta u\left[(pv')'+qv dx\right]=p(\alpha)v'(\alpha)a-p(\beta)v'(\beta)b
$$
dove l'integranda è nulla.
$v(pu')' \rightarrow -v'pu' \rightarrow (v'p)'u$\\
Se il problema (63) deve avere una soluzione, la funzione f e le due costanti a,b devono soddisfare
$$
p(\alpha)v'(\alpha)a-p(\beta)v'(\beta)b=-\int_\alpha^\beta v(x)f(x)dx
$$
altrimenti non ci può essere una soluzione del problema.\\
$\Rightarrow$ nel caso D=0, il problema (63) può avere nessuna soluzione o molte soluzioni, ma mai una sola soluzione.
\underline{Problemi ai valori al contorno ancora piu generali:}
\begin{equation}
\begin{matrix}
(pu')' + qu = -f(x), \quad \alpha<x<\beta\\
-\mu_1 u'(\alpha) + \sigma_1 u(\alpha)=a,\\
\mu_2 u'(\beta) + \sigma_2 u(\beta)=a
\end{matrix} (65)
\end{equation}

(condizione al contorno di Robin).\\
La funzione di Green $G(x,\xi)$ si ricava come prima se
$$
D:=\left[-\mu_1 v_1'(\alpha)+\sigma_1 v_1(\alpha)\right]\cdot \left[\mu_2 v_2'(\beta)+\sigma_2 v_2(\beta)\right] - \left[-\mu_1 v_2'(\alpha)+\sigma_1 v_2(\alpha)\right] \cdot \left[\mu_2 v_1'(\beta)+\sigma_2 v_1(\beta)\right]\neq 0
$$
(esercizio)
%nota che è in pratica un determinante
%prendo sol part dell'omogeneo con la R, aggiungo la generale del problema omogeneo (c1v1(x)+c2v2(x)). devo determinare questa ultima con c.c. diverse, anziche avere D=\= 0 ho uesta condizione sopra
$G(x,\xi)$ è la soluzione del problema
\begin{equation}
\begin{matrix}
\dfrac{d}{dx}\left(p(x)\dfrac{dG}{dx}\right) + q(x)G = 0 \quad x\neq \xi \\
-\mu_1\dfrac{dG}{dx}|_{x=\alpha}+\sigma_1 G|_{x=\alpha}=\mu_2\dfrac{dG}{dx}|_{x=\beta} + \sigma_2 G|_{x=\beta}=0 \\
G|_{x=\xi+0}=G|_{x=\xi-0}
\dfrac{dG}{dx}|_{x=\xi+0}-\dfrac{dG}{dx}|_{x=\xi-0}=-\dfrac{1}{p(\xi)}
\end{matrix} (66)
\end{equation}

(paragona con le (62), qui le condizioni al contorno su G sono più generali). G soddisfa ancora $G(x,\xi)=G(\xi,x)$\\
-Soluzione di (65):
\begin{equation}
u(x)=\int_\alpha^\beta G(x,\xi)f(\xi)d\xi + \dfrac{p(\alpha)}{\mu_1}aG(x,\alpha) + \dfrac{p(\beta)}{\mu_2}bG(x,\beta) (67)
\end{equation}

$(\mu_1,\mu_2\neq 0)$
se $\mu_1=0$ sostituisci $\dfrac{1}{\mu_1}G(x,\alpha) con \dfrac{1}{\sigma_1}\dfrac{\partial G}{\partial \xi}(x,\alpha)$\\
se $\mu_2=0$ sostituisci $\dfrac{1}{\mu_2}G(x,\beta) con \dfrac{1}{\sigma_2}\dfrac{\partial G}{\partial \xi}(x,\alpha)$\\
(Siccome $D\neq 0$ non può essere $\mu_1=\sigma_1=0$ oppure $\mu_2=\sigma_2=0$)\\
Caso D=0: il problema (65) avrà nessuna soluzione o molte soluzione ed è quindi ben bosto se e solo se $D\neq 0$. In tal caso la soluzione è data dalla (67)

%%%%%%%%%%%%%%%%%%%%%%%%%%%%%%%%%%%%%
%% inizio lezione 26/10
%%%%%%%%%%%%%%%%%%%%%%%%%%%%%%%%%%%%%%%%
NB:\\
i) In un problema ai valori al contorno, il valore di u in un dato punto dipende dai valori di f(x) nell'intero intervallo $\alpha,\beta)$. $\rightarrow$ Comportamento simile a quello delle equazionni alle derivate parziali ellittiche (vedi piu tardi).\\ %inserisci riferimento capitolo
ii) Se una soluzione particolare dell'equazione differenziale non omogenea può essere indovinata, non è necessario usare le funzioni di Green

\subsection{Applicazione del formalismo imparato}
Conduzione di calore in un intervallo con sorgente
\begin{equation}
\begin{matrix}
\partial_t T = \chi \partial_x^2 T + S(x,t), \quad x \in [0,L], \quad t\geq 0\\
T(0,t)=T(L,t)=0, \quad T(x,0)=0
\end{matrix} (68)
\end{equation}

Sviluppa $T(x,t)$ in serie di Fourier
\begin{equation}
T(x,t)=\sum_{n=1}^\infty b_n(t)\sin\dfrac{n\pi x}{L} (69)
\end{equation}

con 
\begin{equation}
b_n(t)=\dfrac{2}{L}\int_0^L T(x,t) \sin \dfrac{n\pi x}{L}dx (70), vedi (25)
\end{equation}

\begin{subequations}
\begin{equation}
S(x,t)=\sum_{n=1}^\infty s_n(t)\sin\dfrac{n\pi x}{L} (71a)
\end{equation}
\begin{equation}
s_n(t)=\dfrac{2}{L}\int_0^L S(x,t) \sin \dfrac{n\pi x}{L}dx (71b)
\end{equation}
\end{subequations}

inserisco (69),(71a) nella (68)
\begin{equation}
\Rightarrow \partial_t b_n(t)=\chi \left(-\dfrac{n^2 \pi^2}{L^2}\right)b_n(t) + s_n(t)
\end{equation}

Questa si risolve facilmente col metodo di variazione delle costanti arbitrarie. Invece con la funzione di Green: Riscrivi la (72) nella forma 
$$
(p(t)u'(t))'=f(t)
$$
con $p(t)=exp\left(\dfrac{\chi n^2 \pi^2}{L^2}t\right), \quad u'(t)=b_n(t),\quad f(t)=p(t)s_n(t)$
$$
T(x,0)=0 \Rightarrow b_n(t)=0 \Rightarrow u'(t)=0
$$
$u(t$) è definita a meno di una costante additiva $\rightarrow$ scegli $u(t)=0 \rightarrow$ problema ai valori iniziali (55). La funzione di Green $R(t,\xi)$ dalla (56):
$$
\dfrac{d}{dt}\left(p(t)\dfrac{dR}{dt}\right)=0, \quad t>\xi
$$
$$
\Rightarrow p(t)\dfrac{dR}{dt}=C(\xi) \Rightarrow R=-C(\xi)\dfrac{L^2}{\chi n^2\pi^2} \exp \left(-\dfrac{\chi n^2 \pi^2}{L^2}t\right)+\tilde{C}(\xi)
$$
$$
R|_{t=\xi}=-C(\xi)\dfrac{L^2}{\chi n^2\pi^2} \exp \left(-\dfrac{\chi n^2 \pi^2}{L^2}\xi \right)+\tilde{C}(\xi) =0
$$
$$
\tilde{C}(\xi)=C\dfrac{L^2}{\chi n^2 \pi^2}\exp \left(-\dfrac{\chi n^2 \pi^2}{L^2}\xi \right)=\gamma(\xi)\exp\left(-\dfrac{\chi n^2 \pi^2}{L^2}\xi \right)
$$
$$
\Rightarrow R(t,\xi)=\gamma(\xi)\left(\exp\left(-\dfrac{\chi n^2 \pi^2}{L^2}\xi \right) - \exp\left(-\dfrac{\chi n^2 \pi^2}{L^2}t \right)\right)
$$
$$
\left.\dfrac{dR}{dt}\right|_{t=\xi}=\gamma \dfrac{\chi n^2 \pi^2}{L^2}\exp\left(-\dfrac{\chi n^2 \pi^2}{L^2}\xi \right)=\dfrac{1}{p(\xi)}=\exp\left(-\dfrac{\chi n^2 \pi^2}{L^2}\xi \right)
$$
$$
\gamma=\dfrac{L^2}{\chi n^2 \pi^2} \implies \Rightarrow R(t,\xi)=\gamma(\xi)\left(\exp\left(-\dfrac{\chi n^2 \pi^2}{L^2}\xi \right) - \exp\left(-\dfrac{\chi n^2 \pi^2}{L^2}t \right)\right)
$$
$$
u(t)=(53)=\int_0^t \dfrac{L^2}{\chi n^2 \pi^2}\left( \exp\left(-\dfrac{\chi n^2 \pi^2}{L^2}\xi \right) - \exp\left(-\dfrac{\chi n^2 \pi^2}{L^2}t \right)\right)e^{\dfrac{\chi n^2 \pi^2}{L^2}\xi} s_n(\xi)d\xi
$$
$$
=\dfrac{L^2}{\chi n^2 \pi^2}\int_0^t\left( 1-\exp\left( -\dfrac{\chi n^2 \pi^2}{L^2}(t-\xi)\right)\right)\cdot s_n(\xi)d\xi \equiv \int_0^t H(t,\xi)d\xi
$$
$b_n(t)=u'(t)$
%inserisci appunti a mano, sono solo una digressione fatta per spiegare i due contributi di questo caso particolare
$$
\int_0^t H(t,\xi)d\xi=H(t,t) + \int_0^t \dfrac{\partial}{\partial t}H(t,\xi)d\xi=\dfrac{L^2}{\chi n^2\pi^2}\int_0^t \dfrac{\chi n^2 \pi^2}{L^2}\exp \left(-\dfrac{\chi n^2 \pi^2}{L^2}(t-\xi)\right) s_n(t)d\xi =
$$
$$
=\int_0^t \exp \left(-\dfrac{\chi n^2 \pi^2}{L^2}(t-\xi)\right) s_n(t)d\xi
$$
Nella (69):
$$
T(x,t)=\sum_{n=1}^{\infty}\int_0^t\exp \left(-\dfrac{\chi n^2 \pi^2}{L^2}(t-\xi)\right) s_n(t)d\xi \sin\dfrac{n\pi x}{L}
$$
riscrivo $s_n(t)$ usando la (71b)
$$
\Rightarrow T(x,t)=\int_0^t \int_0^L \sum_{n=1}^{\infty}\dfrac{2}{L}\exp \left(-\dfrac{\chi n^2 \pi^2}{L^2}(t-\xi)\right) \sin \dfrac{n\pi x}{L}\sin \dfrac{n\pi y}{L}S(y,\xi)dyd\xi
$$
\begin{equation}
\Rightarrow T(x,t)=\int_0^t \int_0^L G(t-\xi,x,y)S(y,\xi)dyd\xi
\end{equation}

\begin{equation}
G(t-\xi,x,y):=\dfrac{2}{L}\sum_{n=1}^{\infty}\dfrac{2}{L}\exp \left(-\dfrac{\chi n^2 \pi^2}{L^2}(t-\xi)\right) \sin \dfrac{n\pi x}{L}\sin \dfrac{n\pi y}{L}
\end{equation}

è identica alla (22).\\%dipende solo dalla differenza t-xi, perchè non è rotta l'invarianza per traslazioni temporali, ma dipende separatamente da x e  perche essendoci un bordo l'invarianza traslazionale spaziale è rotta
NB: se $T(x,0)=T_0(x)$ anziché 0, sostituisci $ b_n(0)=0$ con
$$
b_n(0)=\dfrac{2}{L}\int_0^L T_0(x)\sin\dfrac{n\pi x}{L}dx
$$
$$
b_n(t)=\int_0^t\exp\left(-\dfrac{\chi n^2 \pi^2}{L^2}(t-\xi)\right)s_n(\xi)d\xi + \underbrace{b_n(0)\exp\left(-\dfrac{\chi n^2 \pi^2}{L^2}(t)\right)}
$$
Soluzione dell'equazione omogenea $b_n'(t)=-\dfrac{\chi n^2 \pi^2}{L^2}b_n(t)$
$$
\Rightarrow T(x,t) = \sum_{n=1}^\infty \int_0^t\exp\left(-\dfrac{\chi n^2 \pi^2}{L^2}(t-\xi)\right)s_n(\xi)d\xi\sin\dfrac{n\pi x}{L}+\sum_{n=1}^\infty b_n(0)\exp\left(-\dfrac{\chi n^2 \pi^2}{L^2}t\right)\sin\dfrac{n\pi x}{L}
$$
\begin{equation}
T(x,t)=\int_0^t \int_0^t G(t-\xi,x,y)S(y,\xi)dyd\xi + \int_0^L G(t,x,y)T_0(y)dy
\end{equation}

-Abbiamo visto che la (74) coincide con la (22), ottenuta dalla (20). Questo vale in generale.: vogliamo risolvere
\begin{equation}
\begin{matrix}
\partial_t T=\chi\Delta T + S(\x,t), \quad \x \in U\\
T(\x,t)|_{\x\in U}=0
\end{matrix} (76)
\end{equation}

A tal fine: funzione di Green tale che 
\begin{equation}
(\partial_t \chi \Delta_{\x})G(t-t',\x,\y)=\delta(t-t')\delta(\x-\y) (77)
\end{equation}

\begin{equation}
\Rightarrow T(\x,t)=\int_{-\infty}^{\infty}dt'\int_{U}d^d\y G(t-t',\x,\y)S(\y,t') (78)
\end{equation}

Sviluppa G e $\delta$ secondo
\begin{equation}
\begin{matrix}
G(t-t',\x,\y)=\dfrac{1}{2\pi}\int d\omega\sum_{n,m}e^{-i\omega(t-t')}\cdot \varphi_n (\x)\varphi_m(\y)\tilde{G}_{n m}(\omega)\\
\delta(t-t')\delta(\x-\y)=\dfrac{1}{2\pi}\int d\omega e^{-i\omega(t-t')}\sum_{n,m}\varphi_n{\x}\varphi_m(\y)\delta_{nm}\end{matrix} (79)
\end{equation}

a causa del bordo l'integrale viene discretizzato.\\
$\varphi$ soddisfa $\Delta \varphi_n+\lambda_n\varphi_n=0$ in U, $\varphi(\x)|_{\x\in\partial U}=0$ formano un sistema completo in U. Motro completezza
$$
\sum_{n,m}\varphi_n(\x)\varphi_m(\y)\delta_n,m=\sum_n\varphi_n(\x)\varphi_n(\y)=\delta(\x-\y)
$$
sostituisco la (79)nella (77)
$$
\dfrac{1}{2\pi}\int d\omega \sum_{n,m}(-i\omega + \chi \lambda_n)e^{-i\omega(t-t')}\cdot \varphi_n(\x)\varphi_m(\y)\tilde{G}_{n,m}(\omega)=\dfrac{1}{2\pi}\int d\omega \sum_{n,m}(-i\omega + \chi \lambda_n)e^{-i\omega(t-t')}\cdot \varphi_n(\x)\varphi_m(\y) \delta_{n m}
$$
$$
\Rightarrow \tilde{G}_{nm}(\omega)=\dfrac{\delta_{nm}}{-i\omega + \chi \lambda_n}
$$
Nella (79): $\Rightarrow$
$$
G(\tau,\x,\y)=\dfrac{1}{2\pi}\int d\omega \sum_{n,m}e^{-i\omega \tau} \varphi_n(\x)\varphi_m(\y)\dfrac{\delta_{nm}}{-i\omega + \chi \lambda_n}
$$
Considera $\int_{-\infty}^{\infty} d\omega \dfrac{e^{-i\omega \tau}}{-i\omega +\chi\lambda_n}$
Affermazione: titti gli autovalori $\lambda_n$ sono positivi:\\
Dimostrazione
$$
0=\int_{U}\varphi_n(\Delta\varphi_n+\lambda_n\varphi_n)d^d\x=\int_{U}\left[\bar{\nabla}\cdot(\varphi_n\bar{\nabla}\varphi_n)-\bar{\nabla}\varphi_n \cdot \bar{nabla}\varphi_n + \lambda_n\varphi_n^2\right]d^d\x
$$
uso Gauss
$$
=\int_{\partial U}\n(\varphi_n\bar{\nabla}\varphi_n)d^{d-1}\x + \int_{U}\left[-(\bar{\nabla}\varphi_n)^2 +\lambda_n\varphi_n^2 \right]d^d\x
$$
il primo integrale è nullo per le condizioni al contorno, $\varphi_n|_{\partial U}=0$
$$
\lambda_n=\dfrac{\int_{U}(\bar{\nabla}\varphi_n)^2d^d\x}{\int_u\varphi_n^2 d^d\x}>0, \quad q.e.d.
$$
$\rightarrow$ polo nel semipiano inferiore:
%immagine di chiusura di percorsi
$$
e^{-i\omega\tau}=e^{-i(Re\omega + i Im\omega)\tau}=e^{-i\tau Re\omega}e^{\tau Im\omega}
$$
Abbiamo gia dimostrato che il contributo del semicerchio è nullo (vedi )%inserisci riferimento
\begin{equation}
\Rightarrow G(\tau,\x,\y)=0, \quad \tau<0 (80)
\end{equation}

$\tau=t-t', \quad \tau<0\Rightarrow t<t' \rightarrow$ contributi solo dal passato nella convoluzione

%%%%%%%%%%%%%%%%%%%%%%%%%%%%%%%%%%%%%%%%%%%%%%%%%%%%%%%%%%%%%%
%% inizio lezione 2/11
%%%%%%%%%%%%%%%%%%%%%%%%%%%%%%%%%%%

$$
\tau >0 : \int_{-\infty}^{\infty} d\omega \dfrac{e^{-i\omega t}}{-i(\omega +i\chi\lambda_n)}=-2\pi i \cdot i e^{-i(-i\chi \lambda_n)\tau} = 2\pi e^{-\chi \lambda_n \tau}
$$
\begin{equation}
\Rightarrow G(\tau,\x,\y)=\sum_{n} e^{-\chi\lambda_n \tau} \varphi_n (\x)\varphi_n (\y) (81) v(20)
\end{equation}

$\Rightarrow$ L'equazione del calore $(\partial_t - \chi \Delta)T=S$ in U ha la soluzione

\begin{equation}
T(\x,t)=\int_0^t dt' \int_U d^d\y G(t-t',\x,\y)S(\y,t') + \int_U d^d\y G(t,\x,\y)T_0(\y) (82)
\end{equation}

Soluzione particolare dell'equazione non omogenea (v.(78)), tenendo conto della (80) e di $S(\y,t')=0$ per $t'<0$ sommata a soluzone dell'equazione omogenea in modo tale che $T(\x,0)=T_0(\x)$

-Altro esempio: equazione di laplace (Poisson) nella palla di raggio R\\
\begin{equation}
\begin{matrix}
\Delta \Phi=-F(r,\theta,\varphi) & r<R\\
\Phi(R,\theta,\varphi=0
\end{matrix} 
\end{equation}
(Elettrostatica: $-F=4\pi\rho$)
$$
\Delta = \partial_{r}^2 + \dfrac{2}{r}\partial_r + \dfrac{1}{r^2\sin\theta}\partial_\theta(\sin \theta \partial_\theta) + \dfrac{1}{r^2\sin^2\theta}\partial^2_{\varphi}
$$
Sviluppo di $\Phi,F$ in armoniche sferiche (reali, non complesse):
$$
\Phi(r,\theta,\varphi)=\sum_{l=0}^{\infty}\left(\dfrac{1}{2}a_{l0}(r)P^0_l(\cos\theta) + \sum_{m=1}^l\left(a_{lm}(r)P_l^m(\cos\theta)\cos m\varphi + b_{lm}(r)P_l^m (\cos\theta) \sin m\varphi\right)\right)
$$
\begin{equation}
F(r,\theta\varphi)=\sum_{l=0}^\infty\left(\dfrac{1}{2} A_{l0}(r)P_l^0(\cos\theta) + \sum_{m=1}^l \left(A_{lm}(r)P_l^m(\cos\theta)\cos m\varphi +B_{lm}(r)P_l^m(\cos\theta)\sin m\varphi\right)\right)
\end{equation}
Sostituisco nella (83)
\begin{equation*}
\sum_{l=0}^\infty \left(\dfrac{1}{2}a''_{l0}(r)P_l^0(\cos\theta) + \dfrac{2}{r}\dfrac{1}{2}a'_{l0}(r)P_l^0(\cos\theta) + \dfrac{1}{r^2}(-l)(l+1)\dfrac{a_{l0}(r)}{2}P_l^0(\cos\theta)\right) +\sum_{l=1}^\infty \sum_{m>0}\left( a''_{lm}(r)P_l^m(\cos\theta)\cos m\varphi + b''_{lm}(r)P_l^m(\cos\theta)\sin m\varphi + \dfrac{2}{r} a'_{lm}(r) P_l^m (\cos\theta)\cos m\varphi + \dfrac{2}{r}b'_{lm}(r)P_l^m(\cos\theta) \sin m\varphi + \dfrac{1}{r^2}a_{lm}(r)(-l)(l+1)P_l^m(\cos\theta)\cos m\varphi + \dfrac{1}{r^2} b_{lm}(r)(-l)(l+1)P_l^m(\cos\theta)\sin m\varphi\right)
\end{equation*}
$$
=-\sum_l \dfrac{1}{2}A_{l0}(r)P_l^0(\cos\theta) - \sum_{l,m>0}\left(A_{lm}(r) P_l^m(\cos\theta)\cos m \varphi + B_{lm}(r)P_l^m(\cos\theta)\sin m\varphi\right)
$$
(usato: $\frac{1}{\sin\theta}\partial_\theta(\sin \theta\partial_\theta(P_l^m(\cos\theta)cos m \varphi)+\frac{1}{\sin^2\theta} \partial^2_\varphi(P_l^m(\cos\theta)\cos m\varphi)=-l(l+1)P_l^m(\cos\theta)\cos m\varphi$. Sfrutto il fatto che le armoniche sferiche $P_l^m(\cos\theta)\cos m\varphi$ e $P_l^m(\cos\theta)\sin m\varphi$ sono linearmente indipendenti
\begin{equation}
\begin{matrix}
\implies a''_{lm}(r) +\dfrac{2}{r}a'_{lm}(r) - \dfrac{l(l+1)}{r^2}a_{lm}(r)=-A_{lm}(r)\\
 b''_{lm}(r) +\dfrac{2}{r}b'_{lm}(r) - \dfrac{l(l+1)}{r^2}b_{lm}(r)=-B_{lm}(r)\\
a_{lm}(R)=b_{lm}(R)=0
\end{matrix}
\end{equation}
Entrambe le equazioni possono essere riscritte nella forma $(pu')' + qu=-f$, con $p=r^2$, $q=-l(l+1)$, $f=A_{lm}(r)r^2$ oppure $B_{lm}(r)r^2$, $u=a_{lm}(r)$ oppure $b_{lm}(r)$. 
Funzione di Green $G(r,\xi)$. Soddisfa l'equazione omogenea $\frac{d}{dr}\left(r^2\frac{dG}{dr}\right)-l(l+1)G=0$. Prova $G(r,\xi)=C(\xi)r^\alpha$, $r\neq \xi$
$$
r^2\dfrac{dG}{dr}=C\alpha r^{\alpha+1}
$$
$$
\implies C\alpha(\alpha+1)r^\alpha - l(l+1)Cr^\alpha=0, \quad \alpha=l \vee \alpha=-l-1
$$
$$
G(r,\xi)=\left\{
\begin{matrix}
c_1(\xi)r^l + c_2(\xi)r^{-l-1} & \xi\leq r\\
\tilde{c}_1(\xi)r^l + \tilde{c}_2(\xi)r^{-l-1} & \xi \geq r
\end{matrix}\right.
$$
N.B. l'estremo r=0 dell'intervallo [0,R] è un punto singolare: $p(r)=r^2$ si annulla in r=0, e la soluzione fondamentale $\sim r^{-l-1}$ diverge.
Al posto di una condizione al contorno inr=0 richiediamo che u (e G) rimangano limitate $\implies \tilde{c}_2=0$
$$
G|_{r=\xi+0}=G|_{r=\xi-0} \implies c_1\xi^l + c_2\xi^{-l-1}=\tilde{c}_1\xi^l
$$
$$
\tilde{c}_10c_1 + c_2 \xi^{-2l-1}
$$
$$
G|_{r=R}=0 \Rightarrow c_1R^l+c_2R^{-l-1}=0 \implies c_2=-c_1R^{2l + 1} \implies \tilde{c}_1=c_1(1-R^{2l+1}\xi^{-2l-1})
$$
$$
\Rightarrow G(r,\xi)=\left\{\begin{matrix}
c_1r^l(1-(\dfrac{R}{r})^{2l+1} & \xi\leq r \\
c_1r^l(1-(\dfrac{R}{\xi})^{2l+1} & \xi\geq r 
\end{matrix}\right.
$$
$$
\dfrac{dG}{dr}|_{r=\xi+0}-\dfrac{dG}{dr}|_{r=\xi-0}=\dfrac{1}{p(\xi)}
$$
$$
c_1=\dfrac{\xi^l}{(2l+1)R^{2l+1}}
$$

\begin{equation}
G_l(r,\xi)=\left\{
\begin{matrix}
\dfrac{1}{(2l+1)R}(\dfrac{\xi}{R})^l[\dfrac{r}{R})^{-l-1} - (\dfrac{r}{R})^l] & \xi\leq r\\
appunti
\end{matrix}\right.
\end{equation}

Inversa della (84):
\begin{equation}
\begin{matrix}
A_{lm}(r)=\dfrac{(2l+1)(l-m)!}{2\pi(l+m)!}\int_{-\pi}^\pi \int_0^\pi F(r,\theta,\varphi)P_l^m(\cos\theta) \cos m\varphi \sin \theta d\theta d\varphi\\

B_{lm}(r)=\dfrac{(2l+1)(l-m)!}{2\pi(l+m)!}\int_{-\pi}^\pi \int_0^\pi F(r,\theta,\varphi)P_l^m(\cos\theta) \cos m\varphi \sin \theta d\theta d\varphi
\end{matrix}(88)
\end{equation}

(usa ortogonalità delle armoniche sferiche).\\
Con la (87) e (88), la (84) diventa
\begin{equation}
\Phi(r,\theta,\varphi) =\int_{-\pi}^\pi \int_0^\pi \int_0^R G(r,\theta,\varphi,r',\theta',\varphi')F(r',\theta',\varphi')r'^2\sin\theta'dr'd\theta'd\varphi' (89)
\end{equation}
dove, per r'<r

\begin{equation*}
G(r,\theta,\varphi,r',\theta',\varphi')=\sum_{l,m>0}\dfrac{(2l+1)(l-m)!}{2\pi(l+m)!}\dfrac{1}{(2l+1)R}(\dfrac{r'}{R})^l[(\dfrac{r}{R})^{-l-1}-(\dfrac{r}{R})^l] P_L^m(\cos\theta)P_l^m(\cos\theta')(\cos m\varphi \cos m\varphi' \sin m\varphi \sin m\varphi') + \sum_l \dfrac{1}{2}\dfrac{2l+1}{2\pi}\dfrac{1}{(2l+1)R}(\dfrac{r'}{R})^l[(\dfrac{r}{R})^{-l-1} - (\dfrac{r}{R})^l]P^0_l (\cos \theta)P_l^0(\cos\theta')
\end{equation*}

(per r'>r : scambia r con r')

\begin{equation*}
\sum_{m=1}^l \dfrac{(2l+1)(l-m)!}{2\pi(l+m)!}P_l^m(\cos\theta)P_l^m(\cos\theta')\underbrace{(\cos m\varphi \cos m\varphi' + \sin m\varphi + \sin m\varphi')} + \dfrac{1}{2}\dfrac{2l+1}{2\pi}P_l^0(\cos\theta)P_l^0(\cos\theta')
\end{equation*}
$$
\dfrac{1}{2}\left(e^{im(\varphi - \varphi')} + e^{-im(\varphi - \varphi')}\right)
$$
\begin{equation*}
Y_l^m(\theta,\varphi)=(-1)^m\sqrt{\dfrac{(2l+1)(l-m)!}{4\pi(l+m)!}}P_l^m(\cos\theta)e^{im\varphi}
\end{equation*}
Riscrivo passando alle armoniche complesse
\begin{equation*}
=\sum_{m=1}^l \left(Y_l^m(\theta,\varphi)Y_l^{m*}(\theta',\varphi') + Y_l^{m*}(\theta,\varphi)Y_l^m(\theta',\varphi')  \right) + Y_l^0(\theta,\varphi)Y_l^{0*}(\theta',\varphi')
\end{equation*}
$$
Y_l^{m*}(\theta,\varphi) = (-1)^m Y_l^{-m}(\theta,\varphi)
$$
$$
Y_l^{m}(\theta',\varphi') = (-1)^m Y_l^{-m*}(\theta',\varphi')
$$
\begin{equation*}
=\sum_{m=-l}^l Y_l^m (\theta,\varphi)Y_l^{m*}(\theta',\varphi')=\dfrac{2l+1}{4\pi}P_l(\cos\gamma)
\end{equation*}

con $\cos\gamma=\cos\theta\cos\theta' + \sin\theta \sin\theta' \cos(\varphi-\varphi')$. Quindi
$$
G(r,\theta,\varphi;r',\theta',\varphi')=\sum_{l=0}^{\infty}\dfrac{1}{(2l+1)R}(\dfrac{r'}{R})^l [(\dfrac{r}{R})^{-l-1} - (\dfrac{r}{R})^l]\dfrac{2l+1}{4\pi}P_l(\cos\gamma)
$$

%%%%%%%%%%%%%%%%%%%%%%%%%%%%%%%%%%%%%%%%%%%%%%%%%%%%%%
%% inizio lezione 8/11
%%%%%%%%%%%%%%%%%%%%%%%%%%%%%%

-funzione generatrice dei polinomi di Legendre:
\begin{equation}
\dfrac{1}{\sqrt{1-2xt + t^2}} = \sum_{n=0}^{\infty} P_n(x)t^n (90)
\end{equation}

\begin{equation*}
G(r,\theta,\varphi;r',\theta',\varphi')=\dfrac{1}{4\pi r} \sum_{l=0}^\infty P_l(\cos \gamma)\left(\dfrac{r'}{r}\right)^l - \dfrac{1}{4\pi R}\sum_{l=0}^\infty P_l(\cos\gamma)\left(\dfrac{rr'}{R^2}\right)^l
\end{equation*}

\begin{equation*}
(90)=\dfrac{1}{4\pi r}\dfrac{1}{\sqrt{1-2\dfrac{r'}{r}\cos\gamma + \dfrac{r'^2}{r^2}}} - \dfrac{1}{4\pi R}\dfrac{1}{\sqrt{1-2\dfrac{rr'}{R^2}\cos\gamma + \dfrac{r^2r'^2}{R^4}}}
\end{equation*}
\begin{equation}
=\dfrac{1}{4\pi}(r^2 + r'^2 - 2rr'\cos\gamma)^{-\dfrac{1}{2}} - \dfrac{1}{4\pi}(R^2 + \dfrac{r^2r'^2}{R^2} - 2rr'\cos\gamma)^{-\dfrac{1}{2}} (91)
\end{equation}
è già simmetrica in $r,r' \rightarrow$ stesso risultato per $r' >r$.\\
in coordinate cartesiane:
\begin{equation}
G(x,y,z,x',y',z')=\dfrac{1}{4\pi}\left\{(x-x')^2 + (y-y')^2 + (z-z')^2\right\}^{-\dfrac{1}{2}} - \dfrac{1}{4\pi}\dfrac{R}{r'}\left\{(x-x'\dfrac{R^2}{r'^2})^2 + (y-y'\dfrac{R^2}{r'^2})^2 + (z-z'\dfrac{R^2}{r'^2})^2\right\}^{-\dfrac{1}{2}} (91')
\end{equation}
usando
\begin{equation*}
R^2  + \dfrac{r^2 r'^2}{R^2} - 2rr'\cos\gamma = \dfrac{r'^2}{R^2}\left(r^2 + \dfrac{R^4}{r'^2} - \dfrac{2 r R^2}{r'}\cos\gamma \right)=\dfrac{r'^2}{R^2}(\bar{r}-\bar{\rho})^2
\end{equation*}

$$
\bar{\rho} := \dfrac{R^2}{r'^2}\bar{r}'
$$
1 termine della (91'): carica puntiforme in $\bar{r} '$.\\
2 termine : carica immagine in $\bar{\rho} =\dfrac{R^2}{r'^2}\bar{r}'$ (che è fuori dalla palla)

%%%%%%  Il capitolo sulle cariche immagine è stato fatto qui, mistero sul perchè lo abbia messo in nuovo capitolo ricominciando la numerazione

\section{Il kernel di Schroedinger}

equazione di Schroedinger

$$
i\hbar \partial_t \psi = \hat{H}\psi =(p. libera)= -\dfrac{\hbar^2}{2m}\Delta \psi 
$$
\begin{equation}
\Rightarrow -i\partial_t \psi =\dfrac{\hbar}{2m}=\chi \Delta \psi (92)
\end{equation}
risulta dall'equazione del calore $\partial_t \psi =\chi \Delta \psi$ tramite la rotazione di Wick $t \rightarrow it$ $(\Rightarrow \partial _t \rightarrow -i\partial_t)$\\
Supponiamo di essere in d dimensioni spaziali. Trasformata di fourier
$$
\psi(\x,t)=\dfrac{1}{(2\pi)^d}\int d^d\kk e^{i\kk \x}\tilde{\psi}(\kk,t)
$$
$$
(92)\Rightarrow -i\partial_t \tilde{\psi}(\kk,t)=\chi (-\kk ^2)\tilde{\psi}(\kk,t)
$$
$$
\Rightarrow \tilde{\psi}(\kk,t) = e^{-i\chi\kk t^2}\tilde{\psi}_0(\kk)
$$
$$
\tilde{\psi}_0(\kk) = \int d^d \y e^{-i\kk \y} \psi_0(\y)
$$
$$
\Rightarrow \psi(\x,t) = \dfrac{1}{(2\pi)^d}\int d^d \y \psi_0(\y) \cdot \int d^d\kk e^{i\kk(\x-\y) - i\chi \kk^2 t}
$$
definisco $\z :=\x - \y$, \quad $(2\pi)^d G(\z,t) :=\int d^d\kk e^{i\kk(\x-\y) - i\chi \kk^2 t}$
$$
i\kk \z - i \chi \kk^2 t = -i\chi t\left(\kk - \dfrac{\z}{2\chi t}\right)^2 + \dfrac{i\z^2}{4\chi t}
$$
$Re(i\chi t)=0$ $\Rightarrow$ regolarizza $t\rightarrow t-i\epsilon$, $\epsilon>0$ $\implies Re(i\chi(t-i\epsilon))=\chi\epsilon >0 \quad \checkmark$
$$
\Rightarrow G(\z,t)=\dfrac{1}{(2\pi)^d}\exp \left(\dfrac{i\z}{4\chi(t-i\epsilon)}\right)\cdot \int d^d\kk e^{-i\chi(t-i\epsilon)\left( \kk -\dfrac{\z}{2\chi(t-i\epsilon)}\right)^2}
$$
$$
=\dfrac{1}{(2\pi)^d}\exp\left(\dfrac{i\z^2}{4\chi(t-i\epsilon)}\right)\sqrt[d]{\dfrac{\pi}{i\chi(t-i\epsilon)}}
$$
\begin{equation}
\rightarrow (\epsilon \to 0) \left(\dfrac{m}{2\pi i \hbar t}\right)^{\dfrac{d}{2}}\exp \left(-\dfrac{m\z^2}{2i\hbar t}  \right) (93)
\end{equation}
"Schroedinger Kernel" esiste per tutti i $t\in \R$. (l'equazione di Schroedinger è reversibile, quella del calore no)
$$
-i\partial_t \psi =\chi \Delta \psi
$$
$$
(t\to -t)\Rightarrow i\partial_t\psi=\chi\Delta \psi
$$
$$
(\psi \to \psi^*)\Rightarrow i\partial_t\psi^*=\chi\Delta \psi^*
$$
$$
(c.contorno)\Rightarrow -i\partial_t\psi=\chi\Delta\psi
$$
Commenti:
i) $G(\x-\y,t)$ soddisfa l'equazione di Schroedinger $-\partial_t \psi = \chi \Delta_{\x}G$\\
ii) $\lim_{t\to 0} G(\z,t)=\delta(\z)$\\
Più in generale (non necessariamente particella libera):\\
cerchiamo $G^R(\x,\y,t) t.c. $
\begin{equation}
\psi(\x,t)=i\int d^d\y \psi_0(\y) G^R(\x,\y,t) (94)
\end{equation}
$G^R(\x,\y,t)$: funzione di Green ritardata, $G^R(\x,\y,t)$=0 per t<0. Rappresenta l'ampiezza di probabilità che la particella cada dal punto $(\y,t=0)$ al punto $(\x,t)$.\\
$\varphi_n$ autofunzioni di $\hat{H}$
\begin{equation}
\hat{H}\varphi_n=E_n \varphi_n (95)
\end{equation}

\begin{equation}
\Rightarrow G^R(\x,\y,t)=-i\sum_n e^{-iE_nt/\hbar}\varphi_n(\x)\varphi_n^* (\y), \quad t\geq 0(96)
\end{equation}
vedi le (20) e (81)
$$
(94)\Rightarrow \psi(\x,t)=\int d^d\y \sum_n e^{-iE_nt/\hbar}\varphi_n(\x)\varphi_n^* (\y)\psi_0(\y)
$$
sooddisfa l'equazione di Schroedinger:
$$
i\hbar\partial_t\psi(\x,t)=\int d^d\y \sum_n i\hbar \dfrac{-iE_nt}{\hbar} e^{-iE_nt/\hbar}\varphi_n(\x)\varphi_n^* (\y)\psi_0(\y)
$$
$$
\hat{H}_{\x}\psi(\x,t)=\int d^d\y \sum_n e^{-iE_nt/\hbar}\underbrace{\hat{H}_{\x}\varphi_n(\x)}\varphi_n^* (\y)\psi_0(\y)
$$
$$
E_n\varphi_n(\x)
$$
Inoltre: $\psi(\x,0)=\int d^d\y \underbrace{\sum_n \varphi_n(\x)\varphi_n^* (\y)}\psi_0(\y)=\psi_0(\x) \qquad \checkmark$
$$
\delta(\x-\y)
$$
Trasformata di Fourier:
$$
\tilde{G}^R(\x,\y,E)=\int_{-\infty}^{\infty}dt e^{iEt/\hbar}G^R(\x,\y,t)= \int_0^\infty dt e^{iEt/\hbar}G^R(\x,\y,t)=(96)=-i\sum_n\int_0^\infty dt e^{i(E-E_n)t/\hbar}\varphi_n(\x)\varphi_n^*(\y)
$$
l'ultimo integrale non è ben definito se $E\in \R \rightarrow$ regolarizza tramite $E\to E+i\epsilon,\quad \epsilon >0$
\begin{equation}
\Rightarrow \tilde{G}^R(\x,\y,E)=\sum_n\dfrac{\varphi_n(\x)\varphi_n^*(\y)}{E+i\epsilon - E_n}(97)
\end{equation}

%%%%%%%%%%%%%%%%%%%%%%%%%%%%%%%%%%%%
%% inizio lezione 9/11
%%%%%%%%%%%%%%%%%%%%%%%%%%%%%%%%%%%%

Abbiamo 
\begin{equation}
(E + i\epsilon - \hat{H}_{\x})\tilde{G}(\x,\y,E)=\sum_n \dfrac{(E + i\varepsilon - \hat{H}_{\x}) \varphi_n(\x)\varphi_n^*(\y)}{E + i\varepsilon - E_n} = \sum_n \varphi_n(\x)\varphi_n^*(\y)=\delta(\x-\y)
\end{equation}
Funzione di Green \underline{avanzata}: 
$$
G^A(\x,\y,t)
$$
$G^A(\x,\y,t)=0$ per t>0. Affinché la sua trasformata di fourier
$$
\tilde{G}^A(\x,\y,E)=\int_{-\infty}^0 dt e^{iEt/\hbar}G^A(\x,\y,t)
$$
esista, dobbiamo sostituire $E\to E-i\varepsilon, \varepsilon >0$
$$
\tilde{G}^A(\x,\y,E)=\sum_n \dfrac{\varphi_n(\x)\varphi_n^*(\y)}{E - i\varepsilon - E_n}
$$
Esercizio: calcolare G per l'oscillatore armonico.
$$
\hat{H}\varphi_n = -\dfrac{\hbar^2}{2m}\varphi''(x) + \dfrac{1}{2}m\omega^2 x^2 \varphi_n(x)=E_n\varphi_n(x)
$$
$$
E_n=\hbar\omega\left(n+\dfrac{1}{2}\right);\quad n=0,1,2,\dots 
$$
$$
\varphi_n(x)=(\sqrt{\pi}2^n n!)^{-\dfrac{1}{2}}\exp\left(-\dfrac{m\omega x^2}{2\hbar}\right)\cdot\underbrace{H_n\left(\left(\dfrac{m\omega}{\hbar}\right)^{\dfrac{1}{2}}x\right)}
$$
polinomi di Hermite
$$
(96)\Rightarrow -i\sum_{n=0}^\infty e^{-iE_n t/\hbar}\varphi_n(x)\varphi_n^*(y)
$$
$$
=-i\sum_{n=0}^\infty e^{-i\hbar\omega(n+1/2)} \dfrac{1}{\sqrt{\pi} 2^n n!} \exp\left(-\dfrac{m\omega}{2\hbar}(x^2+y^2)\right) H_n\left(\left(\dfrac{m\omega}{\hbar}\right)^{\dfrac{1}{2}}x\right)H_n\left(\left(\dfrac{m\omega}{\hbar}\right)^{\dfrac{1}{2}}y\right)
$$
Usa 
\begin{equation}
\sum_{n=0}^\infty \dfrac{H_n (x)H_n (y)}{n!} \left(\dfrac{u}{2}\right)^n=\dfrac{1}{\sqrt{1-u^2}}\exp\{\dfrac{2u}{1+u}xy - \dfrac{u^2}{1-u^2}(x-y)^2\}(99)
\end{equation}
$$
(u=e^{-i\omega t}) \Rightarrow -i\sum_{n=0}^\infty e^{-iE_n t/\hbar} \varphi_n(x)\varphi_n^*(y) = 
$$
$$
-i e^{-i\omega t/2}\pi^{-\dfrac{1}{2}}\exp\left\{-\dfrac{m\omega}{2\hbar}(x^2+y^2)\right\}\dfrac{1}{\sqrt{1-e^{-2i\omega t}}}\exp\left\{\dfrac{2e^{-i\omega t}}{1+e^{-i\omega t}} \dfrac{m\omega}{\hbar}xy - \dfrac{e^{-2i\omega t}}{1-e^{-2i\omega t}}\dfrac{m\omega}{\hbar}(x-y)^2\right\}
$$
\begin{equation}
=-\sqrt{\dfrac{i}{2\pi\sin \omega t}}\exp\left\{ \dfrac{i\omega t}{\hbar}\left[\dfrac{x^2+y^2}{2}\cot \omega t - xy \mathrm{cosec} \omega t\right]\right\}(100)
\end{equation}
"Mehler Kernel"

\section{L'equazione dei telegrafisti} %ha detto lui stesso che è un nuovo paragrafo
equazioni di Maxwell in materia, sistema CGS:
\begin{subequations}
\begin{equation}
\nabla \cdot \bar{D} = 4\pi \rho_{macr.}(101a)
\end{equation}
\begin{equation}
\nabla \times \bar{H}=\dfrac{4\pi}{c}\bar{j}_{macr.} + \dfrac{1}{c}\dfrac{\partial \bar{D}}{\partial t}(101b)
\end{equation}
\begin{equation}
\nabla \cdot\bar{B}=0 (101c)
\end{equation}
\begin{equation}
\nabla \times\bar{E}=-\dfrac{1}{c}\dfrac{\partial \bar{B}}{\partial t}(101d)
\end{equation}
\end{subequations}
Legge di materiale (caso più semplice)
\begin{equation}
\begin{matrix}
\bar{D} = \varepsilon \bar{E}
\bar{B} = \mu \bar{H}
\end{matrix} (102)
\end{equation}
legge di Ohm, $\sigma$ rappresenta la conducibilità elettrica
\begin{equation}
\bar{j}_{macr} = \sigma \bar{E}(103)
\end{equation}
$$
(101b)\Rightarrow \nabla \times \nabla \times \bar{H}  \dfrac{4\pi}{c}\nabla \times \bar{j}_macr + \dfrac{i}{c}\dfrac{\partial}{\partial t}\nabla \times \bar{D}
$$
=0
$$
(103),(101d),(101c)\Rightarrow\overbrace{grad \nabla \cdot \bar{H}} - \Delta \bar{H} = \dfrac{4\pi\sigma}{c}\underbrace{\nabla \times\bar{E}} - \dfrac{\varepsilon}{c^2}\dfrac{\partial^2}{\partial t^2} \bar{B}
$$
$-\dfrac{1}{c}\dfrac{\partial }{\partial t}\bar{B}$
\begin{subequations}
\begin{equation}
\Rightarrow \Delta \bar{B}=\dfrac{4\pi\sigma \mu}{c^2}\dfrac{\partial \bar{B}}{\partial t} + \dfrac{\epsilon \mu}{c^2}\dfrac{\partial^2 \bar{B}}{\partial t?2}(104a)
\end{equation}
\text{equazione dei telegrafisti}
$$
(101d)\Rightarrow \nabla \times \nabla \times \bar{E} = -\dfrac{1}{c}\dfrac{\partial}{\partial t}\nabla \times\bar{B}=-\dfrac{\mu}{c}\dfrac{\partial}{\partial t}\nabla \times \bar{H}
$$
$$
(101b)=-\dfrac{\mu}{c}\dfrac{\partial}{\partial t}\left(\dfrac{4\pi}{c}\sigma \bar{E} + \dfrac{\epsilon}{c}\dfrac{\partial}{\partial t}\bar{E}\right)=grad\underbrace{\nabla \cdot \bar{E}} - \Delta \bar{E}
$$
\text{=0 se $j_{macr}=0$}

\begin{equation}
\Rightarrow \Delta \bar{E}=\dfrac{4\pi\sigma\mu}{c^2}\dfrac{\partial \bar{E}}{\partial t} + \dfrac{\varepsilon \mu}{c^2}\dfrac{\partial^2 \bar{E}}{\partial t^2}(104b)
\end{equation}
\end{subequations}
è equivalente all'equazione delle onde, con l'aggiunta del termine dissipativo.\\
Casi limite: $\sigma=0$ equazione delle onde. $\epsilon=0$: equazione del calore con $\chi=\dfrac{c^2}{4\pi \sigma \mu}$

Esercizio: risolvere l'equazione dei telegrafisti $\Delta u = a\partial_t u + b \partial^2_t u$ (a,b costanti positive) nell'intervallo $0\leq x\leq L$, $t\geq 0$, con $u(0,t)=u(L,t)=0$, $u(x,0)=0$, $\dfrac{\partial u}{\partial t}(x,0) = g(x)$
Soluzione: serie di Fourier
$$
u(x,t)=\sum_{n=1}^\infty u_n(t) \sin \dfrac{n\pi x}{L}
$$
con 
$$
u_n(t)=\dfrac{2}{L}\int_0^Lu(x,t)\sin \dfrac{n\pi x}{L}dx \Rightarrow -\dfrac{n^2\pi^2}{L^2}u_n(t)=au_n'(t) + b u_n''(t)
$$

ansatz $u_n(t)=\alpha e^{\lambda_n t}$
$$
\Rightarrow -\dfrac{n^2\pi^2}{L^2}=a\lambda_n + b \lambda_n^2
$$
$$
\Rightarrow \lambda_n = -\dfrac{a}{2b} \pm \sqrt{\dfrac{a^2}{4b^2}-\dfrac{n^2\pi^2}{L^2 b}}\equiv \lambda_n^\pm
$$
$\rightarrow$ comportamento oscillatorio se $\dfrac{n^2\pi^2}{L^2}>\dfrac{a^2}{4b}$
$$
u_n(t)=\alpha_n^+ e^{\lambda_n^+ t} + \alpha_n^-e^{\lambda_n^- t}
$$
$$
u(x,0)=0 \Rightarrow u_n(0)=0 \Rightarrow \alpha_n^- = \alpha_n^+
$$
$$
\Rightarrow u(x,t)=\sum_{n=1}^\infty\alpha_n^+(e^{\lambda_n^+ t}-e^{\lambda_n^- t})\sin\dfrac{n\pi x}{L}
$$
$$
\dfrac{\partial u(x,t)}{\partial t} =\sum_{n=1}^\infty \left(\lambda_n^+ e^{\lambda_n^+ t}-\lambda_n^-e^{\lambda_n^- t}  \right)\sin \dfrac{n\pi x}{L}
$$
$$
\left.\dfrac{\partial u(x,t)}{\partial t}\right|_{t=0}= \sum_{n=1}^\infty \alpha_n^+(\lambda_n^+ - \lambda_n^-)\sin\dfrac{n\pi x}{L} =: g(x) = \sum_{n=1}^\infty g_n \sin \dfrac{n\pi x}{L}
$$
$$
\alpha_n^+ = \dfrac{g_n}{\lambda_n^+ - \lambda_n^-}=\dfrac{g_n}{2\sqrt{\dfrac{a^2}{4b^2}-\dfrac{n^2\pi^2}{L^2 b}}}=\dfrac{1}{2\sqrt{\dfrac{a^2}{4b^2}-\dfrac{n^2\pi^2}{L^2 b}}} \dfrac{2}{L}\int_{0}
^{L}g(x')\sin\dfrac{n\pi x'}{L} dx'
$$

e quindi

$$
u(x,t)=\dfrac{1}{L}\int_0^L dx' \sum_{n=1}^\infty \dfrac{e^{\lambda_n^+ t} - e^{\lambda_n^- t}}{\sqrt{\dfrac{a^2}{4b^2}-\dfrac{n^2\pi^2}{L^2 b}}}\sin\dfrac{n\pi x'}{L}\sin \dfrac{n\pi x}{L}g(x')=\int_0^L G(x,x',t)g(x')dx'
$$

con la funzione di Green
$$
G(x,x',t)=\dfrac{1}{L} \sum_{n=1}^\infty \dfrac{e^{\lambda_n^+ t} - e^{\lambda_n^- t}}{\sqrt{\dfrac{a^2}{4b^2}-\dfrac{n^2\pi^2}{L^2 b}}}\sin\dfrac{n\pi x'}{L}\sin \dfrac{n\pi x}{L}
$$
N.B. se esiste $n_0$ tale che
$$
\lambda_{n_0}^+ = \lambda_{n_0}^-
$$
cioè 
$$
\sqrt{\dfrac{a^2}{4b^2}-\dfrac{n^2\pi^2}{L^2 b}}=0
$$
scrivi
$$
\sqrt{\dfrac{a^2}{4b^2}-\dfrac{n^2\pi^2}{L^2 b}}= \dfrac{e^{\left(-\dfrac{a}{2b}+\sqrt{\cdots}\right)t}- e^{\left(-\dfrac{a}{2b}-\sqrt{\cdots}\right)t}}{\sqrt{\cdots}}
$$
$$
\dfrac{e^{-\dfrac{at}{2b}} 2\sinh (\sqrt{\cdots} t)}{\sqrt{\cdots}}\xrightarrow{\sqrt{\cdots}\to 0} \dfrac{e^{-\dfrac{at}{2b}}2\sqrt{\cdots}t}{\sqrt{\cdots}}=2te^{-\dfrac{at}{2b}}
$$

\chapter{Classificazione delle equazioni differenziali alle derivate parziali lineari del 2 ordine}
per semplicità ci limitiamo alle equazioni in 2 variabili, 
\begin{equation}
a_{11}(x,y)u_{xx} + 2a_{12}(x,y)u_{xy} a_{22}(x,y)u_{yy} + b_{1}(x,y)u_{x} + b_2(x,y)u_y + c(x,y)u = d(x,y) (105)
\end{equation}
La classificazione è basata sulla sola parte contenente le derivate seconde, $a_{11}u_{xx} + 2a_{12}(x,y)u_{xy} a_{22}(x,y)u_{yy}$, detta \underline{parte principale} dell'equazione.\\
\underline{Definizione}: Sia $\delta(x,y)=a_{12}^2 - a_{11}a_{22}$, il \underline{discriminante} della parte principale\\
Se $\delta >0$ l'equazione (105) si dice \underline{iperbolica}\\
Se $\delta =0$ l'equazione (105) si dice \underline{parabolica}\\
Se $\delta <0$ l'equazione (105) si dice \underline{ellittica}\\
\underline{N.B.} la definizione si estende anche alle equazioni \underline{quasilinearli} (nelle quali i coefficienti $a_{ij}$ possono dipendere anche da $u, u_x,u_y$) se si riesce a stabilire il segno di $\delta$.\\
Esempio: \underline{superfici minime}\\
Considera superficie in $\R^3$ definita da $z=u(x,y)$, $x,y \in \Omega \subset \R^2$
%immagine superfici minime
$$
\R^3 \ni \x = \left( \begin{matrix}
x\\
y\\
u(x,y)
\end{matrix}\right)
$$
vettori tangenti:
$$
\partial_x \x = \left( \begin{matrix}
1\\
0\\
u_x
\end{matrix}\right) ,\quad 
\partial_y \y = \left( \begin{matrix}
0\\
1\\
u_y
\end{matrix}\right)
$$
vettore normale
$$
\bar{N}=\partial_x \x \times \partial_y\y=\left| \begin{matrix}
e_x & e_y & e_z \\
1 & 0 & u_x \\
0 & 1 & u_y
\end{matrix}\right| = e_z-u_xe_x - u_y e_y = \left( \begin{matrix}
-u_x \\
-u_y \\
1
\end{matrix}\right)
$$

Elemento di superficie:
$$
dA = |\bar{N}|dxdy=\sqrt{u_x^2 u_y^2 +1}dxdy 
$$

$\Rightarrow$ area della superficie:
\begin{equation}
A=\int_{\Omega} \sqrt{1+(\nabla u)^2} dxdy\equiv \int_{\Omega} \mathcal{L}dxdy (106)
\end{equation}

Superfici minime: $\delta A=0$
Equazione di Eulero-Lagrange:
\begin{equation}
\partial_i \dfrac{\mathcal{L}}{\partial \partial_i u}-\dfrac{\partial \mathcal{L}}{u}=0 (107)
\end{equation}
(somma su i da 1 a 2 sottintesa)
$$
\partial_i=\dfrac{\partial}{\partial_i} \qquad i=1,2 \qquad x_1=x,\quad x_2=y
$$
[paragona con $\dfrac{d}{dt}\dfrac{\partial \mathcal{L}}{\partial \dot{u}}-\dfrac{\partial\mathcal{L}}{\partial u}=0$]
$$
(107)\Rightarrow \partial_i \left( \dfrac{\partial_i u}{\sqrt{1+(\nabla u)^2}}\right)=0
$$
%pezzo che non ricordo dove vada, credo qui.
$$
\dfrac{\partial\mathcal{L}}{\partial \partial_i u} = \dfrac{1}{2}(1 +(\nabla u)^2)^{-\dfrac{1}{2}} 2\partial_i u=\dfrac{\partial_i u}{\sqrt{(1+(\nabla u)^2}}, \qquad \dfrac{\partial \mathcal{L}}{\partial u}=0 
$$
oppure
\begin{equation}
\nabla \cdot \left( \dfrac{\partial_i u}{\sqrt{1+(\nabla u)^2}}\right)=0 (108)
\end{equation}
"equazione delle superfici minime"

%%%%%%%%%%%%%%%%%%%%%%%%%%%%%%%%%%%
%% inizio lezione 15/11
%%%%%%%%%%%%%%%%%%%%%%%%%%%%%%%%%%
(Senza usare la (107):
$$
\delta\mathcal{L}=\delta\sqrt{1+(\nabla u)^2}=\dfrac{1}{2}\left(1+(\nabla u)^2\right)^{-\dfrac{1}{2}}2\nabla u \underset{\nabla \delta u}{\underbrace{\delta\nabla u}} \overset{\text{int. parti}}{\underset{\delta u|_{bordo}=0}{\longrightarrow}}
 -\delta u \nabla \left(\dfrac{\nabla u}{\sqrt{1+(\nabla u)^2}}\right)\overset{!}{=}0
$$
$$
(108)\Leftrightarrow \partial_i \left( \dfrac{\partial_i u}{\sqrt{1+(\nabla u)^2}}\right) =0 \Rightarrow \dfrac{\partial_i \partial_i u}{\sqrt{1+(\nabla u)^2}} + \partial_i u \left(-\dfrac{1}{2}\right) (1+(\nabla u)^2)^{-\frac{3}{2}} \cdot \underset{\partial_i(\partial_j u \partial_j u)}{\underbrace{\partial_i (\nabla u)^2}} =0
$$
$$
\partial_i(\partial_j u \partial_j u)=2(\partial_i\partial_j u)\partial_j u 
$$
$$
\Rightarrow (1+\underset{(\partial_x u)^2 + (\partial_y u)^2}{\underbrace{(\nabla u)^2}})\underset{\partial_x^2 u + \partial_y^2 u}{\underbrace{\Delta u}} =- \partial_x u \partial_x u \partial_x^2 u - \partial_x u\partial_y u \partial_x\partial_y u - \partial_yu \partial_x u \partial_y \partial_x u - \partial_y u\partial_y u \partial_y^2 u =0
$$
$$
\Rightarrow u_{xx} + u_{yy} + u_x^2 u_{yy} + u_y^2 u_{xx} - 2u_xu_y u_{xy}=0
$$
\begin{equation}
\Rightarrow (1+u_y^2)u_{xx} + (1+u_x^2) u_{yy} - 2u_x u_y u_{xy}=0 (108')
\end{equation}
$$
a_{11}=1+u_y^2, \qquad a_{22}=1+u_x^2, \qquad a_{12}=-u_x u_y
$$
$$
\delta = a_{12}^2 -a_{11}a_{22} = u_x^2 u_y^2 - (1+u_x^2 u_y^2 + u_x^2u_y^2)=-(1+(\nabla u)^2) <0
$$
$\Rightarrow$  equazione ellittica

def. $ A:=\left(\begin{matrix}
a_{11} & a_{12}\\
a_{12} & a_{22}
\end{matrix}\right)
$, $det A=-\delta$\\
iperbolica ($\delta >0$): $detA<0$, 2 autovalori di segno opposto\\
parabolica ($\delta =0$): 1 autovalore  =0\\
ellittica ($\delta <0$): 2 autovalori di segno uguale\\
$\grave{E}$ il parallelismo col comportamento della forma quadratica generata da A a suggerire la medesima denominazione delle coniche. \\
Esempi:\\
Equazioni ellittiche: laplace, Poisson (parte principale $\Delta u$)\\
Equazioni paraboliche: equazione del calore $\partial_t u = \chi \Delta u$ nel caso di 2 variabili: $\partial_t u = \chi \partial_x^2 u \Rightarrow a_{11}=\chi, \quad a_{12}=0 = a_{22} $\\
Iperboliche: equazione delle onde e dei telegrafisti ( quest'ultima per $\epsilon \neq 0$, altrimenti è parabolica), equazione di Klein-Gordon.

\section{Il problema di Cauchy}
La classificazione ha una prima motivazione nel problema di Cauchy: trovare una funzione $u\in C^2$ che soddisfi la (105) e i dati di Cauchy, ossia:
\begin{equation}
u|_{\gamma}=\Phi|_{\gamma}, \quad \left.\dfrac{\partial u}{\partial n}\right|_{\gamma}=\left.\Psi\right|_{\gamma} (109)
\end{equation}

dove $\Phi$, $\Psi$ sono funzioni assegnate e $\gamma$ è una data urva regolare con normale $\bar{n}$, $\dfrac{\partial n}{\partial n}=\bar{n}\cdot\bar{\nabla} u$.\\
%immagine curva regolare con normale%
Possibile modo di tentare di risolvere il problema: supporre che le $a_{ij}$, la $\gamma$ e i dati $\Phi$, $\Psi$ siano analitici e cercare \\
i) di calcolare tutte le derivate di u su $\gamma$\\
ii) di dimostrare che lo sviluppo di Taylor di u è convergente in un intorno di $\gamma$.\\
Se il procedimento ha successo si è costruita una soluzione analitica. \\
Questo è l'obiettivo del \underline{teorema di Cauchy - Kowalevski}: se i coefficienti $a_{ij}$, $b_i$, c, d sono analitici in un dominio D, se $\gamma \subset D$ è analitica e i dati di Cauchy $\Phi$, $\Psi$ sono analitici in D, il problema di cauchy (105) con dati iniziali (109) ammette una e una sola soluzione analitica in un opportuno intorno $I\subset D$ di $\gamma$, purché la normale $\bar{n}$ a $\gamma$ verifichi la condizione 
\begin{equation}
\bar{n}\cdot A \bar{n}\neq 0 (110).
\end{equation}
(senza dim.)\\
$\rightarrow$ la (110) chiama in causa la classificazione.
Il ruolo della (110) è quello di consentire il calcolo delle derivate seconde:
$$
\bar{n}\equiv (\alpha,\beta)
$$
Deriva tangenzialmente il primo dato di cauchy
%immagine derivata tangenzialmente gamma%
$$
\dfrac{\partial}{\partial\bar{\tau}} \equiv \bar{\tau}\cdot \bar{\nabla}, \qquad \bar{\tau}= (-\beta,\alpha), \qquad \left.\dfrac{\partial u}{\partial \tau}\right|_{\gamma} = \left. \dfrac{\partial \Phi}{\partial \tau}\right|_{\gamma}, \qquad \alpha^2 + \beta^2 =1
$$
\begin{equation}
\Rightarrow (-\beta u_x +\alpha u_y)|_{\gamma}=\left.\dfrac{\partial \Phi}{\partial \tau}\right|_{\gamma} (111)
\end{equation}
Secondo dato di Cauchy:
\begin{equation}
(\alpha u_x + \beta u_y)|_{\gamma} = \Psi|_{\gamma} (112)
\end{equation}
$$
(111),(112) \underset{\alpha^2 + \beta^2 \neq 0}{\implies} 
$$
\begin{eqnarray}
u_x|_{\gamma}=p_0(s),\\
u_y|_{\gamma}=q_0(s) (113)
\end{eqnarray}
dove $p_0$, $q_0$ sono espresse tramite i dati e s è il parametro naturale su $\gamma$ (lunghezza della curva).\\
Deriva (113) lungo $\gamma$:
$$
\bar{\tau}\cdot \bar{\nabla} u_x|_{\gamma}= p_0'(s)
$$
$$
\bar{\tau}\cdot \bar{\nabla}=\dfrac{d}{ds}
$$
$$
\bar{\tau}\cdot \bar{\nabla}u_y|_{\gamma}=q_0'(s),
$$
$$
\Leftrightarrow \left\{\begin{matrix}
(-\beta u_{xx} + \alpha u_{xy})|_{\gamma}=p_0'\\
(-\beta u_{xy} + \alpha u_{yy})|_{\gamma}=q_0'
\end{matrix}\right.
$$
Anche la parte principale nella (105) è esprimibile su $\gamma$ in base ai dati, perchè tutti i termmini rimanenti sono calcolabili:
$$
(a_{11}u_{xx}+2a_{12}u_{xy}+a_{22}u_{yy})|_{\gamma}=r_0(s)
$$
$\Rightarrow$ sistema di 3 equazioni lineare con determinante
\begin{equation}\left| \begin{matrix}

a_{11} & 2a_{12} & a_{22}\\
-\beta & \alpha & 0 \\
0 & -\beta & \alpha 
\end{matrix}\right| = \bar{n}\cdot A \bar{n}(114)
\end{equation}
$\Rightarrow$ la condizione (110) consente di calcolare le derivate seconde. Nello stesso modo di calcolano, sempre grazie alla (110), tutte le derivate successive.\\
Interpretazione geometrica della negazione della (110)?
\begin{equation}
\bar{n}\cdot A\bar{n}=a_{11}\alpha^2+2a_{12}\alpha\beta + a_{22}\beta^2=0 (115)
\end{equation}
Autovalori di A: $\underset{\in\R}{\underbrace{\lambda_1,\lambda_2}}$; con autovettori ortogonali $\bar{\Theta_1},\bar{\Theta_2}$ (A simmetrica).
$$
\bar{n}=: \eta_1 \bar{\Theta_1} + \eta_2\bar{\Theta_2} \Rightarrow \bar{n}\cdot A \bar{n}=\lambda_1\eta_1^2 + \lambda_2\eta_2^2
$$
$\rightarrow$ nel caso ellittico, questo è sempre $\neq 0$.\\
Caso parabolico: $A\bar{n}=0$ per $\bar{n}$ nell'autospazio corrispondente all'autovalore nullo.\\
Caso iperbolico: $\bar{n}\cdot A\bar{n} = 0$ per $\dfrac{\eta_1}{\eta_2}=\pm \sqrt{\left|\dfrac{\lambda_1}{\lambda_2}\right|} \rightarrow$ 2 campi vettoriali
$$
\bar{w}_1 = \sqrt{|\lambda_2|}\bar{\Theta}_1 + \sqrt{|\lambda_1|}\bar{\Theta}_2, \qquad (\eta_2=\sqrt{|\lambda_1|}, \eta_1=\sqrt{|\lambda_2|})
$$ 
$$
\bar{w}_2 = \sqrt{|\lambda_2|}\bar{\Theta}_1 + \sqrt{|\lambda_1|}\bar{\Theta}_2, \qquad (\eta_2=\sqrt{|\lambda_1|}, \eta_1=-\sqrt{|\lambda_2|})
$$
$\cdot$ curve $\perp$ a questi campi:
$$
\chi_{1,2}(x,y)=\text{cost}
$$
%immagine curva perp%

$\implies \bar{n} \propto \bar{\nabla}\chi \Rightarrow (\alpha,\beta) \propto (\chi_x, \chi_y)$ e quindi (115) $\Rightarrow$
\begin{equation}
a_{11}\chi_x^2 + 2a_{12}\chi_x\chi_y + a_{22}\chi_y^2 =0 (116)
\end{equation}
Poni $\chi =: f(x)-y \Rightarrow \chi_x = f', \quad \chi_y=-1 $
$$
(116)\Rightarrow a_{11}f'^2 - 2 a_{12}f' + a_{22}=0 (116')
$$
equazione di 2 grado per f'.\\
nel caso parabolico ci si riduce ad una sola famiglia, nel caso ellittico non ci sono curve di questo genere.\\
\underline{Definizione}: le curve $\chi(x,y)=$cost. soddisfacenti la (116) si chiamano \underline{curve caratteristiche} dell'equazione (105)\\
Esempi:\\
i) equazione d'onda in 1+1 dimensioni
$$
c^2 u_{xx}-u_{tt}=0, \qquad a_{11}=c^2, \quad a_{12}=0, \quad a_{22}=-1
$$
$$
(116')\Rightarrow c^2f'^2 =1 \implies f=\pm \dfrac{x}{c} + f_0
$$
$$
\chi=f(x) - t= \pm \dfrac{x}{c}+f_0 -t \overset{!}{=}cost \Rightarrow x-x_0 \pm c(t-t_0)=0
$$
ii) equazione del calore 
$$
\mathcal{H}u_{xx}- u_t=0, \qquad a_{11}=\mathcal{H}, \quad a_{12}=a_{22}=0
$$
$$
(116')\Rightarrow \mathcal{H} f'^2 =0 \Rightarrow f=cost=:f_0, \quad \chi = f_0-t=cost \Rightarrow t=cost. \text{ho delle rette}
$$
-Cambiamento di coordinate:
Sia $\xi=\xi(x,y)$, $\eta=\eta(x,y)$ cambiamento di coordinate con determinante Jacobiano $\neq 0$.
$$
v(\xi,\eta):=u(x,y)
$$
Come si trasforma la parte principale della (105)?
$$
u_x=v_\xi \xi_x + v_\eta \eta_x,\quad u_y=v_\xi \xi_y + v_\eta \eta_y
$$
$$
u_{xx}=v_{\xi \xi}\xi_x^2 + 2v_{\xi\eta}\xi_x\eta_x + v_{\eta\eta}\eta_x^2 + v_xi\eta_{xx} +v_\eta \eta_{xx}
$$
$$
u_{xy}= v_{\xi \xi}\xi_x\xi_y + 2v_{\xi\eta}(\xi_x\eta_y + \xi_y\eta_x) + v_{\eta\eta} \eta_x\eta_y + v_\xi \xi_{xy} + v_\eta \eta_{xy}
$$
$$
u_{yy}= v_{\xi\xi }\xi_y^2 + 2v_{\xi\eta}\xi_y\eta_y + v_{\eta\eta}\eta_y^2 + v_\xi \xi_{yy} + v_\eta \eta_{yy}
$$
$$
\dfrac{\partial}{\partial x} = \dfrac{\partial \xi}{\partial x}\dfrac{\partial}{\partial \xi} + \dfrac{\partial \eta}{\partial x}\dfrac{\partial}{\partial \eta}
$$
$$
\dfrac{\partial}{\partial y} = \dfrac{\partial \xi}{\partial y}\dfrac{\partial}{\partial \xi} + \dfrac{\partial \eta}{\partial y}\dfrac{\partial}{\partial \eta}
$$
%%%%%%%%%%%%%%%%%%%%%%%%%%%%%%%%%%%%%%%%%%
% inizio lezione 16/11
%%%%%%%%%%%%%%%%%%%%%%%%%%%%%%%%%%%%%%%%%%%%

$\rightarrow$ la parte principale della (105) diventa $ \tilde{a}_{11} v_{\xi\xi} + 2 \tilde{a}_{12}v_{\xi\eta} + \tilde{a}_{22}v_{\eta\eta}$, con 
$$
\tilde{a}_{11}= a_11 \xi_x^2 + 2a_{12}\xi_x\xi_y + a_{22} \xi_y^2 = \bar{\nabla}\xi \cdot A\bar{\nabla}\xi
$$
$$
\tilde{a}_{12}= a_11 \xi_x\eta_x + a_{12}(\xi_x\eta_y+\xi_y\eta_x) + a_{22} \xi_y\eta_y = \bar{\nabla}\eta \cdot A\bar{\nabla}\xi = \bar{\nabla}\xi \cdot A\bar{\nabla}\eta
$$
$$
\tilde{a}_{22}= a_11 \eta_x^2 + 2a_{12}\eta_x\eta_y + a_{22} \eta_y^2 = \bar{\nabla}\eta \cdot A\bar{\nabla}\eta
$$
\begin{equation}
\implies \tilde{A} = J A J^{-1}, \quad \text{J = Jacobiano} (117)
\end{equation}
$$
\Rightarrow det \tilde{A} = det A\underset{>0}{\underbrace{(det J)^2}}
$$
$\Rightarrow$ le trasformazioni invertibili di coordinate lasciano invariante il carattere dell'equazione.\\
Esempio: equazione d'onda $c^2u_{xx} - u_{tt}=0$. Prendi $\xi=x-ct$, $\eta = x+ct$, ("coordinate di cono luce")  poni $v(\xi,\eta) = u(x,y) \implies \tilde{A}=\left(\begin{matrix}
0 & 1\\
1 & 0
\end{matrix}\right)$, e l'equazione diventa $v_{\xi\eta}=0$. ($\rightarrow$ soluzione generale: $v(\xi,\eta)=v_1(\xi) + v_2(\eta)$, con $v_1$, $v_2$ funzioni arbitrarie.\\
\underline{- riduzione alla forma canonica}:
forme canoniche:\\
$u_{xx} + u_{yy} \quad$ ellittica\\
$u_{xx} (- u_t) \quad$ parabolica\\
$u_{xx} - u_{tt} \quad$ iperbolica (in alternativa $u_{xt}$)\\
N.B. costanti come per esempio $c^2$ in $c^2 u_{xx}-u_{tt}$ possno essere riassobite nella scala delle variabili.\\
Trasformazione di una parte principale generica nella forma canonica:\\
\underline{Caso iperbolico}:\\
L'esempio sopra suggerisce di prendere le curve caratteristiche come nuove linee coordinate $\xi=\chi_1(x,y),\quad \eta=\chi_2(x,y)$
%questa matrice era scritta in forma piu compatta, con uan croce a separare i 4 blocchi
$$
(117)\Rightarrow \tilde{A}\left(\begin{matrix}
a_{11}\chi_{1x}^2 + 2a_{12}\chi_{1x}\chi_{1y} + a_{22}\chi_{1y}^2  & a_{11}\chi_{1x}\chi_{2x} + a_{12}(\chi_{1x}\chi_{2y} + \chi_{1y}\chi_{2x}) + a_{22}\chi_{1y}\chi_{2y} \\
a_{11}\chi_{1x}\chi_{2x} + a_{12}(\chi_{1x}\chi_{2y} + \chi_{1y}\chi_{2x}) + a_{22}\chi_{1y}\chi_{2y}  & a_{11}\chi_{2x}^2 + 2a_{12}\chi_{2x}\chi_{2y} + a_{22}\chi_{2y}^2
\end{matrix} \right)=
$$
$$
\overset{(116)}{=}\left(\begin{matrix}
0 & *\\
* & 0
\end{matrix}\right) \rightarrow \text{parte principale riconducibile alla forma canonica $u_{xy}$}
$$
$$
[\Rightarrow 2 * v_{\xi\eta} + \dots =0|: (2*) \Rightarrow \text{parte princ }v_{\xi\eta} ]
$$
\underline{Caso parabolico}: \\
prendi $\xi=x$, $\eta=\chi(x,y) \Rightarrow J=\left(\begin{matrix}
\xi_x & \xi_y \\
\eta_x & \eta_y
\end{matrix}\right)=\left( \begin{matrix}
1 & 0\\
\chi_{x} & \chi_y
\end{matrix}\right)$\\
N.B. detA=0 $\Rightarrow$ se uno dei coefficienti $a_{11}, a_{22}$ è sero lo è anche $a_{12}$ ($a_{11}a_{22}-a_{12}^2=0$). In tal caso siamo già nella forma canonica.\\
$\rightarrow $ supponiamo $a_{11}a_{22}>0$ ( non può essere $<0$, perchè $a_{12}^2 = a_{11}a_{22}$).\\
Equazione delle caratteristiche (116):
$$
a_{11}\chi_x^2+2\sqrt{a_{11}a_{22}}\chi_x\chi_y + a_{22}\chi_y^2 =0
$$
(se $a_{12}=-\sqrt{a_{11}a_{22}}$: manda y in -y).\\
Senza perdere la generalità:\\
$a_{11}>0$ (altrimenti moltiplica la (105) con -1) $\Rightarrow a_{22}>0$. $\rightarrow$ abbiamo $(\sqrt{a_{11}}\chi_x + \sqrt{a_{22}}\chi_y)^2=0$
\begin{equation}
\Rightarrow\chi_y = - \sqrt{\dfrac{a_{11}}{a_{22}}}\chi_x (118)
\end{equation}
$$
\tilde{A}=\left(\begin{matrix}
1 & 0\\
\chi_x & \chi_y
\end{matrix}\right)\left(\begin{matrix}
a_11 & \sqrt{a_{11}a_{22}}\\
\sqrt{a_{11}a_{22}} & a_22
\end{matrix}\right)\left(\begin{matrix}
1 & \chi_x \\
0 & \chi_y
\end{matrix}\right)
$$
%anche questa matrice sotto era con la croce in mezzo inn forma compatta
$$
=\left(\begin{matrix}
a_{11} & a_{11}\chi_x + \sqrt{a_{11}a_{22}}\chi_y \\
a_{11}\chi_x + \sqrt{a_{11}a_{22}}\chi_y & a_{11}\chi_x^2 + 2\sqrt{a_{11}a_{22}}\chi_x\chi_y + a_{22}\chi_y^2
\end{matrix}\right)\overset{(118)}{=}\left(\begin{matrix}
a_{11} & 0\\
0 & 0
\end{matrix}\right) \rightarrow \text{forma canonica}
$$
\underline{Caso ellittico}: il metodo seguito per le equazioni iperboliche e paraboliche non è applicabile (non disponiamo di curve caratteristiche) $\rightarrow$ torna alla (117) e imponi direttamente $\tilde{a}_{12}=0,\tilde{a}_{11}=\tilde{a}_{22}$
\begin{equation}
a_{11}\xi_x \eta_x + a_{12}(\xi_x\eta_y + \xi_y\eta_x) + a_{22}\xi_y\eta_y=0(119)
\end{equation}
\begin{equation}
a_{11}(\xi_x^2 - \eta_x^2) + 2a_{12}(\xi_x\xi_y - \eta_x\eta_y) + a_{22}(\xi_y^2 - \eta_y^2)=0
(119')
\end{equation}
$$
2i\cdot(119) + (119'):
$$
\begin{equation}
\implies a_{11}(\xi_x + i\eta_x)^2 + 2a_{12}(\xi_x + i\eta_x)(\xi_y + i\eta_y) + a_{22}(\xi_y + i\eta_y)^2 =0 (119'')
\end{equation}
Si ha $a_{11}a_{22}>0$ nel caso ellittico (altrimenti $\delta =a_{12}^2 - a_{11}a_{22}$ non può mai essere <0). $\rightarrow$ la (119'') è un'equazione di 2 grado non degenere per $\rho:= \dfrac{\xi_x + i\eta_x}{\xi_y + i\eta_y}$, con soluzioni $\rho_{\pm}=-(a_{12} + i\sqrt{|\delta|})/a_{11}$
Prendi per esempio $\rho_+$:
$$
\Rightarrow a_{11} (\xi_x + i\eta_x)=-(a_{12}+i\sqrt{|\delta|})(\xi_y + i \eta_y)
$$
e quindi 
\begin{equation}
\begin{matrix}
\xi_x=(a_{12}\eta_x + a_{22}\eta_y)/\sqrt{|\delta|} \\
\xi_y=-(a_{11}\eta_x + a_{12}\eta_y)/\sqrt{|\delta|} 
\end{matrix}(120)
\end{equation}
"equazioni di Beltrami", caratterizzano le trasformazioni che conducono la parte principale ad un'espressione proporzionale all'operatore di Laplace. $\grave{E}$ stato dimostrato (risultato non banale) che le (120) ammettono soluzioni nell'intera regione di ellitticità.\\
N.B. le (120) possono essere scritte nella forma\\
\begin{equation}
\dfrac{\partial w}{\partial \z}=\mu\dfrac{\partial w}{\partial z} (120')
\end{equation}
con $z=x+iy$, $w==\xi+i\eta$, $\mu=\dfrac{a_{11}+i a_{12 - \sqrt{|\delta|}}}{-a_{11}+i a_{12 - \sqrt{|\delta|}}}$ (esercizio)\\
Commento: un'applicazione $w(z,\z)$ che soddisfa la (120') viene chiamata mappa \underline{quasiconforme}\\
(Caso particolare $\mu=0$: mappa conforme, $w=w(z)$ funzione olomorfa).

\section{La questione della buona posizione del problema di Cauchy}
\underline{Definizione}: un prblema al contorno per un equazione alle derivate parziali di sice \underline{ben posto secondo Hadamard} se possiede una e una sola soluzione ed essa dipende in modo continuo dai dati.\\
Commenti:\\
i) L'enunciato corretto di un problema al contorno contiente non solo l'equazione differenziale e i dati al contorno, ma anche la precisazione dello spazio funzionale in cui si cerca la soluzione. Questa scelta è in grado di influenzare ad esempio l'unicità.\\
ii) Cosa si intende per dipendenza continua? $\rightarrow$ richiede una metrica nello spazio delle soluzione e una metrica nello spazio dei dati\\
X: spazio delle soluzioni, con norma $\| \cdot \|_X$\\
Y: spazio dei dati, con norma $\| \cdot \|_Y$\\
$\rightarrow$ dipendenza continua significa:\\
Sia $\{\bar{\delta}_n\}$ successione di dati (ordinabili in un vettore) t.c.  $\lim_{n\to\infty} \| \bar{\delta}_n - \bar{\delta}\|_Y=0$, e sia $\{u_n\}$ successione di soluzioni, u: soluzione corrispondente al dato limite $\bar{\delta}$. Si ha dipendenza continua se $\lim_{n\to \infty}\|u_n - u \|=0$.\\
-Le curve caratteristiche di un'equazione differenziale hanno un ruolo critico nel problema di Cauchy, perchè limitano la scelta delle curve portanti i dati.\\
$\rightarrow$ equazioni ellittica avvantaggiate dall'assenza di curve caratteristiche? E' così per la questione dell'esistenza e unicità, ma non per la dipendenza continua (idem per le equazioni paraboliche).\\
N.B. dipendenza continua dai dati è proprietà irrinunciavile per un modello matematico sensato. Infatti i dati sono di solito sperimentali e ciò non deve essere causa di un comportamento imprevedibile della soluzione.\\
Inoltre: mancanza della dipendenza continua impedisce il calcolo numerico della soluzione, a causa della ripercussione incontrollabile degli errori di arrotondamento.\\
-La non buona posizione del problema di Cauchy per equazioni ellittiche e paraboliche è messa in luce dai seguenti esempi:\\
i)Successione di problemi di Cauchy\\
$$ u_{xx}^{(n)} + u_{yy}^{(n)}=0, \quad u^{(n)}(0,y)=0, \quad u_x^{(n)}(0,y)=A_n\cos ny; n=1,2,\dots $$
$$
\bar{n}\cdot \bar{\nabla} = \left(\begin{matrix}
1 \\ 0
\end{matrix}\right) \cdot \left( \begin{matrix}
\partial_x \\ \partial_y
\end{matrix}\right)=\partial_x
$$
$\rightarrow$ unica soluzione: $u^{(n)}(x,y)=\dfrac{A_n}{2n}(e^{nx}-e^{-nx})\cos ny $
(si può trovare con un ansatz di separazione delle variabili, $u^{(n)}(x,y)=f(x)g(y)$ ($\rightarrow$ esercizio!)). Prendi per esempio $A_n=e^{-\sqrt{n}} \Rightarrow n\to \infty$ \\
-i dati di Cauchy tendono uniformemente a 0.\\
-per qualunque $x\neq 0$ le $u^{(n)}$ non solo non tendono a 0, ma non restano limitate.
ii) Successione 
$$
u_{xx}^{(n)} - u_t^{n}=0, \quad u^{(n)}(0,t)=A_n\sin nt, \quad u_x^{(n)}(0,t)=0;n=1,2,\dots
$$
$\rightarrow$ soluzione: 
$$
u^{(n)}(x,t) =\dfrac{1}{2}A_n\left(e^{\sqrt{\dfrac{n}{2}}x}\sin\left(nt+\sqrt{\dfrac{n}{2}x}\right) + e^{-\sqrt{\dfrac{n}{2}}x}\sin\left(nt-\sqrt{\dfrac{n}{2}x}\right)\right) 
$$
(esercizio!)\\
Prendi per esempio $A_n=n^{-k}, k>0 \rightarrow$ stessa situazione dell'esempio i)

%%%%%%%%%%%%%%%%%%%%%%%%%%%%%%%%%%%%%%%%%%%%%%%%%%%%%%%
%% inizio lezione 22/11 
%%%%%%%%%%%%%%%%%%%%%%%%%%%%%%%%%%%%%%%%%%%%%%%%%

\underline{Problema}: calcolare le derivate di ordine >2 nel problema di Cauchy (105),(109).\\
\underline{Soluzione}: conosciamo su $\gamma$
$$
u_{xx}:=\omega_{2,0}(s), \quad u_{xy}:=\omega_{1,1}(s), \quad u_{yy}=\omega_{0,2}(s)
$$
$$
\dfrac{d}{ds}=\bar{t}\cdot \bar{\nabla} = -\beta \partial_x + \alpha \partial_y
$$
%% immagine gamma con n,t
$$
\bar{n}=\left(\begin{matrix}
\alpha \\ \beta
\end{matrix}\right), \qquad \bar{t}=\left(\begin{matrix}
-\beta \\ \alpha
\end{matrix}\right)
$$
Derivate tangenziali:ho 4 incognite
\begin{equation}
\begin{matrix}
-\beta u_{xxx} + \alpha u_{xxy} = \omega'_{2,0}(s) \\
-\beta u_{xxy} + \alpha u_{xyy} = \omega'_{1,1}(s) \\
-\beta u_{xyy} + \alpha u_{yyy} = \omega'_{0,2}(s)
\end{matrix} (121)
\end{equation}
Inoltre: deriva la (105) rispetto a x e otieni conto che termini del tipo $(a_{11,x}u_{xx})|_{\gamma}$ sono noti
\begin{equation}
\Rightarrow a_{11}u_{xxx} + 2a_{12}u_{xxy} + a_{22}u_{xyy}=r_0^{(3)}(s) (122.0)
\end{equation}
oppure deriva rispeto a y
\begin{equation}
\Rightarrow a_{11}u_{xxy} + 2a_{12}u_{xyy} + a_{22}u_{yyy}=r_1^{(3)}(s) (122.1)
\end{equation}
($|_{\gamma}$ sempre sottinteso).\\
Per (121),(122.0) abbiamo il determinante:
$$
J_0^{(3)}=\left|\begin{matrix}
-\beta & \alpha & 0 & 0\\
0 & -\beta & \alpha & 0 \\
0 & 0 & -\beta & \alpha \\
a_{11} & 2a_{12} & a_{22} & 0
\end{matrix}\right| = a_{11}\alpha^3 + 2a_{12}\beta\alpha^2 + a_{22}\alpha\beta^2 = \alpha \bar{n}\cdot A \bar{n}
$$
Per (121),(122.1) invece: 
$$
J_1^{(3)}=\left|\begin{matrix}
-\beta & \alpha & 0 & 0\\
0 & -\beta & \alpha & 0 \\
0 & 0 & -\beta & \alpha \\
0 & a_{11} & 2a_{12} & a_{22} 
\end{matrix}\right| = a_{11}\beta\alpha^2 + 2a_{12}\alpha\beta^2 + a_{22}\beta^3 = \beta \bar{n}\cdot A \bar{n}
$$
Siccome $\alpha^2 + \beta^2 =1$, $\bar{n}\cdot A \bar{n}\neq 0$, non possono essere entrambi nulli.\\
Derivate di ordine $>3$: supponiamo note su $\gamma$ le derivate di grado minode nella forma $\partial_x^{n-i}\partial_y^i u = \omega_{n-i,i} (s)$, con $n\geq 3$, $i=0,1,\dots, n$ $(\partial_x^0 = \partial_y^0 \equiv 1)$\\
Deriviamole tangenzialmente:
\begin{equation}
-\beta \partial_x^{n+1-i}\partial_y^{i}u +\alpha \partial_x^{n-i}\partial_y^{i+1}u = \omega'_{n-i,i}(s) (123)
\end{equation}
ho $n+1$ equazioni, $n+2$ incognite. Equazione mancante: applica alla (105) l'operatore $\partial_x^{n-1-k}\partial_y^k (k=0,\dots,n-1)$ per far comparire le derivate (n+1)-esime:\\
\begin{equation}
\Rightarrow a_{11}\partial_x^{n+1-k}\partial_y^k u + 2a_{12}\partial_x^{n-k}\partial_y^{k+1}u + a_22 \partial_x^{n-1-k}\partial_y^{k+2}u = r_k^{(n+1)}(s) (124)
\end{equation}
$\rightarrow$ troviamo dei determinanti $J_k^{(n+1)}$, che hanno in comue le prime n+1 righe (corrispondenti alle (123))
$$
\begin{matrix}
-\beta & \alpha & 0 & 0 & \dots & 0 & 0\\
0 & -\beta & \alpha & 0 & \dots & 0 & 0\\
0 & 0 &-\beta & \alpha  & \dots & 0 & 0\\
\dots & \dots & \dots & \dots & \dots & \dots & \dots \\
0 & 0 & 0 & 0 &  \dots & -\beta & \alpha\\
\end{matrix}
$$
mentre l'ultima riga (corrispondente a una delle (124)) ha la terna $ a_{11},2a_{12},a_{22}$ che occupa i rispettivi posti (k+1)-esimo, (k+2)-esimo, (k+3)-esimo.\\
Per :
$$
\begin{matrix}
k=0 : J_0^{(n+1)} = \alpha^{n-1}\bar{n}\cdot A\bar{n}\\
k=1 : J_1^{(n+1)} = \beta\alpha^{n-2}\bar{n}\cdot A\bar{n}\\
\dots \\
k=n-1 : J_{n-1}^{(n+1)} = \beta^{n-1}\bar{n}\cdot A\bar{n}\\
\end{matrix}
$$

Almeno uno tra $J_0^{(n+1)}$ e $J_{n-1}^{(n+1)}$ è diverso da zero.

\section{Classificazione di equazioni differenziali di ordine superiore e in più variabili}
%questa sezione è da sapere solo se si vuole il 30 e/o la lode, altrimenti fa niente

Considera equazioni differenziali:
\begin{equation}
\sum_{|j|\leq m}a_j D^j u = f \qquad (\text{in } \Rn) (125)
\end{equation}

$$ j=(j_1,\dots,j_n) \quad \text{ multi-indice}$$
$$|j| \equiv j_1 + \dots + j_n \quad \text{ m ordine}$$
$$D^j \equiv \left(\dfrac{\partial}{\partial x_1}\right)^{j_1} \cdot \dots \cdot \left(\dfrac{\partial}{\partial x_n}\right)^{j_n}$$
parte principale: 
\begin{equation}
\sum_{|j| = m} a_j D^j u (126)
\end{equation}

Esempio m=3: 
$$
a_{300}(x,y,z)\dfrac{\partial ^3 u}{\partial x^3} + a_{020}(x,y,z)\dfrac{\partial ^2 u}{\partial y^2} + a_{201}(x,y,z)\dfrac{\partial ^3 u}{\partial x^2\partial z} + a_{001}(x,y,z)\dfrac{\partial u}{\partial x} = f(x,y,z) (*)
$$

$\rightarrow$ parte principale:

$$
a_{300}(x,y,z)\dfrac{\partial ^3 u}{\partial x^3} + a_{201}(x,y,z)\dfrac{\partial ^3 u}{\partial x^2\partial z}
$$

\underline{Definizione}: una sottovarietà di dim n-1 ed equazione $S(x_1,\dots,x_n)=0$ viene chiamata \underline{varietà caratteristica} dell'equazione differenziale a derivate parziali se soddisfa l'equazione differenziale del 1 ordine

\begin{equation}
\sum_{|j| =m} a_j p^j=0 , \qquad p^j=\left(\dfrac{\partial S}{\partial x_1}\right)^{j_1} \cdot \dots \cdot \left(\dfrac{\partial S}{\partial x_n}\right)^{j_n} = p_1^{j_1}\cdot \dots \cdot p_n^{j_n} (127)
\end{equation}

Esempi:\\
i) per la (*): 
$$
\sum_{|j| = 3}a_{j_1 j_2 j_3}\left(\dfrac{\partial S}{\partial x}\right)^{j_1}\left(\dfrac{\partial S}{\partial y}\right)^{j_2}\left(\dfrac{\partial S}{\partial z}\right)^{j_3}=0
$$
$$
\Rightarrow a_{300}\left(\dfrac{\partial S}{\partial x}\right)^3 + a_{201}\left(\dfrac{\partial S}{\partial x}\right)^2\left(\dfrac{\partial S}{\partial y}\right)=0
$$

ii) per la (105):
$$
a_{20}(x,y)u_{xx}+a_{11}(x,y)u_{xy} + a_{02}u_{yy} + a_{10}(x,y)u_x + a_{01}(x,y)u_y + a_{00}(x,y)u = f(x,y)
$$
(rinominato i coefficienti)\\
m=2\\
parte principale: $a_{20}u_{xx} + a_{11}u_{xy} + a_{02}u_{yy} $\\
varietà caratteristica:
$$
a_{20}\left(\dfrac{\partial S}{\partial x}\right)^2 + a_{11}\dfrac{\partial S}{\partial y} + a_{02}\left(\dfrac{\partial S}{\partial y}\right)^2=0
$$
è la (116) (con $\chi \rightarrow S|_{S(x,y)=0}$)
\underline{Problema di Cauchy}: trovare una soluzione $C^m$ della (125), la quale, assieme alle sue derivate dell'ordine $\leq m-1$, assume dei valori fissati su una sottovarietà di dimensione n-1 ed equazione $S(x_1,\dots,\x_n)=0$.\\
Esempio: $n=m=2 \Rightarrow $ dim S=1 $\rightarrow$ curva sulla quale fissiamo u e la sua derivata prima.\\
Il problema viene semplificato dal seguente cambiamento di coordinate:
$$
x_1'=S(x_1,\dots,x_n), x_j'=x_j (j=2,\dots,n)
$$
$$
\dfrac{\partial}{\partial x_i} = \dfrac{\partial x_1'}{\partial x_i}\dfrac{\partial}{\partial x'_1}+\sum_{j=2}^n \dfrac{\partial x'_j}{\partial x_i}\dfrac{\partial}{\partial x'_j}=p_i\dfrac{\partial}{\partial x'_1} + \sum_{j=2}^n \delta_{ij} \dfrac{\partial}{\partial x'_j}
$$
$\Rightarrow$ la (125) diventa:
%%equazione incasinata con frecce, da completare vedi appunti
$$
\sum_{|j| \leq m} a_j\left(\dfrac{\partial}{\partial x_1}\right)^{j_1}\left(\dfrac{\partial}{\partial x_2}\right)^{j_2}\cdot \dots\cdot \left(\dfrac{\partial}{\partial x_n}\right)^{j_n}u
$$
$$
\left(\dfrac{\partial}{\partial x_1}\right)^{j_1} \rightarrow p_1\dfrac{\partial}{\partial x'_1} + \sum_{j=2}^n \delta_{j1}\dfrac{\partial}{\partial x'_j} = p_1\dfrac{\partial}{\partial x'_1}
$$
$$
\left(\dfrac{\partial}{\partial x_2}\right)^{j_2} \rightarrow p_2\dfrac{\partial}{\partial x'_1} + \sum_{j=2}^n \delta_{j2}\dfrac{\partial}{\partial x'_j} = p_2\dfrac{\partial}{\partial x'_1} + \dfrac{\partial}{\partial x'_2}
$$
$$
\left(\dfrac{\partial}{\partial x_n}\right)^{j_n} \rightarrow
$$
\begin{equation}
\Rightarrow \sum_{|j|=m} a_j p_1^{j_1}\cdot \dots \cdot p_n^{j_n}\left(\dfrac{\partial}{\partial x'_1}\right)^{j_1 + \dots  + j_n} + \sum_{\underset{j_1<m}{|j|\leq m}}b_j\left(\dfrac{\partial}{\partial x'_1}\right)^{j_1} \cdot \dots \cdot \left(\dfrac{\partial}{\partial x'_n}\right)^{j_n}u=f (128)
\end{equation}
Per alleggerire la notazione sopprimo i primi ' in quello che segue.\\
Dati di Cauchy per la (128):
\begin{equation}
\left.\dfrac{\partial^k u}{(\partial x_1)^k}\right|_{x_1=0}=\varphi_k(x_2,\dots, x_n) , \qquad k=0,\dots,m-1 (129)
\end{equation}

Se le funzioni $\varphi_k$ sono di classe $C^{m-k}$, determinano 
\begin{equation}
D^j u(0,x_2,\dots,x_n)=\left(\left(\dfrac{\partial}{\partial x_1}\right)^{j_1}\cdot \dots \cdot \left(\dfrac{\partial}{\partial x_n}\right)^{j_n}u\right)(0,x_2,\dots,x_n) (130)
\end{equation}
per $|j|\leq m, \quad j_1<m$ (deriva la (129) al massimo m-k volte).\\
Le (130), attraverso la (128), determinano $\dfrac{\partial^m u}{(\partial x_1)^m} (0,x_2,\dots, x_n)$ se $\sum_{|j|=m} a_j p^j \neq 0$, cioè se S non è una varietà caratteristica. Se S è una varietà caratteristica, la (128) valutata in $x_1=0$ dà una reatione fra i dati di Cauchy (129). Su una varietà caratteristica i dati di Cauchy possono essere prescritti indipendentemente!
%%%%%%%%%%%%%%%%%%%%%%%%
%% inizio lezione 23/11
%%%%%%%%%%%%%%%%%%%%%%%%%%%%%%%%%

\underline{Teorema di Cauchy - Kowalevski}:\\
La (125) con f e $\{a_j\}$ analitiche, e dati di Cauchy analitici su una varietà analitica S non caratteristica, ha una e una sola soluzione analitica in un intorno di S.\\
\underline{Classificazione}: le proprietà di soluzioni di equazioni differenziali alle derivate parziali dipendono dalla natura delle variabili caratteristiche $\rightarrow$ seguente classificazione:\\
-La (125) viene chiamata \underline{ellittica} se l'equazione in $\underset{\equiv p}{\underbrace{(p_1,\dots,p_n)}}$
$$
\sum_{|j|=m}a_jp^j=0 \quad 
$$
non ammette soluzioni reali per $p\neq 0$\\
-La (125) è $x^1-$\underline{iperbolica} se l'equazione in $p_1$
$$
\sum_{|j|=m}a_jp^j=0
$$
ha m radici reali distinte per ogni sistea di numero reali $(p_1,\dots,p_n)$\\
Esempi:\\
i) Equazione di Laplace $\Delta u=\sum_{i=1}^n\dfrac{\partial^2}{\partial x_i^2}u=0$, m=2
$$
\sum_{|j|=2} a_{j_1,\dots,j_n}p_1^{j_1}\cdot \dots \cdot p^{j_n}_n=0
$$
%c'era una freccia che andava dall'equazione qui sopra alla parte che viene dopo la * nell'equazione per il delta u
$$
\Rightarrow p_1^2 + p_2^2 + \dots + p_n^2 =0
$$
$$
\left[\Delta u = \overset{1}{\overbrace{a_{20\dots 0}}}\dfrac{\partial^2}{\partial x_1^2}u + \underset{1}{\underbrace{a_{020\dots 0}}}\dfrac{\partial^2}{\partial x_2^2} + \dots + \underset{1}{\underbrace{a_{0\dots 02}}}\dfrac{\partial^2}{\partial x_n^2}  + \overset{1}{\overbrace{a_{20\dots 0}}}p_1^2 p_2^0 \dots p_n^0 + \dots + \underset{1}{\underbrace{a_{00\dots 02}}}p_1^0p_2^0\dots p_n^2 \right]\Rightarrow \text{ellittica}
$$
ii) Equazione d'onda
$$
\Delta u =\left(\dfrac{\partial^2}{\partial x_1^2}-\sum_{i=2}^n\dfrac{\partial^2}{\partial x_i^2} \right)u=0, \qquad m=2
$$
$$
\sum_{|j|=2} a_{j_1 \dots j_n}p_1^{j_1}\cdot \dots \cdot p_n^{j_n}=0
$$
$$
\Rightarrow p_1^2 - p_2^2 - \dots - p_n^2=0
$$
$ \Rightarrow$ 2 radici reali per $p_1$, $\forall (P_1,\dots,p_n) \Rightarrow$ iperbolica.\\
-La (125) è $x^1-$\underline{parabolica} se può essere scritta
\begin{equation}
\dfrac{\partial u}{\partial x_1} - \sum_{\underset{j_1=0}{|j|\leq m}}a_jD^j=f (131)
\end{equation}

con $\sum_{\underset{j_1=0}{|j|\leq m}} a_jp^j >0 \forall p\neq 0$, cioé $\sum_{\underset{j_1=0}{|j|\leq m}} a_jD^j$ è un operatore differenziale ellittico in $\R^{n-1}$.\\
Esempio: l'equazione del calore $\dfrac{\partial u}{\partial t} - \Delta u$ è t-parabolica.\\
-Le equazioni paraboliche rappresentano una classe importante di equazioni che non sono né ellittiche né iperboliche.

\section{La formula di Green}
Sia $U\subseteq \Rn$ insieme aperto limitato con bordo $C^1 \partial U$, $u,\varphi$ funzioni $C^2$ in U.
%immagine dell'insieme U
$$
\Rightarrow u\Delta \varphi = \sum_{i=1}^n\dfrac{\partial}{\partial x_i}\left(u\dfrac{\partial \varphi}{\partial x_i} \right) - \sum_{i=1}^{n}\dfrac{\partial u}{\partial x_i}\dfrac{\partial \varphi}{\partial x_i}
$$
$$
\int_{U}u\Delta \varphi d^nx=\underset{\overset{Gauss}{=}\int_{\partial U}\sum_i u\dfrac{\partial \varphi}{\partial x_i}d^{n-1}S^i}{\underbrace{\int_{U}\sum_{i=1}^n\dfrac{\partial}{\partial x_i}\left(u\dfrac{\partial \varphi}{\partial x_i} \right)d^nx}} - \int_{U} \sum_{i=1}^{n} \dfrac{\partial u}{\partial x_i}\dfrac{\partial \varphi}{\partial x_i}d^nx
$$
$\sum_i \dfrac{\partial }{\partial x_i}v_i = \nabla \cdot \bar{v}$ posso applicare il teorema di Gauss al primo integrale.
$$
\left| d^{n-1}\vec{S}\right| = \text{elemento di "superficie" (n-1)-dim.}
$$
\begin{equation}
\overset{\varphi=u}{\Rightarrow} \int_U u\Delta u d^n x = \int_{\partial U} \sum_i u\dfrac{\partial u}{\partial x_i}d^{n-1}S^i - \int_{U}\sum_i \left(\dfrac{\partial u}{\partial x_i}\right)^2 d^n x (132)
\end{equation}

("integrale di energia"), e 

\begin{equation}
\int_U \left(u\Delta \varphi - \varphi \Delta u\right)d^n x=\int_{\partial U} \sum_i \left( u\dfrac{\partial \varphi}{\partial x_i} - \varphi \dfrac{\partial u}{\partial x_i}\right) d^{n-1}S (133)
\end{equation}
"formula di Green" (2 identità di Green)\\
$\Rightarrow$ \underline{teorema di unicità}: l'equazione $\Delta u = f$ ha al massimo una soluzione in $C^2(U)$ che assume dei dati valori su $\partial U$.\\
\underline{Dimostrazione}: siano $v_1$ e $v_2$ soluzioni. Dimostreremo che $u=v_1 - v_2=0$
$$
\Delta u=\Delta v_1 - \Delta v_2 = f-f =0, \qquad u|_{\partial U} = v_1 |_{\partial U} - v_2 |_{\partial U}=0
$$
$$
(132)\Rightarrow \sum_i\int_U \left(\dfrac{\partial u}{\partial x_i}\right)^2 d^n x=0
$$
$$
\Rightarrow \dfrac{\partial u}{\partial x_i}=0 \text{ in }U
$$
$$
\Rightarrow u= \text{costante in }U
$$
$$
\left.u\right|_{\partial U}=0 \Rightarrow u=0 \text{ in }U \qquad \blacksquare
$$
La formula di Green (133) permette di ottenere un'espressione per $ u(\y),\y \in U$, in termini di f $(=\Delta u)$ in U e di $u,\dfrac{\partial u}{\partial n}$ su $\partial U$:\\
Nella (133), poni $\varphi(\x)=-K_n|\x-\y|^{2-n},$ con $K_n:= \dfrac{\Gamma (n/2)}{(n-2)2\pi ^{n/2}}, n\neq 2$\\
$\varphi$ è $C^\infty$ nell'insieme aperto $\x \neq \y$. Per il dominio di integrazione prendiamo $U_\epsilon$, uguale a U senza la palla $B_\epsilon(\y)$ con raggio $\epsilon$ e centro $\y$:
%immagine dell'insieme descritto
$$
(133)\Rightarrow \int_{U_\epsilon} \left(u\Delta\varphi - \varphi \Delta u\right)d^n x
$$
$\Delta \varphi = 0$ in $U_\epsilon$ (compito: verificare!). $\partial U_\epsilon=\partial U \cup \partial B_\epsilon$
$$
\int_{\partial U_\epsilon}\sum_i \left( u\dfrac{\partial \varphi}{\partial x_i} - \varphi \dfrac{\partial u}{\partial x_i}\right)d^{n-1}S^i
$$
$$
\Rightarrow K_n \underset{\overset{\epsilon \to 0}{\rightarrow}\int_U |\x - \y|^{2-n}f d^n x}{\underbrace{\int_{U_\epsilon}|\x-\y|^{2-n}\underset{f}{\underbrace{\Delta u}} d^n x}}
$$
\begin{equation}
=-K_n \int_{\partial U \cup \partial B_\epsilon}\sum_i \left(u\dfrac{\partial}{\partial x_i}|\x-\y|^{2-n} - |\x-\y|^{2-n}\dfrac{\partial u}{\partial x_i} \right)d^{n-1}S^i (134)
\end{equation}
$$
\dfrac{\partial}{\partial x_i}|\x - \y|^{2-n}=\dfrac{\partial}{\partial x_i}\left[ (x_1-y_1)^2 + \dots + (x_n - y_n)^2\right]^{\dfrac{2-n}{2}}=\dfrac{2-n}{2}[\dots]2(x_i-y_i)=(2-n)(x_i-y_i)|\x-\y|^{-n}
$$
Integrale su $\partial B_\epsilon$:
$$
-K_n\int_{S_\epsilon}\sum_i\left(u(2-n)\overset{n_i |\x - \y|}{\overbrace{(x_i-y_i)}}|\x - \y|^{-n}-|\x-\y|^{2-n}\dfrac{\partial u}{\partial x_i}\right) d^{n-1}S^i
$$
con $S_\epsilon^{n-1}$ (n-1)-sfera con raggio $\epsilon$ (bordo di $B_\epsilon$), $|\x-\y|=\epsilon$ per $\x \in \partial B_\epsilon$\\
%immagine palla B epsilon 
vettore normale: $\bar{n}=\dfrac{\x - \y}{|\x-\y|}, |\bar{n}|=1$ (punta verso l'esterno della palla)\\
$d^{n-1}S^i= - n_idV_\epsilon$. Il vettore normale deve puntare verso l'\underline{esterno} di $U_\epsilon$ , che è l'interno della palla. $dV_\epsilon$ elemento di "superficie" su $S^{n-1}_\epsilon$\\
%immagine di U epsilon
$ \Rightarrow $ integrale su $\partial B_\epsilon=$
$$
=+K_n\int_{S_\epsilon^{n-1}} \sum_i \left(u(2-n)n_i\epsilon^{1-n}-\epsilon^{2-n}\dfrac{\partial u}{\partial x_i}\right)n_i\underset{\epsilon^{n-1}dV_1}{\underbrace{dV_\epsilon}}
$$
dove $dV_1$= elemento di volume sulla $S_1^{n-1}$ (raggio 1)
\begin{equation}
\overset{\epsilon \to 0}{\underset{\sum_i (n_i)^2=1}{\longrightarrow}}+K_n u(\y)(2-n)\int_{S_1^{n-1}}dV_1=-u(\y) (135)
\end{equation}
con $\int_{S_1^{n-1}}dV_1=$"superficie" della sfera (n-1)-dimensionale con raggio 1,  $dV_1=\dfrac{2\pi^{n}}{\Gamma(n/2)}$, volume $V_n(R)$

$$
u(\x) \qquad \x \in \partial B_\epsilon
$$
$$
u(\x)\to u(\y) \text{ per } |\x - \y|=\epsilon
$$
\underline{Parentesi: volume della palla n-dim e area della $S^{n-1}$}
$$
B^n(R)=\{(x_1,\dots,x_n)\in \R |x_1^2 + \dots + x_n^2\leq \R^2\}
$$
%immagine bolla raggio R
$$
S^{n-1}(R)=\{(x_1,\dots,\x_n)\in\Rn | x_1 + \dots + x_n^2=R^2 \}, \text{ area $A_{n-1}(R)$}
$$
\begin{equation}
V_n(R)=\underset{x_1^2 + \cdots + x_n^2 \leq R}{\int \dots \int} dx_1\dots dx_n = C_n R^n (136)
\end{equation}
$C_n$ coefficienti da calcolare, $R^n$ segue da analisi dimensionale
%%%%%%%%%%%%%%%%%%%%%%%%%%%%%%%%%%%%%%%%%%%%55
%% inizio lezione 29 Novembre / Annachiara Filippini
%%%%%%%%%%%%%%%%%%%%%%%%%%%%%%%%%%%%%%%%%%%%%%
Posso calcolare $V_n(R)$ sommando infiniti gusci sferici di spessore infinitesimo.
\begin{equation}
V_n(R)=\int_0^R A_{n-1}(r)dr (137)
\end{equation}
\begin{equation}
\Rightarrow A_{n-1}=\dfrac{dV_n(R)}{dR}\overset{(136)}{=} nC_n R^{n-1}(138)
\end{equation}
Uguagliando (136) e (137), e usando la (138)
\begin{equation}
\underset{\text{integrale in cartesiane}}{\underbrace{\underset{x_1^2 + \cdots + x_n^2 \leq R}{\int \dots \int}}}=\underset{\text{in sferiche (già integrato su angoli)}}{\underbrace{C_n \int_0^R r^{n-1}dr}} (139)
\end{equation}
In n dimensioni: 
\begin{equation}
dx_1 \cdots dx_n=r^{n-1}drd\Omega_{n-1} (140)
\end{equation}
$ \Rightarrow d\Omega_{n-1} $ contiene la parte angolare.\\
n=2: $d\Omega=d\theta$; n=3: $d\Omega=\sin\theta d\theta d\varphi$, ecc$\dots$ (potrei dare esplicitamente la n-1-esima variabile angolare)
Confrontando (139) e (140)
\begin{equation}
\int \cdots \int d\Omega_{n-1}=nC_n (141)
\end{equation}
Si può calcolare $C_n$ senza introdurre esplicitamente le coordinate sferiche in n dimensioni $\Rightarrow$ Trucco: considero $f(x_1,\dots,x_n)=\exp(-x_1^2 - \dots - x_n^2)=e^{-1^2}$. Integro f su $\Rn$, sia in cartesiane che in sferiche.
$$
\underset{n volte}{\underbrace{\int_{-\infty}^\infty \cdots \int_{-\infty}^{\infty}}} e^{-(x_1^2 + \dots + x_n^2)}dx_1 \dots dx_n = \int_0^{\infty}r^{n-1}dr\int d\Omega_{n-1}e^{-r^2}
$$
$$
\Rightarrow \underset{\text{Gauss = $\sqrt{\pi}$}}{\underbrace{\left(\int_{-\infty}^{\infty}e^{-x_1^2}dx_1 \right)^n}}= \underset{\text{esprimibile tramite $\Gamma$ di Eulero $\Rightarrow = \dfrac{1}{2}\Gamma\left(\dfrac{n}{2}\right)$}}{\underbrace{n C_n\int_0^\infty r^{n-1}e^{-r^2}dr}} \Rightarrow C_n = \dfrac{2\pi^{n/2}}{n\Gamma\left(\dfrac{n}{2}\right)}
$$
\begin{equation}
\overset{(138)}{\Rightarrow}A_{n-1}(R) = \dfrac{2\pi^{n/2}}{\Gamma\left(\dfrac{n}{2}\right)}R^{n-1} (142)
\end{equation}
\begin{equation}
V_n(R)=\dfrac{2\pi^{n/2}}{n\Gamma\left(\dfrac{n}{2}\right)}R^n (143)
\end{equation}
Se $n=2, A_1(R)=\dfrac{2\pi}{\underset{=1}{\underbrace{\Gamma(1)}}} R=2\pi R$ (superficie sfera in $\R^2$);\\
Se $n=3, A_2(R)=\dfrac{2\pi^{3/2}}{\underset{\Gamma\left(\dfrac{3}{2}\right)=\Gamma\left(\dfrac{1}{2}+1\right)=\dfrac{1}{2}\Gamma\left(\dfrac{1}{2}\right)=\dfrac{1}{2}\sqrt{\pi}}{\underbrace{\Gamma\left(\dfrac{3}{2}\right)}}}R^2=4\pi R^2$ (superficie della sfera in $\R^3$)\\
Cosa divertente: $\lim_{n\to \infty} V_n(R)=0=\lim_{n\to \infty}A_{n-1}(R)$ cosa inaspettata (me li aspettavo divergenti)\\
Torniamo alle identità di Green. Inserisco (135) in (134), e faccio il $\lim_{\epsilon\to 0}\Rightarrow$ ciò che resta della (134) è 
$$
K_n\int_U |\x - \y|^{2-n}fd^nx=-u(y) - K_n\int_{\partial U}\sum_i \left(u\dfrac{\partial}{\partial x_i}|\x - \y|^{2-n} - |\x - \y|^{2-n} \dfrac{\partial u}{\partial x_i} \right)n_i dV
$$
$dV$ è "l'elemento di volume" indotto su $\partial U \Rightarrow$ sarebbe in realtà una superficie in n-1 dimensioni
\begin{equation}
\Rightarrow u(y)=-K_n\int_U |\x - \y|^{2-n}f d^n x - K_n\int_{\partial U} \left(u\dfrac{\partial}{\partial n}|\x - \y|^{2-n} - |\x - \y|^{2-n} \dfrac{\partial u}{\partial n} \right)dV (144)
\end{equation}
Formula generale del potenziale (III identità di Green)\\
$\Rightarrow$ importante in elettrostatica/dinamica, per esemio da soluzione dell'equazioni di Poisson (u sarebbe il potenziale elettricom integro sul dominio la densità di carica; $\dfrac{\partial u}{\partial u}$ sarebbe il campo elettrico).\\
Questa formula ha alcune conseguenze: per esempio, sia $\varphi$ di supporto compatto e $\y=0$. Sia $\partial U$ fuori dal compatto in cui $\varphi \neq 0$ (integrale sul bordo non contribuirà)\\
(144) ci dà: 
$$
\varphi(0)=-K_n\int |\x|^{2-n}\Delta \varphi d^n x=<\delta,\varphi>=\int \delta(\x)\varphi(\x)d^n x
$$
Integro per parti due volte, non ho contributo di bordo perchè il supporto è compatto)
$$
-K_n\int\varphi \Delta |\x|^{2-n}d^nx=\int \varphi(\x)\delta(\x) d^nx
$$
devo quindi avere
\begin{equation}
\Delta(\underset{\text{funzione di Green del }\Delta}{\underbrace{-K_n|\x|^{2-n}}})=\delta(\x) (145)
\end{equation}
Abbiamo concluso la parte sulle equazioni lineari! Adesso parleremo delle equazioni non lineari (importanti perchè il nostro mondo è altamente non lineare).

\section{Equazioni differenziali non lineari}
I metodo per risolverle: metodo delle caratteristiche (o Lagrange-Charpit).\\
Considero equazioni quasi lineari del I ordine (dipendenza da U ma non dalla derivata)
\begin{equation}
a(x,y,u)u_x + b(x,y,u)u_y + c(x,y,u)=0 (146)
\end{equation}
Suppongo di conoscere una soluzione $u(x,y)$ e considero la curva $(x(t),y(t))$ che soddisfa
\begin{equation}
\begin{matrix}
\dfrac{dx}{dt}=a(x(t),y(t),u(x(t),y(t))) \\
\dfrac{dy}{dt}=b(x(t),y(t),u(x(t),y(t)))
\end{matrix} (147)
\end{equation}
Lungo questa curva ho 
\begin{equation}
\dfrac{d}{dt}u(x(t),y(t))=u_x \dfrac{dx}{dt} + u_y\dfrac{dy}{dt}\overset{(147)}{=}au_x + bu_y \overset{(146)}{=}-c (148)
\end{equation}
La curva $(x(t),y(t))$ è una curva caratteristica.\\
Dimostrazione: le curve caratteristiche sono date da 
\begin{equation}
S(x,y)=0,\quad a\dfrac{\partial S}{\partial x} + b\dfrac{\partial S}{\partial y}=0 (149)
\end{equation}
$\Rightarrow$ soddisfa (127): $\sum_{|j|=m} a_jp^j =0$, con $p^j=\left(\dfrac{\partial S}{\partial x_1}\right)^{j_1}\cdots\left(\dfrac{\partial S}{\partial x_n}\right)^{j_n}$ ( in questo caso $m=1,n=2,x_{1,2}=x,y$). Lungo essa 
\begin{equation}
S=0 \Rightarrow dS=0 \Rightarrow \dfrac{\partial S}{\partial x}\dot{x}+\dfrac{\partial S}{\partial y}\dot{y} =0 (150)
\end{equation}
Da (149), il vettore $(a,b) \perp \underset{\nabla S}{\underbrace{(S_x,S_y)}}$; d'altra parte, da (150) segue che anche $(\dot{x},\dot{y}) \perp \underset{\nabla S}{\underbrace{(S_x,S_y)}}$ e quindi $(a,b)\propto (\dot{x},\dot{y})$. Riparametrizzo la curva per assorbire il fattore di proporzionalità, cioè in modo che $(a,b)=(\dot{x},\dot{y})$ $\left( \text{N.B. :} \dfrac{dx}{dt}=\dfrac{dx}{dt'}\dfrac{dt'}{dt}\right)$
%immagine curva caratteristica con x.,y. e Sx Sy
Le (147) sono soddisfatte. $\qquad \blacksquare$\\
Tutto ciò suggerisce un metodo per risolvere (146). Suppogo di avere una curva $\Gamma$ data da $(x_0(\lambda),y_0(\lambda))$ su cui sono definiti dei calori iniziali $U=U_0(\lambda)$. Supponiamo che sulla curva valga
\begin{equation}
\det \left(\begin{matrix}
\dfrac{dx_0}{ds} & a \\
\dfrac{dy_0}{ds} & b
\end{matrix}\right)\neq 0(151)
\end{equation}
cioè $\Gamma$ non è caratteritica (altrimenti $\nabla S \propto (a,b)$ e potrei riparametrizzare come prima). $\forall s$, risolvo 
\begin{equation}
\begin{matrix}
\dfrac{dx}{dt}=a(x,y,u) \\
\dfrac{dy}{dt}=b(x,t,u) \\
\dfrac{du}{dt}=-c(x,y,u)
\end{matrix} (152)
\end{equation}
con condizioni iniziali
\begin{equation}
\begin{matrix}
x(t=0,s)=x_0(s) \\
y(t=0,s)=y_0(s) \\
u(t=0,s)=u_0(s) \\
\end{matrix} (153)
\end{equation}
Dal parametro s dipende la soluzione (vedi esempio dopo). Ho espresso $x,y,u$ in termini di $t,s$.
(151) $\underset{\underset{\text{teo. funzione inversa}}{\uparrow}}{\Rightarrow}$ posso esprimere  t,s in funzione di x,y in un intorno di $\Gamma$.\\
Controllo che questo metodo dia soluzioni di (146)
$$
\dfrac{\partial}{\partial t}u(x(t,s),y(t,s)=u_x \dfrac{\partial x}{\partial t} + u_y \dfrac{\partial y}{\partial t}\overset{(152)}{=}au_x + bu_y
$$
d'altra parte
$$
(152) \Rightarrow  \dfrac{\partial }{\partial t} u(x(t,s),y(t,s))=-c
$$
$\Rightarrow$ soddisfatta (146).\\
Esempio per chiarire: $xu_x + (x+y)u_y = u+1$ (lineare). \\
Come curva, uso $u(x,0)=x^2$. Quali sono le equazioni caratteristiche (!52)?
$$
\Rightarrow \dot{x}(=a)=x; \quad \dot{y}(=b)=x+y; \quad \dot{u}(=-c)=u+1
$$
N.B.: ho trasformato in equazioni differenziali ordinarie, al prezzo di averne 3 al posto di 1.
%immagine curva su cui prescrivo la u=x2, testo a fianco
Curva iniziale può essere parametrizzzata nella forma $x_0(s)=s, y_0(s)=0, u_0(s)=s^2$. Per vedere che la curva sia caratteristica calcolo
$$
\left|\begin{matrix}
\dfrac{dx_0}{ds} & a \\
\dfrac{dy_0}{ds} & b
\end{matrix}\right| = \left| \begin{matrix}
1 & x_0 \\
0 & x_0+y_0
\end{matrix} \right| = x_0 + y_0 = x_0 \neq 0 \Rightarrow \text{ok}
$$
Integrando l'equazione per x, ho $\dot{x}=x \Rightarrow x=Ce^t, x(t=0):=x_0 =s \Rightarrow C=s \Rightarrow x=se^t$. Poi ho $\dot{y}-y=x=se^t$.\\
La soluzione generale dell'equazione omogenea è $y=Ae^t$; per ricavare quella particolare della non omogenea, uso il metodo di variazioen delle costanti: scrivo $y=A(t)e^t \Rightarrow \dot{y}=\dot{A}e^t + Ae^t := y+se^t = Ae^t + se^t \Rightarrow \dot{A}=s \Rightarrow A=st$. La soluzione generale dell'equazione non omogenea diventa $y=ce^t + st e^t$.\\
impongo che $y(t=0,s)=y_0(s)=0 \Rightarrow c=0 \Rightarrow y=ste^t$. Infine risolvo $\dot{u}=u+1\Rightarrow \dfrac{du}{u+1}=dt \Rightarrow ln(u+1)=t+\ln C \Rightarrow \exp (u+1) = Ce^t$ e impongo condizioni iniziali $u(t=0,s)=u_0(s)=s^2 \Rightarrow c-1=s^2 \Rightarrow u=(s^2 + 1)e^t -1$. Ora esprimo t,s in funzione di x,y: $y=ste^t =tx \Rightarrow t=y/x$ e quindi $s=xe^{-t}=xe^{-y/x}$. Sostituisco nell'espressione per u: $u(x,y)=x^2e^{-y/x}+e^{-y/x}-1$\\
%%%%%%%%%%%%%%%%%%%%%%%%%%%%%%%%%%%%%%%%%%%%
%% inizio lezione 30 novembre
%%%%%%%%%%%%%%%%%%%%%%%%%%%%%%%%%%%%%%%%%%%%
\underline{Compito}: generalizzare il metodo descritto sopra al problema
$$
\sum_{i=1}^n a_i(\x,u)u_{x_i} + c(\x,u)=0
$$
con $\x = (x_1,\dots,x_n)$, e i dati iniziali sono dati nella forma $\x=\x_0(\bar{s}), u=u_0(\bar{s})$ con $\bar{s}=(s_1,\dots,s_{n-1})$. L'equazione per $\x$ defnisce un'ipersuperficie di dimensione n-1 in $\Rn$.\\
-Ora vorremmo generalizzare questa costruzione al problema non-quasilineare
\begin{equation}
F(x,y,u,p,q)=0, \quad p=u_x, q=u_y (154)
\end{equation}
Assumiamo che F sia una funzione $C^1$ dei suoi argomenti. Sia u una funzione $C^2$ che risolve la (154). Vorremmo trovare un sistema caratteristico di equazioni differenziali ordinarie simile a (152).\\
A tal fine poniamo
\begin{equation}
\dfrac{dx}{dt}=\dfrac{\partial F}{\partial p}=F_p, \quad \dfrac{dy}{dt}=\dfrac{\partial F}{\partial q}=F_q (155)
\end{equation}
(Applicando questo alla (146), cioé a $a(x,y,u)p + b(x,y,u)q + c(x,y,u)=0$, dà le (152).)\\
equazione per u:
\begin{equation}
\dfrac{d}{dt}u(x(t),y(t)) = u_x \dfrac{dx}{dt}+u_y\dfrac{dy}{dt}=p F_p + q F_q (156)
\end{equation}
(Applicando alla (146), questo da $\dfrac{d}{dt}u(x(t),y(t))=-c$, che è l'ultima delle (152).)\\
Vorremmo usare (155),(156) come sistema caratteristico.\\
problema: questo sistema (contrariamente alle (152)) non dipende solo da x,y,u, ma anche da $u_x$ e $u_y$. \\
$\rightarrow$ servono delle equazioni per $p=u_x$ e $q=u_y$.\\
Si ha:
\begin{equation}
\dfrac{dp}{dt}=\dfrac{d}{dt}u_x = u_{xx}\dfrac{dx}{dt} + u_{xy}\dfrac{dy}{dt}=u_{xx}F_p + u_{xy}F_q (157)
\end{equation}
deriva la $\underset{F(x,y,u,p,q)}{\underbrace{(154)}}$ rispetto a x:
\begin{equation}
\Rightarrow F_x + F_u u_x + F_p u_{xx} + F_q u_{xy}=0 (158)
\end{equation}
Usando la (158), la (157) diventa
\begin{equation}
\dfrac{dp}{dt}=-F_x - p F_u (159)
\end{equation}
In modo molto simile (compito!)
\begin{equation}
\dfrac{dq}{dt}=-F_y - qF_u (159')
\end{equation}
$\rightarrow$ abbiamo il seguente sistema chiuso di equazioni differenziali ordinarie:
\begin{equation}
\begin{matrix}
\dfrac{dx}{dt}= F_p,\quad \dfrac{dy}{dt}=F_q, \quad \dfrac{du}{dt}=pF_p + qF_q,\\
\dfrac{dp}{dt}= - p F_u - F_x, \quad \dfrac{dq}{dt}=-q F_u - F_y 
\end{matrix}(160)
\end{equation}
"Sistema di Lagrange-Charpit"\\
$\rightarrow$ seguente metodo per risolvere la (154):\\
Sia data una curva $\Gamma$ non caratteristica $(x_0(s),y_0(s))$, assieme a dei valori $u=u_0(s)$ su $\Gamma$.\\
Contrariamente al caso qusilineare abbiamo bisogno di condizioni iniziali $p=p_0(s)$ e $q=q_0(s)$ per risolvere le (160). Le condizioni iniziali devono risolvere la (154): 
\begin{equation}
F(x_0,y_0,u_0,p_0,q_0)=0 (161)
\end{equation}
Serve un'altra condizione per determinare $p_0$, e $q_0$. Per ottenerla, notiamo che la soluzione $u(x,y)$ della (!54) soddisfa $\dfrac{d}{ds}u(x_0(s), y_0(s))=u_x\dfrac{dx_0}{ds}+u_y\dfrac{dy_s}{ds} \Rightarrow$ richiediamo che $p_0, q_0$ soddisfino anche
\begin{equation}
\dfrac{du_0}{ds}=p_0(s)\dfrac{dx_0}{ds}+ q_0(s)\dfrac{dy_0}{ds} (162)
\end{equation}

(161) e (162) forniscono le funzioni iniziali $p_0(S), q_0(s)$.\\
Richiediamo inoltre
\begin{equation}
\left| \begin{matrix}
x'_0(s) & F_p\\
y'_0(s) & F_q
\end{matrix}\right| \neq 0 \text{ su } \Gamma (163)
\end{equation}
(vedremo sotto a cosa serve)\\
Risolvi le (160) con condizioni iniziali date su $\Gamma$.\\
$\rightarrow$ produce le funzioni $x(t,s), y(t,s), u(t,s), p(t,s), q(t,s)$\\
Grazie alla (163) possiamo invertire $x(t,s), y(t,s)$ per ottenere $t(x,y), s(x,y)$ in un intorno di $\Gamma$. Sostituito in $u(t,s)$ da una soluzione $u(x,y)$.\\
Dimostriamo che questa ricetta fornisce infatti una soluzione della (154):\\
Dimostriamo prima che 
$$
G(t,s) \equiv F(x(t,s),y(t,s),u(t,s),p(t,s),q(t,s))=0
$$
N.B. : $G(0,s)=0$ a causa della $\underset{F(x_0,y_0,u_0,p_0,q_0)=0}{\underbrace{(161)}}$
$$
\dfrac{\partial G}{\partial t} = F_x \dfrac{\partial x}{\partial t} + F_y \dfrac{\partial y}{\partial t} + F_u \dfrac{\partial u}{\partial t} + F_p \dfrac{\partial p}{\partial t} + F_q \dfrac{\partial q}{\partial t}
$$
$$
\overset{(160)}{=}F_xF_p + F_y F_q + F_u(pF_p + qF_q) - F_p(F_x + pF_u) - F_q(F_y + qF_u)=0
$$
$$
\Rightarrow G(t,s)=0
$$
Prossimo passo: dobbiamo dimostrare che 
$$
p(x,y)=\dfrac{\partial u(x,y)}{\partial x}, \quad q(x,y)=\dfrac{\partial u(x,y)}{\partial y}
$$
A tal fine dimostriamo che 
\begin{equation}
H(t,s)=\dfrac{\partial u}{\partial s} - p(t,s)\dfrac{\partial x}{\partial s} - q(t,s) \dfrac{\partial y}{\partial s} =0 (164)
\end{equation}
(il motivo diventerà chiaro fra qualche istante\dots)\\
N.B. $H(0,s)=0$ per la $\underset{\dfrac{du_0}{ds} = p_0(s)\dfrac{dx_0}{ds} + q_0(s) \dfrac{dy_0}{ds}}{\underbrace{(162)}}$
$$
\dfrac{\partial H}{\partial t} = \dfrac{\partial^2 u}{\partial t \partial s} + \dfrac{\partial p}{\partial t}\dfrac{\partial x}{\partial s} - p\dfrac{\partial^2 x}{\partial t \partial s} + \dfrac{\partial q}{\partial t}\dfrac{\partial y}{\partial s} - q\dfrac{\partial^2 y}{\partial t \partial s}
$$
$$
\overset{(160)}{=} \dfrac{\partial}{\partial s}(pF_p + qF_q) + (F_x pF_u)\dfrac{\partial x}{\partial s} - p\dfrac{\partial}{\partial s}F_p + (F_y + qF_u)\dfrac{\partial y}{\partial s} - q \dfrac{\partial}{\partial s} F_q 
$$
\begin{equation}
=F_p \dfrac{\partial p}{\partial s} + F_q \dfrac{\partial q}{\partial s} + (F_x + pF_u)\dfrac{\partial x}{\partial s} + (F_y + qF_u)\dfrac{\partial y}{\partial s} (165)
\end{equation}
Inoltre: $G(t,s)=0 \forall (t,s)$
$$
\Rightarrow \dfrac{\partial G}{\partial s }=F_x \dfrac{\partial x}{\partial s} + F_y \dfrac{\partial y}{\partial s} + F_u \dfrac{\partial u}{\partial s} + F_p\dfrac{\partial p}{\partial s} + F_q \dfrac{\partial q}{ \partial s}=0
$$
Usando questa nella (165) otteniamo
$$
\dfrac{\partial H}{\partial t} = - F_u\left(\dfrac{\partial u}{\partial s} - p\dfrac{\partial x}{\partial s} - q\dfrac{\partial y}{\partial s}\right)=-F_u H
$$
La soluzione di questa equazione differenziale ordinaria con condizione iniziale $H(0,s)=0$ è $H(t,s)=0$
[Dimostrazione in appendice
$$
\dfrac{\dfrac{\partial H}{\partial t}}{H}=-F_u \Rightarrow \dfrac{\partial}{\partial t}\ln H = -F_u
$$
$$
\Rightarrow \ln H(t) = -\int F_u(s,t)dt + \ln C
$$
$$
\Rightarrow H(t,s)=C\underset{=:h(t,s)}{\underbrace{\exp (-\int F_u dt)}}
$$
$$
\Rightarrow H(0,s)= C h(0,s) \overset{!}{=}0 \Rightarrow C=0 \Rightarrow H(t,s)=0]
$$
Dobbiamo ancora far vedere che $H(t,s)=0$ implica $p=u_x, q=u_y$. A tal fine nota che :
$$
\dfrac{\partial u}{\partial t}\overset{(160)}{=} p F_p + q F_q \overset{(160)}{=} p\dfrac{\partial x}{\partial t} + q \dfrac{\partial y}{\partial t}
$$
Questa, assieme alla (164) dimostra che 
$$
\left(\begin{matrix}
\dfrac{\partial u}{\partial t}\\
\dfrac{\partial u}{\partial s}
\end{matrix}\right) = J \cdot \left(\begin{matrix}
p\\
q
\end{matrix}\right)
$$
con
\begin{equation}
J=\left(\begin{matrix}
\dfrac{\partial x}{\partial t} & \dfrac{\partial y}{\partial t} \\
\dfrac{\partial x}{\partial s} & \dfrac{\partial y}{\partial s}
\end{matrix}\right) (166)
\end{equation}
Considerando u una funzione di x e , abbiamo 
\begin{equation}
J\left(\begin{matrix}
\dfrac{\partial u}{\partial x} \\
\dfrac{\partial u}{\partial y}
\end{matrix}\right)=\left(\begin{matrix}
\dfrac{\partial u}{\partial t} \\
\dfrac{\partial u}{\partial s}
\end{matrix}\right) (167)
\end{equation}
[ p.e. prima riga: $\dfrac{\partial x}{\partial t}\dfrac{\partial u}{\partial x} + \dfrac{\partial y}{\partial t}\dfrac{\partial u}{\partial y}=\dfrac{\partial u}{\partial t} \quad \checkmark$]
$$
(166)=\left(\begin{matrix}
p\\
q
\end{matrix}\right) = \underset{\text{è invertibile a causa della (163)}}{\underbrace{J^{-1}}} \cdot \left(\begin{matrix}
\dfrac{\partial u}{\partial t}\\
\dfrac{\partial u}{\partial s}
\end{matrix}\right)\overset{(167)}{=}\left(\begin{matrix}
\dfrac{\partial u}{\partial x} \\
\dfrac{\partial u}{\partial y}
\end{matrix}\right) q.e.d.
$$
N.B. eliminando il parammetro t dalle equazioni di Lagrange-Charpit (160) si ottiene
\begin{equation}
\dfrac{dx}{F_p}=\dfrac{dy}{F_q}=\dfrac{du}{pF_p + qF_q}=\dfrac{dp}{-F_x - pF_u} = \dfrac{dq}{-F_y - qF_u} (168)
\end{equation}
A volte le equazioni si vedono scritte in questa forma.\\
-Generalizzazione al caso di n variabili:
\begin{equation}
F(x_1,\dots,x_n,u,p_1,\dots,p_n)=0 (169)
\end{equation}
(prima n=2, $p_1=1,p_2=q,x_1=x,x_2=y$)
\begin{equation}
\Rightarrow \begin{matrix}
p_i=\dfrac{\partial u}{\partial x_i} \\
\dot{x_i} = F_{p_i} \\
\dot{p_i}=-F_{x_i} - p_i F_u \\
\dot{u}=\sum_{i=1}^n p_i F_{p_i}
\end{matrix} (170)
\end{equation}
oppure 
$$
\dfrac{dx_i}{F_{p_i}}=-\dfrac{dp_i}{F_{x_i} + p_iF_u}=\dfrac{du}{\sum_{j=1}^n p_j F_{p_j}}
$$
Se u è data su una superficie iniziale, $u=u_0(\bar{s})$, per $\x = \x_0(\bar{s})$ dove $\bar{s}=(S_1,\dots,s_{n-1}), \x=(x_1,\dots,x_n)$, le condizioni iniziali per $\bar{p}=(p_1,\dots,p_n)$ sono le soluzioni del sistema
\begin{equation}
\begin{matrix}
F(\x_0(\bar{s}),u_0(\bar{s}),p_0(\bar{s})=0\\
\dfrac{\partial u_0(\bar{s}}{\partial s_j}=\sum_{k=1}^n p_{k,0}(\bar{s})\dfrac{\partial x_{k,0}(\bar{s})}{\partial s_j}, \quad j=1,\dots,n-1
\end{matrix} (170')
\end{equation}
che generalizzano le (161),(162).\\
%%%%%%%%%%%%%%%%%%%%%%%%%%%%%%%%%%%%%%
%% inizio lezione 6/12
%%%%%%%%%%%%%%%%%%%%%%%%%%%%%%%%%%%%%%
\underline{Esempi per Lagrange-Charpit}:\\
i) $u=u_x^2 - 3u_y^2$, $u(x,0)=x^2$
$$
\Rightarrow F(x,y,u,p,q)=p^2-3q^2-u, \qquad F_x=0=F_y, \quad F_p=2p, \quad F_q=-6q, F_u-1
$$
Sia $x_0(s)=s \Rightarrow y_0(s)=0, u_0(s)=s^2$, $p_0(s),q_0(s)$ sono soluzioni del sistema
\begin{equation}
\begin{matrix}
p_0^2 - 3q_0^2=u_0=s^2 (vedi (161))\\
\dfrac{du_0}{ds}=p_0 \dfrac{dx_0}{ds}+q_0\dfrac{dy_0}{ds}(162)
\end{matrix}
\end{equation}
$$
\begin{matrix}
\Rightarrow 2s=p_0\\
\Rightarrow 4s^2 - 3q_0^2 = s^2 \Rightarrow q_0=\pm s
\end{matrix}
$$
Prendiamo per esempio $q_0=s$
$$
(160): \dot{x}=2p, \dot{y}=-6q, \dot{u}=2p^2 - 6q^2 = 2(p^2-3q^2)=2u, \dot{p}=p, \dot{q}=q
$$
$$
\dot{u}=2u \Rightarrow u=s^2e^{2t}
$$
$$
\dot{p}=p \Rightarrow p=2se^t
$$
$$
\dot{q}=q \Rightarrow q=se^t
$$
Sostituisci nelle equazioni caratteristiche per x e y
$$
\dot{x}=2p=4se^t \Rightarrow x= 4s(e^t -1) +s
$$
Simile:
$$
y=-6s(e^t -1)
$$
Invertire:
$$
e^t-1=-\dfrac{y}{6s} \Rightarrow x=4s\left(-\dfrac{y}{6s}\right) + s=-\dfrac{2}{3}y + s \Rightarrow s=x+\dfrac{2}{3}y
$$
$$
e^t=1-\dfrac{y}{6s}=1-\dfrac{y}{6x+4y} 
$$
$$
\Rightarrow u=s^2 e^{2t}= \left(x+\dfrac{2}{3}y\right)^2\left(1-\dfrac{y}{6x+4y}\right)^2=\left(x+\dfrac{y}{2}\right)^2
$$
Se avessimo scelto $q_0(S)=-s$, avremmo ottenuto $u=\left(x-\dfrac{y}{2}\right)^2$\\
ii)
$$
u_x^2 - u_y^2 = x^2 + y^2, u(x,0)=x \Rightarrow F = p^2 - q^2 - x^2 + y
$$
$$
F_p=2p, F_q=2q, F_x = -2x, F_y=1, F_u=0
$$
$$
x_0(s)=s, y_0(s)=0, u_0(s)=s, p_0^2 - q_0^2 = s^2
$$
$$
\dfrac{du_0}{ds}=p_0\dfrac{dx_0}{ds}+q_0\dfrac{dy_0}{ds} \Rightarrow 1=p_0
$$
$$
\Rightarrow q_0^2=1-s^2 \Rightarrow \text{ prendi per esempio } q_0=\sqrt{1-s^2}
$$
$$
(160): \dot{x}=2p, \dot{y}=-2q, \dot{u}=2p^2 - 2q^2, \dot{p}=2x, \dot{q}=-1
$$
$$
q=-t+\sqrt{1-s^2}
$$
$$
\dot{y}=-2q=2t-2\sqrt{1-s^2} \Rightarrow y= t^2-2t\sqrt{1-s^2}
$$
$$
\ddot{p}=2\dot{x}=4p \Rightarrow p=Ae^{2t} + Be^{-2t} \Rightarrow \dot{p}=2Ae^{2t}-2Be^{-2t}=2x \Rightarrow x=Ae^{2t}-Be^{-2t}
$$
$$
p_0=A+B=1, x_0=A-B=s \Rightarrow A=\dfrac{1+s}{2}, \\B=\dfrac{1-s}{2}
$$
$$
p=\dfrac{1+s}{2}e^{2t}+\dfrac{1-s}{2}e^{-2t}, x=\dfrac{1+s}{2}e^{2t}-\dfrac{1-s}{2}e^{-2t}
$$
$$
\dot{u}=2(p^2-q^2)=2(x^2-y) = \dfrac{(1+s)^2}{2}e^{4t} - (1-s)^2+\dfrac{(1-s)^2}{2}e^{-4t}-2t^2 + 4t\sqrt{1-s^2}
$$
$$
\Rightarrow u=\dfrac{(1+s)^2}{8}e^{4t} - t(1-s^2) - \dfrac{(1-s)^2}{8}e^{-4t}-\dfrac{2}{3}t^3 + 2t^2\sqrt{1-s^2}+\dfrac{s}{2}
$$
Invertire: $\rightarrow$ sono equazioni trascendenti, è possilile dare la soluzione in forma parametrica, $u=u(t,s), x=x(t,s), y=y(t,s)$.\\
iii)
$$
u_xu_y=1, u(x,0)=\ln x \rightarrow esercizio
$$
$$
\left(soluzione: u(x,y)=-1+\sqrt{1+4xy}+\ln \left(\dfrac{2x}{1+\sqrt{1+4xy}}\right)\right)
$$
iv) esempio di equazione senza parametro t
$$
u=pqxy \Rightarrow F=pqxy-u
$$
$$
(168)\Rightarrow \dfrac{dx}{qxy}=\dfrac{dy}{pxy} =\dfrac{du}{2pqxy}=\dfrac{dp}{-pqy+p}=\dfrac{dq}{-pqx+q}
$$
(non c'è una ricetta standard per risolvere questo tipo di equazioni)\\
osservo che $2pqxy=2u$
$$
\dfrac{dx}{qxy}=\dfrac{dp}{p(1-qy)}\Rightarrow \dfrac{d\ln p}{d\ln x})=\dfrac{1}{qy}-1 (*)
$$
$$
\dfrac{dy}{pxy}=\dfrac{dq}{q(1-px)}\Rightarrow \dfrac{s \ln q}{d \ln y}=\dfrac{1}{px}-1 (**)
$$
$$
\dfrac{dx}{qxy}=\dfrac{du}{2u}\overset{(*)}{\Rightarrow} \left(\dfrac{d\ln p}{d\ln x}+1\right)d\ln x = \dfrac{du}{2u}
$$
[conti espliciti
$$
\text{sx: }\dfrac{dx}{x}\cdot \dfrac{1}{qy}\overset{(*)}{=}\dfrac{dx}{x}\left(\dfrac{d\ln p}{d\ln x}+1\right)=d\ln x\left(\dfrac{d\ln p}{d\ln x}+1\right)
$$
]
$$
\Rightarrow d\ln p + d\ln x=\dfrac{1}{2}d\ln u \Rightarrow xp Cu^{\dfrac{1}{2}} (***)
$$
$$
\dfrac{dy}{pxy}=\dfrac{du}{2u}\overset{(**)}{\Rightarrow}\left(\dfrac{d\ln q}{d\ln y}+1\right)d\ln y=\dfrac{1}{2}d\ln u
$$
$$
\Rightarrow yq=\tilde{C}u^{\dfrac{1}{2}} (****)
$$
$$
pqxy=u \Rightarrow \tilde{C}=\dfrac{1}{C}
$$
$$
(***) \Rightarrow xu_x = Cu^{\dfrac{1}{2}}\overset{\text{separando variabili}}{\Rightarrow}\ln x=\dfrac{2}{C}\sqrt{u}-\underset{\text{"cost." di integrazione}}{\ln f(y)}
$$
In modo analogo:
$$
(****)\Rightarrow \ln y = 2C \sqrt{u} - \underset{\text{"cost." di integrazione}}{\ln g(x)}
$$
$$
\Rightarrow 2\sqrt{u} = C\ln (xf(y) = \dfrac{1}{C}\ln (yg(x)) \Rightarrow C \underset{\text{=cost. }\equiv \ln a}{\underbrace{\ln x - \dfrac{1}{c}\ln g(x)}} = \underset{\ln a}{\underbrace{\dfrac{1}{C}\ln y - C\ln f(y)}}
$$
$$
\ln \dfrac{x^C}{g(x)^{1/C}}=\ln a, \ln \dfrac{y^{1/C}}{f(y)^C} = \ln a
$$
$$
g(x)=\dfrac{x^{C^2}}{a^C}, f(y)=\dfrac{y^{1/C^2}}{a^{1/C}}
$$
$$
\Rightarrow 2\sqrt{u}=\dfrac{1}{C} (\ln y + \ln g(x)) = \dfrac{1}{C}(\ln y + C^2 \ln x - C \ln a)
$$
$$
\Rightarrow u=\dfrac{1}{4}\left(C\ln x + \dfrac{1}{C}\ln y + b\right)^2
$$
i dati iniziali sono contenuti nelle due costanti di integrazione.\\
Commento: equazioni di Hamilton-Jacobi
$$
\underset{\text{"tempo"}}{\dfrac{\partial S}{\partial x_1}}+H(\underset{\text{coord. spaziali}}{\underbrace{x_1,\dots,x_n}};\underset{\text{impulsi}}{\underbrace{\dfrac{\partial S}{\partial x_2}, \dots,\dfrac{\partial S}{\partial x_n}}}; x_1)=0
$$
S: funzione principale di Hamilton, corrisponde a $F= p_1 + H(x_1,x_2,\dots,x_n;p_2,\dots,p_n)$
$$
\Rightarrow F_{p_1}=1, F_{p_j}=\dfrac{\partial H}{\partial p_j} \quad (j=2,\dots,n), F_{x_i}=\dfrac{\partial H}{\partial x_i} \quad (i=1,\dots,n), F_u=0 \quad (u\equiv S)
$$
$$
(170) \Rightarrow \dot{x_1}=1, \dot{p_1}=-F_{x_1}-p_1F_u = -\dfrac{\partial H}{\partial x_1}
$$
$$
\left.\begin{matrix}
\dot{x_j}=\dfrac{\partial H}{\partial p_j}, \quad j=2,\dots,n \\
\dot{p_j}=-F_{x_j}-p_jF_u=-\dfrac{\partial H}{\partial x_j}, j=2,\dots,n
\end{matrix}\right\} \text{equazioni canoniche di Hamilton}
$$
$$
\dot{u}=\underset{-H}{\underbrace{p_1}}\underset{1}{\underbrace{F_{p_1}}} + \sum_{j=2}^n p_j \underset{\dfrac{\partial H}{\partial p_j}=\dot{x_j}}{\underbrace{F_{p_j}}} 
$$
$$
\Rightarrow \dot{u}=-H + \sum_{j=2}^n p_j\dot{x}_j=L \quad \text{(Lagrangiana)}
$$
$$
\Rightarrow u=\int L dt \quad \text{azione}
$$
\section{L'esponenziale ordinato cronologicamente}
- Vogliamo risolvere 
\begin{equation}
i\hbar \dfrac{d}{dt}\psi = H \psi (171a)
\end{equation}
(che include equazioni difficili matriciali del tipo 
\begin{equation}
\dot{\x}(t)=At)\x(t) (171b)
\end{equation}
con A matrice finito o infinito dimensionale
\begin{equation}
\psi(t)=\underset{\text{operatore di evoluzione temporale}}{\underset{\uparrow}{U(t,t_0)}}\psi(t_0) (172)
\end{equation}
$$
\Rightarrow i\hbar \dfrac{d}{dt}\psi = i\hbar \dfrac{\partial}{\partial t}U(t,t_0)\psi(t_0)\overset{!}{=}H\psi = HU(t,t_0)\psi(t_0)
$$
\begin{equation}
\Rightarrow i\hbar \dfrac{\partial }{\partial t}U(t,t_0)=HU(t,t_0) (173)
\end{equation}
i) H non dipende esplicitamente dal tempo: \\
Soluzione della (173):
$$
U(t,t_0)=\exp \left(-\dfrac{i}{\hbar}(t-t_0)H\right)
$$
ii) H dipende esplicitamente dal tempo ma $[H(t),H(t')]=0$:\\
(per esempio campo magnetico omogeneo $\bar{B}=B(t)\bar{e_z}$)
$$
\Rightarrow U(t,t_0)=\exp \left(-\dfrac{i}{\hbar}\int_{t_0}^t dt' H(t')\right)
$$
iii) $H=H(t)$ e $[H(t),H(t')]\neq 0$ per $t\neq t'$:\\
(caso piu interessante e piu realistico)\\
-Riscrivi la (173) come equazione integrale:
\begin{equation}
U(t,t_0) = I - \dfrac{i}{\hbar}\int_{t_0}^t dt' H(t')U(t',t_0) (174)
\end{equation}
($\rightarrow$ condizione iniziale $U(t,t_0)=1$ già implementata).\\
Risolvi la (174) iterativamente: \\
Sostituisci il membro destro in se stesso:
\begin{equation}
\begin{matrix}
\Rightarrow U(t,t_0)=\sum_{n=0}^\infty U^{(n)}(t,t_0)\\
U^{(n)}(t,t_0)=-\dfrac{i}{\hbar}\int dt' H(t') U^{(n-1)}(t',t_0)\\
u^{(0)}(t,t_0)=I
\end{matrix} (175)
\end{equation}

%%%%%%%%%%%%%%%%%%%%%%%%%%%%%%%%%%%%%%%%%%%%%
%% inizio lezione 13/12
%%%%%%%%%%%%%%%%%%%%%%%%%%%%%%%%%%%%%%%%%%%%%

$\rightarrow$ relazione di ricorrenz, risolta da
\begin{equation}
U^{(n)}(t,t_0)=\left(-\dfrac{i}{\hbar}\right)^n\int_{t_0}^tdt_1 \int_{t_0}^{t_1}dt_2 \cdots \int_{t_0}^{t_{n-1}}dt_n \cdot H(t_1)H(t_2)\cdot \dots \cdot H(t_n) (176)
\end{equation}
N.B. :1) i limiti superiori degli integrali sono tutti diversi\\
2) l'ordine delle H non è arbitrario. Abbiamo $t_n<t_{n-1}<\dots <t_1<t$\\
-Lo sviluppo 
$$
U(t,t_0)=1+\sum_{n=1}^\infty \left(-\dfrac{i}{\hbar}\right)^n \int_{t_0}^tdt_1 \int_{t_0}^{t_1}dt_2 \cdots \int_{t_0}^{t_{n-1}}dt_n \cdot H(t_1)\cdot \dots \cdot H(t_n)
$$
si chiama \underline{serie di Dyson}.\\
\underline{L'esponenziale ordinato cronologicamente} è definito da
\begin{equation}
T\exp\left(-\dfrac{i}{\hbar}\int_{t_0}^t dt' H(t')\right):=1+\sum_{n=1}^\infty \left(-\dfrac{i}{\hbar}\right)^n \int_{t_0}^tdt_1 \int_{t_0}^{t_1}dt_2 \cdots \int_{t_0}^{t_{n-1}}dt_n \cdot H(t_1)\cdot \dots \cdot H(t_n) = U(t,t_0) (178)
\end{equation}

%soluzione di problema di compito
$$
u_{xx}-u_t =0, \qquad u(0,t)=A_n\sin nt ,\quad u_x(0,t)=0
$$
%immafine problema parabolico soluzione
Più in generale: poniamo $u(0,t)=g(t)$\\
Ansatz di separazione:
$$
u(x,t)=v(x)w(t)
$$
$$
u_{xx}=v''(x)w(t), \quad u_t=v(x)w'(t)
$$
$$
u_{xx}=u_t \Rightarrow v'' w=vw' \Rightarrow \underset{\text{dip. solo da x}}{\dfrac{v''}{v}} = \underset{\text{dip. solo da t}}{\dfrac{w'}{w}} = k \in C= cost.
$$
$$
v''=kv \Rightarrow v(x)=Ae^{\sqrt{k}x} + B e^{-\sqrt{k} x}
$$
$$
w'=kw \Rightarrow w(t)=Ce^{kt}
$$
$$
\Rightarrow u_k(x,t)=e^{kt}\left(A_k e^{\sqrt{k} x}  + B_k e^{-\sqrt{k}x}\right)
$$
(senza perdere generalità ho posto C=1 assorbendolo in A e B)
$ \Rightarrow$ soluzione generale: sovrapposizione delle $u_k(x,t)$
$$%int sarebbe simbolo di somma e integrale sovrapposto
u(x,t)=\int_{k\in C} e^{kt}\left(A_k e^{\sqrt{k} x}  + B_k e^{-\sqrt{k}x}\right) 
$$
$$
\Rightarrow u_x(x,t)=\int_{k\in C} e^{kt}\sqrt{k}\left(A_k e^{\sqrt{k} x}  - B_k e^{-\sqrt{k}x}\right) 
$$
$=0$ in $x=0$ se $A_k = B_k$
$$
\Rightarrow u(x,t)=\int_{k\in C} 2A_ke^{kt}\cosh \sqrt{k}x
$$
$$
u(0,t)=\int_{k\in C} 2A_ke^{kt}\overset{!}{=}g(t)=\dfrac{1}{2\pi}\int_{-\infty}^{\infty}e^{i\chi t}\underset{\text{trasf. di Fourier di g(t)}}{\underset{\uparrow}{\tilde{g}}}(\chi)d\chi
$$
$\Rightarrow$ k immaginaria, $k=i\chi$, $2A_k = \dfrac{\tilde{g}(\chi)}{2\pi}$
$$
\Rightarrow u(x,t) =\dfrac{1}{2\pi}\int_{-\infty}^{\infty}\tilde{g}(\chi)e^{i\chi t}\cosh(\sqrt{i\chi}x d\chi
$$
N.B. $g(t)=g(t)^* \Rightarrow \tilde{g}(\chi)=\tilde{g}(-\chi)^*$ questo garantisce la realtà di $u(x,t) (\rightarrow \text{ esercizio!})$\\
Nel nostro esempio: $g(t)=A_n\sin nt\rightarrow$ trasformata di Fourier:
$$
\tilde{g}(\varkappa)=\int_{-\infty}^{\infty}dt e^{-i\varkappa t}A_n \sin nt
$$
$$
\dfrac{A_n}{2i}\int_{-\infty}^{\infty}dt e^{-i\varkappa t}\left(e^{int}-e^{-int}\right)
$$
$$
=\dfrac{A_n}{2i}\underset{2\pi \delta(n-\varkappa)}{\int_{-\infty}^\infty dt e^{it(n-\varkappa)}}-\dfrac{A_n}{2i}\underset{2\pi \delta(n+\varkappa)}{\int_{-\infty}^\infty dt e^{-it(n+\varkappa)}}
$$
$$
=i\pi A_n\left(\delta(n+\varkappa)-\delta(n-\varkappa)\right)
$$
$$
\Rightarrow u(x,t)=\dfrac{iA_n}{2}\left(e^{-int}\cosh(\sqrt{-in}x) - e^{int}\cosh(\sqrt{in}x)\right)
$$
Senza perdere generalità: $n>0$ ($\sin(-nt)=-\sin(nt)$, riassorbile il - in $A_n$)\\
Usa 
$$
\sqrt{\pm in} =\dfrac{n}{2}(1\pm i)
$$
$$
\cosh(\alpha + \beta)=\cosh \alpha \cosh \beta + \sinh \alpha \sinh \beta
$$
$$
\sin(\alpha \pm \beta)=\sin \alpha \cos \beta \pm \cos \alpha \sin \beta 
$$
$$
\cosh(iz)=\cos z, \qquad \sinh(iz)=i\sin z
$$
$$
\Rightarrow u(x,t)=A_n\left[\cos nt \sinh \left(\sqrt{\dfrac{n}{2}}x \right)\sin \left(\sqrt{\dfrac{n}{2}}x \right) + \sin nt \cosh \left(\sqrt{\dfrac{n}{2}}x \right) \cos \left(\sqrt{\dfrac{n}{2}}x \right)\right]
$$
oppure
$$
u(x,t)=\dfrac{A_n}{2}\left[e^{\sqrt{\dfrac{n}{2}}x}\sin\left(nt+\sqrt{\dfrac{n}{2}}x\right) + e^{-\sqrt{\dfrac{n}{2}}x}\sin\left(nt-\sqrt{\dfrac{n}{2}}x  \right) \right], \quad \text{q.e.d.}
$$

\subsection{L'equazione di Korteweg - De Vries (KdV)}
\begin{equation}
\partial_t u + 6u\partial_xu+\partial^3_x u=0 (179)
\end{equation}
con $u=u(x,t), \quad x\in \R, t>0$\\
-descrive delle onde in acqua non troppo profonda, ha delle applicazioni in molti rami della fisica, p.e. sistemi integrabili, teoria delle stringhe $\dots$.\\
-equazione del 3 ordine\\
-ha delle soluzioni chiamate \underline{onde solitarie}\\
Ansatz: 
\begin{equation}
u(x,t)=f(x-\underset{\text{velocità di propagazione dell'onda}}{\underset{\uparrow}{c}}t) (180)
\end{equation}
nuova variabile $\xi:= x-ct$, $u(x,t)=f(\xi)$
$$
(179)= \dfrac{\partial \xi}{\partial t}\dfrac{df}{d\xi}+6f(|xi)\dfrac{\partial \xi}{\partial x}\dfrac{df}{d\xi}+ \left(\dfrac{\partial \xi}{\partial x}  \right)^3\dfrac{\partial^3 f}{\partial \xi ^3}=0
$$
$$
\dfrac{\partial \xi}{\partial t}=-c, \quad \dfrac{\partial \xi}{\partial x}=1, \quad f'=\dfrac{df}{d\xi}
$$
\begin{equation}
\Rightarrow -cf' + 6ff' + f'''=0 (181)
\end{equation}

Imponiamo ora una condizione fondamentale sul comportamento di f, ovvero che $f\to 0, f'\to 0, f''\to 0$ per $\xi \to \pm \infty$
%immagine condizione su f
\begin{equation}
(181)\Rightarrow -cf' + (3f^2)' + f'''=0(182)
\end{equation}
con le nostre ipotesi
\begin{equation}
\Rightarrow -cf + 3f^2 + f''=cost=0 (183)
\end{equation}
Moltiplica la (183)con $f'$:
$$
\Rightarrow f'f'' + 3f^2f'-cf'f=0 
$$
\begin{equation}
\Rightarrow \left(\dfrac{1}{2}f'^2\right) + (f^3)' - \dfrac{c}{2}(f^2)'=0 (184)
\end{equation}
$$
\Rightarrow \dfrac{1}{2}f'^2+f^3-\dfrac{c}{2}f^2 = cost=0
$$
a causa delle nostre condizioni su $f,f',f''$
\begin{equation}
\Rightarrow f'=\pm f\sqrt{c-2f}, \quad f\le \dfrac{c}{2} (185)
\end{equation}
Scegliamo il segno meno (altrimenti mandi $\xi$ in $-\xi$)
$$
\Rightarrow \dfrac{df}{d\xi}=-f\sqrt{c-2f}\Rightarrow\int_{\dfrac{c}{2}}^{f(\xi)}\dfrac{-df}{f\sqrt{x-2f}}=\int_{\xi_0}^{\xi}d\xi' \text{(sep. di var.)}
$$
$$
\Rightarrow\int_{f(\xi)}^{\dfrac{c}{2}}\dfrac{df}{f\sqrt{c-2f}}=\xi - \xi_0
$$
$$
f=\dfrac{c}{2\cosh^2 z},\quad -\infty<z<\infty, \quad df=-\dfrac{c\sinh z}{\cosh^3 z}dz
$$
$$
\dfrac{df}{f\sqrt{c-2f}}=\dfrac{-c \sinh z}{\cosh^3 z\dfrac{c}{2\cosh^2z}\sqrt{c-\dfrac{c}{\cosh^2 z}}}dz
$$
$$
=\dfrac{-2\sinh z dz}{\sqrt{\underset{c\sinh^2 z}{c\cosh^2z - c}}}
$$
$$
\int_{f(\xi)}^{c/2}\dfrac{df}{f\sqrt{c-2f}}=\xi - \xi_0
$$
$$
\Rightarrow - \int_z^0 \dfrac{2}{\sqrt{c}}dz'=\xi - \xi_0 \Rightarrow\dfrac{2}{\sqrt{c}}z=\xi - \xi_0
$$
$$
\Rightarrow f(\xi)=\dfrac{c}{2\cosh^2 \left(\dfrac{\sqrt{c}}{2}(\xi-\xi_0)\right)}
$$
$$
\Rightarrow u(x,t)=\dfrac{c}{2}\dfrac{1}{\cosh^2 \left[\dfrac{\sqrt{c}}{2}(x-ct-\xi_0)\right]}
$$
N.B.: $f(\xi)=\dfrac{c}{2}$\\
Dall'espressione (183) $f'' + 3f^2 - cf=0 \Rightarrow f''(\xi_0)=-\dfrac{c^2}{4}$
%immagine crescita campana
$\Rightarrow$ al crescere della velocità, l'onda diventa più alta e più sottile.
\subsection{Le trasformazioni di B$\ddot{\textbf{a}}$cklund}
Per ricavare le soluzioni delle equazioni di Liouville e di sine-Gordon introduciamo uno strumento molto utile per la risoluzione di olte e quaioni differenziali a derivate parziali non lineari, ovvero le trasformazioni di B$\ddot{a}$cklund.\\
Per semplicità ci limitiamo al caso di 2 variabili $(x,y)$.\\
L'idea è quella di cercare le soluzioni di un'equazione differenziale del tipo
$$
L u=0
$$
tramite una funzione ausiliare $v(x,y$). Questa funzione è legata a $u(x,y)$ tramite 2 equazioni
$$
B_i(u,v,\partial_x u, \partial_y u, \partial_x v, \partial_y v,x,y)=0, \quad i=1,2
$$
Queste 2 equazioni garantiscono che u sia una soluzione dell'equazioe differenziale in esame se e solo se $v(x,y)$ è soluzione di un'altra equazione differenziale
$$
Mv=0
$$
Le equazioni $B_i=0$ costituiscono la trasformazione di B$\ddot{\text{a}}$cklund. Nel caso M=L, le trasformazioni di B$\ddot{\text{a}}$cklund costituiscono un'auto-trasformazione.

%%%%%%%%%%%%%%%%%%%%%%%%%%%%%%%%%%%%%%%%%%%%%%%%%%%%%%%%%%%%%%%%%%%%%%%%%%%%%
%% inizio lezione 14/12
%%%%%%%%%%%%%%%%%%%%%%%%%%%%%%%%%%%%%%%%%%%%%%%%%%%%%%%%%%%%%%%%%%%%%%%%

\subsection{L'equazione di Liouville}

$$
\partial_x\partial_y u = e^u
$$
Considero le trasformazioni di \Backlund  date da:
$$
\left\{\begin{matrix}
\partial_x u + \partial_x v = \sqrt{2}e^{\dfrac{u-v}{2}} \\
\partial_y u - \partial_y v = \sqrt{2}e^{\dfrac{u+v}{2}}
\end{matrix}\right.
$$
Derivando la I rispetto a y e la II rispetto a x otteniamo
$$
\left\{\begin{matrix}
\partial_x\partial_y u + \partial_x \partial_y v = \dfrac{\sqrt{2}}{2}\left(\partial_y u - \partial_y v \right)e^{\dfrac{u-v}{2}} \\
\partial_x\partial_y u - \partial_x \partial_y v = \dfrac{\sqrt{2}}{2}\left(\partial_x u + \partial_x v \right)e^{\dfrac{u+v}{2}}
\end{matrix}\right.
$$
Riapplicando la trasformazione di \Backlund al 2 membro di entrambe le equazioni otteniamo
$$
\left\{\begin{matrix}
\partial_x\partial_y u + \partial_x \partial_y v = e^u \\
\partial_x\partial_y u - \partial_x \partial_y v = e^u
\end{matrix}\right.
$$
$$
I - II \Rightarrow\partial_x\partial_y v=0\quad \text{equazione delle onde}
$$
Soluzione generale: $v(x,y)=F(x)+G(y)$, con F,G funzioni arbitrarie.\\
Dalla soluzione per v andiamo a ricostruire la soluzione per u. Definiamo $z:=u+v$\\
Utilizzando le trasformazioni di \Backlund e l'equazione $\partial_x \partial_y v=0$, otteniamo
$$
\partial_x\partial_y u -\underset{=0}{\underbrace{\partial_x \partial_y v}} = \dfrac{1}{\sqrt{2}}\left(\partial_x u + \partial_x v\right)e^{\dfrac{u+v}{2}}=\dfrac{1}{\sqrt{2}}\partial_x z \cdot e^{\dfrac{z}{2}} (*)
$$
$$
\partial_x\partial_y u = \partial_x \partial_y (z-v)\underset{\partial_x \partial_Y v=0}{\underset{\uparrow}{=}}\partial_x \partial_y z \overset{(*)}{=}\partial_x \left(\sqrt{2}e^{\dfrac{z}{2}}\right)
$$
Integriamo rispetto a x e otteniamo
$$
\partial_y z = \sqrt{2}e^{\dfrac{z}{2}}+ G_1'(y) (**)
$$
con $G_1(y)$ funzione incognita da determinare.\\
Definizialo $z_1:=z-G_1$
$$
\Rightarrow \partial_y z_1 = \partial_y z- \underset{G_1'(y)}{\underbrace{\partial_y G_1}}=\sqrt{2}e^{z/2}
$$
In termini di $z_1$, l'equazione (**) diventa
$$
\partial_y z_1=\sqrt{2}e^{\frac{z}{2}}=\sqrt{2}e^{\frac{z_1}{2}}e^{\frac{G_1}{2}}
$$
Quest'equazione può essere integrata separando le variabili
$$
\underset{-2\partial_y\left(e^{-\frac{z_1}{2}} \right)}{\underbrace{e^{-\frac{z_1}{2}}\partial_y z_1}}=\sqrt{2}e^{\frac{G_1}{2}}
$$
$$
\partial_y\left(e^{-\frac{z_1}{2} }\right) = -\dfrac{1}{\sqrt{2}}e^{\frac{G_1}{2}}
$$
$$
e^{-\frac{z_1}{2}}=\dfrac{1}{\sqrt{2}}\left(F_1(x) - \int dy e^{\frac{G_1}{2}}\right)
$$
con $F_1(x)$ funzione da determinare.\\
Deriva quest'equazione rispetto a x:
$$
\Rightarrow e^{-\frac{z_1}{2}}\left(-\dfrac{1}{2}\partial_x z_1\right)=\dfrac{1}{\sqrt{2}}F_1'(x) (***)
$$
$$
\Rightarrow \partial_x z_1 = -\sqrt{2}e^{\frac{z_1}{2}}F_1'
$$
Deriva questa rispetto a y:
$$
\Rightarrow \partial_y \partial_x z_1 = -\dfrac{1}{\sqrt{2}}e^{\frac{z_1}{2}}\partial_y z_1 \cdot F_1' \Rightarrow \partial_x\partial_y z = \dfrac{1}{2}\partial_x z_1 \partial_y z_1
$$
Dato che $\partial_x\partial_y u = \partial_x\partial_y z$ possiamo riscrivere l'equazione precedente nel modo
$$
\partial_x\partial_y u =\dfrac{1}{2} \partial_x z_1 \partial_y z_1 \overset{(\Delta),(***)}{=} -F_1' e^{z_1 + \frac{G_1}{2}} \underset{z_1=z-G_1=u+v-G_1=u+F+G-G_1}{=} -F_1'e^{u+F+G-\frac{G_1}{2}}
$$
Se scegliamo $G_1=2G$ e $F_1=-\int dx e^{-F(x)}$ $(\Rightarrow F_1'=e^{-F} \Rightarrow -F_1' e^F = 1)$, troviamo che u soddisfa l'equazione di Liouville.\\
Dall'equazione
$$
e^{-\frac{z_1}{2}}=\dfrac{1}{\sqrt{2}}\left(F_1(x)-\int dy e^{\frac{G_1}{2}}\right)
$$
otteniamo
$$
z_1(x,y)=-2\log \left[-\dfrac{\int dx e^{F} + \int dy e^G}{\sqrt{2}} \right]
$$

Sappiamo che $u=z_1+G-F$ e quindi 
$$
u(x,y)=G(y)-F(x) - \log\left(\dfrac{1}{2}\left(\int dx e^{-F(x)} + \int dy e^{G(y)}\right)^2 \right)
$$

\subsection{Equazione di sine-Gordon}
$$
\partial_x\partial_t u = \sin u
$$
(Con la trasformazione di coordinate $x \mapsto \dfrac{x+t}{2}, t\mapsto\dfrac{x-t}{2}$ otteniamo $\partial_x^2 u - \partial_t^2 u = \sin u$)\\
(Paragona con Klein-Gordon $\partial_x\partial_t u = m^2 u$, per assonanza viene chiamata sine-Gordon, esiste anche sinh-Gordon)\\
-Le trasformazioni di \Backlund sono date da:
$$
\left\{
\begin{matrix}
\partial_x u + \partial_x v = 2a \sin\left(\dfrac{u-v}{2}\right) \\
\partial_t u - \partial_t v = \dfrac{2}{a} \sin\left(\dfrac{u+v}{2}\right)
\end{matrix}
\right.
$$
Analogamente a quanto fatto nel caso dell'equazione di Liouville, deriviamo la I rispetto a t e la II rispetto a x
$$
\left\{
\begin{matrix}
\partial_t\partial_x u + \partial_t\partial_x v = a (\partial_t u - \partial_t v)\cos\left(\dfrac{u-v}{2}\right)=2\sin \left(\dfrac{u+v}{2}\right) \cos \left(\dfrac{u-v}{2}\right) \\
\partial_x\partial_t u - \partial_x\partial_t v = \dfrac{2}{a}(\partial_x u + \partial_x v)\cos\left(\dfrac{u+v}{2}\right)=2\sin\left(\dfrac{u-v}{2}\right)\cos\left(\dfrac{u+v}{2}\right)
\end{matrix}
\right.
$$
usando le formule trigonometriche ottengo
$$
I + II \Rightarrow \partial_x\partial_t u = \sin u
$$
$$
I-II \Rightarrow \partial_x\partial_t v=\sin v
$$
$\Rightarrow$ in questo caso la trasformazione di \Backlund è un'autotrasformazione.\\
$\Rightarrow$ se si conosce una soluzione v dell'equazione di sine-Gordon se ne può costruire un'ealtr. Consideriamo quindi la soluzione più banale per v, ovvero $v=0$\\
$\Rightarrow$ la trasformazione di \Backlund assume la forma
$$
\left\{
\begin{matrix}
\partial_x u = 2a\sin\left(\dfrac{u}{2}\right) \\
\partial_t u = \dfrac{2}{a}\sin\left(\dfrac{u}{2}\right)
\end{matrix}
\right.
$$
Integriamo entrambi separando le variabili. p.e.
$$
\dfrac{\partial_x u }{\sin\left(\dfrac{u}{2}\right)}=2a \Rightarrow \int\dfrac{du}{\sin\left(\dfrac{u}{2}\right)}=2\log\left|\tan\dfrac{u}{4}\right| \Rightarrow \partial_x 2\log \left| \tan \dfrac{u}{4}\right|
$$
$$
\Rightarrow \left\{ \begin{matrix}
\log \left| \tan \dfrac{u}{4}\right|=2ax + f(t) \\
\log \left| \tan \dfrac{u}{4}\right|= \dfrac{2}{a} t + g(x)
\end{matrix}
\right.
$$
con $f, g$ funzioni da determinare.
Perché le due equazioni siano compatibili dobbiamo avere
$$
2ax + f(t)=\dfrac{2}{a}t + g(x) \Rightarrow 2ax-g(x)=\dfrac{2}{a}t - f(t)=k \text{ costante}
$$
$$
\Rightarrow 2 \log \left| \tan \dfrac{u}{4}\right| = 2ax + \dfrac{2}{a}t - k
$$
$$
\Rightarrow  \left| \tan \dfrac{u}{4}\right| = e^{-\frac{k}{2}}e^{ax + \frac{1}{a}t}
$$
$$
\tan \dfrac{u}{4} = \underset{=:C}{\underbrace{\pm e^{-\frac{k}{2}}}}e^{ax + \frac{t}{a}}=Ce^{ax + \frac{t}{a}}
$$
La soluzione ottenuta è data quindi da
$$
u(x,t)=4\arctan \left[Ce^{a\left(x + \frac{t}{a^2}\right)}\right] \qquad (\Box)
$$
u ha la struttura di un'onda che si propaga alla velocità $\dfrac{1}{a^2}$ (solitone)\\
Esericzio: ricavare la $(\Box)$ dall'equazione di sine-Gordon, usando l'ansatz (v. KdV) $u(x,t)=f(\underset{\xi}{\underbrace{x+ct}})$
$$
\Rightarrow \partial_x u = \dfrac{df}{d\xi}\dfrac{\partial \xi}{\partial x}=f'(\xi)
$$
$$
\partial_t\partial_xu=\partial_tf'(\xi)=f''(\xi)\underset{c}{\underbrace{\dfrac{\partial \xi}{\partial t}}} = cf''(\xi)
$$
$\Rightarrow$ sine-Gordon diventa
$$
cf''(\xi)=\sin f | f' \Rightarrow cf''f'=f'\sin f \Rightarrow \dfrac{1}{2}cf'^2 = -\cos f + \underset{\text{cost. di integrazione}}{A}
$$
$\rightarrow$ risolvi separando le variabili.\\
Infine, notiamo che applicando la trasformazione di \Backlund alla soluzioen ottenula, si può ottenere un'altra soluzione. Iterando questo procedimento si possono quindi produrre molti solitoni. Queste soluzioni vengono chiamate multisolitoniche, molto importanti nelle applicazioni nell'ottica nonlineare, come ad esempio le fibre ottiche.
%%%%%%%%%%%%%%%%%%%%%%%%%%%%%%%%%%%%%%%%%%
%% inizio lezione 20/12
%%%%%%%%%%%%%%%%%%%%%%%%%%%%%%%%%%%%%%%%%%%
\subsection{Sistemi integrabili e coppie di Lax}
-La formulazione di Lax dell'equazione di KdV: $u_t + u u_x + u_{xxx}=0$
[$u\to \lambda u, t\to \varkappa t, x\to \rho x, \lambda,\varkappa,\rho \text{ costanti} $]\\
Considera gli operatori differenziali
\begin{equation}
L=-6\dfrac{d^2}{dx^2}-u,\quad B=-4\dfrac{d^3}{dx^3} - u\dfrac{d}{dx} - \dfrac{1}{2}u_x (186)
\end{equation}
Peter Lax notò nel 1968 che l'equazione
\begin{equation}
\dfrac{dL}{dt} + \left[L,B\right]=0 (187)
\end{equation}
è equivalente all'equazione di KdV.\\
Dim:
$$
[L,B]=\left(-6\dfrac{d^2}{dx^2}-u\right)\left(-4\dfrac{d^3}{dx^3} - u\dfrac{d}{dx} - \dfrac{1}{2}u_x\right) - \left(-4\dfrac{d^3}{dx^3} - u\dfrac{d}{dx} - \dfrac{1}{2}u_x\right)\left(-6\dfrac{d^2}{dx^2}-u\right) = 
$$
$$
= 24\dfrac{d^5}{dx^5} + 6\left(u_{xx}\dfrac{d}{dx}+2u_x\dfrac{d^2}{dx^2} + u\dfrac{d^3}{dx^3}\right) + 3\left(u_{xxx} + 2u_{xx}\dfrac{d}{dx}+u_x\dfrac{d^2}{dx^2}\right) + 4u\dfrac{d^3}{dx^3} + u^2\dfrac{d}{dx} + \dfrac{1}{2}u y_x 
$$
$$
- 24\dfrac{d^5}{dx^5} - 4\left(u_{xxx} + 3u_{xx}\dfrac{d}{dx} + 3u_x\dfrac{d^2}{dx^2}+u\dfrac{d^3}{dx^3}\right) - 6u\dfrac{d^3}{dx^3} - u u_x - u^2\dfrac{d}{dx} - 3u_x \dfrac{d^2}{dx^2} - \dfrac{1}{2}u_x u
$$
conti accessori
$$
-d\dfrac{d^2}{dx^2}\left(-u\dfrac{d}{dx}f\right)=6\dfrac{d}{dx}\left(u_xf_x + u f_{xx}\right)=6\left(u_{xx}f_x + u_xf_{xx} + u_xf_{xx} uf_{xxx}\right)-6\dfrac{d^2}{dx^2}\left(-\dfrac{1}{2}u_xf\right)
$$
$$
=3\dfrac{d}{dx}\left(u_{xx}f + u_x f_x\right) =3 \left(u_{xxx} + u_{xx}\dfrac{d}{dx}+ u_{xx}\dfrac{d}{dx} +u_x\dfrac{d^2}{dx^2}\right)
$$
$$
- 4\dfrac{d^3}{dx^3} \left(uf\right) = -4\dfrac{d^2}{dx^2}\left( u_x f u f_x \right) = -4\dfrac{d}{dx}\left(u_{xx}f + 2u_x f_x + uf_{xx} \right) = -4\left( u_{xxx} + u_{xx}f_x +2u_{xx}f_x + 2 u_xf_{xx} + u_xf_{xx} + uf_{xxx} \right) 
$$
$$
- u\dfrac{d}{dx}(uf) = u(u_xf + uf_x)
$$
ottengo quindi
$$
\Rightarrow [L,B] = 3u_{xxx} - 4u_{xxx}-uu_x = -u_{xxx} - u u_x
$$
$$
\dfrac{dL}{dt}=\dfrac{d}{dt}\left(-6\dfrac{d^2}{dx^2} - u\right) = -u_t
$$
$\left[\dfrac{d}{dt}\left(-6\dfrac{d^2}{dx^2}\right)\equiv 0\right.$ perchè l'operatore $\dfrac{d^2}{dx^2}$ agisce su funzioni che dipendono solo da x$\left. \right]$.\\
La (187) viene chiamata \underline{Equazione di Lax}.\\
Come vedremo sotto in un esempio concreto, questa forma dell'equazione di KdV implica (fra l'altro) che gli autovalori di L non dipendono da t. (ndr in pratica gli autovalori di L mettono a disposizione delle costanti del moto, analoghe al caso della maccenica classica). Un altro modo per ottenere la (187) è come condiione di compatibilità dei 2 problemi lineari
\begin{equation}
L\phi =\lambda \phi \qquad \text{ e } \qquad \phi_t = B\phi (188)
\end{equation}
dove $\lambda$ è un parametro fissato.\\
Infatti: $(188)\Rightarrow \dfrac{\partial}{\partial t}(L\phi) = \dfrac{dL}{dt}\phi + L\phi_t = \dfrac{dL}{dt}\phi + LB\phi$, e inoltre $\dfrac{\partial}{\partial t}(L\phi) = \dfrac{\partial}{\partial t}(\lambda \phi) = \lambda \phi_t = \lambda B \phi = B\lambda\phi = BL\phi$
$$
\Rightarrow \dfrac{dL}{dt} + LB = BL
$$
-L'importanza chiave dell'osservazione di Lax è che qualsiasi equazione differenziale che può essere scritta nella forma (187) ( per certi operatori L e B) ha automaticamente molte delle proprietà dell'equazione di KdV, come p.e. un numero infinito di leggi di conservazione.\\
$\left[u_i(t) i=1,\dots,N\right.$ \\
Per avere integrabilità nel senso di Liouville, servono n costanti del moto $I_i$ in involuzione, cioè  deve valere 
$$ 
\left\{I_i,H \right\}_{PB}=0, \left\{I_i,I,j\right\}_{PB}=0
$$
Facciamo diventare continuo l'indice i, $i\mapsto x, u_i(t)\mapsto u(x,t) \left.\right]$\\
-KdV come condizione di curvatura zero:\\
L'equazione$L\phi = \lambda \phi $ è del 2 ordine. Riscrivila come equazione del 1 ordine nel seguente modo:
$$
\phi_1:=\phi, \quad \phi_2:=\phi_x
$$
\begin{equation}
L\phi = \lambda \phi \Leftrightarrow \partial_x\left(\begin{matrix}
\phi_1 \\
\phi_2
\end{matrix}\right) = \left(\begin{matrix}
0 & 1 \\
-\dfrac{u+\lambda}{6} & 0
\end{matrix}\right)\left( \begin{matrix}
\phi_1 \\
\phi_2
\end{matrix} \right) \equiv U\left(\begin{matrix}
\phi_1 \\
\phi_2
\end{matrix}\right) (189)
\end{equation}
Dim:
$$
\left(\begin{matrix}
0 & 1 \\
-\dfrac{u+\lambda}{6} & 0
\end{matrix}\right)\left( \begin{matrix}
\phi_1 \\
\phi_2
\end{matrix} \right) = \left(\begin{matrix}
\phi_2 \\
-\dfrac{u+\lambda}{6}\phi_1
\end{matrix}\right) = \left(\begin{matrix}
\phi_x \\
-\dfrac{u+\lambda}{6}\phi
\end{matrix}\right)\overset{!}{=} \partial_x \left(\begin{matrix}
\phi \\
\phi_x
\end{matrix}\right)
$$
2 riga: 
$$
-\dfrac{u+\lambda}{6}\phi = \dfrac{\partial^2 \phi}{\partial x^2} \Leftrightarrow \underset{L}{\underbrace{\left(-6\dfrac{\partial^2}{\partial x^2} - u\right)}}\phi = \lambda \phi \quad \checkmark
$$
Inoltre:
$$
\phi_{1t}=\phi_t=B\phi \overset{(186)}{=} - 4\phi_{xxx} - u\phi_x - \dfrac{1}{2}u_x\phi =-4 \left(-\dfrac{u+\lambda}{6}\phi\right)_x - u\phi_x - \dfrac{1}{2}u_x\phi 
$$
$\left[\text{ ho usato } \phi_{xx}=-\dfrac{u+\lambda}{6} \phi\right]$
$$
=\dfrac{1}{6} u_x \phi + \left(\dfrac{2}{3}\lambda - \dfrac{1}{3}u\right)\phi_x
$$
\begin{equation}
=\dfrac{1}{6} u_x \phi_1 + \left(\dfrac{2}{3}\lambda - \dfrac{1}{3}u\right)\phi_2 (190)
\end{equation}
$$
\phi_{2t}=\phi_{xt}=\phi_{1xt}=\phi_{1tx} \overset{(190)}{=} \dfrac{1}{6}u_x\phi_{1x} + \dfrac{1}{6}u_{xx}\phi_1 + \left(\dfrac{2}{3}\lambda - \dfrac{1}{3}u\right)\phi_{2x} - \dfrac{1}{3}u_x\phi_2
$$
$$
\overset{(189)}{=}\dfrac{1}{6}u_x\phi_2 + \dfrac{1}{6}u_{xx}\phi_1 - \left(\dfrac{2}{3}\lambda - \dfrac{1}{3}u\right)\left(\dfrac{1}{6}\lambda + \dfrac{1}{6}u \right)\phi_1-\dfrac{1}{3}u_x\phi_1
$$
\begin{equation}
\left(-\dfrac{1}{9}\lambda^2 - \dfrac{1}{18} \lambda u + \dfrac{1}{18}u^2 + \dfrac{1}{6}u_{xx}\right)\phi_1 - \dfrac{1}{6}u_x\phi_2 (191)
\end{equation}
Le (190),(191) possono essere scritte in forma matriciale
%la matrice ha la croce in mezzo a separare i 4 componenti, quella piu lunga era messa su due righe
\begin{equation}
\partial_t\left(\begin{matrix}
\phi_1 \\
\phi_2
\end{matrix} \right) = \left( \begin{matrix}
\dfrac{1}{6}u_x & \dfrac{2}{3}\lambda - \dfrac{1}{3}u \\
-\dfrac{1}{9}\lambda^2 - \dfrac{1}{18} \lambda u + \dfrac{1}{18}u^2 + \dfrac{1}{6}u_{xx} & - \dfrac{1}{6}u_x
\end{matrix}\right)=\left(\begin{matrix}
\phi_1 \\
\phi_2
\end{matrix} \right) \equiv V \left(\begin{matrix}
\phi_1 \\
\phi_2
\end{matrix} \right)(192)
\end{equation}
Condizione di compatibilità fra le (189) e (192):
$$
(189)\Rightarrow \partial_t \partial_x \left(\begin{matrix}
\phi_1 \\
\phi_2
\end{matrix} \right) = \dfrac{\partial U}{\partial t}\left(\begin{matrix}
\phi_1 \\
\phi_2
\end{matrix} \right) + U\partial_t \left(\begin{matrix}
\phi_1 \\
\phi_2
\end{matrix} \right) = \dfrac{\partial U}{\partial t}\left(\begin{matrix}
\phi_1 \\
\phi_2
\end{matrix} \right) + UV \left(\begin{matrix}
\phi_1 \\
\phi_2
\end{matrix} \right)
$$
$$
(192) \Rightarrow \partial_x\partial_t\left(\begin{matrix}
\phi_1 \\
\phi_2
\end{matrix} \right)=\dfrac{\partial V}{\partial_x}\left(\begin{matrix}
\phi_1 \\
\phi_2
\end{matrix} \right) + V\partial_x \left(\begin{matrix}
\phi_1 \\
\phi_2
\end{matrix} \right) =\dfrac{\partial V}{\partial x}\left(\begin{matrix}
\phi_1 \\
\phi_2
\end{matrix} \right) + VU \left(\begin{matrix}
\phi_1 \\
\phi_2
\end{matrix} \right)
$$
e quindi
\begin{equation}
\dfrac{\partial U}{\partial t} - \dfrac{\partial V}{\partial x} + \left[U,V\right]=0(193)
\end{equation}
I coefficienti di $\lambda^3,\lambda^2 e \lambda$ nella (193) sonoo identicamente zero (esercizio). Il coefficiente di $\lambda^0$ dà l'equazione matriciale
$$
\left(\begin{matrix}
0 & 0 \\
-\dfrac{1}{6}\left(u_t + u_x u + u_{xxx}\right) & 0
\end{matrix} \right) = \bar{\bar{0}} \rightarrow \text{ KdV}
$$

$\left[\right.$ queste due equazioni sono chiamate \emph{"coppia di Lax"}, le parentesi tonde identificano delle derivate covarianti

$$
\begin{matrix}
\left(\partial_x - U \right)\left(\begin{matrix}
\phi_1 \\
\phi_2
\end{matrix} \right) =0 \\
\left(\partial_t - V \right)\left(\begin{matrix}
\phi_1 \\
\phi_2
\end{matrix} \right) = 0
\end{matrix}
$$
U,V sono connessioni, la (193) la curvatura di questa connessione
$\left.\right]$

L'idea centrale nella teoria dei sistemi integrabili è di rappresentare un problema nonlineare come condizione di compatibilità fra 2 equazioni lineari di una coppia di Lax.\\
Alcuni altri esempi:\\
Scelta comune per U: 
\begin{equation}
U=\left(\begin{matrix}
i\lambda & q \\
r & -i\lambda
\end{matrix}\right) \equiv \lambda U_1 + U_0 (194
\end{equation}
(probema spettrale di AKNS ( Ablowitz, Kamp,Newell,Segur))
$q=q(x,t), r=r(x,t).$\\
Supponiamo per esempio che V sia un polinomio quadratico in $\lambda$
\begin{equation}
V=\lambda^2 V_2 + \lambda V_1 + V_0 (195)
\end{equation}
(194),(195) nella (193): $\Rightarrow$ coefficienti di $\lambda^3$ :
\begin{equation}
\left[U_1,V_2\right]=0 (196a)
\end{equation}
coefficienti di $\lambda^2$
\begin{equation}
-\dfrac{\partial V_2}{\partial x} + \left[U_1,V_1\right] + \left[U_0,V_2\right]=0 (196b)
\end{equation}
coefficienti di $\lambda$
\begin{equation}
\dfrac{\partial U_1}{\partial_t }-\dfrac{\partial V_1}{\partial x} + \left[U_1,V_0\right] + \left[U_0,V_1\right]=0 (196c)
\end{equation}
coefficienti di $\lambda^0$
\begin{equation}
\dfrac{\partial U_0}{\partial_t }-\dfrac{\partial V_0}{\partial x} + \left[U_0,V_0\right] =0 (196d)
\end{equation}
noto che $U_1=i\sigma_3$, con la matrice di Pauli
$$
\sigma_3=\left(\begin{matrix}
1 & 0 \\
0 & -1
\end{matrix}\right)
$$
La (196a) può essere risolta ponendo $V_2=U_1$\\
(196b): scegli $V_1=U_0 \rightarrow$ soddisfatta.\\
c) d) soddisfatte per 
$$
V_0=\left(\begin{matrix}
iqr/2 & - iq_x/2 \\
ir_x/2 & - iqr/2
\end{matrix}\right)
$$
se vale 
$$
\left(\begin{matrix}
0 & q_t + iq_{xx}/2 - iq^2 r \\
r_t - ir_{xx}/2 + ir^2 q & 0
\end{matrix}\right)=\bar{\bar{0}}
$$
$\rightarrow$ sistema nonlineare accoppiato $ iq_t - \dfrac{1}{2}q_{xx} + q^2 r=0, ir_t - \dfrac{1}{2} r_{xx} r^2 q =0 $.
Questo sistema è consistente con $r=\pm q^*$ che dà $ iq_t - \dfrac{1}{2}q_{xx}\pm |q|^2 q=0$ (e il suo complesso coniugato)$\rightarrow$ equazione di Schr$\ddot{\text{o}}$dinger  nonlineare.





\chapter{Carica immagine}
problema: sfera con raggio R, messa a terra, con carica puntiforme q in $\bar{r}_0$\\
%immagine sfera a terra con cariche, un casino

Per motivi di simmetria, la carica immagine giace sulla retta che collega O e q.\\
Potenziale in P
\begin{equation}
\Phi = \dfrac{q}{a}+\dfrac{q'}{b} = 0
\end{equation}
prendi p=A:
\begin{equation}
\dfrac{q}{R-r_0} + \dfrac{q'}{d}=0
\end{equation}

$$
(2)\Rightarrow \dfrac{q'}{q}=-\dfrac{d}{R-r_0}
$$
prendo p=B:
\begin{equation}
\dfrac{q}{R+r_0}+\dfrac{q'}{d+2R}=0
\end{equation}
$$
(3)\Rightarrow \dfrac{q'}{q}=\dfrac{d+2R}{R+r_0}
$$
$$
\Rightarrow \dfrac{d}{R-r_0}= \dfrac{d+2R}{R+r_0}
$$
$$
\Rightarrow d(R+r_0)=(d+2R)(R-r_0)%manca roba?
$$
$$
\Rightarrow dR + 2dr_0 = dR - dr_0 - 2R^2 - 2Rr_0
$$
\begin{equation}
\Rightarrow d=\dfrac{R(R-r_0)}{r_0}
\end{equation}
e quindi
\begin{equation}
q' = \dfrac{dq}{Rr_0} = -\dfrac{q}{Rr_0}\dfrac{R(R-r_0)}{r_0}=-\dfrac{qR}{r_0}
\end{equation}
Verificare che la (1) vale $\forall p$
$$
(1)\Leftrightarrow \dfrac{q'}{q} = -\dfrac{b}{a}=(5)= \dfrac{R}{r_0} \Leftrightarrow \dfrac{b}{a}=\dfrac{R}{r_0}
$$
Calcolo $\alpha$ : $a^2=r_0^2 + R^2 = 2r_0 R\cos\alpha \implies \cos\alpha =\dfrac{r_0^2 + R^2 - a^2}{2r_0R}$\\
D'altra parte: 
$$
\cos\alpha=\dfrac{r_0+l}{R} \implies \dfrac{r_0+l}{R}=\dfrac{r_0^2 +R^2 - a^2}{2r_0 R} \implies l=\dfrac{-r_0^2 + R^2 - a^2}{2r_0}
$$
$$
c^2=R^2 - (r_0+l)^2 = R^2 - \left(\dfrac{r_0^2 + R^2 - a^2}{2r_0} \right)^2=b^2 - (d +R-r_0 -l)^2 = b^2 -(d+R)^2 - (r_o+l)^2 + 2(d+R)(r_0+l)
$$
$$
\Rightarrow R^2 = b^2 - (d+R)^2 + 2\underbrace{(d+R)}\underbrace{(r_0+l)}
$$
$$
\dfrac{R^2}{r_0^2} \qquad \qquad \dfrac{r_0^2 + R^2 - a^2}{2r_0R}
$$
$$
\Rightarrow R^2 = b^2 - \dfrac{R^4}{r_0^2} + \dfrac{R^2}{r_0^2}(r_0^2 + R^2 - a^2) \implies \dfrac{R}{r_0^2}a^2 = b^2 \implies \dfrac{b}{a}=\dfrac{R}{r_0} \qquad \checkmark
$$






\end{document}









